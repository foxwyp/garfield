\documentclass[doublespacing]{elsarticle}
\usepackage{amssymb}
\usepackage{amsmath}
\usepackage{bigdelim}
\usepackage{multirow}
\usepackage{hyperref}
\usepackage{graphics}
\usepackage{algorithm}
\usepackage{algorithmic}
\usepackage{subfigure}
\usepackage{booktabs}
\usepackage{url}
\usepackage{natbib}
\usepackage{todonotes}

% \usepackage{algorithmicx}
\journal{Expert Systems with Applications}

\newtheorem{definition}{Definition}


\begin{document}

\begin{frontmatter}

  \title{A Business Process Driven Model For Automating Process Knowledge Extraction}

  \author[buaa]{Jun Wang\corref{cor1}}
  \ead{king.wang@buaa.edu.cn}

  \author[buaa]{Yunpeng Wu}
  \ead{yunpeng.wu@sem.buaa.edu.cn}

  \author[buaa]{Guannan Liu}
  \ead{wd33082208@126.com}

  \author[buaa]{Huili Wang}
  \ead{wd33082208@126.com}

  \cortext[cor1]{Corresponding author}

  \address[buaa]{School of Economics and Management, Beihang University, \par
    Beijing 100083, P.R. China}

  \begin{abstract}
    Knowledge Management Process (KMP) is the core issue of Knowledge
    Management (KM) in organizations. It needs continuous improvement
    and is closely related with corresponding Business Process
    (BP). Many knowledge management practices in organizations, however,
    are less effective since organizations failed to integrate KMP with
    BP. This study proposes a methodology for knowledge management
    process modeling based on business process-driven, providing better
    support for the collaboration between BP and KMP and in turn
    continuously optimize knowledge management process. This methodology
    includes the architecture of the business process driven knowledge
    management process modeling system, the depiction of business
    process, a knowledge management process modeling methodology based
    on XML, the depiction and storage for the driven-rules from business
    process to knowledge management process and so on. Finally we give a
    case study of an aviation manufacturing enterprise to illustrate how
    this methodology can support the collaboration between its
    production design process and knowledge management process, in which
    an implementation evaluation will also be given to verify the
    methodology.

  \end{abstract}

  \begin{keyword}
    Knowledge management process; Business-driven; Process modeling;
    Driven rules
  \end{keyword}
\end{frontmatter}

\section{Introduction}
\label{sec:introduction}
Among many organization assets,knowledge is treated as a critical
driving force for attaining organization performance goals because
knowledge facilitates the better business decision makings in a timely
fashion[1]. There has been a growing interest in treating knowledge as
significant organizational resource[2].


Knowledge management characterizes a deliberate and systematic
approach to ensure the full utilization of the knowledge base of an
organization[1]. Effective knowledge management is an increasingly
important source of competitive advantage and a key to the success of
modern organizations[4], especially for intelligence-intensive
enterprises. Therefore knowledge management is receiving increasing
focus from scholars and organizers.  Many knowledge management efforts have been largely concerned with capturing, codifying, storing and disseminating knowledge that is held by people in organizations in the pursuit of strategic competitiveness. However, since knowledge is created and utilized during the execution of business processes, if knowledge is separated from the business process context, it does not lead to the ability to take the right action for target performance[1]. Knowledge management is an evolving practice [6]. Desouza and Awazu argued that the rare combination with organizations has limited the use of knowledge management when considered in the context of organizations [7, 8].  Without appropriate contextual
information, knowledge can be isolated from other relevant
knowledge, consequently resulting in a limited or distorted
understanding[9].

The key issue of combining shared context knowledge into existing
enterprise knowledge is to  to describe and organize
content so that intended end users are aware of its existence and can
easily access and apply this content(Dalkir 2005). Usually these
context knowledge reside in the business process, that is,  a collection of
interdependent activities or tasks organized to achieve specific
business goals. These process-related knowledge are important to
effective knowledge management. However, extracting
process knowledge is not easy. The first problem is the rapid changing
business process leads to fast expiring of the process knowledge. In order to survive in the current
competitive and global business environment, most organizations are
struggling to change their existing business processes into more agile,
product- and customer-oriented structures[11]. While continuing changing
business process create new business knowledge, process knowledge need
to be renewed to adaptive the change. The second problem is process
knowledge are easily lost. Process konwledge are created or generated
during business process. These konwledge are more  implicit than the
direct business knowledge output. Knowledge works may unintentionally
neglect process knowledge or just do not know how to extract these
knowledge. Moreover, codifying process knowledge burdens knowledge
workers since these knowledge are not directly related to business,
thus knowledge works are less stimulated.  Also most organizations are lack of
well defined meta-knowledge to represent the process
knowledge. Without help of automating tools, extracting process
knowledge is a strenuous and  inefficient.

In this paper, we propose a business process driven methodology
\todo{or framework or model}
for  process knowledge extracting and modeling, providing better support for
the integration between business process and knowledge management
process and in turn continuous optimizing knowledge management
process. This framework models the business process and a set of
pre-defined driven rules to elicit process knowledge. When business
process change, corresponding process knowledge are automated
recreated according to driven rules and stored in knowledge base. 
 We also
give a case study of an aviation manufacturing enterprise to illustrate
how this methodology can support the integration between its
production design process and knowledge management process, in which an
implementation evaluation will also be given to verify the
methodology.

The remainder of the paper is organized as follows. Section 2
identifies the needs for better
business driven automation support for  process knowledge
modeling as well as briefly reviews the related works. Section 3
describes the architecture of the business process driven process knowledge modeling system. Section 4 represented the business
process models using UML. Section 5 depicts the knowledge management
process modeling methodology based on XML as well as its
implementation. The driven rules classification, inference principles
and the storage of driven-rules based on the
relational database between
business process and knowledge management process are discussed in
Section 6. To better understand the methodology, section 7 depicts a
case study taking an aerial product enterprise as the background. We
conclude the paper in Section 8 by summarizing our research
contributions and pointing out future research directions.



\section{ The Architecture Of Business  Driven Process Knowledge
    Management System}
\label{sec:backgr-liter}
The interests in the notion of process-oriented knowledge management
(PKM) from academia and industry have been significantly
increased[15]. Process konwledge are organizational related konwledge. It can be
divided to three parts:1)the role of organization memory, 2)structure
of the organization, and 3)organization incentives[nissen]. The change
of  business
process affect the  contextual factors, giving rise to consequent
alternation of roles, policies and rules of organization.
 The advantage of process-oriented
knowledge management is that it can help users avoid information
overload and concentrate on important information which is essential
for company value chains[31] It can also improve the usability of
knowledge in company and the efficiency of implementing knowledge
management system[32]. 



The process-oriented knowledge management system should extract the
process knowledge  in real time, storing these knowledge for future
retrieve and application. 
 Ideally, when new
business requirements come into being, changing business rule should
intrigue process knowledge re-generating immediately  To achieve the goal, organizations have to
collect knowledge of driven-rules, adjusting process knowledge  to
reflect new business process. Process knowledge can be embedded into
the organization ontology[1], so that when new business knowledge object created
the system can  obtain the process knowledge either. The weakness of
this method is the system can not fully integrated with workflow
engine, which is a important source of process knowledge(lai and
fan). In this paper, we adopt a
relatively "dynamic" method. The system is seem as a expert system,
extracting knowledge from, instead of human computer, business process.  

There have been various efforts to introduce the process concept to knowledge management (Knowledge Management) or the knowledge concept to business process management (BPM) in order to combine the advantages of the two paradigms[16,17,18,19,20].In the context of IPM, I. Choi, J. Jung and M. Song proposed a comprehensive framework for integrating Knowledge Management and BPM was proposed[19]. Lai \& Fan, suggested the importance of integrating knowledge management and BPM (business process management) towards PCKM (Process Centered Knowledge Management) in order to maximize and optimize business performance [13, 14]. As well Kwan Hee Han, Jun Woo Park (2009) proposed a concept of process-centered knowledge management (PCKM) as well as a framework for process-centered knowledge model[1]. K.H. Han and Maurer, F et al. proposed a MILOS system in order to integrate Knowledge Management with project-planning and process coordination[26,27]. Jung et al., J. Jung, I. Choi and M. Song, 2007 further proposed integration architecture for integrating knowledge management systems (KMSs) and business process management systems (BPMSs) to combine the advantages of the two paradigms based on a comprehensive framework that reflects lifecycle requirements of both Knowledge Management and BPM。The architecture, which is comprehensive since it is derived from the extended requirements from the lifecycle perspective, will provide a basis for research and development of process-oriented knowledge management systems. A prototype system is presented to demonstrate the feasibility of the proposed architecture[15]. 
Jablonski, Horn, and Schlundt (2001) combined business process modeling with Knowledge Management and developed process-oriented KMS prototype for car manufacturing[28]. Woitsch and Karagiannis, 2005 proposed a service-based approach in which the Knowledge Management services separate technical implementation and conceptual requirements. However, the relationship between Knowledge Management services and business process was not shown in their study[29]. 

\end{document}