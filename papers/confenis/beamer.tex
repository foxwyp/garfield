\documentclass[slidestop,compress,mathserif,table]{beamer}
\usepackage{pgf}
\usetheme[blue,compress,numbers,mylogo]{Trondheim}
\usepackage{CJK}


   
\begin{document}

\title{Knowledge Inventory Management Using Actuarial Method}
    \author{Yunpeng Wu, Lu Liu, Yin Guo}
    \institute[buaa]{Beihang University}
    \begin{frame}
      \titlepage
    \end{frame}

\section{Introduction}

\begin{frame}
  \frametitle{Why knowledge inventories are important}
Knowledge inventory:
\begin{quote}
 A small number of specialized competencies maintianed by the
 individuals and groups that make up the organization.
\end{quote}
Facing challengies:
\begin{itemize}
   \item Changing envioronment
     \item Proper responses
       \item Managers' myopia
         \begin{itemize}
         \item Temporal myopia
           \item Spatial myopiax
         \end{itemize}
\end{itemize}
\begin{block}{Knowledge inventory management}
 Knowledge inventory management involves acquiring, retaining, deploying, idling, and abandoning knowledge over time.
\end{block}
\end{frame}


\section{Review}
\begin{frame}
\frametitle{Former methodologies and shortcoming}
\begin{itemize}
\item DCF method

Unable to deal with investment decision under uncertainty.
\item Real option method
  Good to deal with investment decision under uncertainty, but the
  restrictive modeling assumptions fail this method to deal with
  knowledge inventories problems.
  \begin{itemize}
  \item  Accident will happen definitely in the future but not at a
      definite day, so there is no maturity day needed in real option
      model.
      \item When an accident happens, there is no right to choose but
        an obligation.
        \item Investment in advance may not bring any profit to
          organizations, but can reduce the contingent loss brought by the accident.
  \end{itemize}
\end{itemize}
\end{frame}

\section{ACTUARIAL MODEL}

\begin{frame}
  \frametitle{Model Assumptions}
  \begin{itemize}
  \item Accident will incur in the future but not a definite day as we
    point out above.
    \item Investing in some specific technology knowledge initially will cover
      the whole loss if the accident incurs. That means there is no
      extra expense needed when it happens.
      \item Every technology knowledge has its lifecycle. Investment in a
        technology will bring no returns after the technology has
        ended its lifecycle.
        \item The loss distributions at any time during n years are identical.
  \end{itemize}
\end{frame}

\begin{frame}
  \frametitle{Math}
We define $T(0)$ as the length from current time to some time in the
future when accident happens, thus $T(0)$ is a random variable.
\begin{exampleblock}{Equation}
  \begin{equation}
    S_0=P[T(0)\ge t], t \ge 0
  \end{equation}
\end{exampleblock}
Characteristics of S:
\begin{itemize}
\item $S(0)=0$ Accident will incur only in the future, so there is no
  uncertainty at current time.
  \item $S_0(t)$ is a decreasing function.
    \item  $S_0(t)$ is usually a continuous function and derivable.
\end{itemize}
\end{frame}


\begin{frame}
  \frametitle{Payment Incurrence Until The End Of Year}

We note $ \ _{t|1}q_0=P[t<T(0) \le t+1], t \ge 0 $
  \begin{exampleblock}{Equation}
    \begin{equation}
      A=\sum_{t=0}^{n-1}v^{t+1} \ _{t|1}q_0E(X)=\sum_{t=0}^{n-1}v^{t+1}[S_0(t)-S_0(t+1)]E(X)
    \end{equation}
Where:

$X$ = the loss of accident. X is a random variable

$E(X)$ = expectation of X

$v^t$ = discounted rate from period t to period 0

n = lifecycle of the technology (n=1, 2, 3 \ldots)
  \end{exampleblock}
\end{frame}

\begin{frame}
  \frametitle{Payment Incurrence At Accident Time}
  For some reasons, when the accident happens, the payment has to be made immediately instead of waiting until the end of the period to maintain operation.

  \begin{exampleblock}{Equation}
    \begin{equation}
      \label{eq:1}
      \mu _0=lim_{dx \rightarrow 0} \frac{P[x< T(0) \le x+dx|T(0)>x]}{dx}
    \end{equation}
    \begin{equation}
      \label{eq:2}
      A=\int_0^nS_0(t)\mu_tE(X)v^tdt
    \end{equation}
    
  \end{exampleblock}
\end{frame}

\begin{frame}
  \frametitle{Deferring Investment}
 If the organization determines to defer the investment to the end of T year, then the actuarial value from 0 to T is lost but capital time value can be achieved.
 \begin{exampleblock}{Actuarial value}
     \begin{equation}
    \label{eq:3}
    A=\sum_{k=0}^Tv^{k+1}[s_0(k)-s_0(k+1)]E(X)
  \end{equation}
\end{exampleblock}

\begin{exampleblock}{Time value}
    \begin{equation}
      \label{eq:4}
      A^1=\sum_{t=1}^Tar/(1+r)^t
    \end{equation}
  \end{exampleblock}
\end{frame}

\section{Discussion}

\begin{frame}
  \frametitle{Future works}
  \begin{itemize}
  \item The insurance term which mapped to knowledge  lifecycle is hard to determine.
    \item we assume interest rate is constant during time period we consider, which is not always true in actual world.
      \item the contingent loss is not easily to estimate and the distribution is unclear because of lack of samples.
  \end{itemize}
\end{frame}

\end{document}

%%% Local Variables: 
%%% mode: latex
%%% TeX-master: t
%%% End: 
