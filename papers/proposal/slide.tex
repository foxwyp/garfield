

\def\pgfsysdriver{pgfsys-dvipdfm.def}
%%%
\documentclass[slidestop,compress,mathserif,blue,compress]{beamer}
\usepackage{pgf,tikz}
\usetikzlibrary{shapes,snakes,arrows}
\usepackage{beamerthemesplit}
      
% For Xetex      %%%%%%%%%%%%%%%%%%
\usepackage{fontspec}
\usepackage{xunicode}
\usepackage{xltxtra}
\setmainfont{Adobe Song Std}

\setsansfont{Adobe Heiti Std}  %beamer 默认模板字体是 sans serif,可以在这里将其设置中文字体
     
\setmonofont{Bitstream Vera Sans Mono}

\font\songti="Adobe Song Std"

%\usecolortheme{lily}
\usetheme{Boadilla}
\setlength{\parskip}{10pt}
\title{实践社区成员知识协同的内部动机因素研究}
\author{吴云鹏}
\institute{北京航空航天大学 \\ 经济管理学院}

\begin{document}

\begin{frame}
\titlepage
\end{frame}

\begin{frame}
  \frametitle{研究背景}
  \begin{enumerate}
  \item 实践社区
   \item 知识协同
   \item  内部动机
  \end{enumerate}
\end{frame}

\begin{frame}[shrink]
  \frametitle{研究现状}
  \begin{enumerate}
  \item  实践社区中的知识协同
  \item  动机理论
    \begin{itemize}
    \item 马斯洛:需要层次
\item 赫兹伯格:激励和保健因子
\item  奥德佛:ERG
\item   弗洛姆:效价/期望
\item     波特-劳勒:绩效-满意度
   \item 费斯廷格:动机失调
\item   亚当斯:公平理论

    \end{itemize}
  
 
   
    
 
    
  \item  实践社区中的动机因素研究
  \end{enumerate}

\end{frame}

\begin{frame}[allowframebreaks]
  \frametitle{文献综述}
  
    \begin{tabular}[center]{|c|p{8cm}|c|c|}
\hline\small

      序号&期刊&时间&检索 \\\hline
     1&Management Science &2007&SCI \\\hline
    2&Management Science &2007& SCI \\\hline
     3&Management Science &2007&SCI \\\hline
     4&MIS quartely&2005&SCI \\\hline
     5&MIS quartely&2005&SCI \\\hline
    6&Journal of Management Studies  &2008&SSCI \\\hline
  7&Journal of Information Science  &2007&SSCI  \\\hline
  8&lecture notes  of computer science &2007&SCI  \\\hline
  9&Information \& Management&2007&SSCI \\\hline
  10&International Journal of Human-Computer Studies&2007&SCI \\\hline


    \end{tabular}
    
\end{frame}

\begin{frame}[allowframebreaks]
  \frametitle{文献综述}
  
    \begin{tabular}[center]{|c|p{8cm}|c|c|}
\hline\small

      序号&期刊&时间&检索 \\\hline
     11&International Journal of Human-Computer Studies&2007&SCI \\\hline
12&Computers in Human Behavior&2007&SCI \\\hline
13&Computers in Human Behavior&2008&SCI \\\hline
 14&Strategic Management Journal &2007&SSCI \\\hline
  15&Omega&2008&SSCI \\\hline
  16&Decision Support System&2006&SCI \\\hline
 17&Hawaii International Conference on System Sciences &2007& \\\hline
18&Hawaii International Conference on System Sciences &2007& \\\hline
19&Journal  of The  American Scociety  for  Information Science and Technology &2008& \\\hline
20&Asia Pacific Journal of Management&2008& \\\hline


    \end{tabular}
    
\end{frame}

\begin{frame}
  \frametitle{基本假设}
\vfill
  \begin{block}{不同类型的人有不同的动机}
     在实践社区中,不同类型的人有不同的共享方式、不同的社会地位,从
     而受到他人协同活动产生的影响也不同。不同的人参与协同活动的动机
    各不相同,甚至同一种动机因素对于不同类型的人作用方式也不尽相同。
  \end{block}
  \begin{block}{理性行为假设}
    每个人在特定环境下的行为与其动机是一致的,也就是说,有什么样的
    动机就会产生相应的行为。因此我们可以通过对行为的研究,分析、推
    导出人的动机。
  \end{block}
  \begin{block}{动机变化假设}
    其他人的行为会引起行为主体动机的变化。
  \end{block}
\end{frame}
\vfill
\begin{frame}
  \frametitle{研究内容}
  
  \begin{enumerate}
  \vfill
   \pause \item 社会网络分析
   \pause \item  动机因素模型
   \pause \item  实证分析
   \begin{itemize}
   \item 主观数据
    \item  客观数据
   \end{itemize}
 \vfill
  \end{enumerate}
\end{frame}

\begin{frame}
  \frametitle{关键问题}
  \begin{enumerate}
\item 建立行为主体参与实践社区知识协同活动的动机因素模型,描述协
           同行为与动机之间的关系,根据该模型提 出相应的假设,进而开展
           实证研究。
\item  分析实践社区中行为主题的社会关系,建立其社会网络模型,区分
            不同类型的协同者,描述出其不同的行为模式。
\item 建立行为主体绩效模型。该模型以内容产出为因变量,协同行为、
            时序因素和行为模式为自变量,以此模型分析协同动机的影响因素。
\end{enumerate}
\end{frame}

\begin{frame}
  \frametitle{研究难点}
\vfill
  \begin{enumerate}
\item 取得足够数量的样本。
  \begin{itemize}
  \item 广泛发送调查邀请。
   \item 控制问卷题目数量。
  \end{itemize}
\item  问卷设计。
  \begin{itemize}
  \item  邀请相关老师和同学参与设计。
  \item 小范围测试。
   \item  采用其他研究设计的问题。
  \end{itemize}
\item  大数据量处理。

\end{enumerate}
\vfill
\end{frame}
\begin{frame}
  \frametitle{研究方法}
  \vfill
本项目拟采取理论研究与实证研究相结合、动态分析和静态分析相结
合的研究思路开展工作。具体的研究方法如下:
\begin{enumerate}
\item  调查问卷(Questionnaires)。通过向受访者发放问卷,调查
            了解他们参与实践社区协同行为的内部动机。
\item 分类研究(Classification)。使用网络数据分析实践社区的参
            与主体的社会网络,区分其不同特征,分析其行为模式。
\item 数学建模(Correlational study)。使用数学模型描述维基百
            科社区的内容创造和协同行为。
\item 因果分析(Causal Research )。协同动机可能会受到多种因
            素的影响,通过因果分析,判定某一因素对于协同动机的影响
            以及作用机制。
\end{enumerate}
\vfill
\end{frame}

\begin{frame}[shrink]
  \frametitle{技术路线}
  \tikzstyle{decision} = [diamond, draw, fill=blue!20,
    text width=4.5em, text badly centered, node distance=3cm, inner sep=0pt]
\tikzstyle{block} = [rectangle, draw, fill=blue!20,
    text width=5em, text centered, rounded corners, minimum
    height=4em,minimum width=50mm]
\tikzstyle{line} = [draw, -latex']
\tikzstyle{cloud} = [draw, ellipse,fill=red!20, node distance=3cm,
    minimum height=2em]
    
    
\begin{tikzpicture}[node distance = 3cm, auto]
    % Place nodes
    \node [block] (init) {文献回顾与案例调查};
   \node [block,below of=init] (factor) {初步建立动机因素模型};
     \node [block, below of=factor,xshift=-35mm] (assumption) {问卷设计
       与测试};
    \node [block, below of=assumption] (survey) {发放问卷,处理数据};
      \node [block, below of=survey] (verify) {验证初始假设,讨论分析结果};
    \node [block, below of=factor,xshift=35mm ] (dataextract) {历史数据的选择与清洗};
    \node [block, below of=dataextract] (sna) {社会网络分析};
    %\node [block, below of=sna] (modelling) {行为模式建模};
     \node [block, below of=sna] (tra) {建立绩效模型模型,分析影响因素};
      \node[block,below of= tra,xshift=-35mm](intention){对比分析};
       \node[block,below of=intention](paper){撰写论文};
      
 
    \path [line] (init) -- (factor);
    \path [line] (factor) -- (assumption);

    \path [line] (factor) -- (dataextract);
    \path [line] (assumption) -- (survey);
     \path[line] (survey)--(verify);
     \path[line] (verify)--(intention);
     \path[line] (dataextract)--(sna);
     \path[line] (sna)--(tra);
     \path[line] (tra)--(intention);
     \path[line] (intention)--(paper);
   
\end{tikzpicture}
\end{frame}

\begin{frame}
  \frametitle{关键技术}
\vfill
  \begin{enumerate}
\item  数据采集和数据处理技术。采用成熟的调查研究的技术和方法,
             确保能够获得足够的、有效的、准确的主观数据,供研究分析
            使用。同时对于客观的海量数据,进行清洗、转换,降低数据 
            处理的成本。
\item  社会网络分析技术。通过社会网络分析,区分出实践社区中不同
 类型的行为主体,以及类型之间的联系。
\item  结构方程建模技术。用于数据的分析与处理。
\item  其他与知识管理、知识协同有关的技术。
\vfill
\end{enumerate}

\end{frame}

\begin{frame}
  \frametitle{可能的创新点}
 \begin{enumerate}
\vfill
\item 在理论上,提出基于静态视角研究社区行为主题参与知识协同的
            内部动机模型。该模型综合考虑了协同行为和时序因素(成员数
            量增长和内容数量增长等)以及成员的类型对协同动机的影响,弥
            补以往的研究只对某一类因素进行研究的不足。
\item  在研究方法上,既通过问卷调查获取第一手数据,验证行为主体
              的动机因素;又使用二手网络数据进行分析建模,同时从主客观
              两个角度研究协同动机,克服使用单一方法进行研究的不足。同
              时两种研究角度可能在分析结果上不一致,理解这种不一致有助
             于更好地研究知识协同的动机因素。
\vfill
\end{enumerate}
\end{frame}

\begin{frame}
  \frametitle{预期目标}
\vfill
   \begin{enumerate}
\item 建立实践社区中知识协同内部动机的理论框架。
\item  总结各类影响因素对协同动机的影响和作用机制。
\item  发表研究相关的学术论文,完成博士论文。
\end{enumerate}  
\vfill
\end{frame}

\begin{frame}
  \frametitle{论文发表}

\vfill
  
\begin{block}{Phisica A}
Social network analysis of knowledge collaboration in 
communities of  practice  2009.1  
\end{block}

\begin{block}{Human Computer Interaction}
Intrinsic motivation of knowledge collaberation in 
communities of practice 2009.5
  
\end{block}

\begin{block}{Human Computer Interaction}
Time factors and knowledge collaboration motivation: 
A Wikipedia study  2009.8  
\end{block}
\vfill
\end{frame}

\begin{frame}
  \frametitle{工作计划}
  \begin{center}
  \begin{tabular}[center]{|c|c|c|}
 \hline
序号&时间&工作内容\\
\hline
1&2008.11-2008.12&动机因素模型初步研究\\
\hline
2&2008.12-2009.1&问卷设计与测试\\
\hline
3&2008.1-2008.3&问卷调查\\
\hline
4&2008.11-2008.12&网络数据选取与处理\\
\hline
5&2009.12-2009.2&社会网络分析\\
\hline
6&2008.3-2008.4&动机因素模型研究\\
\hline
7&2009.4-2009.6&问卷处理与统计分析\\
\hline
8&2009.7-2009.10&行为建模与网络数据分析\\
\hline
9&2009.10-2009.11&撰写博士论文并准备答辩\\   
\hline
  \end{tabular}
\end{center}

\end{frame}
\end{document}
%%% Local Variables: 
%%% mode: latex
%%% TeX-master: t
%%% End: 
