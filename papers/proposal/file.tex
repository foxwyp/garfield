\input{ctex4xetex.cfg}
\documentclass[12pt,a4paper]{ctexart}

\usepackage{fontspec}
\usepackage[CJKaddspaces]{xeCJK}
%\usepackage[numbers,sort\&compress]{natbib}
    %    \setlength{\bibsep}{0.5ex}  % vertical spacing between references
\setmainfont{Times New Roman}
\setCJKmainfont{Adobe Song Std}
\setCJKfamilyfont{hei}{Adobe Heiti Std}
\setCJKfamilyfont{song}{Adobe Song Std}

\CTEXsetup[format={\CJKfamily{hei}\zihao{3}\centering}]{chapter}
\CTEXsetup[format+={\CJKfamily{hei}\zihao{4}\flushleft}]{section}
\CTEXsetup[format+={\CJKfamily{hei}\zihao{-4}\flushleft}]{subsection}


\usepackage{geometry}
\usepackage[dvips]{xcolor}
\geometry{vmargin={25mm},hmargin={30mm,20mm}}


\newcommand{\chuhao}{\fontsize{42pt}{\baselinestrench}\selectfont} % 字号设置
\newcommand{\xiaochuhao}{\fontsize{36pt}{\baselineskip}\selectfont} % 字号设置
\newcommand{\yihao}{\fontsize{28pt}{\baselineskip}\selectfont} % 字号设置
\newcommand{\erhao}{\fontsize{21pt}{\baselineskip}\selectfont} % 字号设置
\newcommand{\xiaoerhao}{\fontsize{18pt}{\baselineskip}\selectfont} % 字号设置
\newcommand{\sanhao}{\fontsize{15.75pt}{\baselineskip}\selectfont} % 字号设置
\newcommand{\sihao}{\fontsize{14pt}{\baselineskip}\selectfont} % 字号设置
\newcommand{\xiaosihao}{\fontsize{12pt}{\baselineskip}\selectfont} % 字号设置
\newcommand{\wuhao}{\fontsize{10.5pt}{\baselineskip}\selectfont} % 字号设置
\newcommand{\xiaowuhao}{\fontsize{9pt}{\baselineskip}\selectfont} % 字号设置
\newcommand{\liuhao}{\fontsize{7.875pt}{\baselineskip}\selectfont} % 字号设置
\newcommand{\qihao}{\fontsize{5.25pt}{\baselineskip}\selectfont} % 字号设置


\begin{document}
% fisrt page
  \title{虚拟实践社区中知识共享的内源性动机因
      素研究 }
  \author {吴云鹏}
   \maketitle
% directory
 \newpage
  \tableofcontents
% text  



\newpage
  \section{论文选题依据}

  \subsection{研究背景}
人类的实践活动始终伴随着各类知识的创造,运用,丰富与发展,人们认为,占
有更多的知识能够提高生产的效率,创造更大的价值。如今,越来越多的企业认
识到,要想获得持续成功就必须管理好组织中的各种知识。正如温特
\cite{Williamson1994}指出的:“企业就是知道如何做事的组织。”不论企业是
生产产品还是提供服务或者二者兼有,都依赖于企业知道“如何做事”,也就是企
业中的知识。在当今知识经济的时代,知识已经成为最重要的经济资源,人们对
知识的追求与竞争正变得愈发激烈起来。对于企业来说,对知识的追求就是对超
额利润的追求。随着社会分工和社会化大生产变得更加精细与完善,传统的生产
资源已经不能再为企业贡献超额利润了,知识几乎成为超额利润的唯一来源。企业来要想获得长
久稳定的竞争优势,就必须在知识的占有和管理上取得领先。

尽管知识总是被不自觉地同信息和数据混淆,但是人们普遍认为知识要比信
息、数据具有更广泛更深刻的内涵和更丰富的内容。企业中的知识构成非常复
杂,它是一种混合
体,包括成体系的经验、企业价值、与环境相关的信息和专家的洞察力等等。企
业中的知识不仅存在于文档和数据库中,也存在于企业每日的例行事务,企业过
程和标准规范中。随着劳动分工的日益精细,导致了知识分工的出现。一方面,
人们在越来越窄的范围内知道的越来越多;另一方面,也随之出现了越来越多的
“知识岛屿”,使得如何进行知识沟通成为管理者面临的重大问题。企业中知识复
杂性为知识管理知识带来了巨大的困难,现有的管理理论和技术手段已经不能满足企
业对管理知识的巨大需求,因此,研究在互联网条件下如何进行有效的知识沟通
就成为知识管理研究者紧迫的任务。

知识管理的主要研究领域目前主要还集中在知识管理的“前端”,即知识获取的方
法和技术上。知识抽取、知识表示、知识存贮等方面是知识管理研究最多的方
面,尽管这些领域很重要,研究也取得了丰硕的成果,但是研究表明,对于企业
来说,进行有效的知识管理仍然是困难的。一份对欧洲和北美的研究报告显示,只有13\%的经理认为
知识在企业各组织单元中有效地进行传递,大部分的知识管理都是失败的\cite{Ruggles1998}。知识管理所面临的困境,根本原因就在于没能更好地理
解人们是如何共享知识的。正如
Sveiby所认为的:“如果我们承认知识是人的一部分,知识管理的目的就是组织
如何最好地培育、利用和激励人们改进和分享他们的行动能
力”\cite{Seviby1997}。因此,研究组织中知识共享的模式和动机因素不论是对理论还是现
实均具有重要意义。

\subsection{选题依据}
当前,对于组织中的知识共享,大部分还集中在对知识共享的模式和共享能力的影响因素方
面。而对于动机因素的研究,也就是人们为什么共享知识,共享的动机有哪些这
类问题的研究较少。尽管在心理学和其他行为科学等学科中,动机因素的研究已
经进行了相当长的时间,但是对知识共享领域的动机因素研究还属于初级阶段。

知识共享可以视为一个“将一组复杂的,有时是含混的惯例重建并延续于另一个
新的环境的过程“\cite{szulanski2000pkt},或者说一个“一个个体受到另一个
个体的经验影响的过程”\cite{Argote2000}。有以上两个定义可以看到,知识共
享的核心在于它是共享双方的有效交互。不论是“重建并延续”还是“受到经验的
影响”,都反映了知识的接受一方所做出的相应。因此,Davenport和Prusak给出
了一个更为简洁的定义:“知识共享=传递+吸收”\cite{davenport1998wko}。

尽管对知识共享的定义准确地描述了一个动态的共享过程,但是当我们涉及到研
究知识共享的动机因素的时候却陷入了矛盾中。首先,不是所有的分享行为都可
以视为知识共享。因为共享需要知识的接收端正向反应:“吸收”,也就是说,接
收端需要理解知识,并且成功地融入自身的知识体系。对于那些有分享的行为,
却没有达到共享的效果的过程,尽管知识的发送端有明确的目的,但是却无法划
归到知识共享的范畴内,其动机因素因此也就不在研究范围内了。另一方面,由
于知识本身的特性,隐性知识的共享往往是在无意识的情况下,潜移默化地为接
收端所吸收的。达到知识共享的结果可能只是其他行为的副产品。Sveiby指出
“问题在于共享不是有意的或者是通过某种正式的渠道进行的$\cdots$ 知识
总是以一种最经济的、无意识无目的的方式一代一代传承下去的
\cite{sveiby1996tka}。

实际上,在以上知识共享的定义中,共享的过程可以大致分为两个部分,即知识
源传递知识和接收端接受知识。我们真正感兴趣的动机因素应该是知识源的动
机,也就是人们为什么愿意向他人分享,传递自己的知识。因此,对于动机因素
的研究应该聚焦在知识源的动作---传递上。本文对知识传递将重新进行定义:
知识传递表示知识源使得某件知识可以为接收端可认而且可得,如果知识仅仅是
可以获得,但是获得的是一种不可认的形式,不能称之为知识传递。只有两个条件同时满足,才
可以成为知识传递。

动机因素可以分为两种:外源性动机和内源性动机。外源性动机是对与他人而言
可以接触和观察到的,通过和外界进行物质、能量和信息的交换而获得的。外源
性动机是由其他人或机构进行分配,包括工资、福利和晋升。同时也包括避免惩罚的驱动力。外在的奖励通常是以一致性为基础,即外在的激励因子需要与绩效的提高保持一致。外源性激励因素经常被用于鼓励员工达到更高的绩效水平或者新的目标。但是外源性动机不能完全解释员工个体所进行的每一项努力。
 内源性动机是自内部产生的。换句话说,是这些激励因子将个体与任务或者工
 作本身联系起来的。内源性回报包括责任感、成功感、成就感,他们是通过经
 验习得的。还包括挑战感或竞争感。他们涉及某些吸引人的任务或目标。长久
 以来,完成有意义的工作总是与内在动机相联系的\cite{Luthans2001}。

对于外源性因素来说,尽管通过复杂的薪酬制度设计可以较长时间地提高员工的
工作动机,但是很多学者认为通过激励制度不能带来长期、稳定、持续的动机。“在过去三十年里至少两打以上的研究论文已经得出的结论表明,希望因为完成一项任务或者成功地完成该项任务而得到奖赏的人就是没有根本不想获取报酬的人干得好“\cite{93120316461993}。
另一方面,一些组织是在几乎完全没有外源性
因素的作用下持续运作的,仅仅依靠诸如热情、信仰之类的内源性因素就可以不
断发展壮大下去。因此,本文的研究将针对内源性因素开展,具体研究有哪些内
源性因素会影响组织中的知识传递以及各个因素之间有哪些联系。

外源性动机与内源性动机并不是完全独立的,而是常常交织在一起并相互作用的。
外源性动机即可以强化内源性动机,也可以削弱内源性动机,有时候二者之间还可
能互不影响\cite{deci1971eem}。为此,为了研究内源性因素,希望能够选择一
个比较“纯净”的研究对象,该对象应该较少或者根本不受外源性因素的形象。同
时,内源性因素往往和心理上的矛盾、冲突相联系,因此如果选取在时空、组织、
和背景上差异很大的虚拟组织应该比实体组织具有更大的研究意义,得到更明显
的研究结果。

目前,对于虚拟社区的知识共享主要集中与两类:开源社区(类似于Apache)和
内容协同社区(例如维基百科)。开源社区是针对创建软件代码而展开的共享协
同,内容协同社区则以创建开放的内容而进行知识共享。对于开源社区中知识共享动机因素的研究要
比对内容协同社区的研究丰富的多,研究结果也要复杂的多。有学者的研究结果
显示外源性动机在开源社区中占主导地位\cite{10.1109/HICSS.2001.927045},
另一些学者研究发现内源性因素起重要作用\cite{Lakhani2003}。Xiaoquan等认
为开源社区不论是从软件的功能性和复杂性还是社区的规模、力度上都迥异不同。
不同的开源社区采用不同的协作方式,不同的开源许可,导致了很难通过对一小
部分社区的研究而获得足够丰富的信息以体现这些差异\cite{Zhang2006}。因
此,本文将以内容协同社区为研究对象,具体深入地研究内源性动机因素对知识
传递的影响。


\section{国内外研究现状}
目前,从事知识管理的学者对知识共享进行了大量的研究;同时,动机因
素研究在心理学和行为科学领域也取得了丰富的理论成果。尽管本文在一开始对
知识传递和知识共享做了区分,但是知识共享的研究成果对与本文的研究主题仍
然具有重要的参考价值。本文拟从这两部分总
结并分析国内外的研究现状。

\subsection{知识共享文献综述}


知识管理的重要目标是力图在最恰当的时间将最恰当的知识传递给最恰当的人,
以期能做出最好的决策,创造最大的价值\cite{Petrash1996}。广义地讲,知识共
享就是个体之间交换知识并协同产生新知识的过程。知识必须被他人共享才能
最大化其价值。在不同的文献中,知识共享可能采用不同的叫法:知识共享
(knowledge sharing),知识传送(knowledge transfer)和知识扩散
(knowledge dissemination)。尽管称呼不同,但是所表示的意思基本一样。
一些文章甚至不加区别地使用这几种叫法。不同类型的知识其共享传递的方式也不相同:隐性知识由于
难于用语言清晰表达,因此共享的方式以示范模式为主。在
示范模式中,知识的转移高度依赖于接受和发送双方的互动。\cite{zhoubo2006}显性知识可以有形
式化系统的语言描述,因此它的传播方式是用品模式。知识通过声、文字、图像
等手段从发送者传递到接收者。用品模式的主要特点是知识可以脱离知识发送者
而“独立”存在\cite{zhoubo2006}。

\subsection{知识的价格}

知识沟通实际上就是知识的传播,它可以具有多种形式。例如以公共品渠道发布
知识,非经济的方式转知识以及知识交易。在这三种方式中,只有知识交易使
能够成为长期稳定的知识传播方式。知识交易也就是知识的买卖,交易双方以知
识本身为商品,进行等价交换。由于知识本身所具有的特殊性,知识交
易自然也同普通的商品交易有所区别。周波在其博士论文中对于知识交易做出了
如下定义:“知识交易就是知识拥有者通过知识转移进行“排他性”控制获得经济
利益的过程,交易结果是实现知识转移。”\cite{zhoubo2006}
尽管前两种方式在现实中并不少见,但是正如达
文波特指出的:“那些认为知识无需要经济激励就可以自
然扩散的想法是乌托邦式的;在没有回报预期的条件下人们不太可能贡献出具有
价值的知识。”\cite{davenport1998wko}每个理性人都会根据自己的判断,对于
自身所掌握的知识估计其价值。


\subsection{交易方式}
知识的交易方式可以分成两类,一种是直接的知识交易。交易的标的就是知
识本身,比如一项技术的转让,一本书籍的销售,都是买卖双方直接根据知识的
价值进行交易的。还有一类交易方式是知识服务。交易的一方利用自己的专业知
识生产出某项知识产品,交易对象是服务本身。

尽管第一类交易是最直接,也是人们最先开始实用的交易方式,但是这种方式越
来越暴露出一些问题:
\begin{enumerate}
\item 知识本身难于定价。
\item 供需矛盾
\item 接受者的意愿与能力
\end{enumerate}
知识资产已经成为企业发展最重要的驱动力。Leventhal和March将其描述为“由
组织内部的个人或团体持有的一组特殊的竞争力\cite{levinthal1993ml}.企业
不但自主进行研发以获得知识资产,也可以从外部获得知识资产。现今各类组织
都进行各种各样的知识管理活动,以促进知识资产的挖掘整合共享,最大限度地
实现知识资产的保值和增值。伴随各种各样的知识服务,知识资产的流动也变得
愈发频繁起来。有效的知识管理方式和恰当的制度是促进知识资产流动的重要因
素。随着知识网格技术的成熟和发展,出现了知识集市 。\cite{Andreas2007}

\section{知识传递}
知识在组织和个人之间的相互流动,就是知识传递的过程。按照Davenport和
Prusak\cite{davenport1998wko}的说法:将知识传递给潜在的接收者,并有接
收者加以利用。知识转移已经成为知识管理领域最重要的研究方面之一。知识传递之所以会受
到研究者的重视,是因为知识传递的模式和质量直接影响知识共享的效果。
Alavi指出:知识创造和编码不一定能够改进企业绩效和企业价值
\cite{alavi2000mok}。知识只有在使用过程中才能发挥其作用。

最初对于知识的传递是停留在信息传递的基础上的,以香农\cite{Shannon1949}为代表的信息论认为,信
息的传递是没有开销和即时的,这种观点在相当长的一段时间内被认为理所应当
的,并其应用于知识传递的理论研究中。然而,大量的事实表明,知识在其组织
和人员之间流动并不是自动发生的,反而面临着各种各样的阻力,需要投入和消
耗大量的时间、人力和其它资源去推动知识的流动。在很大程度上,知识的流动
性是非常“粘滞”的。Szulanski\cite{szulanski2000pkt}给出了一个知识传递的
模型。他认为知识传递分为4个过程:初始阶段、实现阶段、跃升阶段与集成阶
段。在初始阶段,知识传递的双方建立“连接”,就知识传递的方式、内容进行交
互并达成一致;在实现阶段,双方开始进行知识的传递,只是由发送方经过一定
的“调制”传递到接收方,接收方进行“解调”后,完成实现的过程。随后,接收方
开始使用新接收的知识,并且试图去解决以前所未遇到过的问题或是无法解决的
问题;最后,接收方从对新知识的运用中,对所接受的知识进一步加深理解,融会贯通,对知识
的应用也更为纯熟,通过新知识的运用圆满地解决问题,知识传递进入了集成阶
段。这四个阶段,刻画了知识传递的整个过程。这个过程不但包括了人们对知识
传递的狭义的理解---从知识的发送方传递到接收方,还包括了接收方对知识的
使用、理解的过程。只有新的知识能够解决发送方的问题,被发送方认可并融入
到自身的知识体系,知识传递的过程才算完成。

这个模型清楚地刻画了知识传递的流程,从而是我们能够对知识传递过程中所遇
到的各种阻碍因素清楚地描述出来。事实上,这些阻碍因素存在于知识传递的各
个阶段,而每个阶段的阻碍因素都有其各自不同的特点。

\subsection{初始阶段}


在传递的起始阶段,传递面临的主要问题是:如何识别出传递的双方以及用什么
方式进行传递。Ounjian和Carne\cite{ounjian1987sfa}认为:在知识传递的初
始阶段,需要双方共同努力确定知识传递的范围、选择传递的时机、评估传递的
成本并建立传递双方的义务。从知识
传递的特点来看,可以分为两种模式:“推”模式和“拉”模式。在“推”模式下,知
识的发送方主动将知识通过某种形式传播出去,传播的对象可以是具体的(比如
说一场讲座的听众),也可能是抽象的(比如一份文件的预期读者)。在“推”模式下,知识
的发送方要做出一系列的相关决策:传播什么样的知识?传递的方式是什么?如
何控制传递的“度”和“量”?如何保证知识的版权和安全?只有当这些问题解决之
后,才可能开始知识的传递过程。在“拉”的模式下,知识的接收方作为共享的发
起者主动寻找知识,这是一种问题驱动的模式。同样,接收方也面临类似的问题:首先,到哪里
寻找相关的知识;其次:当面临多种选择的时候,以什么样的标准选择知识的传
递方;第三:如果没有找到合适的传递方,应该如何做下一步的决策。从以上的
分析可以看出,知识传递的开始阶段,必然有一个传递过程的发起者,而这个发
起者面临着一系列的决策问题,任何一个问题如果不能圆满解决,都会直接影响
知识传递过程的进行。

既然知识共享是传递双方共同作用的结果,那么双方不仅面临这自己各自需要考
虑的担任决策问题,还需要同时考虑与对方相互所产生的决策问题。对于知识的
发送方来说,他需要考虑接收方是否愿意接收知识。在实际的知识传递过程中,
接收方可能处于各种原因,不愿以接收知识。随着知识分工的不断深入和知识总
体数量的日益庞大,人们所掌握的专业知识也限制在越来越狭窄的范围内。而不
断发展的社会分工有要求多学科知识的交叉和集成,在这种情况下,知识的接收
者往往不愿意去学习和自己的工作及知识结构不同的新知识、或者是迫于各种工
作压力没有时间去学习,因为学习和掌握知识是要消耗自身的时间和精力的。另
一方面,学习使用其他人传递的知识,会在一定程度上影响接收方的社会地位和
自我认知。在一个组织中,员工对于不同来源的知识具有不同的反映。Menon等\cite{2203456620060801}通过对两家公司在合并前后员工对
待对方知识的态度研究发现:员工对于来源于组织内部的知识往往持有否定态度。
因为他们认为,吸取采纳别人的知识会使自己作为一个“跟随者”从而影响在组织
中的地位,特别是当双方水平能力接近的时候,员工往往处于维护自己的自我认
知和组织地位的需要,贬低对方知识的价值,阻止其知识在组织中传递。相反
的,组织内部的员工往往对于组织外部的知识报以欢迎的态度,这是因为外部知
识对与员工自身的利益不具有什么威胁,因此愿意使用这些知识。

对于接收方,他对知识发送方的判断是影响传递决策的重要因素。他需要足够的
信息来决定是否要相信发送方,从而采纳他的知识。如果接收方不能做出肯定的
判断,或者判断所需的信息不足,那么即使发送方有能力提供正确的知识,接收
方也不会采纳,甚至知识传递根本就不会发生。

\subsection{实现阶段}
初始阶段的完成为知识共享提供了必要的准备,在实现阶段,知识开始从源端发
送到接收端。显然,这个阶段所面临的最大问题是沟通问题。知识传递的两端可
能具有不同的背景、文化、经验和阅历等不同,如何跨越这些鸿沟是知识顺利传
递的基本要求。大量的实证研究表明,越是具有相似的应用环境,知识的转移就
越有效。同时,接收别人的知识往往有可
能同自己的知识结构、经验、直觉甚至信仰发生冲突。如果接收方认为这种矛盾
不能解决,或者需要较大成本融合进本身的知识体系,传递过程也不会发生。


\subsection{跃升阶段}

在知识被接收方获取后,接收方开始使用新的知识解决问题。如果需要解决的问
题是简单的,那么应用新知识很容易产生效果,满足接收方的需要。如果为体本
身比较复杂,那么新知识可能不能简单地直接使用,何时需要使用者进行一定的
处理加工,使之适应新环境的需要。如果在使用的过程中不断出现新的问题或者
使用者未预料到的情况,那么新知识的应用就更加困难。这就要求使用者本身要
有深厚的知识积累和知识的创新能力。对于新掌握的知识
自然会需要一定的练习后才能取得满意的效果,如果在这个阶段出现意外的情
况,接收方会由于不满新知识带来的收益而否定甚至放弃。应用新的知识解决问
题还可能同自己的日常工作行为和习惯矛盾,更改习惯所产生的“切换成本”也会
对接受者对新知识发生的效益降低。总之,如果新知识不能达到接收者的期望,
则他将排斥新知识,使其不能发挥其应有的作用,知识传递的过程也就此中断。

\subsection{集成阶段}




\subsection{知识传递}
从知识传递的模型中,我们可以看到,知识从发送方(源)的大脑中、以某种形
式在知识的接收方的大脑中重构出了同样的内容。但是,这个模型并没有解释在
传递的过程中、传递的对象到底是什么:是信息,还是内容,或者就是知识?还
有没有其他的部分在传递过程中一并传送?如果传递的对象是知识,那么它和信
息的传播又有什么不同?Polanyi\cite{polanyi1998pkt}在研究个人知识的过程
中认为知识本身本身是个人的、主观的、隐形的。所有的知识要么本身就是隐性
知识,要么源于隐性知识,其显性的程度取决于人的表达能力。人在学习、使用、
表达知识的时候,总会带入自己的感情因素,因此,事实上并不存在所谓的客观
知识。同时,知识又是在一个特定的社会环境下产生、传递的,所以在传递的过
程中、不仅仅是个人知识拥有的个体知识、还包括个人拥有的社会知识也同时进
行了传递。按照Polanyi的看法,传递的对象远不止一个静态的“知识”那么简单,更
重要的是传递“认知的过程”,他称之为通过“传承”(tradition)传递知识。在
这个传递过程中,行为模式、准则、价值观以及知识本身统统是传递的对象,而
只是传递也并不是把知识通过某种媒介向信号一样发送出去。知识的发送端试图
帮助接收端建立一个特定的过程,接收端会根据掌握的经验、知识构建一个自己
的版本。发送端期望通过这个构建过程,可以使接收方达到对一件知识同等程度
的认识和理解。Sveiby\cite{sveiby1996tka}非常支持Polanyi的观点。他认为
知识通过信息传输是错误的,并且和很可能是无收益的。只有在社会互动的环境
下,人与人之间面对面的交流,才能真正地实现知识的传递。而信息应该仅仅被
实为一种“等价且无意义”的信号,如何解释这种信号取决与接受者自身。这就解
释了为什么不同的人对同样的知识理解的程度和角度有所区别。发送方和接收方
拥有的背景知识的结构越是相似,对转送内容的解释也就越相似。

Dixon\cite{dixon2000ckc}在Common konwledge一书中通过对经典的案例分析
也得出类似的结论:仅仅依靠技术手段进行沟通是不够的,必须和面对面的交流
结合使用,才能建立更有效的知识共享系统。技术手段和面对面交流\footnote{这里
  面对面并不一定是在实际环境中的交流,随着虚拟技术的发展,利用视频会议
  等手段同样可以达到面对面交流的作用。员工经过培训后会逐渐适应这种虚拟
  的面对面交流,不但节省了交流的费用,也提高了效率。}各有优缺
点,相互之间不能替代。Dixon将知识转移归纳为五种类型:连续转移、近转移、
远转移、战略转移和专家转移。连续转移是指团队在重复完成某项工作时不断总
结经验,内省不足与有点。近转移是指工作环境于成员背景相同或相似的团队间
转移共享知识。当近转移条件下任务的性质变为非常规的、并且转移的知识常常
是隐性知识的情况下,近转移变为了远转移。战略转移是一个团队向另一个团队
转移非常复杂的知识。专家转移是针对工作中偶尔出现的明晰的、可以使用符号、
公式等表述的问题进行解决或者解答。其中连续转移、远转移、战略转移这三种类型都
是面对面沟通为主要手段、辅以信息技术为支持;而近转移和专家转移则是通过
系统,形式化地表示转移内容,辅以面对面的沟通来完成的。这是因为前三种转
移的知识大多是隐性的,而后两种转移的知识往往是显性的。

知识的转移不仅仅是信息的转移。Lesser\cite{lesser2000kac}指出知识管理最
大的问题在于使用信息管理的手段和理念去设计知识管理系统。很多企业试图建
立一个完善的知识系统,员工从系统里学习知识、寻找解决问题的方法;同时将
自己在实践中所获得的知识记录下来并提交到知识库以供分享。然而典型的后果
是:员工在领导的压迫下撰写完毕知识文档,知识库里很快累积了大量的经验总
结,但是就是没有人去使用。一方面,员工在复杂的工作流程里很难提取出究竟
哪些是有价值的知识,另一方面,读者普遍感到知识文档太空太泛,缺少实际的
指导作用和实用价值。知识的转移需要与转移的对象、环境相配合,然而信息技
术的发展是人们将目光集中在信息而不是知识的处理上,进而忽略了转移的对象
和转移的环境。在这种情况下,共享者不知道预期读者是谁,他会解决什么样的
问题,而知识的消费者也难于描述自己的知识需求,找到合适的知识。Lesser进
一步总结了知识与信息的六个不同:
\begin{enumerate}
\item 认知是一种人类的行为
\item 知识是思考的结果
\item 知识总是在当前产生的
\item 知识属于社区
\item 知识在社区内流转的形式多种多样
\item 新知识产生于旧知识的边界
\end{enumerate}


\section{虚拟组织}

虚拟组织是近年来兴起的一种组织形式。和传统的组织形式不同,虚拟组织的成
员不必在同一地点开展工作。随着全球化的趋势越来越明显,企业面临这越来越
大的竞争压力,有效利用各类资源,降低成本,满足用户需求是每个企业面临的
最重要的问题。信息技术的发展打破了企业发展地域的局限,在新的IT技术的支
持下,企业可以在全球范围内开展业务,吸引各地的人才协同工作,最优化地配
置人力资源。这种组织形式为企业提供了前所未有的灵活性和响应性。

\section{虚拟组织中的知识共享}
员工在组织中需要完成各种任务。这些任务往往需要具有不同能力和专业知识的
员工协作完成,因此在完成任务的过程中,知识的转移会经常发生。由于组织的
虚拟性,知识转移需要跨越时空的间隔,进而带来了各种问题。
Kecmanovic\cite{cecezkecmanovic2001eap}研究了TODO。
Davenport\cite{davenport1998wko}将这些问题归因为知识的本地性:人们总是
从身边的人获取知识,而这些人往往是他们信任的人。当人们进行非面对面的交流
时,这种信任往往消失掉,从而减弱了知识的转移。

虚拟组织的发展使得面对面的交流逐渐减少,通过其他信息媒介交流已经成为必
然。

虚拟组织是随着产业竞争的加剧和全球化程度的加深而产生的新的组织形式。虚
拟组织打破了地域、时间的阻隔和人员流动的障碍,充分利用了企业内外部的各
类资源,减少对市场信号的反应时间,改进企业服务水平,提高顾客的满意度,
从而保持企业的竞争优势。

关于虚拟组织的定义,目前流行的定义有以下几种:一种认为如果一个企业所从
事的生产活动是跨地域、跨周期、跨文化的,则该企业是一个虚拟企业
\cite{1189855}。另一种观点认为虚拟组织的临时性是其显著特征。虚拟组织关
注于市场上的短期获利机会,当这个机会显示后则随之解散并准备重新组织
\cite{655269}。虽然对虚拟组织的定义各有不同,但是学者普遍同意任何团队都或多或少地具有一定的虚拟性。虚拟性是一个团队所
固有的特征,是一个连续的变量,而不是一种分类标准。不同程度的虚拟性形成
了不同类型的虚拟组织。

虚拟性可以从多个维度进行定义,目前主要的观点有3种:
ZIGU RS[ 11 ]认为团队分散的维度越多, 团队的虚拟性就越大。她提出了4 个影
响虚拟性的维度:
地理分散、时间分散、组织分散和文化分散。K IRKMAN [ 7 ]提出的虚拟性3 维度是团队对虚
拟工具的依赖程度、虚拟工具所提供的信息价值, 以及成员之间沟通的异步程度。GR IFF ITH
等[ 6 ]提出的虚拟性定义包含以下3 个维度: 团队成员物理分散程度、团队中所使用的技术支持
和成员花费在虚拟工作上的时间比例。可以看出,zigurs从虚拟组织形成的基础
来定义虚拟的维度,每一个因素都影响着组织的虚拟度,但却不是虚拟性的本
身;kirkman从虚拟组织的表面特征来定义虚拟性,每个虚拟组织尽管都具有这
一类特征,但是同样不是虚拟度本身。可见,他们两个人在虚拟度的定义上都是
兜了大圈子,走了两个极端,分别用虚拟度的影响因素和外部表现来反映虚拟度。
大多数学者认为,虚拟性应该同时包含这两个方面:即人员的分散和对现代通信
技术的使用。黄攸力等\cite{huangyouliandliutuanjie}根据以前的研究成果提
出了虚拟性的三个维度:①团队成员对虚拟工具(包括通讯工具如电子邮件、视频会议系统以及工作工具如
群体决策支持系统) 的依赖程度; ②沟通的信息丰富程度, 包括沟通频率和同步程度, 用沟通的
信息丰富程度来说明成员投入的程度以及成员之间协作的程度; ③虚拟团队成员的多样性(包
括地理分散、时间分布、文化多元和组织多样)。

尽管虚拟性还未能有一个普遍接受的定义,还有学者不断从新的角度提出论点,
但是以上内容已经足以反映出,虚拟组织中的知识共享面临着比在传统组中织更
大的挑战。虚拟性是有效的知识共享所面临的最大难题。

\subsection{空间虚拟性}

空间上的分割使得组织中的成员无法进行面对面的交流,只能通过现代通信工具
完成信息交互,或者说,基于信息的知识共享\cite{sveiby1996tka}。尽管视频
工具可以让组织成员在一定程度上消除空间的阻隔,但是其作用有限。由于视频
方式的成本、安全等问题,使得其应用的时间受到限制,更重要的是,它并没有
真正将需要沟通的双方置于一个公共环境中去,而只是让双方看到对方而已。基
于传统(面对面)的知识共享传递的对象是一整套流程和内容,而基于信息的传
递仅仅是一种“潜力”,接收者据此来构建自己的知识版本,却几乎不能对信息本
身加以任何影响。

同空间的虚拟性一样,工作时间的区别同样导致沟通受阻。全球化的变革导致团
队各个成员的工作时间均不相同,工作与交流有同步沟通变为异步沟通。沟通的
工具也主要以文字沟通为主。在这种情况下,有效的知识共享愈发困难。

相对来说,虚拟组织中成员的文化与背景的差异是知识共享最大的障碍。组织的
成员可能来自世界各地,对事物的理解、认知有很大差异,处理问题的方式也千
差万别。不同的工作经历于阅历形成了不同的世界观和方法论。这些差别使得人
们对某一问题取得共识更加困难:东方人可能很难理解西方人的思维,坚持某一
理论、技术的人可能对持其他理念的人不以为然,如果通的双方不能对对方建立
有效的信任关系,有则么可能指望知识能够顺利地转移呢?另一方面,组织中的
成员可能还各自抱有不同的目的,既可能是私人的目的,也可能是代表自己实体
组织的利益。尽管成员在总的大方向上目标一致,但是每个人都会有自己的目
标,并且和组织的整体利益或多或少有一些冲突。一旦这些矛盾显现出来并激
化,将会严重影响知识共享的过程。
Ai Ling Chua
Sarker等\cite{sarker2000uvt}对虚拟社区的发展最早进行了比较充分和细致的
研究分析。他们调差了12个由美国和加拿大学生组成的信息系统开发的虚拟团
队,研究分析后认为:不同领域背景和专业水平的组织成员要想有效地开展协
作,仅仅通过广泛地沟通是不够的。学习过程的发生不但需要信息的交互,更重
要的是将信息置于每日工作的实践环境中,将结果及时反馈,才能是接收方有效
掌握相关知识。理解团队的知识协同必须同时理解团队的结构、沟通的模式和业
务形态三个方面。这些因素应该同沟通方式匹配且均是有效的,才能保证知识协
同的有效性。


\subsection{虚拟组织中知识共享的推动因素}
Davenport指出,人们倾向于从身边的人获取“本地”知识,“远程”知识的可信度
往往受人质疑,而访问远程知识的机制要么根本不存在,要么作用甚微\cite{davenport1998wko}。随着虚
拟组织的发展和技术的进步,地理上的间隔已经不再是阻碍虚拟组织中知识共享
的主要障碍了。在组织成员难以开展面对面交流的情况下,研究他们在共享知识
时面临的阻碍因素和推动因素,是非常必要的。
Sarker等\cite{1435630}研究了虚拟系统开发团队之间的知识共享,提出了四个
影响因素:信任、能力、沟通和文化背景。通过对团队成员间的沟通情况进行分
析,作者认为信任、沟通和文化背景这三个因素对于知识共享有重要作用。
信任是知识共享开始的必要因素,面对组织中不同专业、不同水平、不同理念的
成员,如果没有信任机制,是不可能开展交互与协作的。信任问题的本质在于
“信任来源于接触”\cite{9506195264n.d.n.d.n.d.}。从未谋面的两个人,互相不了解对方的背景情况,要想建立起互信机制是非常困难的。当一个知识源被判定为
不可信时,接受方会认为源一方的知识是不可靠的,因而不太可能去使用对方的
知识。Jarvenpaa等通过案例研究了虚拟组织中的信任问题。研究结果表明即使
是在一个初始信任度很低的虚拟组织中,信任也是可以建立的。通过不断的沟通
交流,组织的信任度会逐渐提高。同时,在组织成立伊始组织的信任度越高,组织越是
能应对不确定性和复杂性,越是能有效把握组织的发展方向\cite{SirkkaL}。

对希望采取的行为进行外部激励是外部动机的特征。当员工只有在获得物质回报
的时候才愿意和他人分享知识时,外部动机就占据了主导地位。而内部动机来自
任务本身,并且以特定活动的内部激励为先决条件。而且,行为并不因报酬的缘
故而发生,而是其活动使人有成就感。

不过,外部动机可以削弱内部动机,这种现象成为抑制(repression)或满溢效
应(spillover effect)。
内部动机比外部动机更难管理。严格来说,内部动机不能自发产生;只能创造合
适的发展环境。大前提是任务的设计,以唤起内部动机。
\cite{KaiMertins2003}

\section{工作动机理论}
工作动机理论一直是心理学和组织行为学
领域的一个重要主题。根据不同的研究侧重,可以分为三个流派:内容理论、过
程理论和当代理论。

\subsection{内容理论}
内容理论试图找出工作中的激励因素,从而解释工作动机与激励因素之间的关系。
马斯洛的需求层次理论是研究组织激励时应用得最广泛的理论。
马斯洛理论把需求分成生理需求、安全需求、社交需求、尊重需求和自我实现需
求五类,依次由较低层次到较高层次。需求层次理论既是解释人格的理论,也是
解释动机的重要理论。马斯洛认为动机是由多种不同层次与性质的需求组成的,
每个层次的需求与满足的程度,将决定个体的人格的发展境界
\cite{Maslow1943}。

Herzberg对马斯洛的理论进行了拓展,建立了一个更为具
体的关于工作动机的内容理论。Herzberg通过实证研究发现:属于工作本身或
者工作内容方面的因素(例如挑战性的工作、认可、责任等)可以满足工人对于成
就、地位、自我认知等方面的需求,从而使工人感到满意;而缺乏这些因素并不
会导致公认的不满。另一方面,工人的不满主要来自于与工作环境或是工作关系
方面的因素,诸如(地位、工作安全感、薪水、福利等)。
Herzberg将前者成为
激励因素,后者称为保健因素。如果管理者希望提高工人的满意度,就应该关注
于激励因素\cite{hertzberg1959mw}。Alferder同样扩展了马斯洛的理论。他确
立了三种核心的需要类型:存在、关系和成长。同马斯洛和Herzberg不同的是,
他并不认为只有在低层次的需求满足之后才可以进一步满足高层次的需求,也不
认可缺失是激发需求的唯一途径\cite{alderfer:era}。

通过对人的需求和动机进行研究,McClelland提出了成就需求理论。McClelland
认为,具有强烈的成就需求的人渴望将事情做得更为完美,提高工作效率,获得
更大的成功,他们追求的是在争取成功的过程中克服困难、解决难题、努力奋斗
的乐趣,以及成功之后的个人的成就感,他们并不看重成功所带来的物质奖励。
个体的成就需求与他们所处的经济、文化、社会、政府的发展程度有关,社会风
气也制约着人们的成就需求。而金钱刺激对高成就需求者的影响很复杂,他们一般总以自己的最高效率工作,所以金钱固然是成就和能力的鲜明标志,但是由于他们觉得这配不上他们的贡献,所以可能引起不满\cite{mcclelland1976am}。

斯金纳的强化理论认为, 人的行为是由外部因素控制的,
控制行为的因素称为强化物。强化物是在行为结果之后紧接
着的一个反应, 它提高了该行为重复的可能性。按照斯金纳的
观点, 当人们因采取某种理想行为而受到奖励时, 他们最有可
能重复这种行为。当这种奖励紧跟在理想行为之后, 则奖励最
为有效; 当某种行为没有受到奖励或者是受到惩罚时, 其重复
的可能性则非常小。因此, 强化理论者们认为行为是其结果的
函数\cite{skinner1938boe}\cite{ferster1957sr}。强化理论的致命弱点是在于它忽视了诸如目标、期望、
需要等个体因素, 而仅仅注重当人们采取行为时会带来什么
样的后果。

强化理论在Bundura的研究中进一步发展,使得强化理论被用来解释许多社会心
理学问题,社会心理学第一次拥有了改造社会的理论。社会学习理论认为,不仅
加诸于个体本身的刺激物可以让其获得或失去某种行为,观察别的个体的社教化
在
学习过程也可以获得同样的效果。Bundura认为观察学习是规则和创造性行为的
主要来源,通过对社会榜样的模仿、学习,个人可以做出创造性的反应
\cite{bundura1977slt}。

X理论和Y理论(Theory X and Theory Y),是管理学中关于人们工作源动力的理
论,由美国心理学家Douglas McGregor1960年提出的。这是一对完全基于两种完
全相反假设的理论,X理论认为人们有消极的工作源动力,而Y理论则认为人们有
积极的工作源动力。根据X理论,企业管理的唯一激励办法,就是以经济报酬来
激励生产,只要增加金钱奖励,便能取得更高的产量。同时也很重视惩罚,认为
惩罚是最有效的管理工具。McGregor认为X理论是极为片面的,这种软硬兼施的
管理办法,其后果是导致职工的敌视与反抗。针对X理论的错误假设,他提出了
相反的Y理论。Y理论指将个人目标与组织目标融合的观点,与X理论相对立。通
过扩大工作范围,尽可能把职工工作安排得富有意义并具挑战性,工作之后引起
自豪,满足其自尊和自我实现的需要,从而使职工达到自己激励。只要启发内
因,员工就可以实行自我控制和自我指导\cite{mcgregor2006hse}。

\subsection{过程理论}
相对内容理论,过程理论更关注动机因素的作用过程。Vroom在1964年提出了期
望模型。该模型包含三个要素:效价、工具性和期望。人的工作动机来源于效价
(对具体结果的偏好)和期望的双重作用。Vroom将其定义为二者的乘积,这两
个因素增大任何一个,都可以提高人的动机。当人们达到了预期的结果后,该结
果则体现出工具性,即该结果是为了达到另一个新的合意结果而达成的。如果新
的结果无法达到,即当前结果的工具性不强,则反过来有影响到动机本身\cite{vroom1964wam}。
Vroom的理论清晰地反映了个体工作动机的差异。

Locke从20实际60年代中期开始从事目标设定理论的研究。该理论认为绩效与目
标是有直接关系的。易达成的目标总是伴随着较低的绩效,而模糊的目标也很难
提高绩效水平。目标通过三种方式影响绩效。首先目标将人的注意力聚焦于目标
相关的活动,人所做的努力大部分是为达成目标;其次目标具有维持努力程度的
作用;最后明确的目标有助于人们会想起相关的知识和经验\cite{locke1990tgs}。

波特-劳勒模型从跟本上解决了工作满意度与绩效关系的问题。他们认为,一个
人在作出了成绩后,得到两类报酬。一是外在报酬,包括工资、地位、提升、安
全感等。按照马斯洛需求层次理论,外在报酬往往满足的是一些低层次的需要。
由于一个人的成绩,特别是非定量化的成绩往往难于精确衡量,而工资、地位、
提升等报酬的取得也包含多种因素的考虑,不完全取决于个人成绩,所以二者并
非直接的、必然的因果关系。另一种报酬是内在报酬。即一个人由于工作成绩良
好而给予自己的报酬,如感到对社会作出了贡献,对自我存在意义及能力的肯定
等等。它对应的是一些高层次的需要的满足,而且与工作成绩是直接相关
的,“内在报酬”与“外在报酬” 经过“所理解的公正报酬”的调节,最终形成了满
意度。也就是说,一个人要把自己所得到的报酬同自己认为应该得到的报酬相比
较。如果他认为相符合,他就会感到满足,并激励他以后更好地努力。如果他认
为自己得到的报酬低于“所理解的公正报酬”,那么,即使事实上他得到的报酬量
并不少,他也会感到不满足,甚至失落,从而影响他以后的努力
\cite{porter1968maa}。 

\subsection{公平和程序公正理论}
公平理论是由Adams于1965年提出的:员工的激励程度来源于对自己和参照对象
的报酬和投入的比例的主观比较感觉。当以下情况发生的时候,员工会感到公平:
$$\frac{本人获得的结果}{本人的投入} = \frac{本人认为的他人获得的结
  果}{本人认为的他人的投入}$$
否则,当比例不相等是,不论是大于还是小于,本人都会通过一定的努力来调整
比率的值力图使其相等。这种努力就可以认为是工作的动机。努力的方式包括改
变自己的投入、改变自己的所得、扭曲对自己的认知、扭曲对他人的认知、改变
参考对象、改变目前的工作这几方面\cite{adams1965ise}。

Festinger提出的认知失调理论是另一个非常重要的动机理论。认知失调是一个心理学上的名词,用来描述在同一时间有着两种相矛盾的想法,因而产生了一种不甚舒适的紧张状态。更精确一点来说,是两种认知中所产生的一种不兼容的知觉,这里的“认知”指的是任何一种知识的型式,包含看法、情绪、信仰,以及行为等。认知失调的理论表示相冲突的认知是一种原动力,会强迫心灵去寻求或发明新的思想或信仰,或是去修改已在心里存在的信仰,好让认知间相冲突的程度减到最低。已有实验试图去量化此一理论上的趋动力\cite{aronson1969tcd}。

\subsection{内源性动机因素}


尽管相关理论和学说众多,但是几乎所有的相
关学者都不约而同地把动机分为两大类:内在动机
和外在动机\cite{sansone2000iae}。内
部动机的主要特征为对活动本身的注意和兴
趣,而外部动机的主要特征为关注外在的奖励,外在认同和外在的指导\cite{collins1999mac}。

动机因素对组织来说至关重要,正如McConnell指出的:”动机是一种软因素,它
难于量化而且总是排在那些次要而容易测度的因素之后。每个组织都知道动机很
重要,但是很少有人想到利用动机。很多日常的管理活动小处精打细算、大处大
手大脚,为了改进一些不重要的管理思想或者节约一些成本就把动机因素所能带来的
巨大价值白白忽略掉了“\cite{663801}。

然而,关于哪种动机因素能够更好地激
励人们完成工作这个问题一直以来都没有确定的答案。从经济学的角度来看,个
体会对外界刺激作出反应,而正向的刺激可望获得正向的结果。在经济学中一个
中心理念是激励能够提升努力的程度和绩效水平,并且在研究中得到证实
\cite{388530120001201}。虽然经济学家们同时也承认内部动机的存在,但是内
部动机相对来说更难为管理者把握,结果也更加不确定,因此管理者在实际上更
偏重于建立各种奖惩制度和规章\cite{argyris1998ee}。心理学领域给出的研究
论断则正好相反,研究表明内部动机可以有效地提升员工的绩效,达到很多正向
激励难以实现的效果。内部动机有助于解决企业的多重任务问题\cite{gibbons1998io}\cite{holmstrom1991mpa}\cite{prendergast1999pif},即员工只关注
于与奖励直接挂钩的工作,而对那些与奖励没有直接关系的任务视而不见。Austin指出:企业在很大程度上是依赖
于内源因素而存续的\cite{austin1996mam}。很多研究表明,在有激励的情况下工作的
人反而没有那些不受任何激励的人做得好
\cite{Deci1975}\cite{wilson1981aas}\cite{kruglanski1971eei}\cite{lepper1973ucs}
。例如:一些学者通过对软件工程师的动机因素进行研究发现,
他们的工作动机并不在外界的奖励和承认方面,而是享受工作本身的乐趣,例如技术上的
新发现或者是挑战技术难题等\cite{1252263}\cite{1125221}。
激励可以改变人的行为,却不能改变人的态度。


外部动机与内部动机并不是互不干涉的,而是常常交织在一起同时做用的。这种
作用被称为群集效应(Crowding Effect)。如果外部动机对内部动机有正向的
影响,则称为挤入效应,反之称为挤出效应。从已有的文献来看,对挤出效应的
研究数量远大于对挤入效应的研究。Kruglanski发现激励
可能在实际上降低绩效水平,尤其是从长期来看,反向强化的作用更加明显
\cite{Kruglanski1978}。Deci等进行了一组全面的实验验证了:所有的奖励,
不论是实物奖励还是期望奖励,都严重地降低了自我选择的内源性动机。尽管奖
励改变了人们的行为,却阻碍了自我约束发挥效用\cite{deci1999mar}。即使员
工开始是因为内在动机而从事工作,当外部的奖励
反复出现时,个体很容易忽视那些重要的内在价值、
需要和道德因素\cite{deci2000agp}。当外部
激励因素强化到一定程度时,个体开始完全将外部动机作为惟一的动机,而取代
内部动机\cite{kasser2002hpm}。在外部动机激励下的员工更倾向于重复已经做
过的工作,而不愿意从事更具创新性的工作\cite{amabile1998kc}\cite{schwartz1993cad}。在组织中,同奖励随之而来
的还有更严格的监督,更频繁的评审以及更激烈的竞争\cite{Kohn1993},同样
损害了内源动机\cite{deci1985ima}。尽管外源性动机会损害内源动机
已经被多次证明,但是仍有研究证明了挤入效应的存在。Eisenberger和Cameron分析认为:通过恰当的激励手段可
以降低外源性动机的负面影响,同时提高个体的创造能力;外部动机
和内部动机之间可以有不显著甚至是正向的关联。
\cite{eisenberger1996der}。Vansteenkiste等认为这是由于尽管实
物性奖励增强了外部控制感,会导致对自主性产生负面影响,但是任务性奖励却对个体的能力进行了
肯定,从而促进了其兴趣水平,因此个体反而会从外部激励中获得正向反应\cite{vansteenkiste2003ccr}。Eisenberger和Armeli发现,对于非
创造性任务非显著绩效的报酬奖励会降低内在动
机,但对于创造性任务高绩效的报酬奖励会增强内
在动机\cite{eisenberger1997csr}。奖励可以增加个人的自决能
力,使人感到着我们不再依赖于外界的施舍,从而提高了人的自主性\cite{eisenberger1999dpp},\cite{eisenberger1999eri}。从以上内容可以看到外部动机与内部动机
之间的复杂关系,看似矛盾的结论实际上说明:适当的外部激励可以提高人的动
机,而不适当的外部激励会降低人的动机。



对于内源性动机,不同的学者给出了不同的定义。Hull认为习得性行为均源自基
本需要的满足,凡是内在驱动的行为都跟当事人的基本需要有密切的关系,许多
场合正是这些基本需要导致了行为的内在驱动的\cite{hull1943pbi}。White认为:人们经常参与某些活动只不过为了
体验效能或能力\cite{white66wmr}。类似的,deCharms提出一个人天生有一种
原始驱动的本性,它关乎自身从事活动的缘由\cite{decharms1968pci}。这类观
点认为内在动机主要与人们的某些精神需要相联系\cite{Kanfer1990}。另外一些学者主要从个体的行为归因来对内在
动机进行界定。他们认为,如果个体认为某活动或
工作是自己本身愿意去做的,那么其主要的工作动
机就是内在动机。受到内在动机激励的员工,他们
往往觉得自身能力在工作中得到了发挥,具有很大
的工作自主性。他们往往不是追求一些明显的外在
报酬而是由于对活动或工作本身感兴趣而产生强烈
的工作动机\cite{chenandwu2008}。Hackman和Oldman将内在动机定义为员工在
工作过程中通过自我激励而达到的有效程度,这种程度越高,那么员工的工作体
验越好\cite{hackman1975djd}。

内在动机的影响因素很多。Amabile对内在动机的相关研究进
行了总结,确定了内在动机的五种主要构成要素,
它们分别是自我决定、胜任感、工作参与、好奇心
和兴趣\cite{amabile1996cc}。。Hackman 和
Oldham指出,技能多样性、工作完整性、重要性、
自主性及回馈性等五种工作特性激励性较高,有助
于提高员工工作动机好\cite{hackman1975djd}。Waterman等人通过实证研究认为,自我决定以及
任务的挑战性与技能的平衡是影响内在动机的前因
变量[\cite{AlanSWaterman11012003}。。Eisenberger, Jones, Stinglhamber等人的实证
研究表明,对于成就导向的员工来说,高技能与挑
战性工作的结合有助于提高其对任务的兴趣和积极
情绪体验;但对于低成就需要的员工来说则不存在
这种现象\cite{Eisenberger2005}。。Deci和Ryan认为自主性(autonomy)即个体的自由选择性是内在动
机的一个关键性因素。无论对于何种行
为,自主性都只是一个程度问题,是内在性与外在
性连续维度上的某一点。当个体的自主性达到内在
性的最高值,这时候的行为动机就是内在动机\cite{ryan2000sdt}。Guay, Boggiano和Vallerand研究表明,
自主性有助于个体自我能力评估的提高,进而提高
个体的内在动机。当自主性被破坏时,员工的内在
动机也随之降低,直接导致其效率降低成本增高,
这种现象在需要创造性和灵活性的任务中尤其明显
\cite{FredericGuay06012001}。

自我效能也是影响内在动机的重要因素。自我
效能是一种个人信念,即个人可以以某种方式达成某个目标
\cite{ormrod2003epd}。自我效能高的人会更加努力地投入某项工作,工作热情
也会更加持久\cite{schunk1990gsa}。研究表明,高自我效能感的员工会对自己的
能力表现出更强的信心也在工作过程中拥有更高的
内在动机。如果员工自我效能感较高,由此激发的
内在动机会促使其选择充满挑战性的工作;相反,
如果员工对自己顺利完成某项任务的能力表示怀
疑,自然倾向于逃避相应工作\cite{David2007}\cite{bandura2003nse}。Brown 和Dutton 的研究发现,
具有较低自我效能感的个体往往对失败具有更强的
消极体验,而且往往倾向于把自己某一方面的失败
泛化到生活中其他非相关领域。而具有较高自我效
能感的个体对于自己在某种特定情境下的失败有清
晰的情境认知,不会轻易泛化到生活中其他情境,
因而其消极体验就相对较弱\cite{brown1995tvc}。即使工作有比较大的难度,
承担起来有一定的压力,自我效能所激发的自信也能抵消这些不利因素。

个人目标也会影响内源性动机因素。个人目标可以大致分为两类:一类主要是通过某项工作任务的完成来展示
自己的工作能力,从而得到领导或同事对其正向的
评价,此类目标属于绩效型目标(performance goal)
或任务导向型目标(task-oriented goal);另一类工
作目标主要是通过从事某项活动来发展和提高自己
的工作技能与能力,此类工作目标属于学习型目标
(learning goal)或掌握型目标(mastery
goal)\cite{LairdJRawsthorne11011999}\cite{Pajares2000}。Utman证明了学
习型目标同内在动机的关系更为密切:学习目标型
员工在工作过程中会体验到更多乐趣,具有较高的
工作满意度;而任务目标型员工则体验到较少甚至
体验不到乐趣,相反他们体验到的是更多的压力\cite{ChristopherHUtman05011997}。Potosk和Ramakrishna研究发现,学习型目
标与自我效能感呈显著的正相关,任务型目标则与
自我效能感呈显著的负相关,学习型或任务型目标
的设置可能通过自我效能感的中介作用对内在动机
产生间接影响\cite{potosky2002mru}。Dweck发现一个人未
来的成功与天分和当前的成就没有太大关系,而是与个人的目标紧密相关
\cite{dweck2000stt}。她提出了两种目标倾向:成绩目标取向(performance
goal orientation),致力于通过寻求关于自身能力的肯定性评价、避免否定性
评价来展示自身能力的高水平;学习目标取向(learning goal
orientation),致力于发展掌握新技能、适应新环境的能力。Elliot进一步提出了三因素的目标取向模型,即把成绩目标再分成两类:进取(proving)或趋近(approach)型成绩目标和回避型(avoidance)成绩目标\cite{elliot1996aaa}。成就趋向型目
标个体关注的是赢得对能力的正向评价,成就逃避
型目标个体逃避对能力的负面评价。只要是希望成功而不是逃避失败,不管个体
的目的是为了掌握任务本身还是为了获取一个好的
结果,都可以提高内在动机水平 。

除了个体的动机因素之外,动机因素还包括个体间的动机因素\cite{Hackman1975}\cite{Hackman1980}。个人动机因素在个人独立工作的时候
起作用,而人际间的动机因素只在个体间相互协作的时候才发挥作用。个体间的动机
因素包括能力直觉和相属感。能力知觉是指一个人相信自己做好了某事,或者能够做好某件事情的
程度\cite{harter1981nsr}\cite{bandura1982sem}。能力知觉越高,个体就越
愿意向组织贡献自己的劳动。反之,如果一个人觉得自己的工作不受到别人的重
视,他的工作动机就会显著下降\cite{Hertel2003}。正向的反馈可以有效提升
工作动机,而负向的反馈则会削弱工作动机。

\section{虚拟社区中知识共享的动机因素}
在不断进步的信息技术的支持下,虚拟社区这种松散的组织形式越来越多地涌现
出来。由于虚拟社区本身有许多特点不同于实体社区,虚拟社区中知识共享的动
机因素也发生了一些变化。当前对于虚拟社区中知识共享的动机因素研究主要以
这两类虚拟社区为目标:一种是开源软件开发社区;另一种是开放的内容生产社
区。前一种以Apache和Sourceforge等为代表,后一种以维基百科为代表。而对
于开源软件社区的研究数量有远远大于对内容生产社区的研究,研究的结果也更
为复杂。


\bibliographystyle{unsrt}
\bibliography{../../bibtex/elsevier,../../bibtex/emerald,../../bibtex/chinese,../../bibtex/jstor,../../bibtex/citeseer,../../bibtex/acm,../../bibtex/wiley,../../bibtex/book,../../bibtex/thesis,../../bibtex/ebsco,../../bibtex/old,../../bibtex/ieee,../../bibtex/internet,../../bibtex/ssrn,../../bibtex/apa,../../bibtex/blackwell,../../bibtex/sage,../../bibtex/springer}

\end{document}


