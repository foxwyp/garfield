\input{Ctex4xetex.Cfg}
\documentclass[12pt,a4paper]{Ctexart}

\usepackage{fontspec}
\usepackage[CJKaddspaces]{xeCJK}
\setmainfont{Times New Roman}
\setCJKmainfont{Adobe Song Std}
\setCJKfamilyfont{hei}{Microsoft YaHei}
\setCJKfamilyfont{song}{SimSun}

\CTEXsetup[format={\CJKfamily{hei}\zihao{3}\centering}]{chapter}
\CTEXsetup[format+={\CJKfamily{hei}\zihao{4}\flushleft}]{section}
\CTEXsetup[format+={\CJKfamily{hei}\zihao{-4}\flushleft}]{subsection}


\usepackage{geometry}
\usepackage[dvips]{xcolor}
\geometry{vmargin={25mm},hmargin={30mm,20mm}}


\newcommand{\chuhao}{\fontsize{42pt}{\baselinestrench}\selectfont} % 字号设置
\newcommand{\xiaochuhao}{\fontsize{36pt}{\baselineskip}\selectfont} % 字号设置
\newcommand{\yihao}{\fontsize{28pt}{\baselineskip}\selectfont} % 字号设置
\newcommand{\erhao}{\fontsize{21pt}{\baselineskip}\selectfont} % 字号设置
\newcommand{\xiaoerhao}{\fontsize{18pt}{\baselineskip}\selectfont} % 字号设置
\newcommand{\sanhao}{\fontsize{15.75pt}{\baselineskip}\selectfont} % 字号设置
\newcommand{\sihao}{\fontsize{14pt}{\baselineskip}\selectfont} % 字号设置
\newcommand{\xiaosihao}{\fontsize{12pt}{\baselineskip}\selectfont} % 字号设置
\newcommand{\wuhao}{\fontsize{10.5pt}{\baselineskip}\selectfont} % 字号设置
\newcommand{\xiaowuhao}{\fontsize{9pt}{\baselineskip}\selectfont} % 字号设置
\newcommand{\liuhao}{\fontsize{7.875pt}{\baselineskip}\selectfont} % 字号设置
\newcommand{\qihao}{\fontsize{5.25pt}{\baselineskip}\selectfont} % 字号设置


\begin{document}
% fisrt page
  \title{企业知识管理研究\\\large{ 网格环境下的知识沟通}}
  \author {吴云鹏}
   \maketitle
% directory
 \newpage
  \tableofcontents
% text  



\newpage
  \section{论文选题依据}

  \subsection{研究背景}
人类的实践活动始终伴随着各类知识的创造,运用,丰富与发展,人们认为,占
有更多的知识能够提高生产的效率,创造更大的价值。如今,越来越多的企业认
识到,要想获得持续成功就必须管理好组织中的各种知识。正如温特
\cite{Williamson1994}指出的:“企业就是知道如何做事的组织。”不论企业是
生产产品还是提供服务或者二者兼有,都依赖于企业知道“如何做事”,也就是企
业中的知识。在当今知识经济的时代,知识已经成为最重要的经济资源,人们对
知识的追求与竞争正变得愈发激烈起来。对于企业来说,对知识的追求就是对超
额利润的追求。随着社会分工和社会化大生产变得更加精细与完善,传统的生产
资源已经不能再为企业贡献超额利润了,知识几乎成为超额利润的唯一来源。企业来要想获得长
久稳定的竞争优势,就必须在知识的占有和管理上取得领先。

尽管知识总是被不自觉地同信息和数据混淆,但是人们普遍认为知识要比信
息、数据具有更广泛更深刻的内涵和更丰富的内容。企业中的知识构成非常复
杂,它是一种混合
体,包括成体系的经验、企业价值、与环境相关的信息和专家的洞察力等等。企
业中的知识不仅存在于文档和数据库中,也存在于企业每日的例行事务,企业过
程和标准规范中。随着劳动分工的日益精细,导致了知识分工的出现。一方面,
人们在越来越窄的范围内知道的越来越多;另一方面,也随之出现了越来越多的
“知识岛屿”,使得如何进行知识沟通成为管理者面临的重大问题。企业中知识复
杂性为知识管理知识带来了巨大的困难,现有的管理理论和技术手段已经不能满足企
业对管理知识的巨大需求,因此,研究在互联网条件下如何进行有效的知识沟通
就成为知识管理研究者紧迫的任务。

知识管理的主要研究领域目前主要还集中在知识管理的“前端”,即知识获取的方
法和技术上。知识抽取、知识表示、知识存贮等方面是知识管理研究最多的方
面,尽管这些领域很重要,研究也取得了丰硕的成果,但是研究表明,对于企业
来说,进行有效的知识管理仍然是困难的。一份对欧洲和北美的研究报告显示,只有13\%的经理认为
知识在企业各组织单元中有效地进行传递,大部分的知识管理都是失败的。那
么,一个问题就很自然地摆在人们面前:是不是进行了知识抽取、表示、存储等
等工作,知识共享(这应该是知识管理的重要目标之一)就自动发生了?答案是
显然的,知识管理应该是一系列管理和控制过程,当前的研究实际上忽略了这个
链条的后端--知识如何在员工中传递和共享--的研究。本文就是致力于回
答这个问题而开展研究工作的。

\section{文献综述}

知识管理的最终目的是促进知识共享,或者说--知识的沟通。\textcolor{red}{广义地讲,知识共
享就是个体之间交换知识并协同产生新知识的过程。}知识必须被他人共享才能
最大化其价值。不同类型的知识其共享传递的方式也不相同:隐性知识由于
难于用语言清晰表达,因此共享的方式以示范模式为主。在
示范模式中,知识的转移高度依赖于接受和发送双方的互动。\cite{zhoubo2006}显性知识可以有形
式化系统的语言描述,因此它的传播方式是用品模式。知识通过声、文字、图像
等手段从发送者传递到接收者。用品模式的主要特点是知识可以脱离知识发送者
而“独立”存在。\cite{zhoubo2006}

\subsection{知识的价格}

知识沟通实际上就是知识的传播,它可以具有多种形式。例如以公共品渠道发布
知识,非经济的方式转知识以及知识交易。在这三种方式中,只有知识交易使
能够成为长期稳定的知识传播方式。知识交易也就是知识的买卖,交易双方以知
识本身为商品,进行等价交换。由于知识本身所具有的特殊性,知识交
易自然也同普通的商品交易有所区别。周波在其博士论文中对于知识交易做出了
如下定义:“知识交易就是知识拥有者通过知识转移进行“排他性”控制获得经济
利益的过程,交易结果是实现知识转移。”\cite{zhoubo2006}
尽管前两种方式在现实中并不少见,但是正如达
文波特指出的:“那些认为知识无需要经济激励就可以自
然扩散的想法是乌托邦式的;在没有回报预期的条件下人们不太可能贡献出具有
价值的知识。”\cite{davenport1998wko}每个理性人都会根据自己的判断,对于
自身所掌握的知识估计其价值。

知识的价格一直是知识交易的难题,显然,建立在等价交换基础上的交易必需对
交易物--知识本身--进行定价。按照马克思的观点:商品的交换价值由生产该商
品的平均的社会必要劳动时间所决定,价格围绕交换价值波动,并最终靠向交换
价值。由于知识本身的特殊性,使得价值理论不太适用于解释知识的价格问题。
首先:从长期来看,知识会逐渐成为公共品,也就是不再会有人宣称对该知识拥
有所有权并收取费用,任何人都可以免费获得该知识;其次,从中期来看,知识
的价格可能会产生剧烈的波动,在某一个时期可能无人问津,在另一个时期又突
然收到人们关注,使价格大大提升;再次,知识的创新已经越来越呈现高投入、
高风险的特征,但是知识本身的价格却不一定与投入相符合。从以上的分析可以
看出,对于知识价格的确定,使用效用理论进行分析更为恰当。效用理论认为:
商品的价值由效用决定,效用直接反映在交易的价格上。或者,商品的价值是该
商品所能够支配的劳动量而不是生产商品本身的劳动量。效用本身是动态的、易
变的,由此价格也会随着客观环境的变化而变动,这也同我们在日常生活中观察
的知识价格的变动相一致。但是,为知识的定价同普通商品的定价由本质的不同
。人们对于某一普通商品的效用会有大致相同或相似的预期和评价,因此普通
商品的价格制定也相对容易,价格的波动方向也大致相同。对于知识来说,知识
生产本身需要大量投入,知识的效用在不同时期会发生巨大的变化,而知识的复
制成本几乎为零,这会使投机行为大量出现。如果知识本身不能以均衡价格交易
的话,供求双方至少会有一方的福利受到损害。

知识价格的确定揭示了两个事实,一是知识的需求和该知识的稀缺程度并没有什么直接联系,一件知识可能只有一个
  人掌握,但是却没有人需要它,比如说茴香豆的茴的几种写法。二是知识创造
  和获取的难度也同价格没有直接联系,如果人们觉得某件知识没有什么用武之
  地,即使创造或获得该知识耗费了大量的资源,也不会有交易的愿望。由于知
  识持有者对知识价格的估计具有极大的主观性,因此他心目中的价格往往和外
  界所接受的只是价格有很大的差距,如果他为某件知识投入了大量的资源而该
  知识却不能以他理想的价格交易出去,那么他会认为交易只是所获得价值不能
  补偿自己的投入而暂停交易,以等待该知识的价
  值能逐渐被外界认识,并以一个理想的价格交易出去。这种知识沟通双方对知
  识价值的认知差距,导致知识往往被少数人掌握而不愿以共享。

\subsection{交易方式}
知识的交易方式可以分成两类,一种是直接的知识交易。交易的标的就是知
识本身,比如一项技术的转让,一本书籍的销售,都是买卖双方直接根据知识的
价值进行交易的。还有一类交易方式是知识服务。交易的一方利用自己的专业知
识生产出某项知识产品,交易对象是服务本身。

尽管第一类交易是最直接,也是人们最先开始实用的交易方式,但是这种方式越
来越暴露出一些问题:
\begin{enumerate}
\item 知识本身难于定价。
\item 供需矛盾
\item 接受者的意愿与能力
\end{enumerate}



尽管几乎所有的人类活动都离不开既有知识对人们行为的指导,但是在经济领域中,知识却不是一开始就作为一种必要的生产资源纳入到研究中来的。在古典经济学中,土地、劳动、资本三位一体是生产的
基本要素,知识根本就没有在研究者的概念中出现。直到二十世纪早期,知识的作用才躲在各具特色的企业家理论背后羞涩地显示出来。其中熊彼得的“创新者”
理论出现最具有代表性。他在《经济发展理论》( 1912 年)一书中提出了“ 创
新理论”,首次将 “ 创新 ” 视为经济增长的内生变。以后又在其他著作里加以
应用和发展。 1942 年,熊彼得《资本主义、社会主义和民主主义》一书出版,
标志着他的 “ 创新理论 ” 体系最后完成。熊彼得认为,”创新”就是建立一种新
的生产要素组合的生产函数,新组合包括: 一 引入一种新产品或提供一种产品的新质量; 2. 采用一种新的生产方式; 3. 开辟一个新的市场; 4. 获得一种原料或半成品的新的供给来源; 5. 实行一种新的企业组织形式,例如建立一种垄断地位或打破垄断地位。熊彼得特别强调组织创新、管理创新、制度创新、社会创新和技术创新之间的联系。创新就意味着一种新的生产函数的建立,通过创新,企业降低了生产成本,提高了生产的数量和质量,从而打破了市场均衡,进而获得超额利润。在“创新理论”中,知识以企业家的智慧、能力、经验、阅历、洞察力、决断力等品质表现出来,成为影响企业获得利润的重要因素。同时代的奈特、柯兹纳提出了类似的理论,认为企业家的个人能力是企业发展的动力和源泉。尽管他们的成果并没有明确地指出知识的根本性作用,但是企业家作为知识的载体所体现出的作用得到了充分的肯定。

但是,这些理论的局限性也是显而易见的。将知识羞涩地隐藏在企业家身后而将发展的动力完全归于企业家个人,随着社会和经济的发展已经不断面临新的困境。比如,如何解释同一企业家在不同企业工作但是企业经营状况却大相径庭的现象?对于某个小企业可能经营状况并不如意,为什么会有大企业愿意花大价钱并购?这些问题运用传统理论几乎无法解释,预示着传统理论的固有缺陷并提出了新的理论问题,除了企业家之外,还有哪些因素是企业获得利润的因素?企业家是否还是根本因素之一,如果不是,那么企业获得利润的根本因素是什么?

社会的进步和发展呼唤生产力的提升,而有效的社会分工是保证生产力提升的生产方式。随着分工的不断细化,各种分工的专业化不断增强,不同分工之间的关系在不断加强的同时也在不断产生“隔阂”。在各个分工中所产生的知识也体现出专门性和深入性的特点。知识爆炸使得人们即使是在本专业内也难以学习掌握所有的知识,更不要说出现“通才”。企业家的个人能力被越来越限制在一个狭窄的社会分工中。
知识资产已经成为企业发展最重要的驱动力。Leventhal和March将其描述为“由
组织内部的个人或团体持有的一组特殊的竞争力\cite{levinthal1993ml}.企业
不但自主进行研发以获得知识资产,也可以从外部获得知识资产。现今各类组织
都进行各种各样的知识管理活动,以促进知识资产的挖掘整合共享,最大限度地
实现知识资产的保值和增值。伴随各种各样的知识服务,知识资产的流动也变得
愈发频繁起来。有效的知识管理方式和恰当的制度是促进知识资产流动的重要因
素。随着知识网格技术的成熟和发展,出现了知识集市 。\cite{Andreas2007}

第一章	引言 (10页)
第二章	知识交易与知识服务 (20页)
第三章	知识服务交易机制  (20页)
第四章	网格环境下知识服务的定价策略 (30页)
第五章	实证研究 (10页)
第六章	结论  (3页)

\section{知识传递}
知识在组织和个人之间的相互流动,就是知识传递的过程。按照Davenport和
Prusak\cite{davenport1998wko}的说法:将知识传递给潜在的接收者,并有接
收者加以利用。知识传递之所以会受
到研究者的重视,是因为知识传递的模式和质量直接影响知识共享的效果。最初
对于知识的传递是停留在信息传递的基础上的,以香农\cite{Shannon1949}为代表的信息论认为,信
息的传递是没有开销和即时的,这种观点在相当长的一段时间内被认为理所应当
的,并其应用于知识传递的理论研究中。然而,大量的事实表明,知识在其组织
和人员之间流动并不是自动发生的,反而面临着各种各样的阻力,需要投入和消
耗大量的时间、人力和其它资源去推动知识的流动。在很大程度上,知识的流动
性是非常“粘滞”的。Szulanski\cite{szulanski2000pkt}给出了一个知识传递的
模型。他认为知识传递分为4个过程:初始阶段、实现阶段、跃升阶段与集成阶
段。在初始阶段,知识传递的双方建立“连接”,就知识传递的方式、内容进行交
互并达成一致;在实现阶段,双方开始进行知识的传递,只是由发送方经过一定
的“调制”传递到接收方,接收方进行“解调”后,完成实现的过程。随后,接收方
开始使用新接收的知识,并且试图去解决以前所未遇到过的问题或是无法解决的
问题;最后,接收方从对新知识的运用中,对所接受的知识进一步加深理解,融会贯通,对知识
的应用也更为纯熟,通过新知识的运用圆满地解决问题,知识传递进入了集成阶
段。这四个阶段,刻画了知识传递的整个过程。这个过程不但包括了人们对知识
传递的狭义的理解---从知识的发送方传递到接收方,还包括了接收方对知识的
使用、理解的过程。只有新的知识能够解决发送方的问题,被发送方认可并融入
到自身的知识体系,知识传递的过程才算完成。

这个模型清楚地刻画了知识传递的流程,从而是我们能够对知识传递过程中所遇
到的各种阻碍因素清楚地描述出来。事实上,这些阻碍因素存在于知识传递的各
个阶段,而每个阶段的阻碍因素都有其各自不同的特点。

\subsection{初始阶段}


在传递的起始阶段,传递面临的主要问题是:如何识别出传递的双方以及用什么
方式进行传递。Ounjian和Carne\cite{ounjian1987sfa}认为:在知识传递的初
始阶段,需要双方共同努力确定知识传递的范围、选择传递的时机、评估传递的
成本并建立传递双方的义务。从知识
传递的特点来看,可以分为两种模式:“推”模式和“拉”模式。在“推”模式下,知
识的发送方主动将知识通过某种形式传播出去,传播的对象可以是具体的(比如
说一场讲座的听众),也可能是抽象的(比如一份文件的预期读者)。在“推”模式下,知识
的发送方要做出一系列的相关决策:传播什么样的知识?传递的方式是什么?如
何控制传递的“度”和“量”?如何保证知识的版权和安全?只有当这些问题解决之
后,才可能开始知识的传递过程。在“拉”的模式下,知识的接收方作为共享的发
起者主动寻找知识,这是一种问题驱动的模式。同样,接收方也面临类似的问题:首先,到哪里
寻找相关的知识;其次:当面临多种选择的时候,以什么样的标准选择知识的传
递方;第三:如果没有找到合适的传递方,应该如何做下一步的决策。从以上的
分析可以看出,知识传递的开始阶段,必然有一个传递过程的发起者,而这个发
起者面临着一系列的决策问题,任何一个问题如果不能圆满解决,都会直接影响
知识传递过程的进行。

既然知识共享是传递双方共同作用的结果,那么双方不仅面临这自己各自需要考
虑的担任决策问题,还需要同时考虑与对方相互所产生的决策问题。对于知识的
发送方来说,他需要考虑接收方是否愿意接收知识。在实际的知识传递过程中,
接收方可能处于各种原因,不愿以接收知识。随着知识分工的不断深入和知识总
体数量的日益庞大,人们所掌握的专业知识也限制在越来越狭窄的范围内。而不
断发展的社会分工有要求多学科知识的交叉和集成,在这种情况下,知识的接收
者往往不愿意去学习和自己的工作及知识结构不同的新知识、或者是迫于各种工
作压力没有时间去学习,因为学习和掌握知识是要消耗自身的时间和精力的。另
一方面,学习使用其他人传递的知识,会在一定程度上影响接收方的社会地位和
自我认知。在一个组织中,员工对于不同来源的知识具有不同的反映。Menon等\cite{2203456620060801}通过对两家公司在合并前后员工对
待对方知识的态度研究发现:员工对于来源于组织内部的知识往往持有否定态度。
因为他们认为,吸取采纳别人的知识会使自己作为一个“跟随者”从而影响在组织
中的地位,特别是当双方水平能力接近的时候,员工往往处于维护自己的自我认
知和组织地位的需要,贬低对方知识的价值,阻止其知识在组织中传递。相反
的,组织内部的员工往往对于组织外部的知识报以欢迎的态度,这是因为外部知
识对与员工自身的利益不具有什么威胁,因此愿意使用这些知识。

对于接收方,他对知识发送方的判断是影响传递决策的重要因素。他需要足够的
信息来决定是否要相信发送方,从而采纳他的知识。如果接收方不能做出肯定的
判断,或者判断所需的信息不足,那么即使发送方有能力提供正确的知识,接收
方也不会采纳,甚至知识传递根本就不会发生。

\subsection{实现阶段}
初始阶段的完成为知识共享提供了必要的准备,在实现阶段,知识开始从源端发
送到接收端。显然,这个阶段所面临的最大问题是沟通问题。知识传递的两端可
能具有不同的背景、文化、经验和阅历等不同,如何跨越这些鸿沟是知识顺利传
递的基本要求。大量的实证研究表明,越是具有相似的应用环境,知识的转移就
越有效。同时,接收别人的知识往往有可
能同自己的知识结构、经验、直觉甚至信仰发生冲突。如果接收方认为这种矛盾
不能解决,或者需要较大成本融合进本身的知识体系,传递过程也不会发生。


\subsection{跃升阶段}

在知识被接收方获取后,接收方开始使用新的知识解决问题。如果需要解决的问
题是简单的,那么应用新知识很容易产生效果,满足接收方的需要。如果为体本
身比较复杂,那么新知识可能不能简单地直接使用,何时需要使用者进行一定的
处理加工,使之适应新环境的需要。如果在使用的过程中不断出现新的问题或者
使用者未预料到的情况,那么新知识的应用就更加困难。这就要求使用者本身要
有深厚的知识积累和知识的创新能力。对于新掌握的知识
自然会需要一定的练习后才能取得满意的效果,如果在这个阶段出现意外的情
况,接收方会由于不满新知识带来的收益而否定甚至放弃。应用新的知识解决问
题还可能同自己的日常工作行为和习惯矛盾,更改习惯所产生的“切换成本”也会
对接受者对新知识发生的效益降低。总之,如果新知识不能达到接收者的期望,
则他将排斥新知识,使其不能发挥其应有的作用,知识传递的过程也就此中断。

\subsection{集成阶段}




\subsection{知识传递}
从知识传递的模型中,我们可以看到,知识从发送方(源)的大脑中、以某种形
式在知识的接收方的大脑中重构出了同样的内容。但是,这个模型并没有解释在
传递的过程中、传递的对象到底是什么:是信息,还是内容,或者就是知识?还
有没有其他的部分在传递过程中一并传送?如果传递的对象是知识,那么它和信
息的传播又有什么不同?Polanyi\cite{polanyi1998pkt}在研究个人知识的过程
中认为知识本身本身是个人的、主观的、隐形的。所有的知识要么本身就是隐性
知识,要么源于隐性知识,其显性的程度取决于人的表达能力。人在学习、使用、
表达知识的时候,总会带入自己的感情因素,因此,事实上并不存在所谓的客观
知识。同时,知识又是在一个特定的社会环境下产生、传递的,所以在传递的过
程中、不仅仅是个人知识拥有的个体知识、还包括个人拥有的社会知识也同时进
行了传递。按照Polanyi的看法,传递的对象远不止一个静态的“知识”那么简单,更
重要的是传递“认知的过程”,他称之为通过“传统”(tradition)传递知识。在
这个传递过程中,行为模式、准则、价值观以及知识本身统统是传递的对象,而
只是传递也并不是把知识通过某种媒介向信号一样发送出去。知识的发送端试图
帮助接收端建立一个特定的过程,接收端会根据掌握的经验、知识构建一个自己
的版本。发送端期望通过这个构建过程,可以使接收方达到对一件知识同等程度
的认识和理解。Sveiby\cite{sveiby1996tka}非常支持Polanyi的观点。他认为
知识通过信息传输是错误的,并且和很可能是无收益的。只有在社会互动的环境
下,人与人之间面对面的交流,才能真正地实现知识的传递。而信息应该仅仅被
实为一种“等价且无意义”的信号,如何解释这种信号取决与接受者自身。这就解
释了为什么不同的人对同样的知识理解的程度和角度有所区别。发送方和接收方
拥有的背景知识的结构越是相似,对转送内容的解释也就越相似。

Dixon\cite{dixon2000ckc}在Common konwledge一书中通过对经典的案例分析
也得出类似的结论:仅仅依靠技术手段进行沟通是不够的,必须和面对面的交流
结合使用,才能建立更有效的知识共享系统。技术手段和面对面交流\footnote{这里
  面对面并不一定是在实际环境中的交流,随着虚拟技术的发展,利用视频会议
  等手段同样可以达到面对面交流的作用。员工经过培训后会逐渐适应这种虚拟
  的面对面交流,不但节省了交流的费用,也提高了效率。}各有优缺
点,相互之间不能替代。Dixon将知识转移归纳为五种类型:连续转移、近转移、
远转移、战略转移和专家转移。连续转移是指团队在重复完成某项工作时不断总
结经验,内省不足与有点。近转移是指工作环境于成员背景相同或相似的团队间
转移共享知识。当近转移条件下任务的性质变为非常规的、并且转移的知识常常
是隐性知识的情况下,近转移变为了远转移。战略转移是一个团队向另一个团队
转移非常复杂的知识。专家转移是针对工作中偶尔出现的明晰的、可以使用符号、
公式等表述的问题进行解决或者解答。其中连续转移、远转移、战略转移这三种类型都
是面对面沟通为主要手段、辅以信息技术为支持;而近转移和专家转移则是通过
系统,形式化地表示转移内容,辅以面对面的沟通来完成的。这是因为前三种转
移的知识大多是隐性的,而后两种转移的知识往往是显性的。

知识的转移不仅仅是信息的转移。Lesser\cite{lesser2000kac}指出知识管理最
大的问题在于使用信息管理的手段和理念去设计知识管理系统。很多企业试图建
立一个完善的知识系统,员工从系统里学习知识、寻找解决问题的方法;同时将
自己在实践中所获得的知识记录下来并提交到知识库以供分享。然而典型的后果
是:员工在领导的压迫下撰写完毕知识文档,知识库里很快累积了大量的经验总
结,但是就是没有人去使用。一方面,员工在复杂的工作流程里很难提取出究竟
哪些是有价值的知识,另一方面,读者普遍感到知识文档太空太泛,缺少实际的
指导作用和实用价值。知识的转移需要与转移的对象、环境相配合,然而信息技
术的发展是人们将目光集中在信息而不是知识的处理上,进而忽略了转移的对象
和转移的环境。在这种情况下,共享者不知道预期读者是谁,他会解决什么样的
问题,而知识的消费者也难于描述自己的知识需求,找到合适的知识。Lesser进
一步总结了知识与信息的六个不同:
\begin{enumerate}
\item 认知是一种人类的行为
\item 知识是思考的结果
\item 知识总是在当前产生的
\item 知识属于社区
\item 知识在社区内流转的形式多种多样
\item 新知识产生于旧知识的边界
\end{enumerate}









\bibliographystyle{unsrt}
\bibliography{../../bibtex/elsevier,../../bibtex/emerald,../../bibtex/chinese,../../bibtex/jstor,../../bibtex/citeseer,../../bibtex/acm,../../bibtex/wiley,../../bibtex/book,../../bibtex/thesis,../../bibtex/ebsco,../../bibtex/old,../../bibtex/ieee}

\end{document}


