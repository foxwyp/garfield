\documentclass{elsarticle}

\begin{document}
\journal{Expert Systems with Applications}

\makeatletter
\newcommand\arraybslash{\let\\\@arraycr}
\makeatother



\begin{frontmatter}

  \title{ An Analyzing Method for Influencing Factors of Enterprise Knowledge Creation Capability}

  \author[buaa]{Yunpeng Wu\corref{cor1}}
  \ead{yunpeng.wu@sem.buaa.edu.cn}

  \author[buaa]{Lu Liu}
  \ead{liulu@buaa.edu.cn}
  

  \cortext[cor1]{Corresponding author}

  \address[buaa]{School of Economics and Management, Beihang University, \par
    Beijing 100191, P.R. China}
  \begin{abstract}
Effective knowledge creation is very important to enterprises to gain and maintain competitive predominance and enterprise knowledge creation capability (KCC) plays a key role in achieving effective knowledge innovation. In this article, we induce influencing factors of knowledge creation capability from five different aspects through analyzing process of knowledge creation. After that, by applying the fuzzy set theory and fuzzy clustering approach to analyze enterprise KCC, a method based on fuzzy clustering is proposed for analyzing influencing factors of enterprise KCC. Using this method, the enterprise can find out the key attribute set of influencing factors of enterprise KCC then the key influencing factors which decide its enterprise KCC can be deduced. Enterprises can adjust their strategies of knowledge management (KM) and knowledge creation according to the analytical result to improve its knowledge creation capability and gain core competence. Finally, we provide an application case to illustrate the application of the presented method.
  \end{abstract}

  \begin{keyword}
  \end{keyword}
\end{frontmatter}

\section{ Introduction}
Measuring  the impact of knowledge management are questions that never
go away. When organizations put so many efforts and resources into
knowledge management practices, they need to know what outcomes
knowledge management will bring about.  Nevertheless,  the measures that are
available to evaluate KM, either tools or models, are still
unsatisfactory.  One of the main reasons is that the effect of KM are
often indirect and hard to isolated from other impact
factors. Although the KM executives can run some semi-controlled
experiments between similar teams or groups in the organization to
determine the differences of  output with and without KM practices, or
take an longitudinal study  to observe the change at varies input
level of KM, there are still shortcomings. The measurements may be not
 agile enough to reflect current state of KM and need lots of efforts
 and resources to take into practice.   

 On the other hand, the other
 perspective of KM measurement is often neglected by scholars, that
 is, to know where to invest more or less.  KM needs organization
 taking great efforts and allocate proper resources which  takes substantial, consistent funding and staffing. Effective KM is
 not cheap. Every organization must deal with the issue how to
 maximize the outcomes of KM with limited budget.  Which 
 investments are more 'profitable' that lead to better outcomes?
 Which are just for those "good enough" perspective  because  it is
 easy to implement? To determine what investments we need to make to
 move to a higher level of performance, it is necessary to know what
 gaps are holding us back. 

\section{Knolwedge collaboration}
\label{sec:knolw-coll}

Collaboration can be proceeded at differnt levels: individual level,
team level and organization level. For different levels, the
collaboration strategy
and  pattern vary a lot. Individual level collaboration does not care
the effect of leadership which is an important impact factor to
organization level collaboration. Until recently, the researcher of 
measurement framework of organization level knowledge collaboration,
to our knowledge, are still limited.     

Interorganizational
collaborations are critical for a firm’s innovation, particularly when firms
lack sufficient internal R\&D resources \cite{lin2003technology}.

Particularly in the high-tech
industry, research and development (R\&D) collaborations with other organizations have
been regarded as means of competitive strategy.
\section{Knowledge capability}
\label{sec:knowledge-capability}


Cohen and Levinthal
(1990) suggest that an organization’s ability to lever-
age new information depends on its “absorptive
capacity,” which is a function of its prior knowledge
in a related area. Organization identify the value of external
knowledge, absorb the knowledge and take it into practices of
organization. Since knowledge creation is a complex process, xxx
argues external knowledge is a key factor of innovation
capacity. While absorptive capacity plays a key role to transform external knowledge
to innovation capacity. An
organization with strong absorptive capacity shows strong ability to
integrate external knowledge, which amplify the internal knowledge
inventory and in turn stimulate knowledge collaboration.   

\section{Organization support}
\label{sec:organization-support}

Knowledge creation needs a lot of organization resoruces to support
every collabortation activites. Knolwdge collaborators have to make
thier decisions of 
what knowlwdge to share, how to codify konwledge so that knowledge
receiver could adopt, how to apply knolwedge, communicate repeatedly to
achieve common belief, listening to  other's appeal  and build mutual
shared value, all at a cost.   

\bibliography{../../bibtex/elsevier,../../bibtex/emerald,../../bibtex/chinese,../../bibtex/jstor,../../bibtex/citeseer,../../bibtex/acm,../../bibtex/wiley,../../bibtex/book,../../bibtex/thesis,../../bibtex/ebsco,../../bibtex/old,../../bibtex/ieee,../../bibtex/internet,../../bibtex/ssrn,../../bibtex/apa,../../bibtex/blackwell,../../bibtex/sage,../../bibtex/springer,../../bibtex/MESharp,../../bibtex/taylor}

\end{document}

%%% Local Variables: 
%%% mode: latex
%%% TeX-master: t
%%% End: 
