\documentclass[adobefonts,UTF8]{ctexart}
\newcommand{\tabincell}[2]{\begin{tabular}{@{}#1@{}}#2\end{tabular}}
\usepackage{supertabular}
\usepackage{amsmath}
\usepackage[super,square,numbers,sort&compress]{natbib}
\usepackage{geometry}
\geometry{vmargin={25mm},hmargin={30mm,20mm}}
\begin{document}

\title{知识型组织知识协同能力模糊综合评价}

\begin{center}
\Large 知识型组织知识协同能力模糊综合评价  
\end{center}

\section{引言}
\label{sec:intro}

知识协同就是指多个协同参与者通过对其拥有知识的共享与交流,共同实践某一知识型任务的过程,是一种知识资源的整合优化方式\cite{karlenzig2002tap}。组织面对日益激烈的竞争压力,通过有效地运用知识协同手段,能够满足企业对知识创新的迫切需求,提升员工运用知识解决问题的水平,进而帮助组织获得可持续发展的竞争优势。知识协同是知识管理理论的延伸和拓展。同知识管理的目的不同,知识协同的核心不在于知识共享,其目标明确而集中。知识协同结果可以是一个新知识,也可能是实践一个项目或解决一个问题,它更强调的是针对某种主题的实践知识的整合与优化,而不是知识的转移和扩散。
随着知识协同行为在组织中越来越普遍,对协同行为的研究也成为管理领域的研
究热点。对于知识型组织来说,组织的协同能力对于协同的结果有重要影响。因此,客观评价组织的协同能力,能够帮助组织发现自身在知识协同方面的不足,进而引导组织合理投入资源改进、提升协同能力。

当前关于知识协同的研究受到许多学者的关
注, 充分显示了其理论研究价值和实际意义,。但有关知识协同研究仍然处在初始阶段,研究文献数量也不够更丰富。文献\cite{fan2007}指出目前已有文献的研究主题相对比较分散, 研究
视角也有局限, 应用领域偏重于信息技术, 尚没有形成一
套比较完善的理论与方法体系。知识协同理论
与方法不够完善成熟, 使得目前组织的知识协同实践缺乏理论依据和指导。特别是,对于组织协同能力的评价的相关研究,目前还处于
空白。即缺乏成熟的评价指标体系,也未出现相应的案例研究。本文根据已有的研究成果,提出了一个知识协同能力的评价体系,从五个方面分析了影响协同能力的因素。应用该评价体系,组织可以系统、全面地分析评价自身的知识协同能力,从而有针对性地制定知识管理战略,促进组织知识协同实践。

目前,有
关评价方法的研究比较成熟,形成了多种评价方法,
如加权平均法、 功效系数法、 模糊综合评判方法、 层次
分析法等。其中层次
分析法(Analytic Hierarchy Process ,AHP)的应用最为广泛。现有文献表明,通过结合模糊理论,基于三角模糊数的AHP方法非常适合知识协同水平评价这类被评价
对象的性能无法具体量化、 指标权重不容易判断的多
层次指标模糊评价问题\cite{wangjun,panxing}。与其他评价方法相比,这种
方法更易于专家对被评价对象的性能和指标权重进
行判断,使用更方便。

\section{知识协同能力评价指标}
知识协同行为本身是一个多方参与、反复交互的过程
。随着知识型组织所面临的问题的复杂性和多样性日益显著,对于知识
协同的需求也就越发迫切。通过知识协同的方式进行知
识创新, 能够弥补知识缺口, 有效的解决知识情景嵌入和
路径依赖的问题, 消除“ 知识孤岛” , 并可获得多主体、 多
目标、 多任务间 的知识协同效应,提升组织的响应能力,加速知识创新过程。知识协同的效果同组织的知识协同能力密切相关。协同能力越强,组织越能充分调动、利用各类知识资源,从而越可能创造更多的新知识。评价组织的知识协同能力,对于组织具有重要的意义。目前知识型组织
面临的主要问题是缺乏全面、适合企业特点的综合评价指标,对影响知识协同的
内外因素缺乏清晰的认识,从而难于从组织的角度对协同过程提供必要的支持以
及对员工进行相应的激励。从目前国内外的研究成果来看,对于知识协同能力的
评价研究非常少,主要集中在对知识管理影响因素和绩效的评价研究。因此,本文
在综合国内外研究的基础上,提出一个知识协同能力的综合评价指标体系。该
指标体系从五个维度对组织协同能力进行评价。
主要内容包括:
\begin{enumerate}
\item 知识水平。知识协同意在创造新的知识,解决复杂的实际问题,是典型的知识密集活动。组织的知识水平越高,知识储备越丰富,知识协同的基础也就越好。组织的水平不仅指其知识深度,还包括其知识广度。知识协同主要发生于知识结构互补的协同主体之间。如果合作双方的知识完全一致, 他们合作的可能性将大大降低, 因为知识相似性的增加意味着伙伴间彼此交流共享知识的必要性减少\cite{cowan2006evolving}。组织所涉及的知识领域越广泛,知识协同越容易开展。
\item 沟通能力。知识系统需要参与主体间的通力协作。协同主体不仅要有良好的沟通技巧表达自己的意图,更重要的是在协同过程中尊重对方的信念和价值,建立合作更方的相互信任。协同过程需要顾及各方的利益和感受,只有在积极融洽的气氛中,协同才能顺利开展。
\item 信息技术。信息技术的应用已经成为组织开展知识协同的重要基础。信息技术不仅为协同提供了实时在线的交互平台,还能提供各种知识服务支持协同活动,帮助组织成员构建社会网络。应用信息技术可以消除组织内部的知识孤岛,将各类信息和知识有效集成起来,为知识协同提供全面的基础性的支持。
\item 组织氛围。良好的协同氛围是协同活动顺利开展的有力条件之一。在一个宽松、崇尚共享与协作的氛围内,协同实践变得更为普遍和深入。良好的协同氛围意味着协同的成本降低,协同主体将更积极地投入协同活动。
\item 管理控制。知识协同的开展开展离不开组织的支持和管理。协同时间不但
  需要组织在人力和财力上的投入,还需要组织提供清晰的协同目标、良好的过
  程控制、高效的管理流程。没有组织的大力支持,协同活动将很难在更高的层次开展。
\end{enumerate}
针对以上五个一级指标,进一步提出21个二级指标。

尽管指标的提出是根据现有研究成果分析、综合得出,但是这些指标毕竟没有在组织中实际应用验证过。一些指标来自于国外的研究和经验,也不一定适合我国的国情。因此,本文特地针对这些指标进行了实证研究,在实际的知识性组织中验证指标的可信性、可行性和有效性。
问卷根据提出的指标体系,假设每一个指标对组
织知识协同能力有正向影响,进而设计量表,测度其显著程度。问卷采用 5点式
李克特量表 , 受访者可以根据所在组织的情况给
出对每个问题的同意程度 , 打分时用的数字越大表示程度越
高 , 1表示 “ 完全不同意” , 5表示 “完全同意 ”。问卷首先在小范围内进行测试,检验其信度和效度。测试对象选取了某大学信息管理专业的硕士生和博士生26名。量表经过Cronbach A系数来检验量表的内在一致性信度。在
一般情况下, Cronbach A系数大于等于0. 65 可
以接受, 对于尚未验证过的变量尺度, 只要其
Cronbach A系数大于0. 60 即可接受。最终检验结果表明问卷的一致性程度较
好,调查所获得的数据可信度很高。量表还同时经过了效度检验,也取得了较好
的结果。

模型中$X_i$分别代表五个一级指标:知识水平($X_1$),沟通能力($X_2$),信息技
术($X_3$),组织氛围($X_4$)和管理控制($X_5$)。$x_{ij}$代表一级指
标下的二级指标。

问卷在某航空企业设计所
内进行发放。航空企业的设计部门是典型的知识密集型组织。该设计所承担了大
量的型号研制任务,每年均产生许多研究成果。同时设计人员素质普遍较高,日
常的协同活动开展非常普遍,是非常合适的调查对象。本次调查共发放问卷113份,回收98份,回收率为86.7\%,其中有效问卷为
95份。问卷调查对象设计该设计所13个科室,包括设计员、科室主任以及部分
总师,基本囊括了该组织内各种类型的知识员工。通过对受访对象的基本数据进
行卡方检验,表明调查结果同受访者的年龄、性别、教育程度、职位等因素没有
显著关联。

在对问卷进行处理后,使用结构方程模型进行数据分析,研究假设是否成
立。结果经LISREL软件进行分析,使用极大似然估计验证假设。在检验假设之
前,首先对模型的拟合程度进行了检验。检验从3个方面进行衡量:基本的拟合
标准、整体模型拟合度以及模式内在结构拟合度\cite{bagozzi1988evaluation}。
检验表明模型的拟合程度较好,可以检验相应的假设。通过对假设进行分析验
证,结果显示:原有的21个假设中有17个假设成立,即有17个指标对组织知识协
同能力有显著正向影响。剔除掉不支持的指标后,重新构建组织知识协同能力的评价
指标体系,
具体的评价指标如下表所示:
\begin{center}
  

\topcaption {知识协同能力评价指标体系}
  \begin{supertabular}[center]{cll}
\hline
    一级指标&二级指标&说明\\\hline
    知识水平&专家水平&组织中知识专家的总体水平\\
          &知识结构&组织成员所涉及的知识领域    \\
          &组织学习能力&组织成员学习吸收新知识的能力\\
          &组织知识资源&组织所拥有的各类组织知识 \\
    沟通能力&交互频率&组织成员沟通交互的频繁程度\\
          &共享能力&组织成员分享、传递知识的能力\\
          &协调能力&组织成员协调各自分工,共同合作的能力\\
    信息技术&协同平台建设&协同者通过协同平台开展协同活动\\
          &信息技术应用水平&信息技术统现有组织实践结合的水平\\
          &信息技术投入水平&组织对信息技术持续投入、改进的水平\\
          &信息技术集成水平&信息技术集成组织内知识、人员、组织过程等的
          水平\\
    组织氛围&参与程度&组织成员对知识协同的积极程度\\
          &领导推动&组织领导对于知识协同活动的推动和指导程度\\
          &竞争压力&组织成员面临的内外部竞争压力\\
    管理控制&协同组织能力&组织为成员开展协同组织各类资源的能力\\
          &过程控制&组织对协同过程能够有效管理,协助提升协同效率\\
          &激励机制&组织有恰当的激励机制,对协同的参与者有相应的评价手段\\\hline
  \end{supertabular}
\end{center}
可以看到,模型中5个一级指标全部得以保留,表明这几个指标得到了普遍的认
可。二级指标中知识水平和信息技术的二级指标得以全部保留,体现了知识员工
对这两个方面的重视。而沟通能力、组织氛围和管理控制3个方面,均有2级指标
没有获得支持。由于初始的指标体系主要来自西方国家的研究成果,这一方面体现了国内同西方国家在文化上的差异,一方面也表明
我们同知识管理实施较好的发达国家的组织相比,在认识和实践上还有一定的差距。
  \section{模糊层次分析法}

  在多准则、多目标的决策问题上,层次分析法(AHP)是目前应用最广泛、最
  受认可的方法\cite{Wallenius2008}。使用AHP方法发表的学术文章也大大多于其他方法。许多领域的决策问题都可以使用层次分析法辅助进行决策。近
  年来,针对层次分析法的改进也逐渐增多。其中,模糊层次分析法是对传统层
  次分析法的一类重要改进。传统的层次分析法中对于那些无法准确度量的评价准则主
  要是使用自然语
  言描述评价结果。由于自然语言本身就带有一定的模糊性,因此将自然语言评
  价结果很难完全转换为精确的数值。模糊层次分析法通过引入模糊理论,在一
  定程度上比不了上述不足。许多文献使用模糊理论分别提出了不同的改进方法\cite{buckley2001fuzzy,cheng1997evaluating,chang1996applications}。
  其中,chang\cite{chang1996applications}提出的范围模糊分析方法具有同传统AHP方法相似,应用简便的特点。 同时,应用该方法还可以克服以往的研究中采用模糊数进行权重评估时,模糊数重叠造成排序结果不一致的问题\cite{chen1997efficient}。
  在本文中,根据这种方法,采用了三角模糊数来表示
  主观评价的数值。三角模糊数的形式为$\tilde{A}=(a,b,c)$,其隶属度函数
  为:
  \[
  \mu_A~(x)=
  \left\{
      \begin{array}{ll}
        \frac{x-a}{b-a}&a\leq x \leq b\\
        \frac{c-x}{c-b}&b \leq x \leq c\\
        0 &  \text{其他}
      \end{array}
    \right.
\]

在进行评价时,需要对评价准则进行两两比较,比较的结果采用三角模糊数标度来衡量相对重
要程度,本文采用文献\cite{Lee2009}提出的标度,如表2所示:
\begin{table}[hbt]
  \centering
 \caption{三角模糊数标度}
  \begin{tabular}{ll}
    语言标度&三角模糊数标度\\\hline
    两个元素完全一致&$(1,1,1)$\\
    两个元素同等程度重要&$(1/2,1,3/2)$\\
    一个元素比另一个元素略重要&$(1,3/2,2)$\\
    一个元素比另一个元素重要一些&$(3/2,2,5/2)$\\
    一个元素比另一元素重要很多&$(2,5/2,3)$\\
    一个元素比另一个元素极端重要&$(5/2,3,7/2)$\\
  \end{tabular}
 
\end{table}

范围分析需要定义模糊综合评估值。设$X={x_1,x_2,\ldots,x_n}$为属性集合,$U={u_1,u_2,\ldots,u_n}$为目标集合。根据范围分析法可以计算每一个目标集合的属性$g_i$的范围值,$m$个目标集合的范围值可表示为:
\[
\tilde{M}^1_{g_i},\tilde{M}^2_{g_i},\ldots,\tilde{M}^m_{g_i} \qquad i=1,2,\ldots,n \quad \text{其中}\tilde{M}^j_{g_i}(j=1,2,\ldots,m)\text{为三角模糊数}
\]

根据第$i$个属性所定义的模糊综合评估值可表示为:
\[
S_i=\sum^m_{j=1}M^j_{i}\odot
\left[\sum_{i=1}^n\sum_{j=1}^nM_{i}^j\right]^{-1}
\]
其中$\odot$表示模糊乘法,$M_{i}^j$是以三角模糊函数表示的评价准则$i$和
评价准则$j$范围分析值。$S_i$则表示第$i$个评价准则的模糊综合评估值。进
一步可以定义$M_1\geq M_2$的可能性程度:
% \[V(M_1 \geq M_2)=
%  \mathop{sup}_{x \geq y}[min(\mu_{M_1}(x),\mu_{M_2}(y)]
% \]
\[V(M_1 \geq M_2)= hgt(M_1\cap M_2)=
\left\{
    \begin{array}{lll}
      1,& if \ b_1 \geq b_2\\
      0,& if \ a_2 \geq c_1\\
      \frac{a_2-c_1}{(b_1-c_1)-(b_2-a_2)},&\mbox{其他}
    \end{array}
  \right.
\]

对于凸模糊数$M$,$M$大于$k$个凸模糊数$M_i(i=1,2,\ldots,k)$的可能性程度可表
示为:
\[
V(M \geq M_1,M_2,\ldots,M_k)=min \ V(M \geq M_i),\ i=1,2,\ldots,k 
\]
令$d^{'} (A_i)=min \ V(S_i \geq S_k) \ k=1,2,\ldots,n; k \not= i$,评
价准则的权重向量可表示为:
\[
W^{'}=(d^{'}(A_1),d^{'}(A_2),\ldots,d^{'}(A_n))^T
\]
$A_i$是$n$个评价准则。
对$W$进行归一化后,可得到权重向量:
\[
W=(d(A_1),d(A_2),\ldots,d(A_n))^T
\]
$W$是一个确定的数值。

\section{协同评价实例}

某航空企业同时开展研发不同类型和型号的飞机,从而形成了不同的研发团队。
为了更好地实施组织知识管理策略,有针对性地提升组织知识管理水平,推动研
发团队的知识协同,需要对各个团队的知识协同水平进行评价。评价的指标体系
采用前文提出的指标。三位专家$E_1,E_2,E_3$分别对3个团队$T_1,T_2,T_3$进
行评价。专家首先对指标体系中的第一层指标进行了成对比较,比较结果根据表2转换为对应的三角模糊数,从而确定了比较矩阵:
\[
\left[
  \begin{array}{lllll}
(1,1,1)&(3/2,2,5/2)&(5/2,3,7/2)&(2,5/2,3)&(1,3/2,2)\\
       &(1/2,2/3,1)&(3/2,2,5/2)&(1,3/2,2)&(2/5,1/2,2/3)\\
       &(1/2,2/3,1)&(1/2,2/3,1)&(1/3,2/5,1/2)&(1/2,2/3,1)\\
\\
(2/5,1/2,2/3)&(1,1,1)&(1,3/2,2)&(3/2,2,5/2)&(2/5,1/2,2/3)\\
(1,3/2,2)&&(2,5/2,3)&(3/2,2,5/2)&(1/2,2/3,1)\\
(1,3/2,2)&&(1,3/2,2)&(1,1,1)&(1,3/2,2)\\
\\
(2/7,1/3,2/5)&(1/2,2/3,1)&(1,1,1)&(1/2,2/3,1)&(1/3,2/5,1/2)\\
(2/5,1/2,2/3)&(1/3,2/5,1/2)&&(1/2,2/3,1)&(2/7,1/3,2/5)\\
(1,3/2,2)&(1/2,2/3,1)&&(1/2,2/3,1)&(1/2,2/3,1)\\
\\
(1/3,2/5,1/2)&(2/5,1/2,2/3)&(1,3/2,2)&(1,1,1)&(1/2,2/3,1)\\
(1/2,2/3,1)&(2/5,1/2,2/3)&(1,3/2,2)&&(1/3,2/5,1/2)\\
(2,5/2,3)&(1,1,1)&(1,3/2,2)&&(1,3/2,2)\\
\\
(1/2,2/3,1)&(3/2,2,5/2)&(2,5/2,3)&(1,3/2,2)&(1,1,1)\\
(3/2,2,5/2)&(1,3/2,2)&(5/2,3,7/2)&(2,5/2,3)&\\
(1,3/2,2)&(1/2,2/3,1)&(1,3/2,2)&(1/2,2/3,1)&\\

  \end{array}
\right]
\]
对专家的打分求均值,得到矩阵:

\[
\left[
\small
  \begin{array}{ccccc}
    (1,1,1)&(0.833,1.111,1.5)&(1.5,1.889,2.333)&(1.111,1.467,1.833)&(0.633,0.889,1.222)\\
    (0.8,1.167,1.556)&(1,1,1)&(1.333,1.833,2.333)&(1.333,1.667,2)&(0.633,0.889,1.222)\\
    (0.562,0.778,1.022)&(0.444,0.578,0.833)&(1,1,1)&(0.5,0.667,1)&(0.373,0.467,0.633)\\
    (0.944,1.189,1.5)&(0.6,0.667,0.778)&(1,1.5,2)&(1,1,1)&(0.611,0.856,1.167)\\
    (1,1.389,1.833)&(1,1.389,1.833)&(1.833,2.333,2.833)&(1.167,1.556,2)&(1,1,1)\\
  \end{array}
\right]
\]
使用该矩阵计算各个评价准则的模糊综合评估值。
\begin{eqnarray*}
  \label{eq:1}
  S_1 =
(5.077,6.356,7.888)\odot(\frac{1}{36.431},\frac{1}{29.291},\frac{1}{23.21})=(0.139,0.217,0.340)\\
S_2 =
(5.099,6.556,8.111)\odot(\frac{1}{36.431},\frac{1}{29.291},\frac{1}{23.21})=(0.140,0.224,0.349)\\
S_3 =
(2.879,3.500,4.488)\odot(\frac{1}{36.431},\frac{1}{29.291},\frac{1}{23.21})=(0.079,0.119,0.193)\\
S_4
=(4.155,5.212,6.445)\odot(\frac{1}{36.431},\frac{1}{29.291},\frac{1}{23.21})=(0.114,0.178,0.278)\\
S_5 =(6.000,7.667,9.499)\odot(\frac{1}{36.431},\frac{1}{29.291},\frac{1}{23.21})=(0.165,0.262,0.409)
\end{eqnarray*}

\begin{eqnarray*}
  \label{eq:2}
V(S_1\geq S_2)=  \frac{0.14-0.34}{(0.217-0.34)-(0.224-0.14)}=0.966\\
V(S_1\geq S_3)=V(S_1\geq S_4)=1\\
V(S_1\geq S_5)=\frac{0.165-0.34}{(0.217-0.34)-(0.262-0.165)}=0.795\\
V(S_2\geq S_1)=V(S_2\geq S_3)=V(S_2\geq S_4)=1\\
V(S_2\geq S_5)=\frac{0.165-0.349}{(0.224-0.349)-(0.262-0.165)}=0.394\\
V(S_3\geq S_1)=0.365 \qquad
V(S_3\geq S_2)=0.339\\
V(S_3\geq S_4)=0.576 \qquad
V(S_3\geq S_5)=0.168\\
V(S_4\geq S_1)=0.780 \qquad
V(S_4\geq S_2)=0.750\\
V(S_4\geq S_3)=1 \qquad
V(S_4\geq S_5)=0.574\\
V(S_5\geq S_1)=V(S_5\geq S_2)=V(S_5\geq S_3)=V(S_5\geq S_4)=1
\end{eqnarray*}

\begin{eqnarray*}
  d^{'}(C_1)=0.795 \qquad
  d^{'}(C_2)=0.394\\
  d^{'}(C_3)=0.168\qquad
  d^{'}(C_4)=0.574\qquad
  d^{'}(C_5)=1
\end{eqnarray*}
经过归一化后,可得到一级指标的权重向量$W$:
\[
W=(0.271,0.134,0.057,0.196,0.341)^T
\]

重复使用该方法应用于二级指标,可得到各个二级指标的局部权重,乘以所属一
级指标的权重即可得到二级指标的全局权重。下表列出了各个二级指标的全局权
重。
\begin{table}[htb]
  \centering
\caption{二级指标权重}
  \small
  \begin{tabular}{cccccccccc}
    \multicolumn{2}{c}{知识水平}&\multicolumn{2}{c}{沟通能
      力}&\multicolumn{2}{c}{信息技术}&\multicolumn{2}{c}{组织氛
      围}&\multicolumn{2}{c}{管理控制}\\\hline
    专家水平&0.091&协作能力&0.049&协同工具&0.023&协同文化&0.103&协同目
    标&0.168\\
    知识结构&0.089&共享能力&0.056&知识服务&0.015&领导推动&0.03&过程控
    制&0.061\\
    组织学习能力&0.056&成员互信&0.028&专家推荐&0.012&竞争压力&0.063&激
    励机制&0.112\\
    组织知识资源&0.035&&&知识孤岛&0.007&&&& \\
  \end{tabular}
  
\end{table}

根据各个评价准则,分别对各个项目组进行比较。下表给出了专家根据“专家水
平”指标给出的评价结果:
\[
\left[
  \begin{array}{lllll}
(1,1,1)&(1,3/2,2)&(1/2,2/3,1)\\
       & (1,3/2,2)&(1/2,2/3,1) \\
       &(3/2,2,5/2)&(1,3/2,2)\\
\\
(1/2,2/3,1)&(1,1,1)&(2/5,1/2,2/3)\\
(1/2,2/3,1)& &(1/3,2/5,1/2)\\
(2/5,1/2,2/3)&&(1,3/2,2)\\
\\
(1,3/2,2)&(3/2,2,5/2)&(1,1,1)\\
(1,3/2,2)&(2,5/2,3)& \\
(1/2,2/3,1)&(1/2,2/3,1)&\\
\end{array}
\right]
\]
经计算可得到三个项目组在指标“专家水平”上的得
分:$W=(0.389,0.178,0.432)$。进一步可计算出三个项目组在其他指标上的得
分,乘以相应的权重,从而得到最终的得分。下表列出了三个项目组在各个一级
指标上的得分和最终的总的分。
\begin{table}[htb]
  \centering
  \caption{项目组评价}
  \begin{tabular}{ccccccc}
\hline
    &知识水平&协作能力&信息技术&组织氛围&管理控制&最终得分\\\hline
项目组1&0.129&0.041&0.026&0.054&0.093&0.343\\
项目组2&0.043&0.049&0.014&0.082&0.119&0.307\\
项目组3&0.099&0.044&0.017&0.061&0.128&0.349\\\hline
  \end{tabular}
\end{table}
从结果可以看出,项目组3的知识协同能力在整体上要优于其他两个项目组。进
一步分析各个一级指标的得分,还可以看出知识水平方面能力较强,而项目组2
在协作能力和组织氛围方面做的最好。更进一步可以分析比较各个项目组在二级
指标上的得分,从而找到项
目组在协同能力上的薄弱环节,从而更有利于今后改进自身的协同能力。

\section{结论}

随着知识协同在知识性组织中日益普及,对组织的协同能力进行评价,找到影响协同的主客观因素,从而找出提升组织的协同能力的策略,合理分配组织的知识资源具有重要和积极的意义。本文根据已有的研究成果,提出了评价知识性组织协同能力的指标体系,从五个方面分析了这些指标因素对于协同能力的影响。由于这些指标因素普遍难于量化,而自然语言描述又比较模糊的特点,使用模糊AHP方法对协同能力进行了评价。该方法即结合了传统AHP方法的优点,同时又利用模糊理论较少了评价过程中的信息损失。从实际应用的情况看,组织既可以得到综合评价的结果,又可以针对具体因素进行比较,从而有利于有针对性地改进自身协同能力。
\bibliographystyle{GBT7714-2005NLang-utf8}
\bibliography{bbb}
\end{document}

%%% Local Variables: 
%%% mode: latex
%%% TeX-master: t
%%% End: 
