\chapter{绪论}
  \section{论文选题的背景意义和根据}

  \subsection{研究背景}

  
面对经济的全球化和竞争的日益加剧,人们越来越认识到,知识已经成为组织保持竞争优
势关键的甚至是唯一的要素\cite{Drucker1995}。如何有效地利用并管理好知
识,已经成为组织所面临的最重要的任务之一。随着知识管理的理论逐渐成熟和
信息技术的进步,通过知识管理提升组织的创新能力已经成为当前研究的热点问
题。

然而,尽管知识管理在知识抽取、知识表示、知识存贮等方面取得了大量的研究
成果,但是大量的实践表明,在组织内部实行有效的知识管理仍然是困难的。一份对欧洲和北美的研究报告显示,只有13\%的经理认为
知识在企业各组织单元中有效地进行共享,大部分的知识管理都是不尽如人意的
\cite{Ruggles1998}。组织中依然出现了大量的“知识孤岛”,知识流动的“粘性”\cite{szulanski2000pkt}
问题长期得不到有效解决。造成这种情况的原因在于,人们未能充分地理解知识
共享的机制。组织中的知识构成非常复
杂,并且大量的知识是以隐形知识的形式存在的,包括成体系的经验、价值观、
与环境相关的信息和专家的洞察力等等。这些知识存在于组织每日的例行事务,过
程和标准规范中。隐性知识的共享往往是在无意识的情况下,潜移默化地进行转
移的。Sveiby指出
“问题在于共享不是有意的或者是通过某种正式的渠道进行的$\cdots$ 知识
总是以一种最经济的、无意识无目的的方式一代一代传承下去的
\cite{sveiby1996tka}。因此,当前以
将隐形知识转化为显性知识作为知识管理主要内容的实践并未取得预期的成功。
隐形知识一旦离开了问题本身和应用环境,就很难与其他人共享。

实践社区的兴起为组织知识管理提供了新的管理手段。实践社区的成员用于共同
关注的主体,一起协作解决问题,在实践的过程中实现知识的共享与协同。实践
社区本身是一种松散的组织结构,它在知识创造、
知识交流、知识共享和促进参与者学习方面具有不可低估的作用,被认为是组织内部和组织之间
支持知识共享和知识创新的一种特别有效的组织形式,是正式的组织机构有益的
补充和延伸。可以说,实践社区本身的特点真正体现了组织在知识管理实践中的
作用:更好地培育、利用和激励人们改进和分享他们的行动能
力\cite{Seviby1997}。

实践社区成员自愿组织起来,围绕特定的知识领域共同工作和学习,通过持续的
相互交流分享知识,目的是为了共同完成某项目标。尽管实践过程中存在着大量
的知识共享行为,但是实践社区的本质在于成员间进行知识协同,共同实践某一
知识型任务,从而增强自己在此领
域的知识和技能。协同的结果可以是一个新知识,也可能是实践一个项目或解决
一个问题。社区中的实践活动强调的是针
对某种主题的实践知识的整合与优化,而不是知识的转移和扩散。

当前有关实践社区中知识协同方面深入、系统的研究很少,已有研究大多关注于实践社
区中知识共享和交流方面,缺乏更深层次的研究。另外,大量的实践社区在实践
上取得了成功,迫切需要从理论上进行科学、系统的归纳。对实践社区知识协同的
机理问题进行深入探讨和研究,已经成为知识管理领域重要的研究课题。

\subsection{选题依据及意义}

动机因素研究是揭示实践社区知识协同活动机理的重要内容。在实践社区这样一
个松散的组织中,组织成员为什么要同其他人共享和协同,有哪些因素会影响成
员的动机,研究这问题对于改善协同的效率,提高个体的创新能力具有重要作用。

尽管在心理学和组织行为学等学科中,对于动机因素的研究已
经进行了相当长的时间,但是对知识协同领域的动机因素研究还属于初级阶段。
以往的研究大都集中在对知识共享的动机因素研究方面,讨论的是个体的、局部
的动机,这些动机大多是稳定的、较少变化的。而在实践社区这种存在大量交互、
协同活动的环境下,群体活动会使得动机因素具有更大的动态性和不稳定性,从
而呈现出和个人动机不一样的表现。对于协同活动的动机因素还需要进行
大量的研究。



动机因素可以分为两种:外部动机和内部动机。外部动机是对他人而言
可以接触和观察到的,通过和外界进行物质、能量和信息的交换而获得的。外部
动机是由其他人或机构进行分配,包括工资、福利和晋升。同时也包括避免惩罚
的驱动力。外在的奖励通常是以一致性为基础,即外在的激励因子需要与绩效的
提高保持一致。外部激励因素经常被用于鼓励员工达到更高的绩效水平或者新的
目标。但是外部动机不能完全解释员工个体所进行的每一项努力。
 内部动机是自内部产生的。换句话说,是这些激励因子将个体与任务或者工
 作本身联系起来的。内源性回报包括责任感、成功感、成就感,他们是通过经
 验习得的。还包括挑战感或竞争感。他们涉及某些吸引人的任务或目标。长久
 以来,完成有意义的工作总是与内在动机相联系的\cite{Luthans2001}。

对于外部动机因素来说,尽管通过复杂的薪酬制度设计可以较长时间地提高员工的
工作动机,但是很多学者认为通过激励制度不能带来长期、稳定、持续的动机。“在过去三十年里至少两打以上的研究论文已经得出的结论表明,希望因为完成一项任务或者成功地完成该项任务而得到奖赏的人就是没有根本不想获取报酬的人干得好“\cite{93120316461993}。
另一方面,一些组织是在几乎完全没有外源性
因素的作用下持续运作的,仅仅依靠诸如热情、信仰之类的内源性因素就可以不
断发展壮大下去。因此,本研究将针对内部动机开展,通过对虚拟实践社区进行
实证研究,具体研究有哪些内
部动机因素会影响实践社区中的知识协同行为并分析各个因素之间的联系。

随着知识协同活动越来越显示出对企业知识能力的重要作用,建立完整
的理论框架已经成为管理者迫切的需求。实践社区为知识协同提供了有效的平
台,而如何调动人的积极因素,发挥其主观能动性,是协同活动有效开展的保障。
本研究从动机因素角度着手,分析影响协同者参与协同活动的动机因素,从人的
行为方面揭示知识协同的本质。本研究对于完善和丰富知识管理理论,指导知识
管理实践具有重要的理论和实际意义。

%%% Local Variables: 
%%% mode: latex
%%% TeX-master: "master"
%%% End: 