
\chapter{维基百科用户分析}
\label{cha:wikipedian}

\section{维基百科简介}
维基百科(Wikipedia),网址:http://www.wikipedia.org/  ,是一个语言、
内容开放的网络百科全书计划。英文的“Wikipedia”是“wiki”(一种可供协
作的网络技术)和“encyclopedia”(百科全书)结合而成的混成词。(wikipedia)

维基百科由来自全世界的自愿者协同写作。自2001年1月15日英文维基百科成立
以来,维基百科不断的快速成长,已经成为最大的资料来源网站之一,而以热门
度来说,则为世界第六大的网站,在2008年吸引了超过684,000,000的访客,目
前在272种的独立语言版本中,共有6万名以上的使用者贡献了超过1000万篇条目。
截至今天,共有314,766篇条目以中文撰写;每天有数十万的访客作出数十万次
的编辑,并建立数千篇新条目以让维基百科的内容变得更完整。(请参见维基百
科统计)

维基百科一直坚持内容的开放性。这时维基百科取得成功的一个重要原因。维基
百科所有的内容,包括文字、图片等均在“知识共享 署名-相同方式共享 3.0协
议”之条款下提供。任何人都可以自由引用维基百科的全部或者部分内容,仅需
要注明其出处即可。维基百科不仅赋予内容使用的开放性,还奉行参与的开放性。
这种开放性同自由软件运动有很大的相似之处。Raymond将这种开放、自由的参
与和使用方式比喻为“集市”似的方式,区别于传统的那种严谨、刻板、集中的
“大教堂”方式。任何人只要愿意遵守维基百科社区的政策,就可以加入到协同
创作过程中来。

维基百科的成功证明了“群体智慧”的力量。Surowiecki指出利用群体的智慧开
展协同
工作对于当今的政治、经济、商业等各个方面具有重要的影响。日益频繁的人际
交互已经超出了时空的阻隔,正在改变人们的生活方式。维基百科的成功不仅仅
是信息技术创新的结果,更是人们相互协作,共同应对面临的各种挑战,利用集
体的力量取得成功的最好体现。

\section{维基百科社区的知识协同平台}

Wiki技术本质上是一种超文本系统。这种超文本系统支持面向社群的协作式写
作,同时也包括一组支持这种写作的辅助工具。用户可以在Web的基础上对 Wiki
文本进行浏览、创建、更改等,而且创建、更改、发布的代价远比HTML文本为
小;同时Wiki系统还支持面向社群的协作式写作,为协作式写作提供必要帮助;
最后,Wiki的写作者自然构成了一个社群,Wiki系统为这个社群提供简单的交流
工具。与其它超文本系统相比,Wiki有使用方便及开放的特点,所以Wiki系统可
以帮助我们在一个社群内共享某领域的知识。 (baidu)

Wiki技术为世界各地的人们在互联网上进行协同工作提供了一个技术平台。应用
Wiki技术开展协同的网站最早可追溯到1995年Ward Cunningham创立的c2.com。
随着WEB 2.0的兴起,协同创作这一新的内容产生方式逐渐受到人们的重视。工
业界不断推出新的技术和标准,而学术界也对协同创作的机理、方式、产生的影
响等进行了研究。维基百科在这样的背景下应运而生。

协同创作是在互联网环境下,创作者借助信息技术超越时间和空间的限制,共同
合作完成特定主体的内容创作。随着知识分工的日益细化,仅凭个人所掌握的知
识越来越难易适应不断提升的知识创新需求。为了弥补自身知识的不足,以协同
方式共同完成新知识的产生成了越来越多的人的必然选择。然而,新的知识创新
方式尽管能积极应对知识创新所带来的压力和挑战,也随之产生了新的问题。
Liccardi等人将这些问题归纳为五个方面\cite{liccardi2007caws}:
\begin{enumerate}
\item 沟通不足。
知识协同是促进知识创新的手段,而持续有效的沟通是知识协同的基础。协同平
台不但要保证协同成员间的沟通顺畅,更重要的是能够追溯思想、灵感的产生过
程,确保这些内容能够完整真实地记录下来。
\item 内容与讨论脱节。
知识创新是一个持续、反复的过程中,需要存于者进行大量的讨论、评价,对有
疑问、不明确的地方进行辨析、扬弃,最终达成一致。讨论是最终内容形成的完
整链条,串起了内容各个版本间的变化过程,对于理解最终内容非常重要。一旦
导论于内容间的关系没有得到紧密关联,那么协同过程中必然会出现大量的误解
和矛盾。
\item 对群组讨论缺乏足够的支持。
对内容的讨论常常是以群组讨论的方式开展的。当参与协同的成员众多时,会出
现大量参与者同时参与多个讨论主体的情况,而讨论本身会进一步衍生出其他的
讨论。这就要求协同平台能够支持有效的
群组讨论,维系特定的讨论线索,并有效解决讨论过程中出现的各类问题。
\item 缺乏旧有版本的回溯功能。
协同参与者需要不时审视以前的内容,并做出决定是否之前某一版本更好,进而
应该恢复到那一版本。如果缺乏回溯功能,这些需求将很难得到满足。
\item 解决冲突。
 当多个协同者对内容进行编辑时,必然会产生冲突。冲突的原因是由于协同者
 “同时”对某一段文字进行了修改,一般来说这类冲突很难通过自动化的手段
 解决。好的协同平台除了要及时给出冲突的提示
 外,还应该根据相应的规则和手段协助协同成员解决冲突。
\end{enumerate}

一个好的协同平台,应该能有效地处理好以上问题,为协同者提供强大的技术支
持和保障
Neuwirth等人归纳了一个优秀的协同平台所应该具有的特征:
\begin{enumerate}
\item 提供适合的方法和手段促进创作者在内容创作过程中开展有效的交互。
\item 将内容创作和对内容的讨论,意见等内容有机地结合起来。
\item 提供有效的工具支持协同创作和内容讨论这两种形式的交互。
\end{enumerate}
这几个特征实际上揭示了协同创新的本质。新的内容(或者知识)来源于个
体、团队、组织等不同层次群体之间的交互和沟通,在思维的碰撞中产生。协同
创作的本意在于激发参与者的沟通欲望,调动参与者的协作热情,共同完成某一
主题的知识或者内容的创新。在创新过程中,共享各自的知识已经不再是主要目
的,因此,支持协同成员间的社会化交互,提供高效、简洁、易用的交流平台就
成为wiki平台成功的关键因素。

维基百科从几个方面来保障协同活动的开展。首先,维基百科对每一个条目都设
置了讨论页。讨论页是特殊的维基百科页面,它包含了所有对主题文章的讨论。
任何的问题、疑虑、怀疑、参考文献、有关文章的论战或者评论都可以在相关的
讨论页提出来。在讨论页中,协同者可以分享自己的思想和观点,整理内容创作
的思路和逻辑,分析内容的取舍,澄清材料的真伪,最大限度地保障协同的质量。
讨论页可以包括多个讨论主体,维基百科平台会对最近的讨论话题和热门的讨论
话题醒目地标识出来,从而方便使用者进行讨论。

维基百科对所有内容的变更都保留了记录。事实上维基百科的内容无法作出任何
改变。你只能增加内容。每一次内容的变更则作为一个历史版本保留下来。读者
阅读的每一个条目均只是好像只是一份当前的草稿一样。这个做法使我们能比较
不同版本之间的差异,或者在需要时将文章回复到旧版本。读者甚至可以引用其
中一个特定的版本。你只要在左方控制列的「工具箱」中点下「永久连结」,就
可以连结到该文章版本的网址,其内容永远不会改变。这种做法的最大目的在于将每一
个协同参与者的活动及其蕴涵的信息能够呈现在每一个人面前(而不仅仅是其他
的协同参与者)。不但如此,条目的历史版本还和讨论有效地整合起来。每一个
历史版本都可以连接到针对此版本的讨论主题,从而将内容变更的原因和结果直
接联系起来。Dourish和Bellotti将其称之为共享反馈。

协同编辑带来的一个重要不利影响就是冲突几乎不可避免。冲突可能来自于两方
面的原因:作者所持观点和立场的不同,以及对内容的恶意破坏。维基百科的一
个重要原则是中立和不偏不倚。维基百科的创始人之一吉米 $\cdot$ 威尔士说,必须保
持中立观点(NPOV)这个原则在维基百科中是绝对的和不可争辩的的编辑原则,
维基百科的章程是,“中立观点意味着应努力让支持者和反对者都同意某种观点
或事实……”维基百科的管理员这样解释中立政策,“我们应该把争论中各方面
的声音都公平地表达出来,而不是在文章中指出或暗示任何一方的观点是正确
的”,“中立的立场,中性的描述”。然而,在现实中并不存在纯粹的中立与客
观。当两位或多位协同参与者意见相左时,如果没有良好的冲突解决机制,那么
不断地互相删改对方的编辑内容以维持己方观点,甚至进行人身攻击就必然会发
生。冲突的第二个来源来自于维基百科参与者的匿名性。由于参与到维基百科社
区几乎不需要任何身份的核实于验证过程,因此会有一小部分带有纯粹恶意的人
参与进来。这些人不断地篡改别人的劳动成果,胡乱添加违反维基百科创作原则
的内容。这种做法不但伤害了维基百科内容的真实性和完整性,也大大损害了其
他社区成员的利益和热情。

冲突的极端后果是导致“编辑战”的产生。“编辑战”是指明知会招致反对时仍然固执己见,采取挑衅性的编辑行为,并且反复使用回退功能。编辑战通常会使条目在短时间内内容
频繁变化,并会造成大量的版本覆盖。维基百科社区为此制定了大量的规则和政
策,防止编辑战的产生。条目可能会被被临时甚至永久性的保护或锁定,以进行
一段时间的强制冷却,等待适当时机再次开放编辑。有时也会采取保留较为合适
的版本并长期锁定的措施。对参与编辑战的人员,社区提供了讨论、投诉和仲裁
等机制来解决冲突。在技术上,维基社区采取“回退不过三”的措施,即“一位
编辑者对于一个维基百科的页面,在24小时内,不可以执行多于三次的回退”。
通过这些手段,维基百科社区在最大程度上避免冲突的发生。同时一旦发生冲
突,能够快速进行处理,将冲突的损失将为最低。

\section{维基百科社区的知识协同}
维基百科的协同活动主要是协同协作,是由一群人一起,而非单独一人完成的写
作工作计划。协同的目标是针对某一主题给出其“百科式”的解释,不仅包括该
主题自身的含义,还可能包括其历史背景和演变过程,其他人的评价,对其他方
面的影响等内容。协同的最终结果是一个个具体的条目。维基百科用户可以
参与到绝大部分条目的协同协作中。条目被分为不同的种类,维基百科成不同的
种类为命名空间。维基百科目前有20个命名空间,其中包括9个基本的名字空间
;此外还有两个虚拟名字空间。
表\ref{tab:namespace}给出了维基百科中基本命名空间的清单:

\LTXtable{13cm}{namespace.tbl}

命名空间对维基百科中所有创建的内容用途角度进行了划分。尽管几乎所有的内
容都是社区成员的协同结果,但是这并不意味着这些内容都会纳入到本文的研究
范围。事实上,协同的主要成果是各个条目页面,而其他几个命名空间的内容均
是为了更好地编写条目而提供辅助功能的。因此,本文将维基百科社区内的知识
协同活动限制为共同编写某一条目内容,而忽略其他内容的协同编写活动。这样
的目的是:一方面保留了协同活动的主体,同时有减少了数据分析和处理的难
度,突出了研究的重点。
协同创作条目还可以分为两个部分,条目的编写和讨论。显然,条目的内容本身
是协同的直接结果,条目的编写是直接的协同活动,参与条目创作的用户是协同的直接参与者;而条目的讨论是
协同过程的间接结果,讨论本身不是协同的目的而是必要手段,那么参与讨论的
用户是否也应该作为协同活动的参与
者?维基
百科的讨论的目的在于解决协同创作过程中所遇到的一些问题,主要的方式是头脑风暴,汇
集各方的思路和意见;而条目的编写在于组织各类材料和内容,完成实际的内容
创作,强调“做”而不是“说”。同讨论过程的松散性不同,内容编写是非常正
式、严谨的。Viégas等认为,用户参与讨论本质上是一种合作行为而非协同行
为。讨论的内容可以分为以下几个部分:
\begin{enumerate}
\item 征集合作者。征集着意识到条目内容本身还有不完善的地方,但是个人又
  无力完成,因此号召具有相关知识的人补充完善条目内容。
\item 寻找信息。一些用户试图从条目中寻找相关信息,但是条目内容并未涉及
  这些信息,因此希望有人能够提供这些信息。
\item 讨论本页面的恶意篡改行为。例如是否暂停页面编辑,或者是否取消对页面
  的保护等内容。
\item 讨论条目内容是否符合维基百科的编写指南和相应的政策。维基百科条目
  的编写有许多规定和准则,如果怀疑某一部分内容同这些规定和准则相违背,
  那么参与者会在讨论页提出自己的质疑。
\item 引用其它维基资源。通过引用其他的条目或者讨论内容来解释自己的编辑
  行为。
\item 与讨论主题无关的内容。用户在讨论页发布广告、交友等垃圾或恶意信息。
\item 投票。当某一部分内容存在争议,且没有压倒性的证据支持某一方的论
  点,用户通过投票方式来解决争端。
\item 征集内容审阅着。条目编写者期望提高内容质量,征集熟悉相关领域的用
  户对内容进行审阅,提出具体意见。
\item 其他讨论。

\end{enumerate}
讨论所涉及的内容是协同创作的辅助手段,是提升知识协同水平,达到预期协同
成果的保障。但是讨论本身并不直接产生协同的结果,而仅仅是协同活动的必要
补充。因此,本研究所讨论的知识协同活动并不包括讨论的内容。在本文中,将
知识协同定义为:用户参与的维基百科条目的协同创作活动。

\section{知识协同的参与者}
\label{sec:participants}
维基百科的开放性决定了任何人都可以参与到协同创作中来。不论是在维基百科
中注册的用户还是未注册的匿名用户,均可以为编写条目贡献自己的力量。维基
百科社区将用户分为不同的角色,每种角色有各自的权限。用户的角色还可以进
一步划分为普通用户和管理用户。表\ref{tab:user}列出了普通用户的角色和权
限。

\LTXtable{13cm}{collaboration.tbl}

\LTXtable{13cm}{user.tbl}
从表中可以看出,协同创作的参与者必然属于某个普通用户角色。即使是权限最
低的匿名用户,也可以参与到已创建条目的编写过程中去。而一旦在维基百科进
行注册,则可以获得更高级的编辑功能。维基百科的内容主要是有不同用户创建
的。

管理用户主要参与到社区成员和内容的管理工作。比如用户权限的授予和回收,
管理内容的创建和编辑工作,对争议和冲突进行调解和仲裁,执行特定的任务,
以及开发各类辅助工具方便用户使用维基百科,协助管理者和学术研究人员进行
数据统计和分析。维基百科中的管理用户主要包括:系统管理员;系统行政员;
系统监督员;回退员;IP封禁例外者;账户创建员;上传者;机器人;程序开发
人员等。

管理用户不直接参与到条目的协同创作中,但是管理用户仍然会给条目的内容带
来一定的变动,这些变动包括:
\begin{itemize}
\item 管理员删除条目。由于条目本身未能达到维基百科自身的要求,或者是条
  目内容被其它条目所取代或覆盖,则管理员将删除该条目。
\item 机器人编辑条目内容。机器人是由用户开发的自动化或者半自动化程序,
  参与各类内容创建与编辑的辅助性工作,主要用于自动处理繁琐的格式或数据。机器人按
照预定的目标和规则对页面内容进行重新编辑,比如调整内容的结构,增加条目
间的链接等。
\end{itemize}
管理用户对条目的变更的主要目的是:保障条目的一致性,清除冗余条目,促进条
目内容更加符合维基百科的编写规范和格式要求。管理用户本身并不创建新的内
容,也并不提升已有内容的编写质量。因此,管理用户不视为协同的参与者,其
对于条目内容的变更也不视为知识协同活动。在本研究中,协同的参与者仅限于
普通用户。

\section{协同贡献度量}
在上一部分中,本研究明确定义了维基百科社区的知识协同:维基百科普通用户
共同参与条目内容的编写。参与同一条目编写的用户可能会多达数百人,尽管每
个人都为内容创建贡献了自己的时间、精力和知识,但是每个人对于条目内容的
贡献程度确实各不相同的。为了反应社区成员参与知识协同活动的积极程度和共
贡献大小,需要对用户的协同行为进行度量。度量协同行为对于研究维基百科社区知识协同的模
式、理解协同行为具有重要的意义。首先,度量用户的协同贡献可以更好地促进
虚拟实践社区的发展。对于贡献程度很大的用户,社区可以通过表彰,激励等手
段促进其更好地参与社区的发展中去,同时调整社区对用户的支持力度,最大限
度发挥这些用户的能力,合理利用社区资源。其次,通过度量协同行为可以帮助
社区发现协同平台,协同政策等方面的不足,促进社区对相关的问题进行改进,
更好地支持用户的协同。在本研究中,度量协同活动可以揭示维基百科用户的协
同模式,区分不同的用户群体,从而为分析用户参与维基百科社区知识协同的动
机因素打下基础。

用户的协同贡献可以从多个方面进行度量。Kittur和Chi认为用户的编辑次数
(Edit count)可以用于衡量用户的贡献。一个用户参与的编辑次数越多,那么
意味着他对此条内容的改进就越大,从而作出的贡献也就大于编辑次数少的用户。
根据维基百科社区进行的一次针对英文维基百科的统计结果,维基百科的内容并非是由社区所有的用户持续不
断地进行小规模的改进,最终汇集成现在的内容规模。实际上,维基百科的大部
分内容是由一小部分的用户完成的。统计结果表明,在所有针对条目的编辑
中,超过50\%的编辑是由0.7\%的用户(524人)完成的,而社区中最活跃的2\%
的用户贡献了总编辑次数的74\%。这也就意味着,维基百科的大部分用户仅仅是
参与少量的内容修订,真正对协同创作做出主要贡献的仅仅是一小部分核心用户。正是这些核心用户的努力使
 得维基百科取得了巨大的成功。但是,有学者认为,编辑次数本身只反应了用
 户参与协同活动的活跃程度,而不是对贡献内容多少(Text count)的反应。
 Swartz随机选取了一些条目,分别统计出编辑次数最多的10位用户,以及贡献
 内容最多的10位用户,发现两组用户的差异极大:编辑次数最多的10位用户均
 进行了至少上千次的编辑,但是几
 乎没有人同时成为内容贡献最多的人;贡献内容最多的10位用户最多只进行了
 25次编辑,最少的甚至只进行了一次编辑,但是却贡献了绝大部分的内容
 \cite{aswartz}。基于编辑次数以及基于贡献内容这两种方式反映了不同研究
 着对贡献的不同理解,不同的度量方式所得到的结果也不尽相同。

尽管这两种类型的贡献度量得到了广泛研究,但是其缺点也非常明显。Adler等批评
 这两种度量方式均是不稳健的\cite{Adler2008}。不论是编辑次数还是内容贡
 献,二者都容易被恶意利用,从而导致最终结果产生偏差。如果根据编辑次数
 统计用户贡献,那么一个用户很容易将一次规模较大的修订分解成数个小修订,从
 而增加编辑次数;或者该用户可以进行错误的编辑,然后利用回退功能取消这
 次编辑,同样可以达到欺骗的目的。由于维基百科的条目众多,使得这种行为
 很难被发现。更重要的是,这种行为严重扰乱了条目的正常编辑过程,从而降
 低了条目质量和内容的稳定性,甚至打击其他协同用户的积极性。因此,使用
 编辑次数来衡量用户贡献在实际中效果并不好。类似地,如果以文字数量作为
 贡献度量,那么恶意的用户可以在一次编辑中集中添加大量的文本,随后删除
 这些内容,从而达到欺骗的目的。

这两种贡献度量的方式的根本问题在于:首先,它们不能抑制恶意用户利用度量
本身的缺陷进行欺骗,而且欺骗的行为也难于识别;其次,这两种度量仅仅反映
了内容贡献的一个侧面,而不是完整,全面的衡量用户的贡献。因此,这两种类
型的度量往往会低估用户的实际贡献。编辑次数反映了一个用户参与协同的频
率,但是却无法衡量该用户的“生产力”;而内容贡献仅仅以新增的文本内容为
统计依据,对于那些重组文章内容,修订文字错误,移除恶意篡改等内容维护工
作则忽略不计。因此,迫切需要一种协同贡献的度量,来真实反应维基百科用户
对知识协同的实际作用。

编辑次数和文本数量均是协同贡献的数量指标。对于贡献的度量,更重要的是衡
量其质量。Alder等将贡献的质量定义为内容文本从加入到移除的时间长度。由
于维基百科的条目编辑对于所有人开放,因此条目的内容会很快发生变动。如果
某位用户再一次编辑中所新增的内容质量很高,那么这些内容就会在很长一段时
间内,历经多次修订而得以保留,除非有用户用更高质量的内容替代之。反之,如果一段内容的质
量很低,那么其实际寿命就会很短,很快就在随后的编辑中被取代。因此,内容
的质量可以用一个位于区间$[-1,1]$的常数来表示。基于以上假
设,因此一个用户的内容贡献可以根据其文字寿命来度量。文本寿命是指在一次
编辑过程中,一个用户实际新增加的文字在随后的各个修订版本中所存续的时间。
文本寿命同新增文本的数量和其存续的时间成正比,因此在度量过程中兼顾了内
容的数量和质量两方面的因素。文本寿命的主要缺陷在于:它难于识别恶意的篡
改和破坏。不论用户的编辑属于何种类型,最终都会被视为是用户贡献,而不是
根据其特征将正常的编辑行为和破坏行为区分开来。因此,可以将回退编辑视为
“负”的贡献,一旦某一段文字被撤销或者被新的内容代替,那么即认为该段内
容的作者的贡献为“负”。对于恶意用户的编辑行为,由于其篡改和破坏的内容
往往能在很短的时间内被修正,因此其贡献可以立即被判定为“负”,从而有效
地将恶意用户和正常用户的实际贡献区隔开来。但是,这两种度量方式均不能有
效地反映出从事内容维护工作的用户的贡献。

尽管上述方法对于衡量用户仍有不足之处,但是该方法实际上揭示了用户贡献认
定的本质:被其他协同互用所认可的内容数量。用户创建内容本身是为了完成协
同目标而进行的,其工作必然要被其他协同者所接受。然而,使用基于时间的判定方
式来判断协同内容的质量本身是不稳定的。一段文字内容即使在较长的时间段
内,尽力数个修订版本后仍得以保留并不意味着内容本身是高质量的。一方面,
由于参与内容编辑的用户完全是根据个人兴趣和热情参与进来的,因此不同条目
的用户活跃程度各不相同,对于哪些相对不活跃的条目来说,每次变更需要花费
更多的时间。另一方面,即使内容经历了多次保本修订仍然保留也并不意味着其
质量是受到认可的,有可能是存在着质量更为低劣的内容吸引了协同者的注意。
尤其是编辑过程中出现编辑战或者恶意破坏的行为时,参与者主要精力都用于恢
复正确的内容而无暇顾及其他内容。因此,衡量真正的内容质量需要一个更稳健、
准确的指标。

维基百科的研究者认为,经过一段较长时间的协同编辑,条目最终可以达到一个
稳定、高质量的状态。这也就意味着,一个条目的最终版本可以被视为是一个高
质量的协同成果,最终版本里的所有内容是经过该条目所有参与者共同认可的,
因此可以将最终版本中的内容视为衡量一个用户参与协同活动
所做出贡献的标准。一个条目从最初创建开始,经过不断的完善与修订,所有高
质量的内容得以保留,而低质量的、错误的内容不断得到替换与改进。最终形成
一个稳定的版本。尽管维基百科的所有条目本质上并不存在一个最终的版本,但
是经过较长时间的演化,条目的内容基本上已经固定,修订则主要集中在新内容
的添加及修正原有内容的错误等方面,在总体上已经十分接近最终版本。因此,
本研究认为,在某个时间点上,对于那些内容十分稳定成熟的来说,可以使用该
事件点上的最后一个版本作为条目最终版本的近似,用以分析协同用户各自的具
体贡献。

记维基百科中所有的内容条目集合为$E=\{e_i|i=1,2,3,\ldots \}$,协同用户
的集合为$A=\{a_i|i=1,2,3,\ldots \}$,则对于任意一个条目$e_i$,在任意一
个时间点上必然存在$n, n>0$个版本,记为$v_{ij} , j=0,1,2,\ldots,n$。其
中,$v_{in}$表示条目$e_i$当前时间点的最后一个版本,而$v_{i0}$表示条目
$e_i$在没有任何内容时的初始状态版本,这时条目$e_i$实际上为空。 


\begin{figure}[htp]
  \centering
 % GNUPLOT: LaTeX picture
\setlength{\unitlength}{0.240900pt}
\ifx\plotpoint\undefined\newsavebox{\plotpoint}\fi
\begin{picture}(1500,900)(0,0)
\sbox{\plotpoint}{\rule[-0.200pt]{0.400pt}{0.400pt}}%
\put(171.0,131.0){\rule[-0.200pt]{4.818pt}{0.400pt}}
\put(151,131){\makebox(0,0)[r]{0}}
\put(1429.0,131.0){\rule[-0.200pt]{4.818pt}{0.400pt}}
\put(171.0,277.0){\rule[-0.200pt]{4.818pt}{0.400pt}}
\put(151,277){\makebox(0,0)[r]{0.2}}
\put(1429.0,277.0){\rule[-0.200pt]{4.818pt}{0.400pt}}
\put(171.0,422.0){\rule[-0.200pt]{4.818pt}{0.400pt}}
\put(151,422){\makebox(0,0)[r]{0.4}}
\put(1429.0,422.0){\rule[-0.200pt]{4.818pt}{0.400pt}}
\put(171.0,568.0){\rule[-0.200pt]{4.818pt}{0.400pt}}
\put(151,568){\makebox(0,0)[r]{0.6}}
\put(1429.0,568.0){\rule[-0.200pt]{4.818pt}{0.400pt}}
\put(171.0,713.0){\rule[-0.200pt]{4.818pt}{0.400pt}}
\put(151,713){\makebox(0,0)[r]{0.8}}
\put(1429.0,713.0){\rule[-0.200pt]{4.818pt}{0.400pt}}
\put(171.0,859.0){\rule[-0.200pt]{4.818pt}{0.400pt}}
\put(151,859){\makebox(0,0)[r]{1}}
\put(1429.0,859.0){\rule[-0.200pt]{4.818pt}{0.400pt}}
\put(171.0,131.0){\rule[-0.200pt]{0.400pt}{4.818pt}}
\put(171,90){\makebox(0,0){0}}
\put(171.0,839.0){\rule[-0.200pt]{0.400pt}{4.818pt}}
\put(278.0,131.0){\rule[-0.200pt]{0.400pt}{4.818pt}}
\put(278,90){\makebox(0,0){1}}
\put(278.0,839.0){\rule[-0.200pt]{0.400pt}{4.818pt}}
\put(384.0,131.0){\rule[-0.200pt]{0.400pt}{4.818pt}}
\put(384,90){\makebox(0,0){2}}
\put(384.0,839.0){\rule[-0.200pt]{0.400pt}{4.818pt}}
\put(491.0,131.0){\rule[-0.200pt]{0.400pt}{4.818pt}}
\put(491,90){\makebox(0,0){3}}
\put(491.0,839.0){\rule[-0.200pt]{0.400pt}{4.818pt}}
\put(597.0,131.0){\rule[-0.200pt]{0.400pt}{4.818pt}}
\put(597,90){\makebox(0,0){4}}
\put(597.0,839.0){\rule[-0.200pt]{0.400pt}{4.818pt}}
\put(704.0,131.0){\rule[-0.200pt]{0.400pt}{4.818pt}}
\put(704,90){\makebox(0,0){5}}
\put(704.0,839.0){\rule[-0.200pt]{0.400pt}{4.818pt}}
\put(810.0,131.0){\rule[-0.200pt]{0.400pt}{4.818pt}}
\put(810,90){\makebox(0,0){6}}
\put(810.0,839.0){\rule[-0.200pt]{0.400pt}{4.818pt}}
\put(917.0,131.0){\rule[-0.200pt]{0.400pt}{4.818pt}}
\put(917,90){\makebox(0,0){7}}
\put(917.0,839.0){\rule[-0.200pt]{0.400pt}{4.818pt}}
\put(1023.0,131.0){\rule[-0.200pt]{0.400pt}{4.818pt}}
\put(1023,90){\makebox(0,0){8}}
\put(1023.0,839.0){\rule[-0.200pt]{0.400pt}{4.818pt}}
\put(1130.0,131.0){\rule[-0.200pt]{0.400pt}{4.818pt}}
\put(1130,90){\makebox(0,0){9}}
\put(1130.0,839.0){\rule[-0.200pt]{0.400pt}{4.818pt}}
\put(1236.0,131.0){\rule[-0.200pt]{0.400pt}{4.818pt}}
\put(1236,90){\makebox(0,0){10}}
\put(1236.0,839.0){\rule[-0.200pt]{0.400pt}{4.818pt}}
\put(1343.0,131.0){\rule[-0.200pt]{0.400pt}{4.818pt}}
\put(1343,90){\makebox(0,0){11}}
\put(1343.0,839.0){\rule[-0.200pt]{0.400pt}{4.818pt}}
\put(1449.0,131.0){\rule[-0.200pt]{0.400pt}{4.818pt}}
\put(1449,90){\makebox(0,0){12}}
\put(1449.0,839.0){\rule[-0.200pt]{0.400pt}{4.818pt}}
\put(171.0,131.0){\rule[-0.200pt]{0.400pt}{175.375pt}}
\put(171.0,131.0){\rule[-0.200pt]{307.870pt}{0.400pt}}
\put(1449.0,131.0){\rule[-0.200pt]{0.400pt}{175.375pt}}
\put(171.0,859.0){\rule[-0.200pt]{307.870pt}{0.400pt}}
\put(50,495){\makebox(0,0){\rotatebox{90}{相似度}}}
\put(810,29){\makebox(0,0){变更版本}}
\put(171,131){\usebox{\plotpoint}}
\multiput(171.58,131.00)(0.499,0.682){211}{\rule{0.120pt}{0.646pt}}
\multiput(170.17,131.00)(107.000,144.660){2}{\rule{0.400pt}{0.323pt}}
\multiput(278.00,277.58)(0.737,0.499){141}{\rule{0.689pt}{0.120pt}}
\multiput(278.00,276.17)(104.570,72.000){2}{\rule{0.344pt}{0.400pt}}
\multiput(384.58,349.00)(0.499,0.682){211}{\rule{0.120pt}{0.646pt}}
\multiput(383.17,349.00)(107.000,144.660){2}{\rule{0.400pt}{0.323pt}}
\multiput(491.00,493.92)(0.727,-0.499){143}{\rule{0.681pt}{0.120pt}}
\multiput(491.00,494.17)(104.587,-73.000){2}{\rule{0.340pt}{0.400pt}}
\multiput(597.00,422.58)(0.733,0.499){143}{\rule{0.686pt}{0.120pt}}
\multiput(597.00,421.17)(105.576,73.000){2}{\rule{0.343pt}{0.400pt}}
\multiput(704.00,493.92)(0.727,-0.499){143}{\rule{0.681pt}{0.120pt}}
\multiput(704.00,494.17)(104.587,-73.000){2}{\rule{0.340pt}{0.400pt}}
\multiput(810.00,422.58)(0.733,0.499){143}{\rule{0.686pt}{0.120pt}}
\multiput(810.00,421.17)(105.576,73.000){2}{\rule{0.343pt}{0.400pt}}
\multiput(917.00,493.92)(0.727,-0.499){143}{\rule{0.681pt}{0.120pt}}
\multiput(917.00,494.17)(104.587,-73.000){2}{\rule{0.340pt}{0.400pt}}
\multiput(1023.00,422.58)(0.733,0.499){143}{\rule{0.686pt}{0.120pt}}
\multiput(1023.00,421.17)(105.576,73.000){2}{\rule{0.343pt}{0.400pt}}
\multiput(1130.58,495.00)(0.499,1.030){209}{\rule{0.120pt}{0.923pt}}
\multiput(1129.17,495.00)(106.000,216.085){2}{\rule{0.400pt}{0.461pt}}
\multiput(1236.00,711.92)(0.744,-0.499){141}{\rule{0.694pt}{0.120pt}}
\multiput(1236.00,712.17)(105.559,-72.000){2}{\rule{0.347pt}{0.400pt}}
\multiput(1343.58,641.00)(0.499,1.030){209}{\rule{0.120pt}{0.923pt}}
\multiput(1342.17,641.00)(106.000,216.085){2}{\rule{0.400pt}{0.461pt}}
\put(171,131){\makebox(0,0){$+$}}
\put(278,277){\makebox(0,0){$+$}}
\put(384,349){\makebox(0,0){$+$}}
\put(491,495){\makebox(0,0){$+$}}
\put(597,422){\makebox(0,0){$+$}}
\put(704,495){\makebox(0,0){$+$}}
\put(810,422){\makebox(0,0){$+$}}
\put(917,495){\makebox(0,0){$+$}}
\put(1023,422){\makebox(0,0){$+$}}
\put(1130,495){\makebox(0,0){$+$}}
\put(1236,713){\makebox(0,0){$+$}}
\put(1343,641){\makebox(0,0){$+$}}
\put(1449,859){\makebox(0,0){$+$}}
\put(171.0,131.0){\rule[-0.200pt]{0.400pt}{175.375pt}}
\put(171.0,131.0){\rule[-0.200pt]{307.870pt}{0.400pt}}
\put(1449.0,131.0){\rule[-0.200pt]{0.400pt}{175.375pt}}
\put(171.0,859.0){\rule[-0.200pt]{307.870pt}{0.400pt}}
\end{picture}

  \caption{不同版本的用户贡献}
  \label{fig:contribution}
\end{figure}


%\LTXtable{13cm}{admin.tbl}

%%% Local Variables: 
%%% mode: latex
%%% TeX-master: "master"
%%% End: 
