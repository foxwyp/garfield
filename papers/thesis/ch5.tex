
\chapter{维基百科用户参与知识协同的动机因素模型}
\label{cha:motivation}



越来越多的学者已经注意到维基百科的日益流行及其取得的巨大成功。维基百科
社区以一种前所未有的运作方式吸引了大量的志愿者自愿为其撰写内容,并为此
付出了巨大的经历,投入了大量的时间。不论是维基百科的研究者,还是社区的
运营者都自然地关注一个问题,到底是什么因素促使如此众多的人员参与到这项
庞大的计划中,并创造出传统方式难于匹敌的数量的内容?

当前已经有一些学者针对维基百科用户参与社区的动机因素进行了研究,并取得
了一定的成果。但是,这些成果主要是关于为何维基用户参与知识共享的动机的
研究。知识共享是以共享为目的,以知识的传递和流动为表象,达到知识的发送
方和接收方共同拥有同一知识的结果。在维基百科社区中,知识共享主要发生于
条目的作者和读者之间。条目的作者用户和条目主题相关的知识,并将这些知识
按照一定的格式撰写出来。当读者对某条目的知识有需求的时候,通过阅读条目
的内容从而吸收和掌握知识,进而完成知识的共享。相对作者于读者之间的知识
共享关系来说,条目作者之间的关系主要是协同关系。每个作者都只拥有关于某
个条目的一部分知识,必须依靠协同的方式来弥补知识的缺口,完成条目的撰写。
协同过程中存在一定的知识共享,但是知识共享不是协同的主要目的,它只是协
同过程中所连带产生的结果。

个人将自身的知识同其他人共享,一般来讲除了纯粹的利他以外,往往是出于以下几个目的:1)为了获得
物质奖励;2)为了获得其他非物质的收益,如名誉,其他人的认同和赞扬
等;3)期望与其他人互惠互利,在将来一旦有知识需求的时候能从他人那里获
得帮助。显然,这几种因素均不是维基百科用户参与知识协同的动机。首先,维
基百科并不能给用户带来现实的或是潜在的物质收益;其次,根据维基基金会所
作的调查,在所有可能的用户参与维基百科的动机因素中,赢得在社区中的声誉
处于最后一位,并且大部分用户不认为这是他们参与到维基百科中的原因;最
后,在维基百科社区中确实存在一定程度的互惠互利现象,但是,对于相当一部
分用户来说,尤其是领导者和领域专家,互惠互利并不能解释他们参与知识协同
的动机。考虑到这两类用户的知识投入和所获得的他人的知识投入极度的不平
衡,可以认为这两类用户并不指望从其他用户那里获得同自身的投入同等程度的
回报。基于以上原因,目前关于对个人参与虚拟实践社区的动机因素的研究并不
能完全用来解释个体间知识协同的动机。因此,有必要对各种动机因素进行分析
和梳理,构建一个适合于知识协同的动机因素模型。

在之前的章节中已经定义了维基百科社区中的知识协同,即对所有条目内容所做
的编辑工作,包括新增内容,删除内容,修改内容等。这些协同行为背后的驱动
力量就是动机。个体的行为往往同时受到多种动机的影响,同时在不同阶段起作
用的动机也不同。更为复杂的是,动机往往易受到外界因素的影响,呈现为增强
或者减弱的趋势。这就需要动机模型不但要反映出各种动机对行为的影响,同时
要考虑时间和外部因素对动机强度的影响,反应各个动机因素间的相互作用。

现有的研究已经提出了很多影响个体行为的动机因素。这些动机因素大致可以分
为两种类型:个体因素和人际因素。个体因素强调个体固有的感受和需要,即使
个体处于一个“独立”的环境下,个体因素仍然可以起作用,促使其从事某种行
为。人际因素则存在于个体之间的互动或称中,是其他人加于个体自身的感受和
需要。在知识协同过程中,个体因素和人际因素同时起作用,共同决定了个体的
行为。

\section{个体因素}
\label{sec:individual-factor}

\subsection{利他主义}
\label{sec:altruism}

利他主义是一种纯粹的、只希望施惠于他人而不谋求从中取得任何收益或者改进自身
状态的动机因素。利他主义认为关注他人的利益胜于关心自身的利益。Stacey认为维基百科中绝大部分活动都可以归结于利他主义。在维基百
科中,知识协同的最终目标是完成一个条目的创作,个体自身并不能从协同中得
到对其劳动的补偿。因此,利他主义是个人参与知识协同的一个重要动机。

\subsection{感知到的意义}
\label{sec:sense-of-meaning}

感知到的意义(sense of meaningfulness)是个体的一种感知。个体意识到自
己所从事的工作或是追寻的目标是有意义、有价值的,即使投入大量时间和精力
也是值得的。这种感知形成了个体的内
部推动力量,促使个体努力达成预定目标。维基百科的目标是“包含人类所有领域
的知识”,并且“始终保持自由”,可以任意地“被复制,修改和再发布”。绝
大多数维基人都认同这个目标,并且认为他们的工作会“改变
世界”,使更多的人能“自由地访问信息“。

\subsection{自我决定}
\label{sec:self-determination}

自我决定(Self determination)也是一种个体的感知状态。个体意识到自己有
权利决定是否要从事某项工作或者完成某种任务,并且这种选择不受任何外部力
量的影响。Pintrich等认为自我决定理论是如今的动机理论中最全面的、广受实
践检验过的理论。自我决定理论的应用范围非常广泛,近年来在对
在线社区的研究方面更是取得很多的成果。自我决定理论认为人有三种基本需
求:自主性、胜任感和归属感。自主性是指个体感受到能够掌控周围
的环境、事件的进程以及资源的获取的分配;胜任感是指个体认为自己有足够
的能力能够胜任某种工作或者任务;归属感是指个体能够感受到来自他人的关
心。当着三种基本需求得到满足时,个体的心理满足程度则得到加强。

自我决定理论将个体的动机分为三种类型:无动机、外部动机和内部动机。这三
种动机不是孤立地存在的,而是一个连续统一体。无动机是指个体缺乏参与活动
的意图,完全没有自我决定的因素;随着动机的逐渐增强(大部分是由于外部力
量的介入),个体的行为转为受外部动机的影响。这时一定程度的外部激励会弥
补个体的被强迫感觉,增
强个体的动机;当动机进一步增强,个体完全受自身意愿和兴趣所驱使参与某项
活动是,个体的主要驱动因素转为内部动机;个体不在受到任何外部力量的驱使
和强迫,这时外部激励反而会降低个体的内部动机。

从维基百科社区为用户提供了极大的自由,任何人都可以不受限制地参与
编辑,不论是否在社区注册过;社区中也不存在等级制度限制用户编辑的自由,
用户随时可以退出。这种开放性使得用户的自主性得到了极大的满足。这种开放
性还表现为维基百科欢迎任何类型的贡献而不计较贡献的大小,大到完成整个条
目的编写,小到对错别字的订正。这就是每个参与的用户都能找到适合自身的任
务,并提升其胜任感,并最终激发用户的内部动机,自愿地为社区做出贡献。

\subsection{自我效能}
\label{sec:self-efficacy}

自我效能(self-efficacy)是指一个人在特定情景中从事某种行为并取得预期结
果的能力。班杜拉将自我效能定义为:个体对有效控制自己的生活行为等诸方面
的能力的知觉或信念\cite{}。自我效能在很大程度上指个体自己对自我有关能
力的感觉,既对自己是否能够成功地进行某一成就行为的主观判断。一般来说,
成功的经验会增强个体的自我效能,反复的失败会降低个体的自我效能。

Raban等认为,一个人的才学是其所拥有的最宝贵的个人财富,能够与他人分享
这种这份财富会增强个人的自我意识,尤其是在对方表达谢意和尊敬的时候增强
的程度更为明显\cite{raban2007investigating}。Chan等研究了虚拟社区中的
知识共享行为,发现那些共享者在从他人那里获得了正向的反馈之后自我效能得
到了很大的提升,这种自我增强反过来又进一步促使他们分享更多的知识。
自我效能高的个体比那些自我效能低的个体更愿意参与到相关活动中去,因为他
们相信自己有能力取得预期的效果。相反的,自我效能低的个体常常缺乏参与的
动力,或者不愿意面对可能出现的困难,从而降低了参与的动机。

个体的自我效能主要受到四种因素的影响:个人成就、替代性经验、言语说服和
情绪激发。个体自身的成功经验是影响自我效能的最重要的因素。一旦个体成功
地完成了某项任务,将大大地提升个体对自身能力的信心,激发自身的自我效
能;而失败会使个体对自身能力产生怀疑,进而降低其参与动机。他人的经验对
于个体的自我效能也有很大的影响。通过观察他人的行动判断任务的特点和难
点,从中吸取经验并学习必要的技巧,个体能够显著地改善和提升自身的绩效。
他人对与个体的鼓励和帮助能增强个体的动机。当个体受到他人的激励,树立自身
能够成功完成某项任务的信念是时,个体的自我效能就会相应得到提升。帮助的
形式可能有言语上的激励、直接的指导以及对个体行为绩效的反馈等形式。个体
的情绪和心理状态对于个体对特定任务的判断起到很大的影响。高昂的情绪和积
极的心理状态往往会提升个体对自身能力的判断,激发个体潜能,改善其对参与
的预期结果的判断。负面的情绪,如焦虑等则会降低个体对自身能力的判断、夸
大任务的难度、降低预期收益,最终导致自我效能的较低。

维基百科的用户主要行为就是知识协同。社区为用户提供了多样的手段来克服对
社区的生疏感和对协同的“恐惧感”:既有详尽的使用说明和新手指南,同时又
可以得到资深用户的指导。新用户既可以通关观察自身用户的行为学习社区内的
协同方式,又可以在宽容的气氛中开始尝试协同而不必担心会收到批评和指责。
一旦个体的内容能够在未来的编辑中得以留存,就意味着个人的贡献受到了其他协同者的认可。个
体在协同中既获得了经验,同时也获得了心理上的满足和愉悦,从而提升其自我
效能。而更高的自我效能又提升了用户的参与水平。自我效能感与成就行为是相互促进的,它是用户参与维基百科重要的内部动机因素。

\subsection{自我肯定}
\label{sec:self-esteem}

自我肯定(Self-esteem)又称为自我价值、自尊等,是个体对于自我形象的主
观感觉,这种感觉可以是过分的或不合理的。自我肯定常常同自我效能相混淆。
自我效能是对特定能力的一种判断,而非自我价值的一般性感受。这种特定能力
总是与特定目标相联系的。有些人具备很高的“自我效能”──努力驱动自我,
但是自我肯定的意识却不强。另一方面,有些人可能对自身的价值非常肯定,但
是当遇到某一特定问题时,却缺乏足够的自我效能。

高自我肯定的人对于自身价值的评价很高,但同时又能清楚地认识到自身的缺点
和不足。高自我肯定的人乐于去学习、改进和提高自己,这类人通常表现为性格
外向、很受别人欢迎、工作认真尽责、善于控制情绪并且喜欢各种不同的体验。
而低自我肯定的人
表现为对自己的不满和自我否定,不喜欢自己当前的样子,这类人通常表现为性
格沉闷、不擅同人交往、情绪波动大。低自我肯定的人试图努力改变当前自身的
形象,为此需要不断地投入到某种活动中来证明自己;高自我肯定的
人通常倾向于维持这种良好的自我感觉,因此也乐于参与到活动中获得自身形象进一步的
提升。这就成
了个体参与某项任务的基本动机。

维基百科为用户强化其自我肯定的意识提供了良好的媒介。它的独特之处就在于
个体可以通过适当的策略尽量提升自身肯定,同时又不必承担不被别人所认同的
风险。
这是因为其他人的冷淡和忽视可以被贡献者自身直接理解为认同。不论条目的内容是否充
实,也不论文字是否有错误,只要没人提出或者修正,对于作者本身来说就是好的。同
时,各个条目之间又是平等的,即使是很小的条目,作者也会获得等量的参与感
和满足感。这使
得在维基百科中获得自我肯定极其容易。Timme等指出在维基百科中提升自我肯
定的最佳策略就是撰写那些受众较小的条目,获得最大收益的同时还尽量降低了
风险。

Tice (1993)认为低自我效能的人因为没有什
么负担,故愿意采用那些风险较大但是见效较快的行动来快速改进自身的形象。
而高自我肯定的人为了维持自身的形象,因此对于参与活动往往持稳健和保守的
态度,力求凡是做到尽善尽美。因此尽管高自我肯定的人和低自我肯定的人都有
很强的参与维基百科的动机,但是他们在社区中的表现是不一致的。高自我肯定
的人会重视条目的质量和内容的编排,努力编写完善的条目,而低自我肯定的人
则是“重在参与”,贡献的内容往往质量不高,主题生僻。

\subsection{成就动机}
\label{sec:achievement}
成就动机(achivement motivation),是个体追求自认为重要的有价值的工作,并
使之达到完美状态的动机。在这种动机的驱动下,个体愿意从事这项工作,并尽
力取得成功。Clark等将成就动机定义为对优秀标准的竞争或个体设定、实现个
体目标的愿望。
阿特金森认为个体的动机水平依赖于3大因素:成功诱因值,即对实现目标的价
值判断;成功的可能性大小以及成就需要,即主体追求成功的动机强度。这3个
因素发生综合影响,决定了个体的主观倾向和投入程度。

麦克利兰的成就动机理论进一步揭示了成就动机如何影响人的行为。个体记忆中存在着与成就相联系的愉快经验,
当情境能引起这些愉快经验时,就能激发人的成就动机欲望。个人的成就动机包
括三个方面:成就需要、权利需要和亲和需要。这三种需要在人们需要结构中有
主次之分,作为人们的主需求在满足了以后往往会要示更多更大的满足,拥有
成就者更追求成就,拥有权力者更追求权力、拥有亲情者更追求亲情。同时,成就需要的高低对人的成长和发展起到特别重要的作用,

成就需要是个体设定目标并努力达成目标的需要,个体从中获得优越感和满足感。
个人体自己认为重要或有价值的工作,不但愿意去做,而且会投入大量的时间和
精力,力
求达到完美地步。在这种动机驱使下,个体不是去追求由于成就本身所带来的报
酬,而是谋求把事情做得更好、更有效果。参与维基百科的用户正是在完成协同
任务的过程中,满足个人对于成就的渴望。

权力需要是指个体试图影响或控制他人且不受他人控制的需要。对于权利需求较
高的人来说,对他人施加影响要比工作本身更值得追求。这类人通常偏好于有竞
争性和地位取向的场合。尽管维基用户本质上不存在等级划分,但是一部分用户仍然通过自己的
表现成为了事实上的领导者。领导者的意见和建议对于社区的其他成员具有很大
的影响,他们的价值取向也往往成为某个用户群体的价值取向。特别是当协同实
践出现争议和矛盾时,社区领导者往往会利用自身的地位加以解决,这反过来又
进一步强化了其领导者的优势。

亲和需求是个体希望同其他人建立友好亲密的人际关系的需要。亲和需求促使个
体追求友善,重视人际关系。个体从他人的交往中获得愉悦感。亲和需求是虚拟
实践社区存在和发展的重要保证。在知识协同过程中,只有协同的各方互相尊
重,互相体谅,才能保证协同的顺利开展,取得最大的协同效应。
The more
people involved, the slower and stupider their union is

the greater overhead imposed by the costs for coordination and
bureaucracy;
%%% Local Variables: 
%%% mode: latex
%%% TeX-master: t
%%% End: 
