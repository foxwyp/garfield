\documentclass[adobefonts,cs4size,a4paper,openany]{ctexbook}

%\usepackage{caption}

\CTEXsetup[nameformat={\zihao{3}\bfseries},titleformat={\zihao{3}\bfseries},beforeskip={0pt},afterskip={10pt}]{chapter}

\setmainfont{Times New Roman}

\usepackage{titlesec}
%\titlespacing*{\chapter}{0pt}{-4\baselineskip}{0.5\baselineskip}
\titlespacing{\section}{0pt}{0.5\baselineskip}{0.5\baselineskip}
\titlespacing{\subsection}{0pt}{0.5\baselineskip}{0.5\baselineskip}
\usepackage{geometry}
\geometry{paper=a4paper,vmargin={25mm},hmargin={30mm,20mm},footskip={10mm},headheight={6mm},headsep={4mm}}

\usepackage{fancyhdr}
\pagestyle{fancy}

\fancyhf{}
\fancyhead[co]{北京航空航天大学博士学位论文}
\fancyhead[ce]{\leftmark}
\cfoot{\thepage}
\renewcommand{\chaptermark}[1]{\markboth{第 \chinese{chapter} 章 \quad
    #1}{}}

%\captiondelim{~~}
%\captionsetup{labelsep=none}





\renewcommand{\baselinestretch}{1.62}


\makeatletter
\newcommand\arraybslash{\let\\\@arraycr}
\makeatother


\begin{document}


\chapter*{摘 \quad 要}

面对经济的全球化和竞争的日益加剧,人们越来越认识到,如何有效地利用并管理好知
识,已经成为组织所面临的最重要的任务之一。随着知识管理的理论逐渐成熟和
信息技术的进步,通过知识管理提升组织的创新能力已经成为当前研究的热点问
题。

实践社区的兴起为组织知识管理提供了新的管理手段。实践社区的成员拥有共同
关注的主题,一起协作解决问题,在实践的过程中实现知识的共享与协同。动机因素研究是揭示实践社区知识协同活动机理的重要内容。在实践社区这样一
个松散的组织中,组织成员为什么要同其他人共享和协同,有哪些因素会影响成
员的动机,研究这些问题对于改善协同的效率,提高个体的创新能力具有重要作
用。

本研究从动机因素角度着手,分析影响协同者参与协同活动的动机因素,从人的
行为方面揭示知识协同的本质。本研究对于完善和丰富知识管理理论,指导知识
管理实践具有重要的理论和实际意义。

本文选取了中文维基百科作为虚拟实践社区的代表,对其内部的协同
行为和内部动机因素开展研究。主要研究内容包括:

\begin{enumerate}
\item 虚拟实践社区中的协同行为研究。通过深入分析虚拟实践社区中的用户行
  为,明确知识协同行为的定义和特点,并确定协同行为的度量方法,将协同行
  为进行量化。进一步,根据量化的行为分析用户的协同参与水平和贡献度。
\item 虚拟实践社区中的用户分类研究。虚拟实践社区中存在着不同类型的用
  户,这些用户间的协同行为和模式各不相同。利用量化的用户贡献度和用户参
  与水平两个维度,将不同类型的用户区分开。同时,利用社会网络分析工具具
  体分析每一类用户的行为特点,提出社区中用户的协同形式。
\item 知识协同的动机因素模型研究。协同活动的特点决定了协同的动机不仅仅
包括个人动机,也包括人际动机。两种动机的共同作用影响了人的实际行为。这
一部分研究将这两类动机融入到动机模型中,建立适合分析协同活动的动机模型。
既考虑个人的主观因素,又考虑协同活动对于动机的影响,综合分析个人动机与
行为、人际动机与行为、个人动机与人际动机之间的关系。利用系统动力学理
论,建立个体参与知识协同的动机模型。
\item 模型仿真与结果分析。使用实际数据验证模型的有效性,并利用模型分析
  动机因素是如何影响用户行为的。在此基础上,提出相应的管理建议,促进社
  区提升管理水平,达到良性发展的目的。

\end{enumerate}

\textbf{关键词:}虚拟实践社区,动机因素,知识协同,知识管理
\end{document}

%%% Local Variables: 
%%% mode: latex
%%% TeX-master: t
%%% End: 

%%% Local Variables: 
%%% mode: latex
%%% TeX-master: t
%%% End: 
