\documentclass[adobefonts,cs4size,a4paper,openany]{ctexbook}

%\usepackage{caption}

\CTEXsetup[nameformat={\zihao{3}\bfseries},titleformat={\zihao{3}\bfseries},beforeskip={0pt},afterskip={10pt}]{chapter}

\setmainfont{Times New Roman}

\usepackage{titlesec}
%\titlespacing*{\chapter}{0pt}{-4\baselineskip}{0.5\baselineskip}
\titlespacing{\section}{0pt}{0.5\baselineskip}{0.5\baselineskip}
\titlespacing{\subsection}{0pt}{0.5\baselineskip}{0.5\baselineskip}
\usepackage{geometry}
\geometry{paper=a4paper,vmargin={25mm},hmargin={30mm,20mm},footskip={10mm},headheight={6mm},headsep={4mm}}

\usepackage{fancyhdr}
\pagestyle{fancy}

\fancyhf{}
\fancyhead[co]{北京航空航天大学博士学位论文}
\fancyhead[ce]{\leftmark}
\cfoot{\thepage}
\renewcommand{\chaptermark}[1]{\markboth{第 \chinese{chapter} 章 \quad
    #1}{}}

%\captiondelim{~~}
%\captionsetup{labelsep=none}





\renewcommand{\baselinestretch}{1.62}


\makeatletter
\newcommand\arraybslash{\let\\\@arraycr}
\makeatother


\begin{document}


\chapter*{摘 \quad 要}

面对经济的全球化和竞争的日益加剧,人们越来越认识到,如何有效地利用并管理好知
识,已经成为组织所面临的最重要的任务之一。随着知识管理的理论逐渐成熟和
信息技术的进步,通过知识管理提升组织的创新能力已经成为当前研究的热点问
题。

实践社区的兴起为组织知识管理提供了新的管理手段。实践社区的成员拥有共同
关注的主题,一起协作解决问题,在实践的过程中实现知识的共享与协同。动机因素研究是揭示实践社区知识协同活动机理的重要内容。在实践社区这样一
个松散的组织中,组织成员为什么要同其他人共享和协同,有哪些因素会影响成
员的动机,研究这些问题对于改善协同的效率,提高个体的创新能力具有重要作
用。

本研究从动机因素角度着手,分析影响协同者参与协同活动的动机因素,从人的
行为方面揭示知识协同的本质。本研究对于完善和丰富知识管理理论,指导知识
管理实践具有重要的理论和实际意义。

本文选取了中文维基百科作为虚拟实践社区的代表,对其内部的协同
行为和内部动机因素开展研究。主要研究内容包括:

\begin{enumerate}
\item 虚拟实践社区中的协同行为研究。通过深入分析虚拟实践社区中的用户行
  为,明确知识协同行为的定义和特点,并确定协同行为的度量方法,将协同行
  为进行量化。进一步,根据量化的行为分析用户的协同参与水平和贡献度。
\item 虚拟实践社区中的用户分类研究。虚拟实践社区中存在着不同类型的用
  户,这些用户间的协同行为和模式各不相同。利用量化的用户贡献度和用户参
  与水平两个维度,将不同类型的用户区分开。同时,利用社会网络分析工具具
  体分析每一类用户的行为特点,提出社区中用户的协同形式。
\item 知识协同的动机因素模型研究。协同活动的特点决定了协同的动机不仅仅
包括个人动机,也包括人际动机。两种动机的共同作用影响了人的实际行为。这
一部分研究将这两类动机融入到动机模型中,建立适合分析协同活动的动机模型。
既考虑个人的主观因素,又考虑协同活动对于动机的影响,综合分析个人动机与
行为、人际动机与行为、个人动机与人际动机之间的关系。利用系统动力学理
论,建立个体参与知识协同的动机模型。
\item 模型仿真与结果分析。使用实际数据验证模型的有效性,并利用模型分析
  动机因素是如何影响用户行为的。在此基础上,提出相应的管理建议,促进社
  区提升管理水平,达到良性发展的目的。

\end{enumerate}

\textbf{关键词:}虚拟实践社区,动机因素,知识协同,知识管理

\chapter*{Abstract}
Efficient manage organization knowledge id becoming the most important task for an organization. With the development of KM theory and information technology, how to advance the innovation capacity of an organization attracting more eyes of scholars.

The appearance of virtual community of practice providing new tool for KM. Members in VCoP have common sense and collaborate  to resolve issues. In practices they achieve the knowledge sharing and collaboration. Motivation factors is one of the key to demonstrate the mechanism of VCoP that why people collaborate with others, which factors will affect members motivation. Understanding  this questions will benefit improving the efficiency of collaboration and appraise individual’s innovation ability.

This research concentrate on motivation factors and analyze which factors affect people’s intention to participate in VCoP. The result thus is important to complement the KM theory and guide KM practices in organization.

This thesis select Chinese Wikipedia as research object. The main research include:
\begin{enumerate}
\item  Research of collaboration behavior in VCoP. The concept of
collaboration in VCoP is defined and the measurement of collaboration
behavior is   proposed. Using the measurement the member’s
contribution can be calculated.
\item Research on category of users in VCoP. There are different kinds of
users in VCoP and each group of users behave differently. According to
collaboration measurement and the widespread of participation the
users can then be divided into categories. Applying SNA the characters
of each group then be studied.
\item Motivation factors of knowledge collaboration. There are tow kinds
of motivation factor: individual factors and inter-personal
factors. Both of them affect human behavior. Using system dynamics
model the casual relationship between motivation factors and behavior
is modeled.
\item  Simulation and result analysis. The reliability of model is
examined. Also the sensitivity of model is checked.  Some advisees to
improve the community are then proposed.

\end{enumerate}

\textbf{Keywords:} knowledge management, knowledge collaboration, virtual communities of practice, motivation factors   

\end{document}


%%% Local Variables: 
%%% mode: latex
%%% TeX-master: t
%%% End: 

%%% Local Variables: 
%%% mode: latex
%%% TeX-master: t
%%% End: 
