\documentclass[slidestop,compress,mathserif,red]{beamer}
\usetheme{Antibes}
\usecolortheme{lily}
\setlength{\parskip}{10pt}
\title{}
\author{Yunpeng Wu}
\institute{BUAA}
\begin{document}
\begin{frame}
\titlepage
\end{frame}

\section{Knowledge Asset}
\begin{frame}
  The world economy mainly has been based on production. The factors creating values
in the production economy are land, labor, capital and physical assets. However, in the
last two decades, knowledge economy the Intellectual Capital (IC) has become more
important to add values when it is compared to physical assets.
\end{frame}

\subsection{What is Knowledge asset}
\begin{frame}[allowframebreaks]
IC is a mix of human capital, customer capital and
structural capital. Human capital generates innovation. Customer capital is the firm��s
value of its franchise, its ongoing relationships with the people or organizations to which
it sells, like market share, and customer retention and defection rates. Finally, structural
capital is the knowledge that belongs to the organization as a whole. It encompasses
technologies, inventions, data, strategy, culture, structures and systems, procedures,
trade secrets, copyright, patent, etc. (Brooking, 1997).
\ 

The key characteristics of IC as strategic assets:
\begin{itemize}{}{}
\item rarity
the asset can be owned or used  by only one company, otherwise it can not bring excess profit because of reducing margin cost of production.
\item inimitability
easy to imitate will stimulate company standing inferior position to imitate and develope the knowledge asset which leads to incentive competence and offset the income.
\item non-substitutability
other asset can not bring profit as efficient as this asset
\item unobservability
more precisely, an knowledge asset can play important role only in specific context, if the environment changes, than it loss it's value.
\end{itemize}
Only structural capital, which is
owned by the firm and is assumed not to be reproduced and shared, is regarded as the
best approximation of IC.

in a competitive
environment, a firm��s structural capital is the key to increasing its value.
\end{frame}
\begin{frame}[allowframebreaks]
Tangible goods:
\begin{quote}
 Tangible goods are goods whose quality can be
judged before purchase and compared against
advertisements, product specifications, and
uncertain nature with a common judgment
standard. Transactions in the market are completed
by the change of ownership and payment of
consideration is by the exchange principle. 
\end{quote}
\\
Information goods
\begin{quote}
  Information goods are goods whose quality cannot
be judged before purchase. Neither advertisements
nor product specifications provide sufficient
judgment standards. Experience in using such
goods is necessary to assess their quality. However,
since consumption of goods is completed after the
consumer sees the goods and understands them,
transactions do not function effectively. We
characterize these information goods by the
impossibility of prior evaluation.
\end{quote}
\\
Knowledge goods
\begin{quote}
  Knowledge goods are kinds of information goods.
We define, for example, the method of solving a
problem and specific know -how as being
knowledge goods. In addition to the characteristics
of information goods, knowledge goods have the
following features.
\end{quote}
\end{frame}

\section{Knowledge service}

\begin{frame}
  \frametitle{knowledge service}
  A knowledge service is the integration of systematic
knowledge and the mechanism of using the knowledge to
perform a task.
\end{frame}

\section{Knowledge grid}

\section{Mechanism of Knowledge Asset Pricing}


\end{document}