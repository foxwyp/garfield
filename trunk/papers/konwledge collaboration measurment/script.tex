\documentclass{elsarticle}

\begin{document}
\journal{Expert Systems with Applications}

\makeatletter
\newcommand\arraybslash{\let\\\@arraycr}
\makeatother



\begin{frontmatter}

  \title{ An Analyzing Method for Influencing Factors of Enterprise Knowledge Creation Capability}

  \author[buaa]{Yunpeng Wu\corref{cor1}}
  \ead{yunpeng.wu@sem.buaa.edu.cn}

  \author[buaa]{Lu Liu}
  \ead{liulu@buaa.edu.cn}
  

  \cortext[cor1]{Corresponding author}

  \address[buaa]{School of Economics and Management, Beihang University, \par
    Beijing 100191, P.R. China}
  \begin{abstract}
Effective knowledge creation is very important to enterprises to gain and maintain competitive predominance and enterprise knowledge creation capability (KCC) plays a key role in achieving effective knowledge innovation. In this article, we induce influencing factors of knowledge creation capability from five different aspects through analyzing process of knowledge creation. After that, by applying the fuzzy set theory and fuzzy clustering approach to analyze enterprise KCC, a method based on fuzzy clustering is proposed for analyzing influencing factors of enterprise KCC. Using this method, the enterprise can find out the key attribute set of influencing factors of enterprise KCC then the key influencing factors which decide its enterprise KCC can be deduced. Enterprises can adjust their strategies of knowledge management (KM) and knowledge creation according to the analytical result to improve its knowledge creation capability and gain core competence. Finally, we provide an application case to illustrate the application of the presented method.
  \end{abstract}

  \begin{keyword}
  \end{keyword}
\end{frontmatter}

\section{ Introduction}
Measuring  the impact of knowledge management are questions that never
go away. When Organizations put so many efforts and resources into
knowledge management practices, they need to know what outcomes
knowledge management will bring about.  Nevertheless,  the measures that are
available to evaluate KM, either tools or models, are still
unsatisfactory.  One of the main reasons is that the effect of KM are
often indirect and hard to isolated from other impact
factors. Although the KM executives can run some semi-controlled
experiments between similar teams or groups in the organization to
determine the differences of  output with and without KM practices, or
take an longitudinal study  to observe the change at varies input
level of KM, there are still shortcomings. The measurements may be not
 agile enough to reflect current state of KM and need lots of efforts
 and resources to take into practice.   

 On the other hand, the other
 perspective of KM measurement is often neglected by scholars, that
 is, to know where to invest more or less.  KM needs organization  

\end{document}

%%% Local Variables: 
%%% mode: latex
%%% TeX-master: t
%%% End: 
