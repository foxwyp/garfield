\documentclass[adobefonts]{ctexart}
\newcommand{\tabincell}[2]{\begin{tabular}{@{}#1@{}}#2\end{tabular}}
\usepackage{supertabular}
\usepackage{amsmath}
\begin{document}
\title{航空企业知识协同能力模糊综合评价}

\section{intro}
\label{sec:intro}

知识协同就是指多个协同参与者通过对其拥有知识的共享与交流,共同实践某一知识型任务的过程,是一种知识资源的整合优化方式[13]。组织面对日益激烈的竞争压力,通过有效地运用知识协同手段,能够满足企业对知识创新的迫切需求,提升员工运用知识解决问题的水平,进而帮助组织获得可持续发展的竞争优势。知识协同是知识管理理论的延伸和拓展。同知识管理的目的不同,知识协同的核心不在于知识共享,其目标明确而集中。知识协同结果可以是一个新知识,也可能是实践一个项目或解决一个问题,它更强调的是针对某种主题的实践知识的整合与优化,而不是知识的转移和扩散。
随着知识协同行为在组织中越来越普遍,对协同行为的研究也成为管理领域的研
究热点。协同行为本身是一个多方参与、反复交互的过程,因此,协同者的参与
程度和协同能力对于协同的结果有重要影响。

\textbf{关于知识协同的研究受到许多学者的关
注, 充分显示了其理论研究价值和实际意义, 知识协同是
一个新的研究方向。但有关知识协同研究的文献数量仍
处于初期增长阶段, 还不多见。已有文献的研究主题相对比较分散, 研究
视角也有局限, 应用领域偏重于信息技术, 尚没有形成一
套比较完善的理论与方法体系。因此, 由于知识协同理论
与方法不够完善, 从而无法指导企业的知识协同运作与
实践。}特别是,对于企业协同能力的评价,目前还处于
空白。即缺乏成熟的评价指标体系,也未出现相应的案例研究。本文通过实证分析的方法,研究了
 组织内知识协同的影响因素。通过对其影响因素的分析研究,能够使组织更系统
地理解组织内部的协同行为,找到影响协同的关键因素,从而更有力地推动协同
活动在组织内部的进行。

目前,有
关评价方法的研究比较成熟,形成了多种评价方法,
如加权平均法、 功效系数法、 模糊综合评判方法、 层次
分析法等。评价方法中指标性能和权重都用三角模
糊数进行判断,不适用于多级指标的评价问题;文献
[7 ]也研究了知识管理绩效评价问题,但其使用层次
分析法(Analytic Hierarchy Process ,AHP)进行指标
权重确定方法不如模糊AHP一致性强。本文在分析
航空制造企业中知识资源构成的基础上,建立了航空
制造企业知识管理水平的多层次综合评价指标体系,
并运用基于三角模糊数的模糊层次分析方法来评价
航空制造企业的知识管理水平。该方法尤其适用于
类似于航空制造企业知识管理水平评价这类被评价
对象的性能无法具体量化、 指标权重不容易判断的多
层次指标模糊评价问题。与其他评价方法相比,这种
方法更易于专家对被评价对象的性能和指标权重进
行判断,使用更方便。

\section{知识协同能力评价指标}
航空制造企业的产品具有高附加值和高风险的
特点 ,属于典型的知识密集型企业[ 8 ] 。飞机的设计研制过程往往需要多个专
业、多个领域的专家联合攻关。随着问题的复杂性和多样性日益显著,对于知识
协同的需求也就越发迫切。通过知识协同的方式进行知
识创新, 能够弥补知识缺口, 有效的解决知识情景嵌入和
路径依赖的问题, 消除“ 知识孤岛” , 并可获得多主体、 多
目标、 多任务间的“ 1+1>2” 的知识协同效应。我国航空企业知识协同
面临的主要问题是缺乏全面、适合企业特点的综合评价指标,对影响知识协同的
内外因素缺乏清晰的认识,从而难于从组织的角度对协同过程提供必要的支持以
及对员工进行相应的激励。从目前国内外的研究成果来看,对于知识协同能力的
评价研究几乎没有,主要集中在对知识管理水平和绩效的评价研究。因此,本文
在综合国内外研究的基础上,提出一个知识协同能力的综合评价指标体系。该
指标体系包括五个维度,25个评价指标。为了进一步验证该指标体系的有效性,本文在
航空企业内部进行了问卷调查。问卷根据提出的指标体系,假设每一个指标对组
织知识协同有影响,进而设计量表,测度其显著程度。问卷在某航空企业设计所
内进行发放,共发放问卷410份,回收398份,回收率为97.1\%,其中有效问卷为
395份。问卷调查对象设计该设计所13个科室,包括设计员、科室主任以及部分
总师。在对问卷进行处理后,使用结构方程模型进行数据分析,研究假设是否成
立。分析结果显示,原有的25个假设中有17个假设成立,即有17个指标对组织知识协
同能力有显著影响。剔除掉不支持的指标后,重新构建组织知识协同能力的评价
指标体系,如下表所示:

  \begin{supertabular}[center]{clp{6.5cm}}
\hline
    一级指标&二级指标&说明\\\hline
    知识水平&专家水平&组织中知识专家的总体水平\\
          &专家数量&组织中知识专家的数量    \\
          &其他知识资源&专家知识以外的其他组织知识资源\\
          &外部资源获取能力&知识专家对于组织外部知识资源的获取能力 \\
    协作能力&组织能力&能够充分调动协同参与者的热情和积极性,推动协同过
    程 \\
          &共享能力&共享能力意味着协同者能否及时有效地共享组织知识\\
          &社会网络&组织中的成员联系的紧密程度\\
    信息技术&协同工具&通过协同平台协同者能够实时方便地进行在线协同\\
          &知识服务&协同者能够方便地找到所需要的知识\\
          &专家推荐&根据问题描述和相关要求,帮助员工找到合适的协同对象\\
          &知识孤岛&运用信息技术消除组织内的知识孤岛\\
    组织氛围&协同文化&组织内部积极分享、积极合作的态度对员工参与知识协
    同有积极影响\\
          &领导推动&组织领导对于知识协同活动的推动和指导\\
          &竞争压力&宽松的竞争环境能够促进协同活动的展开\\
    管理控制&协同目标&组织提供清晰的协同目标是协同活动开展的先决条件\\
          &过程控制&组织对协同过程能够有效管理,协助提升协同效率\\
          &激励机制&组织有恰当的激励机制,对协同的参与者有相应的评价手段\\\hline
  \end{supertabular}

  \section{模糊层次分析法}

  在多准则、多目标的决策问题上,层次分析法(AHP)是目前应用最广泛、最
  受认可的方法。许多领域的决策问题都可以使用层次分析法辅助进行决策。近
  年来,针对层次分析法的改进也逐渐增多。其中,模糊层次分析法是对传统层
  次分析法的一类重要改进。传统的层次分析法中对于那些无法准确度量的评价准则主
  要是使用自然语
  言描述评价结果。由于自然语言本身就带有一定的模糊性,因此将自然语言评
  价结果很难完全转换为精确的数值。模糊层次分析法通过引入模糊理论,在一
  定程度上比不了上述不足。许多文献使用模糊理论分别提出了不同的改进方法。
  其中,chang提出的程度分析方法具有同传统AHP方法相似,应用简便的特点,
  在本文中,根据这种方法,采用了三角模糊数来表示
  主观评价的数值。三角模糊数的形式为$\tilde{A}=(a,b,c)$,其隶属度函数
  为:
  \[
  \mu_A~(x)=
  \left\{
      \begin{array}{ll}
        \frac{x-a}{b-a}&a\leq x \leq b\\
        \frac{c-x}{c-b}&b \leq x \leq c\\
        0 &  \text{其他}
      \end{array}
    \right.
\]

在进行评价时,需要对评价准则进行两两比较,比较的结果采用三角模糊数标度来衡量相对重
要程度,本文采用zhang提出的标度,如表1所示:
\begin{table}
  \centering
 \caption{标度}
  \begin{tabular}{lll}
    语言标度&精确标度&三角模糊数标度\\\hline
    两个元素同等重要&1&$(1/2,1,3/2)$\\
    一个元素比另一个元素重要一些&2&$(1,3/2,2)$\\
    一个元素比另一个元素重要得多&3&$(3/2,2,5/2)$\\
    一个元素比另一元素重要太多&4&$(2,5/2,3)$\\
    一个元素比另一个元素绝对重要&5&$(5/2,3,7/2)$\\
  \end{tabular}
 
\end{table}
程度分析需要定义模糊综合程度值,该值可表示为:
\[
S_i=\sum^m_{j=1}M^j_{i}\odot
\left[\sum_{i=1}^n\sum_{j=1}^nM_{i}^j\right]^{-1}
\]
其中$\odot$表示模糊乘法,$M_{i}^j$是以三角模糊函数表示的评价准则$i$和
评价准则$j$程度分析值。$S_i$则表示第$i$个评价准则的模糊综合程度值。进
一步可以定义$M_1\geq M_2$的可能度:
% \[V(M_1 \geq M_2)=
%  \mathop{sup}_{x \geq y}[min(\mu_{M_1}(x),\mu_{M_2}(y)]
% \]
\[V(M_1 \geq M_2)= hgt(M_1\cap M_2)=
\left\{
    \begin{array}{lll}
      1,& if \ b_1 \geq b_2\\
      0,& if \ a_2 \geq c_1\\
      \frac{a_2-c_1}{(b_1-c_1)-(b_2-a_2)},&\mbox{其他}
    \end{array}
  \right.
\]

对于凸模糊数$M$,$M$大于$k$个凸模糊数$M_i(i=1,2,\ldots,k)$的可能度可表
示为:
\[
V(M \geq M_1,M_2,\ldots,M_k)=min \ V(M \geq M_i),\ i=1,2,\ldots,k 
\]
令$d^{'} (A_i)=min \ V(S_i \geq S_k) \ k=1,2,\ldots,n; k \not= i$,评
价准则的权重向量可表示为:
\[
W^{'}=(d^{'}(A_1),d^{'}(A_2),\ldots,d^{'}(A_n))^T
\]
$A_i$是$n$个评价准则。
对$W$进行归一化后,可得到权重向量:
\[
W=(d(A_1),d(A_2),\ldots,d(A_n))^T
\]
$W$是一个确定的数值。

\section{协同评价实例}

某航空企业同时开展研发不同类型和型号的飞机,从而形成了不同的研发团队。
为了更好地实施组织知识管理策略,有针对性地提升组织知识管理水平,推动研
发团队的知识协同,需要对各个团队的知识协同水平进行评价。评价的指标体系
采用前文提出的指标。三位专家$E_1,E_2,E_3$分别对3个团队$T_1,T_2,T_3$进
行评价。专家首先对指标体系中的第一层指标进行了成对比较,确定了比较矩阵:
\[
\left[
  \begin{array}{lllll}
(1,1,1)&(3/2,2,5/2)&(5/2,3,7/2)&(2,5/2,3)&(1,3/2,2)\\
       &(1/2,2/3,1)&(3/2,2,5/2)&(1,3/2,2)&(2/5,1/2,2/3)\\
       &()&()&()&()\\
\\
(2/5,1/2,2/3)&(1,1,1)&(1,3/2,2)&(3/2,2,5/2)&(2/5,1/2,2/3)\\
(1,3/2,2)&&(2,5/2,3)&(3/2,2,5/2)&(1/2,2/3,1)\\
()&&()&()&()\\
\\
(2/7,1/3,2/5)&(1/2,2/3,1)&(1,1,1)&(1/2,2/3,1)&(1/3,2/5,1/2)\\
(2/5,1/2,2/3)&(1/3,2/5,1/2)&&(1/2,2/3,1)&(2/7,1/3,2/5)\\
()&()&&()&()\\
\\
(1/3,2/5,1/2)&(2/5,1/2,2/3)&(1,3/2,2)&(1,1,1)&(1/2,2/3,1)\\
(1/2,2/3,1)&(2/5,1/2,2/3)&(1,3/2,2)&&(1/3,2/5,1/2)\\
()&()&()&&()\\
\\
(1/2,2/3,1)&(3/2,2,5/2)&(2,5/2,3)&(1,3/2,2)&(1,1,1)\\
(3/2,2,5/2)&(1,3/2,2)&(5/2,3,7/2)&(2,5/2,3)&\\
()&()&()&()&\\

  \end{array}
\right]
\]
\end{document}

%%% Local Variables: 
%%% mode: latex
%%% TeX-master: t
%%% End: 
