\input{ctex4xetex.cfg}
\documentclass[12pt,a4paper]{ctexart}
\usepackage{fontspec}
\usepackage[CJKaddspaces]{xeCJK}
\setmainfont{Times New Roman}
\setCJKmainfont{SimSun}
\setCJKfamilyfont{hei}{Microsoft YaHei}
\setCJKfamilyfont{song}{SimSun}
\setCJKfamilyfont{youyuan}{YouYuan}

\CTEXsetup[format={\CJKfamily{hei}\zihao{3}\centering}]{chapter}
\CTEXsetup[format+={\CJKfamily{hei}\zihao{4}\flushleft}]{section}
\CTEXsetup[format+={\CJKfamily{hei}\zihao{-4}\flushleft}]{subsection}


\usepackage{geometry}
\usepackage[dvips]{xcolor}
\geometry{vmargin={25mm},hmargin={30mm,20mm}}

\begin{document}
% fisrt page
  \title{企业知识管理研究\\\large{ 网格环境下的知识沟通}}
  \author {吴云鹏}
   \maketitle
% directory
 \newpage
  \tableofcontents
% text  



\newpage
  \section{论文选题依据}

  \subsection{研究背景}
人类的实践活动始终伴随着各类知识的创造,运用,丰富与发展,人们认为,占
有更多的知识能够提高生产的效率,创造更大的价值。如今,越来越多的企业认
识到,要想获得持续成功就必须管理好组织中的各种知识。正如温特
\cite{Williamson1994}指出的:“企业就是知道如何做事的组织。”不论企业是
生产产品还是提供服务或者二者兼有,都依赖于企业知道“如何做事”,也就是企
业中的知识。在当今知识经济的时代,知识已经成为最重要的经济资源,人们对
知识的追求与竞争正变得愈发激烈起来。对于企业来说,对知识的追求就是对超
额利润的追求。随着社会分工和社会化大生产变得更加精细与完善,传统的生产
资源已经不能再为企业贡献超额利润了,知识几乎成为超额利润的唯一来源。企业来要想获得长
久稳定的竞争优势,就必须在知识的占有和管理上取得领先。

尽管知识总是被不自觉地同信息和数据混淆,但是人们普遍认为知识要比信
息、数据具有更广泛更深刻的内涵和更丰富的内容。企业中的知识构成非常复
杂,它是一种混合
体,包括成体系的经验、企业价值、与环境相关的信息和专家的洞察力等等。企
业中的知识不仅存在于文档和数据库中,也存在于企业每日的例行事务,企业过
程和标准规范中。随着劳动分工的日益精细,导致了知识分工的出现。一方面,
人们在越来越窄的范围内知道的越来越多;另一方面,也随之出现了越来越多的
“知识岛屿”,使得如何进行知识沟通成为管理者面临的重大问题。企业中知识复
杂性为知识管理知识带来了巨大的困难,现有的管理理论和技术手段已经不能满足企
业对管理知识的巨大需求,因此,研究在互联网条件下如何进行有效的知识沟通
就成为知识管理研究者紧迫的任务。

知识管理的主要研究领域目前主要还集中在知识管理的“前端”,即知识获取的方
法和技术上。知识抽取、知识表示、知识存贮等方面是知识管理研究最多的方
面,尽管这些领域很重要,研究也取得了丰硕的成果,但是研究表明,对于企业
来说,进行有效的知识管理仍然是困难的。一份对欧洲和北美的研究报告显示,只有13\%的经理认为
知识在企业各组织单元中有效地进行传递,大部分的知识管理都是失败的。那
么,一个问题就很自然地摆在人们面前:是不是进行了知识抽取、表示、存储等
等工作,知识共享(这应该是知识管理的重要目标之一)就自动发生了?答案是
显然的,知识管理应该是一系列管理和控制过程,当前的研究实际上忽略了这个
链条的后端--知识如何在员工中传递和共享--的研究。本文就是致力于回
答这个问题而开展研究工作的。

\section{文献综述}

知识管理的最终目的是促进知识共享,或者说--知识的沟通。\textcolor{red}{广义地讲,知识共
享就是个体之间交换知识并协同产生新知识的过程。}知识必须被他人共享才能
最大化其价值。不同类型的知识其共享传递的方式也不相同:隐性知识由于
难于用语言清晰表达,因此共享的方式以示范模式为主。在
示范模式中,知识的转移高度依赖于接受和发送双方的互动。\cite{zhoubo2006}显性知识可以有形
式化系统的语言描述,因此它的传播方式是用品模式。知识通过声、文字、图像
等手段从发送者传递到接收者。用品模式的主要特点是知识可以脱离知识发送者
而“独立”存在。\cite{zhoubo2006}

\subsection{知识的价格}

知识沟通实际上就是知识的传播,它可以具有多种形式。例如以公共品渠道发布
知识,非经济的方式转知识以及知识交易。在这三种方式中,只有知识交易使
能够成为长期稳定的知识传播方式。知识交易也就是知识的买卖,交易双方以知
识本身为商品,进行等价交换。由于知识本身所具有的特殊性,知识交
易自然也同普通的商品交易有所区别。周波在其博士论文中对于知识交易做出了
如下定义:“知识交易就是知识拥有者通过知识转移进行“排他性”控制获得经济
利益的过程,交易结果是实现知识转移。”\cite{zhoubo2006}
尽管前两种方式在现实中并不少见,但是正如达
文波特指出的:“那些认为知识无需要经济激励就可以自
然扩散的想法是乌托邦式的;在没有回报预期的条件下人们不太可能贡献出具有
价值的知识。”\cite{davenport1998wko}每个理性人都会根据自己的判断,对于
自身所掌握的知识估计其价值。

知识的价格一直是知识交易的难题,显然,建立在等价交换基础上的交易必需对
交易物--知识本身--进行定价。按照马克思的观点:商品的交换价值由生产该商
品的平均的社会必要劳动时间所决定,价格围绕交换价值波动,并最终靠向交换
价值。由于知识本身的特殊性,使得价值理论不太适用于解释知识的价格问题。
首先:从长期来看,知识会逐渐成为公共品,也就是不再会有人宣称对该知识拥
有所有权并收取费用,任何人都可以免费获得该知识;其次,从中期来看,知识
的价格可能会产生剧烈的波动,在某一个时期可能无人问津,在另一个时期又突
然收到人们关注,使价格大大提升;再次,知识的创新已经越来越呈现高投入、
高风险的特征,但是知识本身的价格却不一定与投入相符合。从以上的分析可以
看出,对于知识价格的确定,使用效用理论进行分析更为恰当。效用理论认为:
商品的价值由效用决定,效用直接反映在交易的价格上。或者,商品的价值是该
商品所能够支配的劳动量而不是生产商品本身的劳动量。效用本身是动态的、易
变的,由此价格也会随着客观环境的变化而变动,这也同我们在日常生活中观察
的知识价格的变动相一致。但是,为知识的定价同普通商品的定价由本质的不同
。人们对于某一普通商品的效用会有大致相同或相似的预期和评价,因此普通
商品的价格制定也相对容易,价格的波动方向也大致相同。对于知识来说,知识
生产本身需要大量投入,知识的效用在不同时期会发生巨大的变化,而知识的复
制成本几乎为零,这会使投机行为大量出现。如果知识本身不能以均衡价格交易
的话,供求双方至少会有一方的福利受到损害。

知识价格的确定揭示了两个事实,一是知识的需求和该知识的稀缺程度并没有什么直接联系,一件知识可能只有一个
  人掌握,但是却没有人需要它,比如说茴香豆的茴的几种写法。二是知识创造
  和获取的难度也同价格没有直接联系,如果人们觉得某件知识没有什么用武之
  地,即使创造或获得该知识耗费了大量的资源,也不会有交易的愿望。由于知
  识持有者对知识价格的估计具有极大的主观性,因此他心目中的价格往往和外
  界所接受的只是价格有很大的差距,如果他为某件知识投入了大量的资源而该
  知识却不能以他理想的价格交易出去,那么他会认为交易只是所获得价值不能
  补偿自己的投入而暂停交易,以等待该知识的价
  值能逐渐被外界认识,并以一个理想的价格交易出去。这种知识沟通双方对知
  识价值的认知差距,导致知识往往被少数人掌握而不愿以共享。

\subsection{交易方式}
知识的交易方式可以分成两类,一种是直接的知识交易。交易的标的就是知
识本身,比如一项技术的转让,一本书籍的销售,都是买卖双方直接根据知识的
价值进行交易的。还有一类交易方式是知识服务。交易的一方利用自己的专业知
识生产出某项知识产品,交易对象是服务本身。

尽管第一类交易是最直接,也是人们最先开始实用的交易方式,但是这种方式越
来越暴露出一些问题:
\begin{enumerate}
\item 知识本身难于定价。
\item 供需矛盾
\item 接受者的意愿与能力
\end{enumerate}



尽管几乎所有的人类活动都离不开既有知识对人们行为的指导,但是在经济领域中,知识却不是一开始就作为一种必要的生产资源纳入到研究中来的。在古典经济学中,土地、劳动、资本三位一体是生产的
基本要素,知识根本就没有在研究者的概念中出现。直到二十世纪早期,知识的作用才躲在各具特色的企业家理论背后羞涩地显示出来。其中熊彼得的“创新者”
理论出现最具有代表性。他在《经济发展理论》( 1912 年)一书中提出了“ 创
新理论”,首次将 “ 创新 ” 视为经济增长的内生变。以后又在其他著作里加以
应用和发展。 1942 年,熊彼得《资本主义、社会主义和民主主义》一书出版,
标志着他的 “ 创新理论 ” 体系最后完成。熊彼得认为,”创新”就是建立一种新
的生产要素组合的生产函数,新组合包括: 一 引入一种新产品或提供一种产品的新质量; 2. 采用一种新的生产方式; 3. 开辟一个新的市场; 4. 获得一种原料或半成品的新的供给来源; 5. 实行一种新的企业组织形式,例如建立一种垄断地位或打破垄断地位。熊彼得特别强调组织创新、管理创新、制度创新、社会创新和技术创新之间的联系。创新就意味着一种新的生产函数的建立,通过创新,企业降低了生产成本,提高了生产的数量和质量,从而打破了市场均衡,进而获得超额利润。在“创新理论”中,知识以企业家的智慧、能力、经验、阅历、洞察力、决断力等品质表现出来,成为影响企业获得利润的重要因素。同时代的奈特、柯兹纳提出了类似的理论,认为企业家的个人能力是企业发展的动力和源泉。尽管他们的成果并没有明确地指出知识的根本性作用,但是企业家作为知识的载体所体现出的作用得到了充分的肯定。

但是,这些理论的局限性也是显而易见的。将知识羞涩地隐藏在企业家身后而将发展的动力完全归于企业家个人,随着社会和经济的发展已经不断面临新的困境。比如,如何解释同一企业家在不同企业工作但是企业经营状况却大相径庭的现象?对于某个小企业可能经营状况并不如意,为什么会有大企业愿意花大价钱并购?这些问题运用传统理论几乎无法解释,预示着传统理论的固有缺陷并提出了新的理论问题,除了企业家之外,还有哪些因素是企业获得利润的因素?企业家是否还是根本因素之一,如果不是,那么企业获得利润的根本因素是什么?

社会的进步和发展呼唤生产力的提升,而有效的社会分工是保证生产力提升的生产方式。随着分工的不断细化,各种分工的专业化不断增强,不同分工之间的关系在不断加强的同时也在不断产生“隔阂”。在各个分工中所产生的知识也体现出专门性和深入性的特点。知识爆炸使得人们即使是在本专业内也难以学习掌握所有的知识,更不要说出现“通才”。企业家的个人能力被越来越限制在一个狭窄的社会分工中。
知识资产已经成为企业发展最重要的驱动力。Leventhal和March将其描述为“由
组织内部的个人或团体持有的一组特殊的竞争力\cite{levinthal1993ml}.企业
不但自主进行研发以获得知识资产,也可以从外部获得知识资产。现今各类组织
都进行各种各样的知识管理活动,以促进知识资产的挖掘整合共享,最大限度地
实现知识资产的保值和增值。伴随各种各样的知识服务,知识资产的流动也变得
愈发频繁起来。有效的知识管理方式和恰当的制度是促进知识资产流动的重要因
素。随着知识网格技术的成熟和发展,出现了知识集市 。\cite{Andreas2007}

第一章	引言 (10页)
第二章	知识交易与知识服务 (20页)
第三章	知识服务交易机制  (20页)
第四章	网格环境下知识服务的定价策略 (30页)
第五章	实证研究 (10页)
第六章	结论  (3页)

\section{知识传递}
知识在组织和个人之间的相互流动,就是知识传递的过程。知识传递之所以会收
到研究者的重视,是因为知识传递的模式和质量直接影响知识沟通的效果。根据
知识的信息属性,以香农的信息论为基础,专家和学者对知识的传递进行了广泛
而深入的研究。



\bibliographystyle{unsrt}
\bibliography{../../bibtex/elsevier,../../bibtex/emerald,../../bibtex/chinese,../../bibtex/jstor,../../bibtex/citeseer,../../bibtex/acm,../../bibtex/wiley,../../bibtex/book,../../bibtex/thesis}

\end{document}


