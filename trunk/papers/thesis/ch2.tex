\chapter{维基百科用户分析}
\label{cha:wikipedian}

\section{维基百科简介}
维基百科(Wikipedia),网址:http://www.wikipedia.org/  ,是一个语言、
内容开放的网络百科全书计划。英文的“Wikipedia”是“wiki”(一种可供协
作的网络技术)和“encyclopedia”(百科全书)结合而成的混成词。(wikipedia)

维基百科由来自全世界的自愿者协同写作。自2001年1月15日英文维基百科成立
以来,维基百科不断的快速成长,已经成为最大的资料来源网站之一,而以热门
度来说,则为世界第六大的网站,在2008年吸引了超过684,000,000的访客,目
前在272种的独立语言版本中,共有6万名以上的使用者贡献了超过1000万篇条目。
截至今天,共有314,766篇条目以中文撰写;每天有数十万的访客作出数十万次
的编辑,并建立数千篇新条目以让维基百科的内容变得更完整。(请参见维基百
科统计)

维基百科一直坚持内容的开放性。这时维基百科取得成功的一个重要原因。维基
百科所有的内容,包括文字、图片等均在“知识共享 署名-相同方式共享 3.0协
议”之条款下提供。任何人都可以自由引用维基百科的全部或者部分内容,仅需
要注明其出处即可。维基百科不仅赋予内容使用的开放性,还奉行参与的开放性。
这种开放性同自由软件运动有很大的相似之处。Raymond将这种开放、自由的参
与和使用方式比喻为“集市”似的方式,区别于传统的那种严谨、刻板、集中的
“大教堂”方式。任何人只要愿意遵守维基百科社区的政策,就可以加入到协同
创作过程中来。

维基百科的成功证明了“群体智慧”的力量。Surowiecki指出利用群体的智慧开
展协同
工作对于当今的政治、经济、商业等各个方面具有重要的影响。日益频繁的人际
交互已经超出了时空的阻隔,正在改变人们的生活方式。维基百科的成功不仅仅
是信息技术创新的结果,更是人们相互协作,共同应对面临的各种挑战,利用集
体的力量取得成功的最好体现。

\section{维基百科社区的知识协同平台}

Wiki技术本质上是一种超文本系统。这种超文本系统支持面向社群的协作式写
作,同时也包括一组支持这种写作的辅助工具。用户可以在Web的基础上对 Wiki
文本进行浏览、创建、更改等,而且创建、更改、发布的代价远比HTML文本为
小;同时Wiki系统还支持面向社群的协作式写作,为协作式写作提供必要帮助;
最后,Wiki的写作者自然构成了一个社群,Wiki系统为这个社群提供简单的交流
工具。与其它超文本系统相比,Wiki有使用方便及开放的特点,所以Wiki系统可
以帮助我们在一个社群内共享某领域的知识。 (baidu)

Wiki技术为世界各地的人们在互联网上进行协同工作提供了一个技术平台。应用
Wiki技术开展协同的网站最早可追溯到1995年Ward Cunningham创立的c2.com。
随着WEB 2.0的兴起,协同创作这一新的内容产生方式逐渐受到人们的重视。工
业界不断推出新的技术和标准,而学术界也对协同创作的机理、方式、产生的影
响等进行了研究。维基百科在这样的背景下应运而生。

协同创作是在互联网环境下,创作者借助信息技术超越时间和空间的限制,共同
合作完成特定主体的内容创作。随着知识分工的日益细化,仅凭个人所掌握的知
识越来越难易适应不断提升的知识创新需求。为了弥补自身知识的不足,以协同
方式共同完成新知识的产生成了越来越多的人的必然选择。然而,新的知识创新
方式尽管能积极应对知识创新所带来的压力和挑战,也随之产生了新的问题。
Liccardi等人将这些问题归纳为五个方面\cite{liccardi2007caws}:
\begin{enumerate}
\item 沟通不足。
知识协同是促进知识创新的手段,而持续有效的沟通是知识协同的基础。协同平
台不但要保证协同成员间的沟通顺畅,更重要的是能够追溯思想、灵感的产生过
程,确保这些内容能够完整真实地记录下来。
\item 内容与讨论脱节。
知识创新是一个持续、反复的过程中,需要存于者进行大量的讨论、评价,对有
疑问、不明确的地方进行辨析、扬弃,最终达成一致。讨论是最终内容形成的完
整链条,串起了内容各个版本间的变化过程,对于理解最终内容非常重要。一旦
导论于内容间的关系没有得到紧密关联,那么协同过程中必然会出现大量的误解
和矛盾。
\item 对群组讨论缺乏足够的支持。
对内容的讨论常常是以群组讨论的方式开展的。当参与协同的成员众多时,会出
现大量参与者同时参与多个讨论主体的情况,而讨论本身会进一步衍生出其他的
讨论。这就要求协同平台能够支持有效的
群组讨论,维系特定的讨论线索,并有效解决讨论过程中出现的各类问题。
\item 缺乏旧有版本的回溯功能。
协同参与者需要不时审视以前的内容,并做出决定是否之前某一版本更好,进而
应该恢复到那一版本。如果缺乏回溯功能,这些需求将很难得到满足。
\item 解决冲突。
 当多个协同者对内容进行编辑时,必然会产生冲突。冲突的原因是由于协同者
 “同时”对某一段文字进行了修改,一般来说这类冲突很难通过自动化的手段
 解决。好的协同平台除了要及时给出冲突的提示
 外,还应该根据相应的规则和手段协助协同成员解决冲突。
\end{enumerate}

一个好的协同平台,应该能有效地处理好以上问题,为协同者提供强大的技术支
持和保障
Neuwirth等人归纳了一个优秀的协同平台所应该具有的特征:
\begin{enumerate}
\item 提供适合的方法和手段促进创作者在内容创作过程中开展有效的交互。
\item 将内容创作和对内容的讨论,意见等内容有机地结合起来。
\item 提供有效的工具支持协同创作和内容讨论这两种形式的交互。
\end{enumerate}
这几个特征实际上揭示了协同创新的本质。新的内容(或者知识)来源于个
体、团队、组织等不同层次群体之间的交互和沟通,在思维的碰撞中产生。协同
创作的本意在于激发参与者的沟通欲望,调动参与者的协作热情,共同完成某一
主题的知识或者内容的创新。在创新过程中,共享各自的知识已经不再是主要目
的,因此,支持协同成员间的社会化交互,提供高效、简洁、易用的交流平台就
成为wiki平台成功的关键因素。

维基百科从几个方面来保障协同活动的开展。首先,维基百科对每一个条目都设
置了讨论页。讨论页是特殊的维基百科页面,它包含了所有对主题文章的讨论。
任何的问题、疑虑、怀疑、参考文献、有关文章的论战或者评论都可以在相关的
讨论页提出来。在讨论页中,协同者可以分享自己的思想和观点,整理内容创作
的思路和逻辑,分析内容的取舍,澄清材料的真伪,最大限度地保障协同的质量。
讨论页可以包括多个讨论主体,维基百科平台会对最近的讨论话题和热门的讨论
话题醒目地标识出来,从而方便使用者进行讨论。

维基百科对所有内容的变更都保留了记录。事实上维基百科的内容无法作出任何
改变。你只能增加内容。每一次内容的变更则作为一个历史版本保留下来。读者
阅读的每一个条目均只是好像只是一份当前的草稿一样。这个做法使我们能比较
不同版本之间的差异,或者在需要时将文章回复到旧版本。读者甚至可以引用其
中一个特定的版本。你只要在左方控制列的「工具箱」中点下「永久连结」,就
可以连结到该文章版本的网址,其内容永远不会改变。这种做法的最大目的在于将每一
个协同参与者的活动及其蕴涵的信息能够呈现在每一个人面前(而不仅仅是其他
的协同参与者)。不但如此,条目的历史版本还和讨论有效地整合起来。每一个
历史版本都可以连接到针对此版本的讨论主题,从而将内容变更的原因和结果直
接联系起来。Dourish和Bellotti将其称之为共享反馈。

协同编辑带来的一个重要不利影响就是冲突几乎不可避免。冲突可能来自于两方
面的原因:作者所持观点和立场的不同,以及对内容的恶意破坏。维基百科的一
个重要原则是中立和不偏不倚。维基百科的创始人之一吉米 $\cdot$ 威尔士说,必须保
持中立观点(NPOV)这个原则在维基百科中是绝对的和不可争辩的的编辑原则,
维基百科的章程是,“中立观点意味着应努力让支持者和反对者都同意某种观点
或事实……”维基百科的管理员这样解释中立政策,“我们应该把争论中各方面
的声音都公平地表达出来,而不是在文章中指出或暗示任何一方的观点是正确
的”,“中立的立场,中性的描述”。然而,在现实中并不存在纯粹的中立与客
观。当两位或多位协同参与者意见相左时,如果没有良好的冲突解决机制,那么
不断地互相删改对方的编辑内容以维持己方观点,甚至进行人身攻击就必然会发
生。冲突的第二个来源来自于维基百科参与者的匿名性。由于参与到维基百科社
区几乎不需要任何身份的核实于验证过程,因此会有一小部分带有纯粹恶意的人
参与进来。这些人不断地篡改别人的劳动成果,胡乱添加违反维基百科创作原则
的内容。这种做法不但伤害了维基百科内容的真实性和完整性,也大大损害了其
他社区成员的利益和热情。

冲突的极端后果是导致“编辑战”的产生。“编辑战”是指明知会招致反对时仍然固执己见,采取挑衅性的编辑行为,并且反复使用回退功能。编辑战通常会使条目在短时间内内容
频繁变化,并会造成大量的版本覆盖。维基百科社区为此制定了大量的规则和政
策,防止编辑战的产生。条目可能会被被临时甚至永久性的保护或锁定,以进行
一段时间的强制冷却,等待适当时机再次开放编辑。有时也会采取保留较为合适
的版本并长期锁定的措施。对参与编辑战的人员,社区提供了讨论、投诉和仲裁
等机制来解决冲突。在技术上,维基社区采取“回退不过三”的措施,即“一位
编辑者对于一个维基百科的页面,在24小时内,不可以执行多于三次的回退”。
通过这些手段,维基百科社区在最大程度上避免冲突的发生。同时一旦发生冲
突,能够快速进行处理,将冲突的损失将为最低。

\section{维基百科社区的知识协同}
维基百科的协同活动主要是协同协作,是由一群人一起,而非单独一人完成的写
作工作计划。协同的目标是针对某一主题给出其“百科式”的解释,不仅包括该
主题自身的含义,还可能包括其历史背景和演变过程,其他人的评价,对其他方
面的影响等内容。协同的最终结果是一个个具体的条目。维基百科用户可以
参与到绝大部分条目的协同协作中。条目被分为不同的种类,维基百科成不同的
种类为命名空间。维基百科目前有20个命名空间,其中包括9个基本的名字空间
;此外还有两个虚拟名字空间。
表\ref{tab:namespace}给出了维基百科中基本命名空间的清单:

\begin{table}[!htp]
  \zihao{5}
\caption{维基百科的命名空间}
\label{tab:namespace}
  \begin{tabularx}{14cm}{|c|c|X|}
    \hline
    编号  & 命名空间  & 内容 \\
    \hline
    1 & 条目&条目命名空间”又称“主命名空间”,包含了维基百科上的所有条目页面,或“百科全书文章”。 \\\hline 
    2 &维基计划 & 这个命名空间提供了有关维基百科的内容信息,包括维基百科自身的信息、方针、指引、论述,以及维基人的讨论空间“互助客栈”、询问处等。\\\hline
    3 & 帮助 & 包含了所有维基百科及MediaWiki软件的使用指南信息。有些内容帮助读者更好地使用维基百科,而另一些内容则为编者准备,用来更好地编写维基百科。有些信息亦是从元维基和MediaWiki 网站上复制而来的。\\\hline
    4 &  用户  &包含了所有用户的个人页面,以及其个人创建的相关页面。 \\\hline
    5 & 分类 &包含了所有的分类页面,内容为该分类之下的页面和子分类列表,以及可选的分类提示信息。 \\\hline
    6 & 文件  &包含了图像和声音的文件描述页,以及指向文件本身的链接。 \\\hline
    7 & MediaWiki & 包含了所有的软件界面文字,例如在一些页面上自动生成的信息和链接。这个名字空间用于定制和翻译MediaWiki的软件界面。\\\hline
    8 &  模板 & 包含了所有的模板。模板是一类特殊的页面,用于嵌入或替换引用进其他的页面,以加入一些标准化的内容,或者信息栏、导航栏等。\\\hline
    9 & 专题 &包含了所有的主题页面。一个主题页面是关于某一方面内容的信息集合,一个相关条目的入口。 \\\hline
    \end{tabularx}
\end{table}

命名空间对维基百科中所有创建的内容用途角度进行了划分。尽管几乎所有的内
容都是社区成员的协同结果,但是这并不意味着这些内容都会纳入到本文的研究
范围。事实上,协同的主要成果是各个条目页面,而其他几个命名空间的内容均
是为了更好地编写条目而提供辅助功能的。因此,本文将维基百科社区内的知识
协同活动限制为共同编写某一条目内容,而忽略其他内容的协同编写活动。这样
的目的是:一方面保留了协同活动的主体,同时有减少了数据分析和处理的难
度,突出了研究的重点。
协同创作条目还可以分为两个部分,条目的编写和讨论。显然,条目的内容本身
是协同的直接结果,条目的编写是直接的协同活动,参与条目创作的用户是协同的直接参与者;而条目的讨论是
协同过程的间接结果,讨论本身不是协同的目的而是必要手段,那么参与讨论的
用户是否也应该作为协同活动的参与
者?维基
百科的讨论的目的在于挖掘内容和材料,辨析真伪,主要的方式是头脑风暴,汇
集各方的思路和意见;而条目的编写在于组织各类材料和内容,完成实际的内容
创作,强调“做”而不是“说”。同讨论过程的松散性不同,内容编写是非常正
式、严谨的。Viégas等认为,用户参与讨论本质上是一种合作行为而非协同行
为。讨论的内容可以分为以下几个部分:

\begin{table}
  \centering
  \caption{协同与合作} 
\zihao{5}
 \begin{tabularx}{13cm}{|X|X|X|X|}
    \hline
  &协调&合作&协同\\\hline
必要条件&有共同目标;多人参与;知道何时由谁做什么。&有共同目标;多人参
与;相互信任与尊重;承认合作是双赢(多赢)。&有共同目标;多人参
与;积极的投入;对协同群体有归属感;开放的沟通和交互;相互信任与尊重;
互补的知识与技能。\\\hline
主要目的&避免工作的重叠或者缺失&在合作过程中各自获得利益&通过集体的努
力完成个体无法
独立完成的工作,并取得合作成果。\\\hline
预期成果&令人满意的工作成果&取得工作成果的同时还节省了时间和投入&除了
合作产生的结果,还取得了创新性的成果以及完成工作的成就感。\\\hline
适用范围&应对简单、独立性高的任务,成员角色和进度安排非常明确。&应用于
在复杂环境下系统地解决问题。&在复杂环境下,需要彼此理解并认可对方,建
立一致的价值观,通力协作解决问题。\\\hline

  \end{tabularx}
  
  \label{tab:collaboration}
\end{table}


\begin{table}[!htp]
\caption{维基百科用户分类}
\label{tab:user}
  \zihao{5}
\begin{tabularx}{13cm}{|c|X|X|}
\hline
用户类型&定义&权限 \\\hline
匿名用户&未在维基百科网站注册账户的用户。&浏览所有页面;编辑所有未经保护
的页面;在任意命名空间下创建讨论页面。 \\\hline
新注册用户&已经在维基百科网站注册了账户,但是还未确认其电子邮件地址。&
创建新页面;给其他已经确认邮件地址的用户发送电子邮件;将某次页面编辑标
注为细微改动;删除页面无须确认;定制维基百科界面和账户信息。\\\hline
自动确认用户&已经在维基百科网站注册了账户,并且已经确认其电子邮件地址;用
户状态由系统自动确认:用户注册4天以上并且进行过10次编辑即成为确认用户。
&移动页面;对部分保护页面进行编辑;上传新文件或者上传已存在文件的新版
本。\\



\hline
\end{tabularx}
\end{table}


\begin{table}[!htp]
\caption{aaa} 
\label{tab:administrator}
\zihao{5} 
\begin{tabularx}{13cm}{|c|X|X|}
\hline
用户类型&定义&权限 \\\hline
系统管理员&管理整个社区各类内容的创作和编辑。管理员由个人提出申请,社区对
其资格进行审查,包括申请者的编辑历史和社区活动的参与程度。对于合格的申
请者将赋予其管理员权限&删除页面;保护页面;组织某个用户或者IP对于特定
页面的修改;修改受保护的页面;赋予或者收回对其他用户的回滚操作权限。 \\\hline
系统行政员&管理用户授权&可以将一名用户变为管理员或行政员(但是不能移去这个权限)。也可以变更其他用户的用户名。
\\\hline
系统监督员&特殊的管理人员。它可以在元维基上将所有计划及其所有语言版本的某位用户设置为这个语言版本的管理员或行政员或其他权限,或者取消这些权限。
&赋予或者收回任意用户任意权限,甚至包括管理员权限和监督员权限。\\\hline
回退员&特定权限用户&可以恢复其他用户对某一个页面的修改。\\\hline
IP封禁例外者&特定权限用户&不受IP封禁或者自动屏蔽IP的影响。\\\hline
账户创建员&特定权限用户&不受每个IP每日最多创建6个IP的限制。创建账户过
程中也可以跳过安全检查。\\\hline
上传者&特定权限用户&由系统监督员赋予新用户,使得新用户可以上传各类合法
文件\\\hline
机器人&自动化或者半自动化程序,用于各类内容创建与编辑的辅助性工作。&按
照预定的目标和规则对页面内容进行重新编辑。一旦发现机器人偏离了预定的目
标,可以自动屏蔽该机器人。\\\hline
开发人员&程序开发人员&可以访问特点的开发人员页面,允许使用与开发相关的
特定功能。\\
\hline
\end{tabularx}

\end{table}


  


%%% Local Variables: 
%%% mode: latex
%%% TeX-master: "master"
%%% End: 
