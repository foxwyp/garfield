\documentclass[slidestop,compress,mathserif,table]{beamer}
\usepackage{ctex}
\usepackage{xcolor}
\usepackage{fontspec}
\usepackage{algorithm}
\usepackage{algorithmic}
\usepackage{slashbox}
\setCJKmainfont{Adobe Song Std}


\usepackage{pgf}
\usetheme{Boadilla}
%\usetheme{Madrid}
%\usetheme{Antibes}
                                                                          
\begin{document}


\title{ 虚拟实践社区知识协同动机因素研究 \vspace{1.5cm}}
    \author{作者姓名 \ 吴云鹏 \\指导教师  \ 刘\ \ \ 鲁\ }
   % \institute[buaa]{北京航空航天大学}
    \begin{frame}
      \titlepage
    \end{frame}

    \begin{frame}
      \frametitle{章节结构}
      \begin{enumerate}
      \item 绪论
       \item 用户协同行为分析
         \item 用户分类研究
           \item 知识协同动机因素模型
\item 模型仿真与决策分析
\item 结论与展望
      \end{enumerate}
    \end{frame}

    \begin{frame}
       \begin{block}{知识协同定义}
   Denise指出:知识协同是一个共同创新的过程。群体利用其互补的知识和能力在交互
过程中建立关于某种事物共同的理解,这种理解任何人之前都不曾拥有过,个人
也不太可能独立产生这种理解。知识协同最终产生出关于某种过程、产品或者事件的
共同知识。
     \end{block}
\pause
\vfill
\begin{block}{维基百科中的知识协同}
在本文中,将
知识协同定义为:由用户参与的维基百科条目内容的协同创作活动。
  
\end{block}
    \end{frame}

    \begin{frame}{协同行为的度量}
      \begin{block}{协同贡献}
        编辑次数?\\
        新增内容数量?
      \end{block}

  \pause \vfill
  \begin{block}{理想的度量方式}
    \begin{enumerate}
     \item 维护公平性。度量结果必须要客观反应所有参与者的工作成绩,并且保证
  工作成效高的成员其度量结果也高。
\item 阻止个体的作弊倾向。
\item 符合评价者的价值取向。
    \end{enumerate}
     
\end{block}
    \end{frame}
    
\begin{frame}{基于文本相似度的协同贡献度量}
  \begin{block}{核心思想}
    个体的协同贡献可以经由比较个体编
辑的内容与最终版本的文本相似度得到。
  \end{block}
\vfill
\begin{block}{基本思路}
  每次用户
的编辑都会增添或者删改一部分内容,这些内容最后都可能在条目的最后版本中
得到保留。因此, 用户在每一次编辑过程所做的贡献可以视为究竟有多少内容
在最后版本中仍然存在。如果两个版本中有一部分文字相同,则两个版本的文字
在一定程度上是相似的,如果存在一个函数
$similarity(v_{i,j_1},v_{i,j_2})$能够将相似程度以数值的程度表示出来,
则相似度的差值可以作为用户在一次编辑中所做出的贡献。
\end{block}
    \end{frame}

    \begin{frame}{相似度的确定}
      \begin{exampleblock}{Ratcliff/Obershelp算法}
         设有两段文本$T_1$、$T_2$,其长度分别为$t_1$、$t_2$,两段文本可以完
全匹配的最长字文本串为$S_1$、$S_2$,其长度为$s$,则两段文本的相似程度
可以表示为:
\[
similarity(T_1,T_2)=\frac{2 \cdot  s}{t_1+t_2}
\]
        
      \end{exampleblock}

\pause
\begin{block}{算法的缺陷}

  \begin{enumerate}
  \item 对于新增文本的位置敏感。
\item 如果在编辑过程中编辑行为是
重组文本顺序,则Ratcliff/Obershelp算法同样显示出极大的不足。
  \end{enumerate}
  
\end{block}
    \end{frame}

    \begin{frame}{算法改进}
    

文本的相似度不应仅考虑最大完全匹配子串的长度,而是应
该考虑所有的完全匹配字串。两段文本中的每一个完全匹配的字串均可以计算得到一个相似度,这
些相似度的线性组合最终成为整段文本的相似度。为此,可以在首次计算相似度
之后,将两段文本中的最大完全匹配子串移除,同时使用一个虚拟字符代替该字串,即保
持原有文本其他字串的相对位置不变。重新计算两个新的文本段落的相似度,并乘以
适当的系数。反复使用该算法进行迭代,直到新生成的文本段落没有完全匹
配字串。则迭代过程中所得到的所有相似度的和即为两段原始文本的相似度。

    \end{frame}

 \begin{frame}{用户在一次编辑中的协同贡献}
   \[
c_{i,j}=s(v_{i,j},v_{i,n})-s(v_{i,j-1},v_{i,n}), \quad 1<j \leq n
\]
   \scalebox{0.85}{% GNUPLOT: LaTeX picture
\setlength{\unitlength}{0.240900pt}
\ifx\plotpoint\undefined\newsavebox{\plotpoint}\fi
\begin{picture}(1500,900)(0,0)
\sbox{\plotpoint}{\rule[-0.200pt]{0.400pt}{0.400pt}}%
\put(171.0,131.0){\rule[-0.200pt]{4.818pt}{0.400pt}}
\put(151,131){\makebox(0,0)[r]{0}}
\put(1429.0,131.0){\rule[-0.200pt]{4.818pt}{0.400pt}}
\put(171.0,277.0){\rule[-0.200pt]{4.818pt}{0.400pt}}
\put(151,277){\makebox(0,0)[r]{0.2}}
\put(1429.0,277.0){\rule[-0.200pt]{4.818pt}{0.400pt}}
\put(171.0,422.0){\rule[-0.200pt]{4.818pt}{0.400pt}}
\put(151,422){\makebox(0,0)[r]{0.4}}
\put(1429.0,422.0){\rule[-0.200pt]{4.818pt}{0.400pt}}
\put(171.0,568.0){\rule[-0.200pt]{4.818pt}{0.400pt}}
\put(151,568){\makebox(0,0)[r]{0.6}}
\put(1429.0,568.0){\rule[-0.200pt]{4.818pt}{0.400pt}}
\put(171.0,713.0){\rule[-0.200pt]{4.818pt}{0.400pt}}
\put(151,713){\makebox(0,0)[r]{0.8}}
\put(1429.0,713.0){\rule[-0.200pt]{4.818pt}{0.400pt}}
\put(171.0,859.0){\rule[-0.200pt]{4.818pt}{0.400pt}}
\put(151,859){\makebox(0,0)[r]{1}}
\put(1429.0,859.0){\rule[-0.200pt]{4.818pt}{0.400pt}}
\put(171.0,131.0){\rule[-0.200pt]{0.400pt}{4.818pt}}
\put(171,90){\makebox(0,0){0}}
\put(171.0,839.0){\rule[-0.200pt]{0.400pt}{4.818pt}}
\put(278.0,131.0){\rule[-0.200pt]{0.400pt}{4.818pt}}
\put(278,90){\makebox(0,0){1}}
\put(278.0,839.0){\rule[-0.200pt]{0.400pt}{4.818pt}}
\put(384.0,131.0){\rule[-0.200pt]{0.400pt}{4.818pt}}
\put(384,90){\makebox(0,0){2}}
\put(384.0,839.0){\rule[-0.200pt]{0.400pt}{4.818pt}}
\put(491.0,131.0){\rule[-0.200pt]{0.400pt}{4.818pt}}
\put(491,90){\makebox(0,0){3}}
\put(491.0,839.0){\rule[-0.200pt]{0.400pt}{4.818pt}}
\put(597.0,131.0){\rule[-0.200pt]{0.400pt}{4.818pt}}
\put(597,90){\makebox(0,0){4}}
\put(597.0,839.0){\rule[-0.200pt]{0.400pt}{4.818pt}}
\put(704.0,131.0){\rule[-0.200pt]{0.400pt}{4.818pt}}
\put(704,90){\makebox(0,0){5}}
\put(704.0,839.0){\rule[-0.200pt]{0.400pt}{4.818pt}}
\put(810.0,131.0){\rule[-0.200pt]{0.400pt}{4.818pt}}
\put(810,90){\makebox(0,0){6}}
\put(810.0,839.0){\rule[-0.200pt]{0.400pt}{4.818pt}}
\put(917.0,131.0){\rule[-0.200pt]{0.400pt}{4.818pt}}
\put(917,90){\makebox(0,0){7}}
\put(917.0,839.0){\rule[-0.200pt]{0.400pt}{4.818pt}}
\put(1023.0,131.0){\rule[-0.200pt]{0.400pt}{4.818pt}}
\put(1023,90){\makebox(0,0){8}}
\put(1023.0,839.0){\rule[-0.200pt]{0.400pt}{4.818pt}}
\put(1130.0,131.0){\rule[-0.200pt]{0.400pt}{4.818pt}}
\put(1130,90){\makebox(0,0){9}}
\put(1130.0,839.0){\rule[-0.200pt]{0.400pt}{4.818pt}}
\put(1236.0,131.0){\rule[-0.200pt]{0.400pt}{4.818pt}}
\put(1236,90){\makebox(0,0){10}}
\put(1236.0,839.0){\rule[-0.200pt]{0.400pt}{4.818pt}}
\put(1343.0,131.0){\rule[-0.200pt]{0.400pt}{4.818pt}}
\put(1343,90){\makebox(0,0){11}}
\put(1343.0,839.0){\rule[-0.200pt]{0.400pt}{4.818pt}}
\put(1449.0,131.0){\rule[-0.200pt]{0.400pt}{4.818pt}}
\put(1449,90){\makebox(0,0){12}}
\put(1449.0,839.0){\rule[-0.200pt]{0.400pt}{4.818pt}}
\put(171.0,131.0){\rule[-0.200pt]{0.400pt}{175.375pt}}
\put(171.0,131.0){\rule[-0.200pt]{307.870pt}{0.400pt}}
\put(1449.0,131.0){\rule[-0.200pt]{0.400pt}{175.375pt}}
\put(171.0,859.0){\rule[-0.200pt]{307.870pt}{0.400pt}}
\put(50,495){\makebox(0,0){\rotatebox{90}{相似度}}}
\put(810,29){\makebox(0,0){变更版本}}
\put(171,131){\usebox{\plotpoint}}
\multiput(171.58,131.00)(0.499,0.682){211}{\rule{0.120pt}{0.646pt}}
\multiput(170.17,131.00)(107.000,144.660){2}{\rule{0.400pt}{0.323pt}}
\multiput(278.00,277.58)(0.737,0.499){141}{\rule{0.689pt}{0.120pt}}
\multiput(278.00,276.17)(104.570,72.000){2}{\rule{0.344pt}{0.400pt}}
\multiput(384.58,349.00)(0.499,0.682){211}{\rule{0.120pt}{0.646pt}}
\multiput(383.17,349.00)(107.000,144.660){2}{\rule{0.400pt}{0.323pt}}
\multiput(491.00,493.92)(0.727,-0.499){143}{\rule{0.681pt}{0.120pt}}
\multiput(491.00,494.17)(104.587,-73.000){2}{\rule{0.340pt}{0.400pt}}
\multiput(597.00,422.58)(0.733,0.499){143}{\rule{0.686pt}{0.120pt}}
\multiput(597.00,421.17)(105.576,73.000){2}{\rule{0.343pt}{0.400pt}}
\multiput(704.00,493.92)(0.727,-0.499){143}{\rule{0.681pt}{0.120pt}}
\multiput(704.00,494.17)(104.587,-73.000){2}{\rule{0.340pt}{0.400pt}}
\multiput(810.00,422.58)(0.733,0.499){143}{\rule{0.686pt}{0.120pt}}
\multiput(810.00,421.17)(105.576,73.000){2}{\rule{0.343pt}{0.400pt}}
\multiput(917.00,493.92)(0.727,-0.499){143}{\rule{0.681pt}{0.120pt}}
\multiput(917.00,494.17)(104.587,-73.000){2}{\rule{0.340pt}{0.400pt}}
\multiput(1023.00,422.58)(0.733,0.499){143}{\rule{0.686pt}{0.120pt}}
\multiput(1023.00,421.17)(105.576,73.000){2}{\rule{0.343pt}{0.400pt}}
\multiput(1130.58,495.00)(0.499,1.030){209}{\rule{0.120pt}{0.923pt}}
\multiput(1129.17,495.00)(106.000,216.085){2}{\rule{0.400pt}{0.461pt}}
\multiput(1236.00,711.92)(0.744,-0.499){141}{\rule{0.694pt}{0.120pt}}
\multiput(1236.00,712.17)(105.559,-72.000){2}{\rule{0.347pt}{0.400pt}}
\multiput(1343.58,641.00)(0.499,1.030){209}{\rule{0.120pt}{0.923pt}}
\multiput(1342.17,641.00)(106.000,216.085){2}{\rule{0.400pt}{0.461pt}}
\put(171,131){\makebox(0,0){$+$}}
\put(278,277){\makebox(0,0){$+$}}
\put(384,349){\makebox(0,0){$+$}}
\put(491,495){\makebox(0,0){$+$}}
\put(597,422){\makebox(0,0){$+$}}
\put(704,495){\makebox(0,0){$+$}}
\put(810,422){\makebox(0,0){$+$}}
\put(917,495){\makebox(0,0){$+$}}
\put(1023,422){\makebox(0,0){$+$}}
\put(1130,495){\makebox(0,0){$+$}}
\put(1236,713){\makebox(0,0){$+$}}
\put(1343,641){\makebox(0,0){$+$}}
\put(1449,859){\makebox(0,0){$+$}}
\put(171.0,131.0){\rule[-0.200pt]{0.400pt}{175.375pt}}
\put(171.0,131.0){\rule[-0.200pt]{307.870pt}{0.400pt}}
\put(1449.0,131.0){\rule[-0.200pt]{0.400pt}{175.375pt}}
\put(171.0,859.0){\rule[-0.200pt]{307.870pt}{0.400pt}}
\end{picture}
}
 \end{frame}

 \begin{frame}{条目内容质量评价}
   \begin{block}{维基百科的评价准则}
     详实的内容;恰
当的遣词;观点中立客观;引用外部资料准确;内容结构编排合理;适当添加图
片说明;符合格式指南;无错别字,且标点符号应用得当;链接适当,没有多余的链接。
   \end{block}
    \vfill
   \begin{block}{其他学者的评价准则}
     Stvilia等提取了七个指标作为评价条目内容质量的度量,包
括:1)内容涉及的范围;2)内容的格式;3)内容的独创性;4)内容的权威
性;5)内容的准确程度;6)内容的时效性以及7)内容的可访问性。
   \end{block}
 \end{frame}

 \begin{frame}{量化指标}
 \scriptsize
    \begin{tabular}{|c|c|c|c|}
 
    \hline
    表层特征&结构特征&可读性度量&语法特征\\\hline
    字符数量&内部链接数\footnotemark[1]&Gunning Fog指标&名词短语数\\\hline
    单词数量&外部链接数\footnotemark[2]&Coleman-Liau指标&限定词数\\\hline
    句子数量&所属内容分类&Flesch-Kincaid指标&形容词数\\\hline
    音节数量&图、表的数量&SMOG index&名词数\\\hline
    分词数量&参考文献数量&Automated Readability指标&副词数\\\hline
    单音节词数量&内容段落数量&FORCAST readability指标&过去式动词数量\\\hline
    复合词数量& & &过去分词数量\\\hline
  \end{tabular}
 \end{frame}

 \begin{frame}{量化指标}
\scriptsize
   \begin{tabular}{|c|c|c|c|}
 \hline
    文字特征&结构特征&演化特征&其他特征\\\hline
    条目内容的长度&条目划分的段落&条目的编辑次数&该条目被其他\\
 &内部链接数量&参与编写的用户数量&内容引用
    次数\\
 &外部链接数量&条目编辑的频率& \\
 &条目中图、表的数量& &\\
  &参考文献的数量&& \\\hline
  \end{tabular}
\pause
\begin{block}{回归分析结果}
  \begin{enumerate}
  \item 条目内容的长度
\item 内部链接数
\item 参考文献数
\item 编辑次数
\item 用户数
量
  \end{enumerate}
\end{block}
 \end{frame}

 \begin{frame}{用户每次编辑的协同贡献}
   \vfill
通过评价条目内容的质量,使得用户贡献有了一致的比较基准。为了同用户贡献
一致,也可以将条目质量归一化。用户每次编辑的贡献最终可
以定义为:
\begin{exampleblock}

  \[
\text{每次编辑的协同贡献}=\text{条目内容质量} \times  
\text{用户当次编辑的贡献值}
\]
\end{exampleblock}
\vfill
 \end{frame}


 \begin{frame}{用户分类研究}
   在虚拟社区中,用户是内容的创建者和分享者。社区的活力取决于用户的参与和
活跃程度,并最终影响到社区能否持续存在。
虚拟实践社区中个人的特质、参与动机、参与行为和资历的各不相同,每人的积
极程度、个人性格、表达能力和知识水平也不同,个体在社区互动中扮演的角色
不同,因而在同一虚拟社区内的用户会在社区中具有不同的行为模方式,并自然分化为多
个不同的群体。

研究用户分类对于分析用户参与知识协同动机非常重要。一方面,对于不同类型的用
户起作用的动机因素有所不同;另一方面,对于协同行为的结果,不同类型的用
户也会有不同的反应。
 \end{frame}

 \begin{frame}{已有分类研究: Kozinets}
    \scalebox{0.88}{ \includegraphics{users.jpg}}
 \end{frame}

 \begin{frame}{已有分类研究: 杨堤雅}
   \tiny \vfill
   \begin{tabular}{|c|c|c|c|c|}
    \hline
\backslashbox{特性}{角色} &参与程度&专业知识 &成员互动& 文章主要内容
\\\hline
成员领袖&高$\backslash$ 中&高&高&提供意见、分享经验\\\hline
意见呼应者&高$\backslash$ 中&中$\backslash$低&高&提供意见、分享经验
\\\hline
自我揭露者&中$\backslash$ 低&低&低&分享经验\\\hline
经验意见分享者&高$\backslash$ 中$\backslash$低$\backslash$路过&高
$\backslash$中&中&提供意见、分享经验\\\hline
查询者&低$\backslash$路过&中
$\backslash$低&中$\backslash$低&提问\\\hline
信息推广者&高$\backslash$ 中$\backslash$低$\backslash$路过&高
$\backslash$中&高$\backslash$ 中$\backslash$低&信息推广、建立关系
\\\hline
浏览者&---&低&低&无\\\hline
干扰者&低$\backslash$路过&低&低&其他\\\hline
  \end{tabular}
\vfill
 \end{frame}

 \begin{frame}{分类维度}
   \begin{block}{条目平均贡献}
     用户所参与的所有条目所获得贡献的平均值,该值反映了用户参与的深度。
   \end{block}
\scalebox{0.77}{ % GNUPLOT: LaTeX picture
\setlength{\unitlength}{0.240900pt}
\ifx\plotpoint\undefined\newsavebox{\plotpoint}\fi
\sbox{\plotpoint}{\rule[-0.200pt]{0.400pt}{0.400pt}}%
\begin{picture}(1500,900)(0,0)
\sbox{\plotpoint}{\rule[-0.200pt]{0.400pt}{0.400pt}}%
\put(191.0,252.0){\rule[-0.200pt]{4.818pt}{0.400pt}}
\put(171,252){\makebox(0,0)[r]{5000}}
\put(1419.0,252.0){\rule[-0.200pt]{4.818pt}{0.400pt}}
\put(191.0,374.0){\rule[-0.200pt]{4.818pt}{0.400pt}}
\put(171,374){\makebox(0,0)[r]{10000}}
\put(1419.0,374.0){\rule[-0.200pt]{4.818pt}{0.400pt}}
\put(191.0,616.0){\rule[-0.200pt]{4.818pt}{0.400pt}}
\put(171,616){\makebox(0,0)[r]{20000}}
\put(1419.0,616.0){\rule[-0.200pt]{4.818pt}{0.400pt}}
\put(191.0,859.0){\rule[-0.200pt]{4.818pt}{0.400pt}}
\put(171,859){\makebox(0,0)[r]{30000}}
\put(1419.0,859.0){\rule[-0.200pt]{4.818pt}{0.400pt}}
\put(191.0,131.0){\rule[-0.200pt]{0.400pt}{4.818pt}}
\put(191,90){\makebox(0,0){-0.3}}
\put(191.0,839.0){\rule[-0.200pt]{0.400pt}{4.818pt}}
\put(287.0,131.0){\rule[-0.200pt]{0.400pt}{4.818pt}}
\put(287,90){\makebox(0,0){-0.2}}
\put(287.0,839.0){\rule[-0.200pt]{0.400pt}{4.818pt}}
\put(383.0,131.0){\rule[-0.200pt]{0.400pt}{4.818pt}}
\put(383,90){\makebox(0,0){-0.1}}
\put(383.0,839.0){\rule[-0.200pt]{0.400pt}{4.818pt}}
\put(479.0,131.0){\rule[-0.200pt]{0.400pt}{4.818pt}}
\put(479,90){\makebox(0,0){0}}
\put(479.0,839.0){\rule[-0.200pt]{0.400pt}{4.818pt}}
\put(575.0,131.0){\rule[-0.200pt]{0.400pt}{4.818pt}}
\put(575,90){\makebox(0,0){0.1}}
\put(575.0,839.0){\rule[-0.200pt]{0.400pt}{4.818pt}}
\put(671.0,131.0){\rule[-0.200pt]{0.400pt}{4.818pt}}
\put(671,90){\makebox(0,0){0.2}}
\put(671.0,839.0){\rule[-0.200pt]{0.400pt}{4.818pt}}
\put(767.0,131.0){\rule[-0.200pt]{0.400pt}{4.818pt}}
\put(767,90){\makebox(0,0){0.3}}
\put(767.0,839.0){\rule[-0.200pt]{0.400pt}{4.818pt}}
\put(863.0,131.0){\rule[-0.200pt]{0.400pt}{4.818pt}}
\put(863,90){\makebox(0,0){0.4}}
\put(863.0,839.0){\rule[-0.200pt]{0.400pt}{4.818pt}}
\put(959.0,131.0){\rule[-0.200pt]{0.400pt}{4.818pt}}
\put(959,90){\makebox(0,0){0.5}}
\put(959.0,839.0){\rule[-0.200pt]{0.400pt}{4.818pt}}
\put(1055.0,131.0){\rule[-0.200pt]{0.400pt}{4.818pt}}
\put(1055,90){\makebox(0,0){0.6}}
\put(1055.0,839.0){\rule[-0.200pt]{0.400pt}{4.818pt}}
\put(1151.0,131.0){\rule[-0.200pt]{0.400pt}{4.818pt}}
\put(1151,90){\makebox(0,0){0.7}}
\put(1151.0,839.0){\rule[-0.200pt]{0.400pt}{4.818pt}}
\put(1247.0,131.0){\rule[-0.200pt]{0.400pt}{4.818pt}}
\put(1247,90){\makebox(0,0){0.8}}
\put(1247.0,839.0){\rule[-0.200pt]{0.400pt}{4.818pt}}
\put(1343.0,131.0){\rule[-0.200pt]{0.400pt}{4.818pt}}
\put(1343,90){\makebox(0,0){0.9}}
\put(1343.0,839.0){\rule[-0.200pt]{0.400pt}{4.818pt}}
\put(1439.0,131.0){\rule[-0.200pt]{0.400pt}{4.818pt}}
\put(1439,90){\makebox(0,0){1}}
\put(1439.0,839.0){\rule[-0.200pt]{0.400pt}{4.818pt}}
\put(191.0,131.0){\rule[-0.200pt]{0.400pt}{175.375pt}}
\put(191.0,131.0){\rule[-0.200pt]{300.643pt}{0.400pt}}
\put(1439.0,131.0){\rule[-0.200pt]{0.400pt}{175.375pt}}
\put(191.0,859.0){\rule[-0.200pt]{300.643pt}{0.400pt}}
\put(30,495){\makebox(0,0){\rotatebox{90}{用户数量}}}
\put(815,29){\makebox(0,0){平均贡献}}
\put(1324.0,131.0){\rule[-0.200pt]{0.400pt}{5.059pt}}
\put(1324.0,152.0){\rule[-0.200pt]{9.154pt}{0.400pt}}
\put(1362.0,131.0){\rule[-0.200pt]{0.400pt}{5.059pt}}
\put(1324.0,131.0){\rule[-0.200pt]{9.154pt}{0.400pt}}
\put(1228.0,131.0){\rule[-0.200pt]{0.400pt}{3.613pt}}
\put(1228.0,146.0){\rule[-0.200pt]{9.154pt}{0.400pt}}
\put(1266.0,131.0){\rule[-0.200pt]{0.400pt}{3.613pt}}
\put(1228.0,131.0){\rule[-0.200pt]{9.154pt}{0.400pt}}
\put(1132.0,131.0){\rule[-0.200pt]{0.400pt}{3.132pt}}
\put(1132.0,144.0){\rule[-0.200pt]{9.154pt}{0.400pt}}
\put(1170.0,131.0){\rule[-0.200pt]{0.400pt}{3.132pt}}
\put(1132.0,131.0){\rule[-0.200pt]{9.154pt}{0.400pt}}
\put(1036.0,131.0){\rule[-0.200pt]{0.400pt}{3.613pt}}
\put(1036.0,146.0){\rule[-0.200pt]{9.154pt}{0.400pt}}
\put(1074.0,131.0){\rule[-0.200pt]{0.400pt}{3.613pt}}
\put(1036.0,131.0){\rule[-0.200pt]{9.154pt}{0.400pt}}
\put(940.0,131.0){\rule[-0.200pt]{0.400pt}{4.818pt}}
\put(940.0,151.0){\rule[-0.200pt]{9.154pt}{0.400pt}}
\put(978.0,131.0){\rule[-0.200pt]{0.400pt}{4.818pt}}
\put(940.0,131.0){\rule[-0.200pt]{9.154pt}{0.400pt}}
\put(844.0,131.0){\rule[-0.200pt]{0.400pt}{8.431pt}}
\put(844.0,166.0){\rule[-0.200pt]{9.154pt}{0.400pt}}
\put(882.0,131.0){\rule[-0.200pt]{0.400pt}{8.431pt}}
\put(844.0,131.0){\rule[-0.200pt]{9.154pt}{0.400pt}}
\put(748.0,131.0){\rule[-0.200pt]{0.400pt}{10.600pt}}
\put(748.0,175.0){\rule[-0.200pt]{9.154pt}{0.400pt}}
\put(786.0,131.0){\rule[-0.200pt]{0.400pt}{10.600pt}}
\put(748.0,131.0){\rule[-0.200pt]{9.154pt}{0.400pt}}
\put(652.0,131.0){\rule[-0.200pt]{0.400pt}{16.622pt}}
\put(652.0,200.0){\rule[-0.200pt]{9.154pt}{0.400pt}}
\put(690.0,131.0){\rule[-0.200pt]{0.400pt}{16.622pt}}
\put(652.0,131.0){\rule[-0.200pt]{9.154pt}{0.400pt}}
\put(556.0,131.0){\rule[-0.200pt]{0.400pt}{27.222pt}}
\put(556.0,244.0){\rule[-0.200pt]{9.154pt}{0.400pt}}
\put(594.0,131.0){\rule[-0.200pt]{0.400pt}{27.222pt}}
\put(556.0,131.0){\rule[-0.200pt]{9.154pt}{0.400pt}}
\put(460.0,131.0){\rule[-0.200pt]{0.400pt}{166.221pt}}
\put(460.0,821.0){\rule[-0.200pt]{9.154pt}{0.400pt}}
\put(498.0,131.0){\rule[-0.200pt]{0.400pt}{166.221pt}}
\put(460.0,131.0){\rule[-0.200pt]{9.154pt}{0.400pt}}
\put(364.0,131.0){\rule[-0.200pt]{0.400pt}{56.371pt}}
\put(364.0,365.0){\rule[-0.200pt]{9.154pt}{0.400pt}}
\put(402.0,131.0){\rule[-0.200pt]{0.400pt}{56.371pt}}
\put(364.0,131.0){\rule[-0.200pt]{9.154pt}{0.400pt}}
\put(268.0,131.0){\rule[-0.200pt]{0.400pt}{4.095pt}}
\put(268.0,148.0){\rule[-0.200pt]{9.154pt}{0.400pt}}
\put(306.0,131.0){\rule[-0.200pt]{0.400pt}{4.095pt}}
\put(268.0,131.0){\rule[-0.200pt]{9.154pt}{0.400pt}}
\put(191.0,131.0){\rule[-0.200pt]{0.400pt}{2.168pt}}
\put(191.0,140.0){\rule[-0.200pt]{4.577pt}{0.400pt}}
\put(210.0,131.0){\rule[-0.200pt]{0.400pt}{2.168pt}}
\put(191.0,131.0){\rule[-0.200pt]{4.577pt}{0.400pt}}
\put(191.0,131.0){\rule[-0.200pt]{0.400pt}{175.375pt}}
\put(191.0,131.0){\rule[-0.200pt]{300.643pt}{0.400pt}}
\put(1439.0,131.0){\rule[-0.200pt]{0.400pt}{175.375pt}}
\put(191.0,859.0){\rule[-0.200pt]{300.643pt}{0.400pt}}
\end{picture}
}  
 \end{frame}

 \begin{frame}{分类维度}
    \begin{block}{参与条目数量}
     用户所参与的所有条目数量,该值反映了用户参与的广度。
   \end{block}
\scalebox{0.77}{ % GNUPLOT: LaTeX picture
\setlength{\unitlength}{0.240900pt}
\ifx\plotpoint\undefined\newsavebox{\plotpoint}\fi
\begin{picture}(1500,900)(0,0)
\sbox{\plotpoint}{\rule[-0.200pt]{0.400pt}{0.400pt}}%
\put(191.0,252.0){\rule[-0.200pt]{4.818pt}{0.400pt}}
\put(171,252){\makebox(0,0)[r]{5000}}
\put(1419.0,252.0){\rule[-0.200pt]{4.818pt}{0.400pt}}
\put(191.0,374.0){\rule[-0.200pt]{4.818pt}{0.400pt}}
\put(171,374){\makebox(0,0)[r]{10000}}
\put(1419.0,374.0){\rule[-0.200pt]{4.818pt}{0.400pt}}
\put(191.0,616.0){\rule[-0.200pt]{4.818pt}{0.400pt}}
\put(171,616){\makebox(0,0)[r]{20000}}
\put(1419.0,616.0){\rule[-0.200pt]{4.818pt}{0.400pt}}
\put(191.0,859.0){\rule[-0.200pt]{4.818pt}{0.400pt}}
\put(171,859){\makebox(0,0)[r]{30000}}
\put(1419.0,859.0){\rule[-0.200pt]{4.818pt}{0.400pt}}
\put(191.0,131.0){\rule[-0.200pt]{0.400pt}{4.818pt}}
\put(191,90){\makebox(0,0){0}}
\put(191.0,839.0){\rule[-0.200pt]{0.400pt}{4.818pt}}
\put(304.0,131.0){\rule[-0.200pt]{0.400pt}{4.818pt}}
\put(304,90){\makebox(0,0){1}}
\put(304.0,839.0){\rule[-0.200pt]{0.400pt}{4.818pt}}
\put(418.0,131.0){\rule[-0.200pt]{0.400pt}{4.818pt}}
\put(418,90){\makebox(0,0){2}}
\put(418.0,839.0){\rule[-0.200pt]{0.400pt}{4.818pt}}
\put(531.0,131.0){\rule[-0.200pt]{0.400pt}{4.818pt}}
\put(531,90){\makebox(0,0){3}}
\put(531.0,839.0){\rule[-0.200pt]{0.400pt}{4.818pt}}
\put(645.0,131.0){\rule[-0.200pt]{0.400pt}{4.818pt}}
\put(645,90){\makebox(0,0){4}}
\put(645.0,839.0){\rule[-0.200pt]{0.400pt}{4.818pt}}
\put(758.0,131.0){\rule[-0.200pt]{0.400pt}{4.818pt}}
\put(758,90){\makebox(0,0){5}}
\put(758.0,839.0){\rule[-0.200pt]{0.400pt}{4.818pt}}
\put(872.0,131.0){\rule[-0.200pt]{0.400pt}{4.818pt}}
\put(872,90){\makebox(0,0){6}}
\put(872.0,839.0){\rule[-0.200pt]{0.400pt}{4.818pt}}
\put(985.0,131.0){\rule[-0.200pt]{0.400pt}{4.818pt}}
\put(985,90){\makebox(0,0){7}}
\put(985.0,839.0){\rule[-0.200pt]{0.400pt}{4.818pt}}
\put(1099.0,131.0){\rule[-0.200pt]{0.400pt}{4.818pt}}
\put(1099,90){\makebox(0,0){8}}
\put(1099.0,839.0){\rule[-0.200pt]{0.400pt}{4.818pt}}
\put(1212.0,131.0){\rule[-0.200pt]{0.400pt}{4.818pt}}
\put(1212,90){\makebox(0,0){9}}
\put(1212.0,839.0){\rule[-0.200pt]{0.400pt}{4.818pt}}
\put(1326.0,131.0){\rule[-0.200pt]{0.400pt}{4.818pt}}
\put(1326,90){\makebox(0,0){10}}
\put(1326.0,839.0){\rule[-0.200pt]{0.400pt}{4.818pt}}
\put(1439.0,131.0){\rule[-0.200pt]{0.400pt}{4.818pt}}
\put(1439,90){\makebox(0,0){11}}
\put(1439.0,839.0){\rule[-0.200pt]{0.400pt}{4.818pt}}
\put(191.0,131.0){\rule[-0.200pt]{0.400pt}{175.375pt}}
\put(191.0,131.0){\rule[-0.200pt]{300.643pt}{0.400pt}}
\put(1439.0,131.0){\rule[-0.200pt]{0.400pt}{175.375pt}}
\put(191.0,859.0){\rule[-0.200pt]{300.643pt}{0.400pt}}
\put(30,495){\makebox(0,0){\rotatebox{90}{用户数量}}}
\put(815,29){\makebox(0,0){参与条目}}
\put(287.0,131.0){\rule[-0.200pt]{0.400pt}{168.148pt}}
\put(287.0,829.0){\rule[-0.200pt]{8.191pt}{0.400pt}}
\put(321.0,131.0){\rule[-0.200pt]{0.400pt}{168.148pt}}
\put(287.0,131.0){\rule[-0.200pt]{8.191pt}{0.400pt}}
\put(401.0,131.0){\rule[-0.200pt]{0.400pt}{49.625pt}}
\put(401.0,337.0){\rule[-0.200pt]{8.191pt}{0.400pt}}
\put(435.0,131.0){\rule[-0.200pt]{0.400pt}{49.625pt}}
\put(401.0,131.0){\rule[-0.200pt]{8.191pt}{0.400pt}}
\put(514.0,131.0){\rule[-0.200pt]{0.400pt}{23.126pt}}
\put(514.0,227.0){\rule[-0.200pt]{8.191pt}{0.400pt}}
\put(548.0,131.0){\rule[-0.200pt]{0.400pt}{23.126pt}}
\put(514.0,131.0){\rule[-0.200pt]{8.191pt}{0.400pt}}
\put(628.0,131.0){\rule[-0.200pt]{0.400pt}{13.731pt}}
\put(628.0,188.0){\rule[-0.200pt]{8.191pt}{0.400pt}}
\put(662.0,131.0){\rule[-0.200pt]{0.400pt}{13.731pt}}
\put(628.0,131.0){\rule[-0.200pt]{8.191pt}{0.400pt}}
\put(741.0,131.0){\rule[-0.200pt]{0.400pt}{9.154pt}}
\put(741.0,169.0){\rule[-0.200pt]{8.191pt}{0.400pt}}
\put(775.0,131.0){\rule[-0.200pt]{0.400pt}{9.154pt}}
\put(741.0,131.0){\rule[-0.200pt]{8.191pt}{0.400pt}}
\put(855.0,131.0){\rule[-0.200pt]{0.400pt}{5.059pt}}
\put(855.0,152.0){\rule[-0.200pt]{8.191pt}{0.400pt}}
\put(889.0,131.0){\rule[-0.200pt]{0.400pt}{5.059pt}}
\put(855.0,131.0){\rule[-0.200pt]{8.191pt}{0.400pt}}
\put(968.0,131.0){\rule[-0.200pt]{0.400pt}{5.059pt}}
\put(968.0,152.0){\rule[-0.200pt]{8.191pt}{0.400pt}}
\put(1002.0,131.0){\rule[-0.200pt]{0.400pt}{5.059pt}}
\put(968.0,131.0){\rule[-0.200pt]{8.191pt}{0.400pt}}
\put(1082.0,131.0){\rule[-0.200pt]{0.400pt}{4.095pt}}
\put(1082.0,148.0){\rule[-0.200pt]{8.191pt}{0.400pt}}
\put(1116.0,131.0){\rule[-0.200pt]{0.400pt}{4.095pt}}
\put(1082.0,131.0){\rule[-0.200pt]{8.191pt}{0.400pt}}
\put(1195.0,131.0){\rule[-0.200pt]{0.400pt}{3.132pt}}
\put(1195.0,144.0){\rule[-0.200pt]{8.191pt}{0.400pt}}
\put(1229.0,131.0){\rule[-0.200pt]{0.400pt}{3.132pt}}
\put(1195.0,131.0){\rule[-0.200pt]{8.191pt}{0.400pt}}
\put(1309.0,131.0){\rule[-0.200pt]{0.400pt}{2.650pt}}
\put(1309.0,142.0){\rule[-0.200pt]{8.191pt}{0.400pt}}
\put(1343.0,131.0){\rule[-0.200pt]{0.400pt}{2.650pt}}
\put(1309.0,131.0){\rule[-0.200pt]{8.191pt}{0.400pt}}
\put(191.0,131.0){\rule[-0.200pt]{0.400pt}{175.375pt}}
\put(191.0,131.0){\rule[-0.200pt]{300.643pt}{0.400pt}}
\put(1439.0,131.0){\rule[-0.200pt]{0.400pt}{175.375pt}}
\put(191.0,859.0){\rule[-0.200pt]{300.643pt}{0.400pt}}
\end{picture}
}  
 \end{frame}

 \begin{frame}{分类维度}
 
\scalebox{0.77}{ % GNUPLOT: LaTeX picture
\setlength{\unitlength}{0.240900pt}
\ifx\plotpoint\undefined\newsavebox{\plotpoint}\fi
\begin{picture}(1500,900)(0,0)
\sbox{\plotpoint}{\rule[-0.200pt]{0.400pt}{0.400pt}}%
\put(171.0,235.0){\rule[-0.200pt]{4.818pt}{0.400pt}}
\put(151,235){\makebox(0,0)[r]{500}}
\put(1419.0,235.0){\rule[-0.200pt]{4.818pt}{0.400pt}}
\put(171.0,339.0){\rule[-0.200pt]{4.818pt}{0.400pt}}
\put(151,339){\makebox(0,0)[r]{1000}}
\put(1419.0,339.0){\rule[-0.200pt]{4.818pt}{0.400pt}}
\put(171.0,547.0){\rule[-0.200pt]{4.818pt}{0.400pt}}
\put(151,547){\makebox(0,0)[r]{2000}}
\put(1419.0,547.0){\rule[-0.200pt]{4.818pt}{0.400pt}}
\put(171.0,859.0){\rule[-0.200pt]{4.818pt}{0.400pt}}
\put(151,859){\makebox(0,0)[r]{3500}}
\put(1419.0,859.0){\rule[-0.200pt]{4.818pt}{0.400pt}}
\put(286.0,131.0){\rule[-0.200pt]{0.400pt}{4.818pt}}
\put(286,90){\makebox(0,0){6-10}}
\put(286.0,839.0){\rule[-0.200pt]{0.400pt}{4.818pt}}
\put(402.0,131.0){\rule[-0.200pt]{0.400pt}{4.818pt}}
\put(402,90){\makebox(0,0){20  }}
\put(402.0,839.0){\rule[-0.200pt]{0.400pt}{4.818pt}}
\put(517.0,131.0){\rule[-0.200pt]{0.400pt}{4.818pt}}
\put(517,90){\makebox(0,0){30\ }}
\put(517.0,839.0){\rule[-0.200pt]{0.400pt}{4.818pt}}
\put(632.0,131.0){\rule[-0.200pt]{0.400pt}{4.818pt}}
\put(632,90){\makebox(0,0){40  }}
\put(632.0,839.0){\rule[-0.200pt]{0.400pt}{4.818pt}}
\put(747.0,131.0){\rule[-0.200pt]{0.400pt}{4.818pt}}
\put(747,90){\makebox(0,0){50  }}
\put(747.0,839.0){\rule[-0.200pt]{0.400pt}{4.818pt}}
\put(863.0,131.0){\rule[-0.200pt]{0.400pt}{4.818pt}}
\put(863,90){\makebox(0,0){60  }}
\put(863.0,839.0){\rule[-0.200pt]{0.400pt}{4.818pt}}
\put(978.0,131.0){\rule[-0.200pt]{0.400pt}{4.818pt}}
\put(978,90){\makebox(0,0){70  }}
\put(978.0,839.0){\rule[-0.200pt]{0.400pt}{4.818pt}}
\put(1093.0,131.0){\rule[-0.200pt]{0.400pt}{4.818pt}}
\put(1093,90){\makebox(0,0){80  }}
\put(1093.0,839.0){\rule[-0.200pt]{0.400pt}{4.818pt}}
\put(1208.0,131.0){\rule[-0.200pt]{0.400pt}{4.818pt}}
\put(1208,90){\makebox(0,0){90  }}
\put(1208.0,839.0){\rule[-0.200pt]{0.400pt}{4.818pt}}
\put(1324.0,131.0){\rule[-0.200pt]{0.400pt}{4.818pt}}
\put(1324,90){\makebox(0,0){>90}}
\put(1324.0,839.0){\rule[-0.200pt]{0.400pt}{4.818pt}}
\put(1439.0,131.0){\rule[-0.200pt]{0.400pt}{4.818pt}}
\put(1439,90){\makebox(0,0){ }}
\put(1439.0,839.0){\rule[-0.200pt]{0.400pt}{4.818pt}}
\put(171.0,131.0){\rule[-0.200pt]{0.400pt}{175.375pt}}
\put(171.0,131.0){\rule[-0.200pt]{305.461pt}{0.400pt}}
\put(1439.0,131.0){\rule[-0.200pt]{0.400pt}{175.375pt}}
\put(171.0,859.0){\rule[-0.200pt]{305.461pt}{0.400pt}}
\put(30,495){\makebox(0,0){\rotatebox{90}{用户数量}}}
\put(805,29){\makebox(0,0){参与条目}}
\put(269.0,131.0){\rule[-0.200pt]{0.400pt}{170.316pt}}
\put(269.0,838.0){\rule[-0.200pt]{8.431pt}{0.400pt}}
\put(304.0,131.0){\rule[-0.200pt]{0.400pt}{170.316pt}}
\put(269.0,131.0){\rule[-0.200pt]{8.431pt}{0.400pt}}
\put(384.0,131.0){\rule[-0.200pt]{0.400pt}{136.349pt}}
\put(384.0,697.0){\rule[-0.200pt]{8.431pt}{0.400pt}}
\put(419.0,131.0){\rule[-0.200pt]{0.400pt}{136.349pt}}
\put(384.0,131.0){\rule[-0.200pt]{8.431pt}{0.400pt}}
\put(500.0,131.0){\rule[-0.200pt]{0.400pt}{49.144pt}}
\put(500.0,335.0){\rule[-0.200pt]{8.191pt}{0.400pt}}
\put(534.0,131.0){\rule[-0.200pt]{0.400pt}{49.144pt}}
\put(500.0,131.0){\rule[-0.200pt]{8.191pt}{0.400pt}}
\put(615.0,131.0){\rule[-0.200pt]{0.400pt}{27.222pt}}
\put(615.0,244.0){\rule[-0.200pt]{8.191pt}{0.400pt}}
\put(649.0,131.0){\rule[-0.200pt]{0.400pt}{27.222pt}}
\put(615.0,131.0){\rule[-0.200pt]{8.191pt}{0.400pt}}
\put(730.0,131.0){\rule[-0.200pt]{0.400pt}{17.104pt}}
\put(730.0,202.0){\rule[-0.200pt]{8.431pt}{0.400pt}}
\put(765.0,131.0){\rule[-0.200pt]{0.400pt}{17.104pt}}
\put(730.0,131.0){\rule[-0.200pt]{8.431pt}{0.400pt}}
\put(845.0,131.0){\rule[-0.200pt]{0.400pt}{12.045pt}}
\put(845.0,181.0){\rule[-0.200pt]{8.431pt}{0.400pt}}
\put(880.0,131.0){\rule[-0.200pt]{0.400pt}{12.045pt}}
\put(845.0,131.0){\rule[-0.200pt]{8.431pt}{0.400pt}}
\put(961.0,131.0){\rule[-0.200pt]{0.400pt}{9.636pt}}
\put(961.0,171.0){\rule[-0.200pt]{8.191pt}{0.400pt}}
\put(995.0,131.0){\rule[-0.200pt]{0.400pt}{9.636pt}}
\put(961.0,131.0){\rule[-0.200pt]{8.191pt}{0.400pt}}
\put(1076.0,131.0){\rule[-0.200pt]{0.400pt}{7.950pt}}
\put(1076.0,164.0){\rule[-0.200pt]{8.191pt}{0.400pt}}
\put(1110.0,131.0){\rule[-0.200pt]{0.400pt}{7.950pt}}
\put(1076.0,131.0){\rule[-0.200pt]{8.191pt}{0.400pt}}
\put(1191.0,131.0){\rule[-0.200pt]{0.400pt}{5.782pt}}
\put(1191.0,155.0){\rule[-0.200pt]{8.431pt}{0.400pt}}
\put(1226.0,131.0){\rule[-0.200pt]{0.400pt}{5.782pt}}
\put(1191.0,131.0){\rule[-0.200pt]{8.431pt}{0.400pt}}
\put(1306.0,131.0){\rule[-0.200pt]{0.400pt}{4.336pt}}
\put(1306.0,149.0){\rule[-0.200pt]{8.431pt}{0.400pt}}
\put(1341.0,131.0){\rule[-0.200pt]{0.400pt}{4.336pt}}
\put(1306.0,131.0){\rule[-0.200pt]{8.431pt}{0.400pt}}
\put(171.0,131.0){\rule[-0.200pt]{0.400pt}{175.375pt}}
\put(171.0,131.0){\rule[-0.200pt]{305.461pt}{0.400pt}}
\put(1439.0,131.0){\rule[-0.200pt]{0.400pt}{175.375pt}}
\put(171.0,859.0){\rule[-0.200pt]{305.461pt}{0.400pt}}
\end{picture}
}  
 \end{frame}

 \begin{frame}{用户分类}
    \scalebox{0.51}{\includegraphics{user-cat.pdf}}
 \end{frame}

 \begin{frame}{用户特点}
   \begin{block}{ 领导者}
    这类用户的特点是广泛深入地参与到社区的协同活动中去。不但
  参与了大量条目的编辑工作,而且对于每个条目都积极投入,是协同编辑主要
  的领导者和最重要的贡献者。这类用户所占的比例极小,但是却起到了引领社
  区前进和发展的作用。
   \end{block}

   \begin{block}{领域专家}
     这类用户的特点是对某些领域的知识精通。对于该领域下的条
  目具有独立撰写、或者领导其他用户共同完成编写条目的能力。对于每个参与
  的条目,他们都能深入地参与并贡献高质量的内容。同领导者群体
  不同,领域专家参与社区活动的热情要小得多,只求做好自己擅长的工作即可。
  因此他们实际参与的条目数量都比较小。
   \end{block}
 \end{frame}

 \begin{frame}{用户特点}
   \begin{block}{ 内容贡献者}
    这类用户是协同活动的积极参与者。尽管他们限于自身的知识水
  平和个人能力还不足以起到领导知识协同的作用,但是他们是领域专家和领导者的追随者。他
  们的工作对于补充、丰富条目的内容起到了重要的作用。对于前两类用户所忽
  视或者涉及不到的内容,都是由这类用户完成的。尽管由于时间和精力的原因
  他们参与的广泛程度不同,但是参与的动机基本是一致的。
   \end{block}
 \end{frame}

 \begin{frame}{用户特点}
   \begin{block}{ 内容维护者}
   内容维护者的主要作用在于修补条目内容的疏漏,及时更新
  过时、无效的信息。对于每个条目,内容维护者参与的程度都不高,但是其参
  与的范围却很广泛。
 \end{block}
 \begin{block}{ 边缘用户}
   边缘用户是所有用户类别中数量最为庞大的群体。他们很少参
  与到协同活动中,即使偶尔参与协同也往往会由于经验不足而贡献低质量的内
  容,很快被回退或被其他用户修改。从主观上讲,边缘用户愿意参与到社区活
  动中,这是他们区别于破坏者和潜水者的显著特征。但是他们的意愿收到了某
  种客观条件的制约,一旦条件成熟,将促进边缘用户向更高级的用户转变。
 \end{block}
 \end{frame}

 \begin{frame}{不同用户间的相互关系}
   \begin{block}{领导者和领域专家}
     领域专家和领导者两类用户都属于能够主导条目编写的类型,用户之间几
     乎没有任何凝聚效应,与其他用户的关联非常稀疏,较少收到其他人行为
     的影响。
   \end{block}
\vfill
   \begin{block}{其他用户}
     与其他用户的关联紧密,容易收到其他人行为的影响。
   \end{block}
 \end{frame}   
   \begin{frame}{两种协同形式}
     \begin{block}{少数人的力量}
       是以领导者或者领域专家
为主导,内容贡献者和维护者参与辅助性工作最终完成条目的编写;
     \end{block}
\vfill
     \begin{block}{多数人的智慧}
       参
与编写的条目中没有真正的主导者,而是由多个内容贡献者和维护者通力合作,
利用集体的力量共同完成的。
     \end{block}
   \end{frame}

   \begin{frame}{知识协同的动机因素}
     \begin{block}{个体动机因素}
       个体动机因素强调个体固有的感受和需要,即使
个体处于一个“独立”的环境下,个体因素仍然可以起作用,促使其从事某种行
为。在这种动机的作用下,个体更看中行为本身所带来的心理上的满足感。
     \end{block}
\vfill
\begin{block}{人际间动机因素}
  人际
间动机则存在于个体之间的互动过程中,是其他人的行为加于个体自身的感受和
需要。这时个体关注的焦点是其他人的行为对自身的影响。在知识协同过程中,
个体动机因素和人际间动机因素同时起作
用,共同影响了个体的
行为。
\end{block}
   \end{frame}

   \begin{frame}{本文所研究的动机因素}
     \scalebox{0.55}{\includegraphics{factors.pdf}}
   \end{frame}

   \begin{frame}{领导者和领域专家动机因素因果图}
     \scalebox{0.5}{\includegraphics{motive1.pdf}}
   \end{frame}

   \begin{frame}{内容贡献者、维护者和边缘用户动机因素因果图}
       \scalebox{0.53}{\includegraphics{motive2.pdf}}
   \end{frame}

   \begin{frame}{内容贡献者、维护者和边缘用户动机因素因果图}
       \scalebox{0.73}{\includegraphics{motive3.pdf}}
   \end{frame}

   \begin{frame}{知识协同个体动机因素的存量流量图}
     \scalebox{0.45}{\includegraphics{io1.pdf}}
   \end{frame}

 \begin{frame}{知识协同人际动机因素的存量流量图}
     \scalebox{0.45}{\includegraphics{io2.pdf}}
   \end{frame}

   \begin{frame}{参数估计}
\scriptsize
    \begin{tabular}{|c|c|c|c|c|c|}
\hline
\multicolumn{ 1}{|c|}{变量名称} &                                     \multicolumn{ 5}{|c|}{初始值} \\
\hline
\multicolumn{ 1}{|c|}{} &        领导者 &       领域专家 &      内容贡献者 &      内容维护者 &       边缘用户 \\
\hline
      利他主义 &      13      &        13    &        10    &   10         &      10      \\
\hline
    感知到的意义 &      10      &     10       &     10       &   10         &      10      \\
\hline
      自我决定 &      20      &     18       &      15      &          10  &    10        \\
\hline
      自我效能 &     25       &      15      &        15    &        8    &       5     \\
\hline
      自我肯定 &      25      &    15        &     10       &      10      &       10     \\
\hline
      成就需求 &     20       &     18       &      18      &      5      &        5    \\
\hline
      认知失调 &     $\slash$       &       $\slash$       &      10      &       15     &      9      \\
\hline
      群体效能 &     $\slash$         &          $\slash$    &    15        &      10      &       5     \\
\hline
       归属感 &       $\slash$       &     $\slash$         &      18      &        18    &        5    \\\hline
\end{tabular}  

   \end{frame}

   \begin{frame}{历史检验}
     领导者用户
\scalebox{0.8}{% GNUPLOT: LaTeX picture
\setlength{\unitlength}{0.240900pt}
\ifx\plotpoint\undefined\newsavebox{\plotpoint}\fi
\begin{picture}(1500,900)(0,0)
\sbox{\plotpoint}{\rule[-0.200pt]{0.400pt}{0.400pt}}%
\put(131.0,204.0){\rule[-0.200pt]{4.818pt}{0.400pt}}
\put(111,204){\makebox(0,0)[r]{5}}
\put(1419.0,204.0){\rule[-0.200pt]{4.818pt}{0.400pt}}
\put(131.0,277.0){\rule[-0.200pt]{4.818pt}{0.400pt}}
\put(111,277){\makebox(0,0)[r]{10}}
\put(1419.0,277.0){\rule[-0.200pt]{4.818pt}{0.400pt}}
\put(131.0,349.0){\rule[-0.200pt]{4.818pt}{0.400pt}}
\put(111,349){\makebox(0,0)[r]{15}}
\put(1419.0,349.0){\rule[-0.200pt]{4.818pt}{0.400pt}}
\put(131.0,422.0){\rule[-0.200pt]{4.818pt}{0.400pt}}
\put(111,422){\makebox(0,0)[r]{20}}
\put(1419.0,422.0){\rule[-0.200pt]{4.818pt}{0.400pt}}
\put(131.0,495.0){\rule[-0.200pt]{4.818pt}{0.400pt}}
\put(111,495){\makebox(0,0)[r]{25}}
\put(1419.0,495.0){\rule[-0.200pt]{4.818pt}{0.400pt}}
\put(131.0,568.0){\rule[-0.200pt]{4.818pt}{0.400pt}}
\put(111,568){\makebox(0,0)[r]{30}}
\put(1419.0,568.0){\rule[-0.200pt]{4.818pt}{0.400pt}}
\put(131.0,641.0){\rule[-0.200pt]{4.818pt}{0.400pt}}
\put(111,641){\makebox(0,0)[r]{35}}
\put(1419.0,641.0){\rule[-0.200pt]{4.818pt}{0.400pt}}
\put(131.0,713.0){\rule[-0.200pt]{4.818pt}{0.400pt}}
\put(111,713){\makebox(0,0)[r]{40}}
\put(1419.0,713.0){\rule[-0.200pt]{4.818pt}{0.400pt}}
\put(164.0,131.0){\rule[-0.200pt]{0.400pt}{4.818pt}}
\put(164,90){\makebox(0,0){ 1}}
\put(164.0,839.0){\rule[-0.200pt]{0.400pt}{4.818pt}}
\put(229.0,131.0){\rule[-0.200pt]{0.400pt}{4.818pt}}
\put(229,90){\makebox(0,0){ 3}}
\put(229.0,839.0){\rule[-0.200pt]{0.400pt}{4.818pt}}
\put(295.0,131.0){\rule[-0.200pt]{0.400pt}{4.818pt}}
\put(295,90){\makebox(0,0){ 5}}
\put(295.0,839.0){\rule[-0.200pt]{0.400pt}{4.818pt}}
\put(360.0,131.0){\rule[-0.200pt]{0.400pt}{4.818pt}}
\put(360,90){\makebox(0,0){ 7}}
\put(360.0,839.0){\rule[-0.200pt]{0.400pt}{4.818pt}}
\put(425.0,131.0){\rule[-0.200pt]{0.400pt}{4.818pt}}
\put(425,90){\makebox(0,0){ 9}}
\put(425.0,839.0){\rule[-0.200pt]{0.400pt}{4.818pt}}
\put(491.0,131.0){\rule[-0.200pt]{0.400pt}{4.818pt}}
\put(491,90){\makebox(0,0){ 11}}
\put(491.0,839.0){\rule[-0.200pt]{0.400pt}{4.818pt}}
\put(556.0,131.0){\rule[-0.200pt]{0.400pt}{4.818pt}}
\put(556,90){\makebox(0,0){ 13}}
\put(556.0,839.0){\rule[-0.200pt]{0.400pt}{4.818pt}}
\put(622.0,131.0){\rule[-0.200pt]{0.400pt}{4.818pt}}
\put(622,90){\makebox(0,0){ 15}}
\put(622.0,839.0){\rule[-0.200pt]{0.400pt}{4.818pt}}
\put(687.0,131.0){\rule[-0.200pt]{0.400pt}{4.818pt}}
\put(687,90){\makebox(0,0){ 17}}
\put(687.0,839.0){\rule[-0.200pt]{0.400pt}{4.818pt}}
\put(752.0,131.0){\rule[-0.200pt]{0.400pt}{4.818pt}}
\put(752,90){\makebox(0,0){ 19}}
\put(752.0,839.0){\rule[-0.200pt]{0.400pt}{4.818pt}}
\put(818.0,131.0){\rule[-0.200pt]{0.400pt}{4.818pt}}
\put(818,90){\makebox(0,0){ 21}}
\put(818.0,839.0){\rule[-0.200pt]{0.400pt}{4.818pt}}
\put(883.0,131.0){\rule[-0.200pt]{0.400pt}{4.818pt}}
\put(883,90){\makebox(0,0){ 23}}
\put(883.0,839.0){\rule[-0.200pt]{0.400pt}{4.818pt}}
\put(949.0,131.0){\rule[-0.200pt]{0.400pt}{4.818pt}}
\put(949,90){\makebox(0,0){ 25}}
\put(949.0,839.0){\rule[-0.200pt]{0.400pt}{4.818pt}}
\put(1014.0,131.0){\rule[-0.200pt]{0.400pt}{4.818pt}}
\put(1014,90){\makebox(0,0){ 27}}
\put(1014.0,839.0){\rule[-0.200pt]{0.400pt}{4.818pt}}
\put(1079.0,131.0){\rule[-0.200pt]{0.400pt}{4.818pt}}
\put(1079,90){\makebox(0,0){ 29}}
\put(1079.0,839.0){\rule[-0.200pt]{0.400pt}{4.818pt}}
\put(1145.0,131.0){\rule[-0.200pt]{0.400pt}{4.818pt}}
\put(1145,90){\makebox(0,0){ 31}}
\put(1145.0,839.0){\rule[-0.200pt]{0.400pt}{4.818pt}}
\put(1210.0,131.0){\rule[-0.200pt]{0.400pt}{4.818pt}}
\put(1210,90){\makebox(0,0){ 33}}
\put(1210.0,839.0){\rule[-0.200pt]{0.400pt}{4.818pt}}
\put(1276.0,131.0){\rule[-0.200pt]{0.400pt}{4.818pt}}
\put(1276,90){\makebox(0,0){ 35}}
\put(1276.0,839.0){\rule[-0.200pt]{0.400pt}{4.818pt}}
\put(1341.0,131.0){\rule[-0.200pt]{0.400pt}{4.818pt}}
\put(1341,90){\makebox(0,0){ 37}}
\put(1341.0,839.0){\rule[-0.200pt]{0.400pt}{4.818pt}}
\put(1406.0,131.0){\rule[-0.200pt]{0.400pt}{4.818pt}}
\put(1406,90){\makebox(0,0){ 39}}
\put(1406.0,839.0){\rule[-0.200pt]{0.400pt}{4.818pt}}
\put(131.0,131.0){\rule[-0.200pt]{0.400pt}{175.375pt}}
\put(131.0,131.0){\rule[-0.200pt]{315.097pt}{0.400pt}}
\put(1439.0,131.0){\rule[-0.200pt]{0.400pt}{175.375pt}}
\put(131.0,859.0){\rule[-0.200pt]{315.097pt}{0.400pt}}
\put(30,495){\makebox(0,0){\rotatebox{90}{用户贡献}}}
\put(785,29){\makebox(0,0){}}
\sbox{\plotpoint}{\rule[-0.400pt]{0.800pt}{0.800pt}}%
\sbox{\plotpoint}{\rule[-0.200pt]{0.400pt}{0.400pt}}%
\put(1279,819){\makebox(0,0)[r]{仿真数据}}
\sbox{\plotpoint}{\rule[-0.400pt]{0.800pt}{0.800pt}}%
\put(1299.0,819.0){\rule[-0.400pt]{24.090pt}{0.800pt}}
\put(164,309){\usebox{\plotpoint}}
\multiput(165.41,309.00)(0.503,0.785){57}{\rule{0.121pt}{1.450pt}}
\multiput(162.34,309.00)(32.000,46.990){2}{\rule{0.800pt}{0.725pt}}
\multiput(196.00,360.41)(0.568,0.504){51}{\rule{1.110pt}{0.121pt}}
\multiput(196.00,357.34)(30.695,29.000){2}{\rule{0.555pt}{0.800pt}}
\multiput(229.00,389.41)(0.793,0.505){35}{\rule{1.457pt}{0.122pt}}
\multiput(229.00,386.34)(29.976,21.000){2}{\rule{0.729pt}{0.800pt}}
\multiput(262.00,410.41)(1.055,0.507){25}{\rule{1.850pt}{0.122pt}}
\multiput(262.00,407.34)(29.160,16.000){2}{\rule{0.925pt}{0.800pt}}
\multiput(295.00,426.41)(1.278,0.509){19}{\rule{2.169pt}{0.123pt}}
\multiput(295.00,423.34)(27.498,13.000){2}{\rule{1.085pt}{0.800pt}}
\multiput(327.00,439.40)(1.586,0.512){15}{\rule{2.600pt}{0.123pt}}
\multiput(327.00,436.34)(27.604,11.000){2}{\rule{1.300pt}{0.800pt}}
\multiput(360.00,450.40)(1.768,0.514){13}{\rule{2.840pt}{0.124pt}}
\multiput(360.00,447.34)(27.105,10.000){2}{\rule{1.420pt}{0.800pt}}
\multiput(393.00,460.40)(1.936,0.516){11}{\rule{3.044pt}{0.124pt}}
\multiput(393.00,457.34)(25.681,9.000){2}{\rule{1.522pt}{0.800pt}}
\multiput(425.00,469.40)(2.752,0.526){7}{\rule{3.971pt}{0.127pt}}
\multiput(425.00,466.34)(24.757,7.000){2}{\rule{1.986pt}{0.800pt}}
\multiput(458.00,476.40)(2.752,0.526){7}{\rule{3.971pt}{0.127pt}}
\multiput(458.00,473.34)(24.757,7.000){2}{\rule{1.986pt}{0.800pt}}
\multiput(491.00,483.39)(3.365,0.536){5}{\rule{4.467pt}{0.129pt}}
\multiput(491.00,480.34)(22.729,6.000){2}{\rule{2.233pt}{0.800pt}}
\multiput(523.00,489.39)(3.476,0.536){5}{\rule{4.600pt}{0.129pt}}
\multiput(523.00,486.34)(23.452,6.000){2}{\rule{2.300pt}{0.800pt}}
\multiput(556.00,495.39)(3.476,0.536){5}{\rule{4.600pt}{0.129pt}}
\multiput(556.00,492.34)(23.452,6.000){2}{\rule{2.300pt}{0.800pt}}
\multiput(589.00,501.38)(5.126,0.560){3}{\rule{5.480pt}{0.135pt}}
\multiput(589.00,498.34)(21.626,5.000){2}{\rule{2.740pt}{0.800pt}}
\put(622,505.34){\rule{6.600pt}{0.800pt}}
\multiput(622.00,503.34)(18.301,4.000){2}{\rule{3.300pt}{0.800pt}}
\multiput(654.00,510.38)(5.126,0.560){3}{\rule{5.480pt}{0.135pt}}
\multiput(654.00,507.34)(21.626,5.000){2}{\rule{2.740pt}{0.800pt}}
\put(687,514.34){\rule{6.800pt}{0.800pt}}
\multiput(687.00,512.34)(18.886,4.000){2}{\rule{3.400pt}{0.800pt}}
\put(720,518.34){\rule{6.600pt}{0.800pt}}
\multiput(720.00,516.34)(18.301,4.000){2}{\rule{3.300pt}{0.800pt}}
\put(752,521.84){\rule{7.950pt}{0.800pt}}
\multiput(752.00,520.34)(16.500,3.000){2}{\rule{3.975pt}{0.800pt}}
\put(785,525.34){\rule{6.800pt}{0.800pt}}
\multiput(785.00,523.34)(18.886,4.000){2}{\rule{3.400pt}{0.800pt}}
\put(818,528.84){\rule{7.709pt}{0.800pt}}
\multiput(818.00,527.34)(16.000,3.000){2}{\rule{3.854pt}{0.800pt}}
\put(850,532.34){\rule{6.800pt}{0.800pt}}
\multiput(850.00,530.34)(18.886,4.000){2}{\rule{3.400pt}{0.800pt}}
\put(883,535.84){\rule{7.950pt}{0.800pt}}
\multiput(883.00,534.34)(16.500,3.000){2}{\rule{3.975pt}{0.800pt}}
\put(916,538.84){\rule{7.950pt}{0.800pt}}
\multiput(916.00,537.34)(16.500,3.000){2}{\rule{3.975pt}{0.800pt}}
\put(949,541.34){\rule{7.709pt}{0.800pt}}
\multiput(949.00,540.34)(16.000,2.000){2}{\rule{3.854pt}{0.800pt}}
\put(981,543.84){\rule{7.950pt}{0.800pt}}
\multiput(981.00,542.34)(16.500,3.000){2}{\rule{3.975pt}{0.800pt}}
\put(1014,546.84){\rule{7.950pt}{0.800pt}}
\multiput(1014.00,545.34)(16.500,3.000){2}{\rule{3.975pt}{0.800pt}}
\put(1047,549.34){\rule{7.709pt}{0.800pt}}
\multiput(1047.00,548.34)(16.000,2.000){2}{\rule{3.854pt}{0.800pt}}
\put(1079,551.84){\rule{7.950pt}{0.800pt}}
\multiput(1079.00,550.34)(16.500,3.000){2}{\rule{3.975pt}{0.800pt}}
\put(1112,554.34){\rule{7.950pt}{0.800pt}}
\multiput(1112.00,553.34)(16.500,2.000){2}{\rule{3.975pt}{0.800pt}}
\put(1145,556.84){\rule{7.709pt}{0.800pt}}
\multiput(1145.00,555.34)(16.000,3.000){2}{\rule{3.854pt}{0.800pt}}
\put(1177,559.34){\rule{7.950pt}{0.800pt}}
\multiput(1177.00,558.34)(16.500,2.000){2}{\rule{3.975pt}{0.800pt}}
\put(1210,561.34){\rule{7.950pt}{0.800pt}}
\multiput(1210.00,560.34)(16.500,2.000){2}{\rule{3.975pt}{0.800pt}}
\put(1243,563.34){\rule{7.950pt}{0.800pt}}
\multiput(1243.00,562.34)(16.500,2.000){2}{\rule{3.975pt}{0.800pt}}
\put(1276,565.34){\rule{7.709pt}{0.800pt}}
\multiput(1276.00,564.34)(16.000,2.000){2}{\rule{3.854pt}{0.800pt}}
\put(1308,567.34){\rule{7.950pt}{0.800pt}}
\multiput(1308.00,566.34)(16.500,2.000){2}{\rule{3.975pt}{0.800pt}}
\put(1341,569.34){\rule{7.950pt}{0.800pt}}
\multiput(1341.00,568.34)(16.500,2.000){2}{\rule{3.975pt}{0.800pt}}
\put(1374,571.34){\rule{7.709pt}{0.800pt}}
\multiput(1374.00,570.34)(16.000,2.000){2}{\rule{3.854pt}{0.800pt}}
\put(1406,573.34){\rule{7.950pt}{0.800pt}}
\multiput(1406.00,572.34)(16.500,2.000){2}{\rule{3.975pt}{0.800pt}}
\sbox{\plotpoint}{\rule[-0.200pt]{0.400pt}{0.400pt}}%
\put(1279,769){\makebox(0,0)[r]{实际数据}}
\put(1299.0,778.0){\rule[-0.200pt]{24.090pt}{0.400pt}}
\put(164,260){\usebox{\plotpoint}}
\multiput(164.58,260.00)(0.497,4.449){61}{\rule{0.120pt}{3.625pt}}
\multiput(163.17,260.00)(32.000,274.476){2}{\rule{0.400pt}{1.813pt}}
\multiput(196.58,533.03)(0.497,-2.597){63}{\rule{0.120pt}{2.161pt}}
\multiput(195.17,537.52)(33.000,-165.516){2}{\rule{0.400pt}{1.080pt}}
\multiput(229.58,372.00)(0.497,0.514){63}{\rule{0.120pt}{0.512pt}}
\multiput(228.17,372.00)(33.000,32.937){2}{\rule{0.400pt}{0.256pt}}
\multiput(262.58,402.52)(0.497,-0.928){63}{\rule{0.120pt}{0.839pt}}
\multiput(261.17,404.26)(33.000,-59.258){2}{\rule{0.400pt}{0.420pt}}
\multiput(295.00,345.58)(1.009,0.494){29}{\rule{0.900pt}{0.119pt}}
\multiput(295.00,344.17)(30.132,16.000){2}{\rule{0.450pt}{0.400pt}}
\multiput(327.58,361.00)(0.497,2.919){63}{\rule{0.120pt}{2.415pt}}
\multiput(326.17,361.00)(33.000,185.987){2}{\rule{0.400pt}{1.208pt}}
\multiput(360.58,543.94)(0.497,-2.322){63}{\rule{0.120pt}{1.942pt}}
\multiput(359.17,547.97)(33.000,-147.968){2}{\rule{0.400pt}{0.971pt}}
\multiput(393.58,400.00)(0.497,0.736){61}{\rule{0.120pt}{0.688pt}}
\multiput(392.17,400.00)(32.000,45.573){2}{\rule{0.400pt}{0.344pt}}
\multiput(425.00,447.58)(0.661,0.497){47}{\rule{0.628pt}{0.120pt}}
\multiput(425.00,446.17)(31.697,25.000){2}{\rule{0.314pt}{0.400pt}}
\multiput(458.58,472.00)(0.497,0.820){63}{\rule{0.120pt}{0.755pt}}
\multiput(457.17,472.00)(33.000,52.434){2}{\rule{0.400pt}{0.377pt}}
\multiput(491.58,526.00)(0.497,1.542){61}{\rule{0.120pt}{1.325pt}}
\multiput(490.17,526.00)(32.000,95.250){2}{\rule{0.400pt}{0.663pt}}
\multiput(523.58,621.87)(0.497,-0.514){63}{\rule{0.120pt}{0.512pt}}
\multiput(522.17,622.94)(33.000,-32.937){2}{\rule{0.400pt}{0.256pt}}
\multiput(556.00,590.58)(0.752,0.496){41}{\rule{0.700pt}{0.120pt}}
\multiput(556.00,589.17)(31.547,22.000){2}{\rule{0.350pt}{0.400pt}}
\multiput(589.00,610.92)(0.549,-0.497){57}{\rule{0.540pt}{0.120pt}}
\multiput(589.00,611.17)(31.879,-30.000){2}{\rule{0.270pt}{0.400pt}}
\multiput(622.00,580.93)(2.841,-0.482){9}{\rule{2.233pt}{0.116pt}}
\multiput(622.00,581.17)(27.365,-6.000){2}{\rule{1.117pt}{0.400pt}}
\multiput(654.58,576.00)(0.497,0.652){63}{\rule{0.120pt}{0.621pt}}
\multiput(653.17,576.00)(33.000,41.711){2}{\rule{0.400pt}{0.311pt}}
\multiput(687.58,612.40)(0.497,-1.877){63}{\rule{0.120pt}{1.591pt}}
\multiput(686.17,615.70)(33.000,-119.698){2}{\rule{0.400pt}{0.795pt}}
\multiput(720.58,496.00)(0.497,0.752){61}{\rule{0.120pt}{0.700pt}}
\multiput(719.17,496.00)(32.000,46.547){2}{\rule{0.400pt}{0.350pt}}
\multiput(752.00,542.93)(2.476,-0.485){11}{\rule{1.986pt}{0.117pt}}
\multiput(752.00,543.17)(28.879,-7.000){2}{\rule{0.993pt}{0.400pt}}
\multiput(785.58,534.77)(0.497,-0.545){63}{\rule{0.120pt}{0.536pt}}
\multiput(784.17,535.89)(33.000,-34.887){2}{\rule{0.400pt}{0.268pt}}
\multiput(818.00,501.58)(1.009,0.494){29}{\rule{0.900pt}{0.119pt}}
\multiput(818.00,500.17)(30.132,16.000){2}{\rule{0.450pt}{0.400pt}}
\multiput(850.00,515.92)(0.589,-0.497){53}{\rule{0.571pt}{0.120pt}}
\multiput(850.00,516.17)(31.814,-28.000){2}{\rule{0.286pt}{0.400pt}}
\multiput(883.00,489.58)(1.290,0.493){23}{\rule{1.115pt}{0.119pt}}
\multiput(883.00,488.17)(30.685,13.000){2}{\rule{0.558pt}{0.400pt}}
\multiput(916.58,502.00)(0.497,0.912){63}{\rule{0.120pt}{0.827pt}}
\multiput(915.17,502.00)(33.000,58.283){2}{\rule{0.400pt}{0.414pt}}
\multiput(949.58,562.00)(0.497,1.036){61}{\rule{0.120pt}{0.925pt}}
\multiput(948.17,562.00)(32.000,64.080){2}{\rule{0.400pt}{0.463pt}}
\multiput(981.00,626.92)(0.549,-0.497){57}{\rule{0.540pt}{0.120pt}}
\multiput(981.00,627.17)(31.879,-30.000){2}{\rule{0.270pt}{0.400pt}}
\multiput(1014.00,596.93)(2.476,-0.485){11}{\rule{1.986pt}{0.117pt}}
\multiput(1014.00,597.17)(28.879,-7.000){2}{\rule{0.993pt}{0.400pt}}
\multiput(1047.58,587.21)(0.497,-1.020){61}{\rule{0.120pt}{0.913pt}}
\multiput(1046.17,589.11)(32.000,-63.106){2}{\rule{0.400pt}{0.456pt}}
\multiput(1079.58,526.00)(0.497,1.663){63}{\rule{0.120pt}{1.421pt}}
\multiput(1078.17,526.00)(33.000,106.050){2}{\rule{0.400pt}{0.711pt}}
\multiput(1112.58,630.41)(0.497,-1.265){63}{\rule{0.120pt}{1.106pt}}
\multiput(1111.17,632.70)(33.000,-80.704){2}{\rule{0.400pt}{0.553pt}}
\multiput(1145.58,552.00)(0.497,0.515){61}{\rule{0.120pt}{0.512pt}}
\multiput(1144.17,552.00)(32.000,31.936){2}{\rule{0.400pt}{0.256pt}}
\multiput(1177.58,585.00)(0.497,0.744){63}{\rule{0.120pt}{0.694pt}}
\multiput(1176.17,585.00)(33.000,47.560){2}{\rule{0.400pt}{0.347pt}}
\multiput(1210.58,627.45)(0.497,-1.862){63}{\rule{0.120pt}{1.579pt}}
\multiput(1209.17,630.72)(33.000,-118.723){2}{\rule{0.400pt}{0.789pt}}
\multiput(1243.58,512.00)(0.497,1.617){63}{\rule{0.120pt}{1.385pt}}
\multiput(1242.17,512.00)(33.000,103.126){2}{\rule{0.400pt}{0.692pt}}
\multiput(1276.58,614.06)(0.497,-1.068){61}{\rule{0.120pt}{0.950pt}}
\multiput(1275.17,616.03)(32.000,-66.028){2}{\rule{0.400pt}{0.475pt}}
\multiput(1308.00,550.59)(3.604,0.477){7}{\rule{2.740pt}{0.115pt}}
\multiput(1308.00,549.17)(27.313,5.000){2}{\rule{1.370pt}{0.400pt}}
\multiput(1341.58,555.00)(0.497,1.586){63}{\rule{0.120pt}{1.361pt}}
\multiput(1340.17,555.00)(33.000,101.176){2}{\rule{0.400pt}{0.680pt}}
\multiput(1374.00,657.92)(0.551,-0.497){55}{\rule{0.541pt}{0.120pt}}
\multiput(1374.00,658.17)(30.876,-29.000){2}{\rule{0.271pt}{0.400pt}}
\multiput(1406.00,628.92)(1.290,-0.493){23}{\rule{1.115pt}{0.119pt}}
\multiput(1406.00,629.17)(30.685,-13.000){2}{\rule{0.558pt}{0.400pt}}
\put(164,260){\makebox(0,0){$+$}}
\put(196,542){\makebox(0,0){$+$}}
\put(229,372){\makebox(0,0){$+$}}
\put(262,406){\makebox(0,0){$+$}}
\put(295,345){\makebox(0,0){$+$}}
\put(327,361){\makebox(0,0){$+$}}
\put(360,552){\makebox(0,0){$+$}}
\put(393,400){\makebox(0,0){$+$}}
\put(425,447){\makebox(0,0){$+$}}
\put(458,472){\makebox(0,0){$+$}}
\put(491,526){\makebox(0,0){$+$}}
\put(523,624){\makebox(0,0){$+$}}
\put(556,590){\makebox(0,0){$+$}}
\put(589,612){\makebox(0,0){$+$}}
\put(622,582){\makebox(0,0){$+$}}
\put(654,576){\makebox(0,0){$+$}}
\put(687,619){\makebox(0,0){$+$}}
\put(720,496){\makebox(0,0){$+$}}
\put(752,544){\makebox(0,0){$+$}}
\put(785,537){\makebox(0,0){$+$}}
\put(818,501){\makebox(0,0){$+$}}
\put(850,517){\makebox(0,0){$+$}}
\put(883,489){\makebox(0,0){$+$}}
\put(916,502){\makebox(0,0){$+$}}
\put(949,562){\makebox(0,0){$+$}}
\put(981,628){\makebox(0,0){$+$}}
\put(1014,598){\makebox(0,0){$+$}}
\put(1047,591){\makebox(0,0){$+$}}
\put(1079,526){\makebox(0,0){$+$}}
\put(1112,635){\makebox(0,0){$+$}}
\put(1145,552){\makebox(0,0){$+$}}
\put(1177,585){\makebox(0,0){$+$}}
\put(1210,634){\makebox(0,0){$+$}}
\put(1243,512){\makebox(0,0){$+$}}
\put(1276,618){\makebox(0,0){$+$}}
\put(1308,550){\makebox(0,0){$+$}}
\put(1341,555){\makebox(0,0){$+$}}
\put(1374,659){\makebox(0,0){$+$}}
\put(1406,630){\makebox(0,0){$+$}}
\put(1439,617){\makebox(0,0){$+$}}
\put(1349,778){\makebox(0,0){$+$}}
\put(131.0,131.0){\rule[-0.200pt]{0.400pt}{175.375pt}}
\put(131.0,131.0){\rule[-0.200pt]{315.097pt}{0.400pt}}
\put(1439.0,131.0){\rule[-0.200pt]{0.400pt}{175.375pt}}
\put(131.0,859.0){\rule[-0.200pt]{315.097pt}{0.400pt}}
\end{picture}
} 
   \end{frame}

   \begin{frame}{历史检验}
     领域专家用户
\scalebox{0.8}{% GNUPLOT: LaTeX picture
\setlength{\unitlength}{0.240900pt}
\ifx\plotpoint\undefined\newsavebox{\plotpoint}\fi
\begin{picture}(1500,900)(0,0)
\sbox{\plotpoint}{\rule[-0.200pt]{0.400pt}{0.400pt}}%
\put(131.0,82.0){\rule[-0.200pt]{4.818pt}{0.400pt}}
\put(111,82){\makebox(0,0)[r]{ 0}}
\put(1419.0,82.0){\rule[-0.200pt]{4.818pt}{0.400pt}}
\put(131.0,341.0){\rule[-0.200pt]{4.818pt}{0.400pt}}
\put(111,341){\makebox(0,0)[r]{ 1}}
\put(1419.0,341.0){\rule[-0.200pt]{4.818pt}{0.400pt}}
\put(131.0,600.0){\rule[-0.200pt]{4.818pt}{0.400pt}}
\put(111,600){\makebox(0,0)[r]{ 2}}
\put(1419.0,600.0){\rule[-0.200pt]{4.818pt}{0.400pt}}
\put(131.0,859.0){\rule[-0.200pt]{4.818pt}{0.400pt}}
\put(111,859){\makebox(0,0)[r]{ 3}}
\put(1419.0,859.0){\rule[-0.200pt]{4.818pt}{0.400pt}}
\put(131.0,82.0){\rule[-0.200pt]{0.400pt}{4.818pt}}
\put(131,41){\makebox(0,0){ 1}}
\put(131.0,839.0){\rule[-0.200pt]{0.400pt}{4.818pt}}
\put(198.0,82.0){\rule[-0.200pt]{0.400pt}{4.818pt}}
\put(198,41){\makebox(0,0){ 3}}
\put(198.0,839.0){\rule[-0.200pt]{0.400pt}{4.818pt}}
\put(265.0,82.0){\rule[-0.200pt]{0.400pt}{4.818pt}}
\put(265,41){\makebox(0,0){ 5}}
\put(265.0,839.0){\rule[-0.200pt]{0.400pt}{4.818pt}}
\put(332.0,82.0){\rule[-0.200pt]{0.400pt}{4.818pt}}
\put(332,41){\makebox(0,0){ 7}}
\put(332.0,839.0){\rule[-0.200pt]{0.400pt}{4.818pt}}
\put(399.0,82.0){\rule[-0.200pt]{0.400pt}{4.818pt}}
\put(399,41){\makebox(0,0){ 9}}
\put(399.0,839.0){\rule[-0.200pt]{0.400pt}{4.818pt}}
\put(466.0,82.0){\rule[-0.200pt]{0.400pt}{4.818pt}}
\put(466,41){\makebox(0,0){ 11}}
\put(466.0,839.0){\rule[-0.200pt]{0.400pt}{4.818pt}}
\put(533.0,82.0){\rule[-0.200pt]{0.400pt}{4.818pt}}
\put(533,41){\makebox(0,0){ 13}}
\put(533.0,839.0){\rule[-0.200pt]{0.400pt}{4.818pt}}
\put(601.0,82.0){\rule[-0.200pt]{0.400pt}{4.818pt}}
\put(601,41){\makebox(0,0){ 15}}
\put(601.0,839.0){\rule[-0.200pt]{0.400pt}{4.818pt}}
\put(668.0,82.0){\rule[-0.200pt]{0.400pt}{4.818pt}}
\put(668,41){\makebox(0,0){ 17}}
\put(668.0,839.0){\rule[-0.200pt]{0.400pt}{4.818pt}}
\put(735.0,82.0){\rule[-0.200pt]{0.400pt}{4.818pt}}
\put(735,41){\makebox(0,0){ 19}}
\put(735.0,839.0){\rule[-0.200pt]{0.400pt}{4.818pt}}
\put(802.0,82.0){\rule[-0.200pt]{0.400pt}{4.818pt}}
\put(802,41){\makebox(0,0){ 21}}
\put(802.0,839.0){\rule[-0.200pt]{0.400pt}{4.818pt}}
\put(869.0,82.0){\rule[-0.200pt]{0.400pt}{4.818pt}}
\put(869,41){\makebox(0,0){ 23}}
\put(869.0,839.0){\rule[-0.200pt]{0.400pt}{4.818pt}}
\put(936.0,82.0){\rule[-0.200pt]{0.400pt}{4.818pt}}
\put(936,41){\makebox(0,0){ 25}}
\put(936.0,839.0){\rule[-0.200pt]{0.400pt}{4.818pt}}
\put(1003.0,82.0){\rule[-0.200pt]{0.400pt}{4.818pt}}
\put(1003,41){\makebox(0,0){ 27}}
\put(1003.0,839.0){\rule[-0.200pt]{0.400pt}{4.818pt}}
\put(1070.0,82.0){\rule[-0.200pt]{0.400pt}{4.818pt}}
\put(1070,41){\makebox(0,0){ 29}}
\put(1070.0,839.0){\rule[-0.200pt]{0.400pt}{4.818pt}}
\put(1137.0,82.0){\rule[-0.200pt]{0.400pt}{4.818pt}}
\put(1137,41){\makebox(0,0){ 31}}
\put(1137.0,839.0){\rule[-0.200pt]{0.400pt}{4.818pt}}
\put(1204.0,82.0){\rule[-0.200pt]{0.400pt}{4.818pt}}
\put(1204,41){\makebox(0,0){ 33}}
\put(1204.0,839.0){\rule[-0.200pt]{0.400pt}{4.818pt}}
\put(1271.0,82.0){\rule[-0.200pt]{0.400pt}{4.818pt}}
\put(1271,41){\makebox(0,0){ 35}}
\put(1271.0,839.0){\rule[-0.200pt]{0.400pt}{4.818pt}}
\put(1338.0,82.0){\rule[-0.200pt]{0.400pt}{4.818pt}}
\put(1338,41){\makebox(0,0){ 37}}
\put(1338.0,839.0){\rule[-0.200pt]{0.400pt}{4.818pt}}
\put(1405.0,82.0){\rule[-0.200pt]{0.400pt}{4.818pt}}
\put(1405,41){\makebox(0,0){ 39}}
\put(1405.0,839.0){\rule[-0.200pt]{0.400pt}{4.818pt}}
\put(131.0,82.0){\rule[-0.200pt]{0.400pt}{187.179pt}}
\put(131.0,82.0){\rule[-0.200pt]{315.097pt}{0.400pt}}
\put(1439.0,82.0){\rule[-0.200pt]{0.400pt}{187.179pt}}
\put(131.0,859.0){\rule[-0.200pt]{315.097pt}{0.400pt}}
\put(30,470){\makebox(0,0){\rotatebox{90}{用户贡献}}}
\put(1279,819){\makebox(0,0)[r]{仿真数据}}
\put(1299.0,819.0){\rule[-0.200pt]{24.090pt}{0.400pt}}
\put(131,460){\usebox{\plotpoint}}
\multiput(131.00,458.92)(0.741,-0.496){43}{\rule{0.691pt}{0.120pt}}
\multiput(131.00,459.17)(32.565,-23.000){2}{\rule{0.346pt}{0.400pt}}
\multiput(165.00,435.92)(1.041,-0.494){29}{\rule{0.925pt}{0.119pt}}
\multiput(165.00,436.17)(31.080,-16.000){2}{\rule{0.463pt}{0.400pt}}
\multiput(198.00,419.93)(2.552,-0.485){11}{\rule{2.043pt}{0.117pt}}
\multiput(198.00,420.17)(29.760,-7.000){2}{\rule{1.021pt}{0.400pt}}
\multiput(232.00,412.93)(2.145,-0.488){13}{\rule{1.750pt}{0.117pt}}
\multiput(232.00,413.17)(29.368,-8.000){2}{\rule{0.875pt}{0.400pt}}
\multiput(265.00,404.93)(2.211,-0.488){13}{\rule{1.800pt}{0.117pt}}
\multiput(265.00,405.17)(30.264,-8.000){2}{\rule{0.900pt}{0.400pt}}
\multiput(299.00,396.93)(3.604,-0.477){7}{\rule{2.740pt}{0.115pt}}
\multiput(299.00,397.17)(27.313,-5.000){2}{\rule{1.370pt}{0.400pt}}
\multiput(332.00,391.95)(7.383,-0.447){3}{\rule{4.633pt}{0.108pt}}
\multiput(332.00,392.17)(24.383,-3.000){2}{\rule{2.317pt}{0.400pt}}
\multiput(366.00,388.93)(3.604,-0.477){7}{\rule{2.740pt}{0.115pt}}
\multiput(366.00,389.17)(27.313,-5.000){2}{\rule{1.370pt}{0.400pt}}
\multiput(399.00,383.95)(7.383,-0.447){3}{\rule{4.633pt}{0.108pt}}
\multiput(399.00,384.17)(24.383,-3.000){2}{\rule{2.317pt}{0.400pt}}
\put(433,380.17){\rule{6.700pt}{0.400pt}}
\multiput(433.00,381.17)(19.094,-2.000){2}{\rule{3.350pt}{0.400pt}}
\multiput(466.00,378.95)(7.383,-0.447){3}{\rule{4.633pt}{0.108pt}}
\multiput(466.00,379.17)(24.383,-3.000){2}{\rule{2.317pt}{0.400pt}}
\put(500,375.17){\rule{6.700pt}{0.400pt}}
\multiput(500.00,376.17)(19.094,-2.000){2}{\rule{3.350pt}{0.400pt}}
\multiput(533.00,373.95)(7.383,-0.447){3}{\rule{4.633pt}{0.108pt}}
\multiput(533.00,374.17)(24.383,-3.000){2}{\rule{2.317pt}{0.400pt}}
\multiput(567.00,370.95)(7.383,-0.447){3}{\rule{4.633pt}{0.108pt}}
\multiput(567.00,371.17)(24.383,-3.000){2}{\rule{2.317pt}{0.400pt}}
\put(601,367.17){\rule{6.700pt}{0.400pt}}
\multiput(601.00,368.17)(19.094,-2.000){2}{\rule{3.350pt}{0.400pt}}
\multiput(634.00,365.95)(7.383,-0.447){3}{\rule{4.633pt}{0.108pt}}
\multiput(634.00,366.17)(24.383,-3.000){2}{\rule{2.317pt}{0.400pt}}
\put(668,362.17){\rule{6.700pt}{0.400pt}}
\multiput(668.00,363.17)(19.094,-2.000){2}{\rule{3.350pt}{0.400pt}}
\multiput(735.00,360.95)(7.160,-0.447){3}{\rule{4.500pt}{0.108pt}}
\multiput(735.00,361.17)(23.660,-3.000){2}{\rule{2.250pt}{0.400pt}}
\put(768,357.17){\rule{6.900pt}{0.400pt}}
\multiput(768.00,358.17)(19.679,-2.000){2}{\rule{3.450pt}{0.400pt}}
\put(701.0,362.0){\rule[-0.200pt]{8.191pt}{0.400pt}}
\multiput(835.00,355.95)(7.383,-0.447){3}{\rule{4.633pt}{0.108pt}}
\multiput(835.00,356.17)(24.383,-3.000){2}{\rule{2.317pt}{0.400pt}}
\put(802.0,357.0){\rule[-0.200pt]{7.950pt}{0.400pt}}
\multiput(902.00,352.95)(7.383,-0.447){3}{\rule{4.633pt}{0.108pt}}
\multiput(902.00,353.17)(24.383,-3.000){2}{\rule{2.317pt}{0.400pt}}
\put(869.0,354.0){\rule[-0.200pt]{7.950pt}{0.400pt}}
\put(969,349.17){\rule{6.900pt}{0.400pt}}
\multiput(969.00,350.17)(19.679,-2.000){2}{\rule{3.450pt}{0.400pt}}
\put(936.0,351.0){\rule[-0.200pt]{7.950pt}{0.400pt}}
\multiput(1037.00,347.95)(7.160,-0.447){3}{\rule{4.500pt}{0.108pt}}
\multiput(1037.00,348.17)(23.660,-3.000){2}{\rule{2.250pt}{0.400pt}}
\put(1003.0,349.0){\rule[-0.200pt]{8.191pt}{0.400pt}}
\put(1104,344.17){\rule{6.700pt}{0.400pt}}
\multiput(1104.00,345.17)(19.094,-2.000){2}{\rule{3.350pt}{0.400pt}}
\put(1070.0,346.0){\rule[-0.200pt]{8.191pt}{0.400pt}}
\multiput(1171.00,342.95)(7.160,-0.447){3}{\rule{4.500pt}{0.108pt}}
\multiput(1171.00,343.17)(23.660,-3.000){2}{\rule{2.250pt}{0.400pt}}
\put(1137.0,344.0){\rule[-0.200pt]{8.191pt}{0.400pt}}
\multiput(1271.00,339.95)(7.383,-0.447){3}{\rule{4.633pt}{0.108pt}}
\multiput(1271.00,340.17)(24.383,-3.000){2}{\rule{2.317pt}{0.400pt}}
\put(1204.0,341.0){\rule[-0.200pt]{16.140pt}{0.400pt}}
\put(1372,336.17){\rule{6.700pt}{0.400pt}}
\multiput(1372.00,337.17)(19.094,-2.000){2}{\rule{3.350pt}{0.400pt}}
\put(1305.0,338.0){\rule[-0.200pt]{16.140pt}{0.400pt}}
\put(1405.0,336.0){\rule[-0.200pt]{8.191pt}{0.400pt}}
\put(1279,768){\makebox(0,0)[r]{真实数据}}
\put(1299.0,768.0){\rule[-0.200pt]{24.090pt}{0.400pt}}
\put(131,634){\usebox{\plotpoint}}
\multiput(131.58,621.91)(0.498,-3.546){65}{\rule{0.120pt}{2.912pt}}
\multiput(130.17,627.96)(34.000,-232.956){2}{\rule{0.400pt}{1.456pt}}
\multiput(165.00,393.93)(3.604,-0.477){7}{\rule{2.740pt}{0.115pt}}
\multiput(165.00,394.17)(27.313,-5.000){2}{\rule{1.370pt}{0.400pt}}
\multiput(232.58,386.21)(0.497,-1.020){63}{\rule{0.120pt}{0.912pt}}
\multiput(231.17,388.11)(33.000,-65.107){2}{\rule{0.400pt}{0.456pt}}
\multiput(265.58,323.00)(0.498,2.461){65}{\rule{0.120pt}{2.053pt}}
\multiput(264.17,323.00)(34.000,161.739){2}{\rule{0.400pt}{1.026pt}}
\multiput(299.58,482.19)(0.497,-1.939){63}{\rule{0.120pt}{1.639pt}}
\multiput(298.17,485.60)(33.000,-123.597){2}{\rule{0.400pt}{0.820pt}}
\put(198.0,390.0){\rule[-0.200pt]{8.191pt}{0.400pt}}
\multiput(366.58,358.16)(0.497,-1.035){63}{\rule{0.120pt}{0.924pt}}
\multiput(365.17,360.08)(33.000,-66.082){2}{\rule{0.400pt}{0.462pt}}
\multiput(399.58,294.00)(0.498,0.647){65}{\rule{0.120pt}{0.618pt}}
\multiput(398.17,294.00)(34.000,42.718){2}{\rule{0.400pt}{0.309pt}}
\put(332.0,362.0){\rule[-0.200pt]{8.191pt}{0.400pt}}
\multiput(466.00,336.92)(0.855,-0.496){37}{\rule{0.780pt}{0.119pt}}
\multiput(466.00,337.17)(32.381,-20.000){2}{\rule{0.390pt}{0.400pt}}
\multiput(500.58,318.00)(0.497,1.173){63}{\rule{0.120pt}{1.033pt}}
\multiput(499.17,318.00)(33.000,74.855){2}{\rule{0.400pt}{0.517pt}}
\multiput(533.58,390.29)(0.498,-1.301){65}{\rule{0.120pt}{1.135pt}}
\multiput(532.17,392.64)(34.000,-85.644){2}{\rule{0.400pt}{0.568pt}}
\multiput(567.58,307.00)(0.498,1.851){65}{\rule{0.120pt}{1.571pt}}
\multiput(566.17,307.00)(34.000,121.740){2}{\rule{0.400pt}{0.785pt}}
\multiput(601.58,426.50)(0.497,-1.540){63}{\rule{0.120pt}{1.324pt}}
\multiput(600.17,429.25)(33.000,-98.251){2}{\rule{0.400pt}{0.662pt}}
\multiput(634.58,331.00)(0.498,0.647){65}{\rule{0.120pt}{0.618pt}}
\multiput(633.17,331.00)(34.000,42.718){2}{\rule{0.400pt}{0.309pt}}
\multiput(668.00,375.58)(0.499,0.497){63}{\rule{0.500pt}{0.120pt}}
\multiput(668.00,374.17)(31.962,33.000){2}{\rule{0.250pt}{0.400pt}}
\multiput(701.58,403.29)(0.498,-1.301){65}{\rule{0.120pt}{1.135pt}}
\multiput(700.17,405.64)(34.000,-85.644){2}{\rule{0.400pt}{0.568pt}}
\multiput(735.58,320.00)(0.497,0.560){63}{\rule{0.120pt}{0.548pt}}
\multiput(734.17,320.00)(33.000,35.862){2}{\rule{0.400pt}{0.274pt}}
\multiput(768.00,355.92)(1.329,-0.493){23}{\rule{1.146pt}{0.119pt}}
\multiput(768.00,356.17)(31.621,-13.000){2}{\rule{0.573pt}{0.400pt}}
\multiput(802.00,344.58)(1.113,0.494){27}{\rule{0.980pt}{0.119pt}}
\multiput(802.00,343.17)(30.966,15.000){2}{\rule{0.490pt}{0.400pt}}
\multiput(835.58,359.00)(0.498,1.078){65}{\rule{0.120pt}{0.959pt}}
\multiput(834.17,359.00)(34.000,71.010){2}{\rule{0.400pt}{0.479pt}}
\multiput(869.58,426.20)(0.497,-1.632){63}{\rule{0.120pt}{1.397pt}}
\multiput(868.17,429.10)(33.000,-104.101){2}{\rule{0.400pt}{0.698pt}}
\multiput(902.58,325.00)(0.498,0.930){65}{\rule{0.120pt}{0.841pt}}
\multiput(901.17,325.00)(34.000,61.254){2}{\rule{0.400pt}{0.421pt}}
\multiput(936.58,383.81)(0.497,-1.142){63}{\rule{0.120pt}{1.009pt}}
\multiput(935.17,385.91)(33.000,-72.906){2}{\rule{0.400pt}{0.505pt}}
\multiput(969.00,313.58)(1.746,0.491){17}{\rule{1.460pt}{0.118pt}}
\multiput(969.00,312.17)(30.970,10.000){2}{\rule{0.730pt}{0.400pt}}
\multiput(1003.58,323.00)(0.498,0.960){65}{\rule{0.120pt}{0.865pt}}
\multiput(1002.17,323.00)(34.000,63.205){2}{\rule{0.400pt}{0.432pt}}
\multiput(1037.58,384.16)(0.497,-1.035){63}{\rule{0.120pt}{0.924pt}}
\multiput(1036.17,386.08)(33.000,-66.082){2}{\rule{0.400pt}{0.462pt}}
\multiput(1070.00,320.58)(0.548,0.497){59}{\rule{0.539pt}{0.120pt}}
\multiput(1070.00,319.17)(32.882,31.000){2}{\rule{0.269pt}{0.400pt}}
\multiput(1104.00,351.58)(0.689,0.496){45}{\rule{0.650pt}{0.120pt}}
\multiput(1104.00,350.17)(31.651,24.000){2}{\rule{0.325pt}{0.400pt}}
\multiput(1137.58,371.41)(0.498,-0.960){65}{\rule{0.120pt}{0.865pt}}
\multiput(1136.17,373.21)(34.000,-63.205){2}{\rule{0.400pt}{0.432pt}}
\multiput(1171.00,310.58)(0.719,0.496){43}{\rule{0.674pt}{0.120pt}}
\multiput(1171.00,309.17)(31.601,23.000){2}{\rule{0.337pt}{0.400pt}}
\multiput(1204.00,333.58)(0.710,0.496){45}{\rule{0.667pt}{0.120pt}}
\multiput(1204.00,332.17)(32.616,24.000){2}{\rule{0.333pt}{0.400pt}}
\multiput(1238.00,355.92)(1.534,-0.492){19}{\rule{1.300pt}{0.118pt}}
\multiput(1238.00,356.17)(30.302,-11.000){2}{\rule{0.650pt}{0.400pt}}
\multiput(1271.58,346.00)(0.498,0.722){65}{\rule{0.120pt}{0.676pt}}
\multiput(1270.17,346.00)(34.000,47.596){2}{\rule{0.400pt}{0.338pt}}
\multiput(1305.00,393.92)(0.635,-0.497){49}{\rule{0.608pt}{0.120pt}}
\multiput(1305.00,394.17)(31.739,-26.000){2}{\rule{0.304pt}{0.400pt}}
\multiput(1338.00,369.59)(3.022,0.482){9}{\rule{2.367pt}{0.116pt}}
\multiput(1338.00,368.17)(29.088,6.000){2}{\rule{1.183pt}{0.400pt}}
\multiput(1372.58,372.72)(0.497,-0.560){63}{\rule{0.120pt}{0.548pt}}
\multiput(1371.17,373.86)(33.000,-35.862){2}{\rule{0.400pt}{0.274pt}}
\multiput(1405.58,338.00)(0.498,0.573){65}{\rule{0.120pt}{0.559pt}}
\multiput(1404.17,338.00)(34.000,37.840){2}{\rule{0.400pt}{0.279pt}}
\put(131,634){\makebox(0,0){$+$}}
\put(165,395){\makebox(0,0){$+$}}
\put(198,390){\makebox(0,0){$+$}}
\put(232,390){\makebox(0,0){$+$}}
\put(265,323){\makebox(0,0){$+$}}
\put(299,489){\makebox(0,0){$+$}}
\put(332,362){\makebox(0,0){$+$}}
\put(366,362){\makebox(0,0){$+$}}
\put(399,294){\makebox(0,0){$+$}}
\put(433,338){\makebox(0,0){$+$}}
\put(466,338){\makebox(0,0){$+$}}
\put(500,318){\makebox(0,0){$+$}}
\put(533,395){\makebox(0,0){$+$}}
\put(567,307){\makebox(0,0){$+$}}
\put(601,432){\makebox(0,0){$+$}}
\put(634,331){\makebox(0,0){$+$}}
\put(668,375){\makebox(0,0){$+$}}
\put(701,408){\makebox(0,0){$+$}}
\put(735,320){\makebox(0,0){$+$}}
\put(768,357){\makebox(0,0){$+$}}
\put(802,344){\makebox(0,0){$+$}}
\put(835,359){\makebox(0,0){$+$}}
\put(869,432){\makebox(0,0){$+$}}
\put(902,325){\makebox(0,0){$+$}}
\put(936,388){\makebox(0,0){$+$}}
\put(969,313){\makebox(0,0){$+$}}
\put(1003,323){\makebox(0,0){$+$}}
\put(1037,388){\makebox(0,0){$+$}}
\put(1070,320){\makebox(0,0){$+$}}
\put(1104,351){\makebox(0,0){$+$}}
\put(1137,375){\makebox(0,0){$+$}}
\put(1171,310){\makebox(0,0){$+$}}
\put(1204,333){\makebox(0,0){$+$}}
\put(1238,357){\makebox(0,0){$+$}}
\put(1271,346){\makebox(0,0){$+$}}
\put(1305,395){\makebox(0,0){$+$}}
\put(1338,369){\makebox(0,0){$+$}}
\put(1372,375){\makebox(0,0){$+$}}
\put(1405,338){\makebox(0,0){$+$}}
\put(1439,377){\makebox(0,0){$+$}}
\put(1349,778){\makebox(0,0){$+$}}
\put(433.0,338.0){\rule[-0.200pt]{7.950pt}{0.400pt}}
\put(131.0,82.0){\rule[-0.200pt]{0.400pt}{187.179pt}}
\put(131.0,82.0){\rule[-0.200pt]{315.097pt}{0.400pt}}
\put(1439.0,82.0){\rule[-0.200pt]{0.400pt}{187.179pt}}
\put(131.0,859.0){\rule[-0.200pt]{315.097pt}{0.400pt}}
\end{picture}
} 
   \end{frame}

   \begin{frame}{历史检验}
     内容贡献者用户
\scalebox{0.8}{% GNUPLOT: LaTeX picture
\setlength{\unitlength}{0.240900pt}
\ifx\plotpoint\undefined\newsavebox{\plotpoint}\fi
\begin{picture}(1500,900)(0,0)
\sbox{\plotpoint}{\rule[-0.200pt]{0.400pt}{0.400pt}}%
\put(131.0,82.0){\rule[-0.200pt]{4.818pt}{0.400pt}}
\put(111,82){\makebox(0,0)[r]{ 0}}
\put(1419.0,82.0){\rule[-0.200pt]{4.818pt}{0.400pt}}
\put(131.0,212.0){\rule[-0.200pt]{4.818pt}{0.400pt}}
\put(111,212){\makebox(0,0)[r]{ 1}}
\put(1419.0,212.0){\rule[-0.200pt]{4.818pt}{0.400pt}}
\put(131.0,341.0){\rule[-0.200pt]{4.818pt}{0.400pt}}
\put(111,341){\makebox(0,0)[r]{ 2}}
\put(1419.0,341.0){\rule[-0.200pt]{4.818pt}{0.400pt}}
\put(131.0,471.0){\rule[-0.200pt]{4.818pt}{0.400pt}}
\put(111,471){\makebox(0,0)[r]{ 3}}
\put(1419.0,471.0){\rule[-0.200pt]{4.818pt}{0.400pt}}
\put(131.0,600.0){\rule[-0.200pt]{4.818pt}{0.400pt}}
\put(111,600){\makebox(0,0)[r]{ 4}}
\put(1419.0,600.0){\rule[-0.200pt]{4.818pt}{0.400pt}}
\put(131.0,730.0){\rule[-0.200pt]{4.818pt}{0.400pt}}
\put(111,730){\makebox(0,0)[r]{ 5}}
\put(1419.0,730.0){\rule[-0.200pt]{4.818pt}{0.400pt}}
\put(131.0,859.0){\rule[-0.200pt]{4.818pt}{0.400pt}}
\put(111,859){\makebox(0,0)[r]{ 6}}
\put(1419.0,859.0){\rule[-0.200pt]{4.818pt}{0.400pt}}
\put(131.0,82.0){\rule[-0.200pt]{0.400pt}{4.818pt}}
\put(131,41){\makebox(0,0){ 1}}
\put(131.0,839.0){\rule[-0.200pt]{0.400pt}{4.818pt}}
\put(198.0,82.0){\rule[-0.200pt]{0.400pt}{4.818pt}}
\put(198,41){\makebox(0,0){ 3}}
\put(198.0,839.0){\rule[-0.200pt]{0.400pt}{4.818pt}}
\put(265.0,82.0){\rule[-0.200pt]{0.400pt}{4.818pt}}
\put(265,41){\makebox(0,0){ 5}}
\put(265.0,839.0){\rule[-0.200pt]{0.400pt}{4.818pt}}
\put(332.0,82.0){\rule[-0.200pt]{0.400pt}{4.818pt}}
\put(332,41){\makebox(0,0){ 7}}
\put(332.0,839.0){\rule[-0.200pt]{0.400pt}{4.818pt}}
\put(399.0,82.0){\rule[-0.200pt]{0.400pt}{4.818pt}}
\put(399,41){\makebox(0,0){ 9}}
\put(399.0,839.0){\rule[-0.200pt]{0.400pt}{4.818pt}}
\put(466.0,82.0){\rule[-0.200pt]{0.400pt}{4.818pt}}
\put(466,41){\makebox(0,0){ 11}}
\put(466.0,839.0){\rule[-0.200pt]{0.400pt}{4.818pt}}
\put(533.0,82.0){\rule[-0.200pt]{0.400pt}{4.818pt}}
\put(533,41){\makebox(0,0){ 13}}
\put(533.0,839.0){\rule[-0.200pt]{0.400pt}{4.818pt}}
\put(601.0,82.0){\rule[-0.200pt]{0.400pt}{4.818pt}}
\put(601,41){\makebox(0,0){ 15}}
\put(601.0,839.0){\rule[-0.200pt]{0.400pt}{4.818pt}}
\put(668.0,82.0){\rule[-0.200pt]{0.400pt}{4.818pt}}
\put(668,41){\makebox(0,0){ 17}}
\put(668.0,839.0){\rule[-0.200pt]{0.400pt}{4.818pt}}
\put(735.0,82.0){\rule[-0.200pt]{0.400pt}{4.818pt}}
\put(735,41){\makebox(0,0){ 19}}
\put(735.0,839.0){\rule[-0.200pt]{0.400pt}{4.818pt}}
\put(802.0,82.0){\rule[-0.200pt]{0.400pt}{4.818pt}}
\put(802,41){\makebox(0,0){ 21}}
\put(802.0,839.0){\rule[-0.200pt]{0.400pt}{4.818pt}}
\put(869.0,82.0){\rule[-0.200pt]{0.400pt}{4.818pt}}
\put(869,41){\makebox(0,0){ 23}}
\put(869.0,839.0){\rule[-0.200pt]{0.400pt}{4.818pt}}
\put(936.0,82.0){\rule[-0.200pt]{0.400pt}{4.818pt}}
\put(936,41){\makebox(0,0){ 25}}
\put(936.0,839.0){\rule[-0.200pt]{0.400pt}{4.818pt}}
\put(1003.0,82.0){\rule[-0.200pt]{0.400pt}{4.818pt}}
\put(1003,41){\makebox(0,0){ 27}}
\put(1003.0,839.0){\rule[-0.200pt]{0.400pt}{4.818pt}}
\put(1070.0,82.0){\rule[-0.200pt]{0.400pt}{4.818pt}}
\put(1070,41){\makebox(0,0){ 29}}
\put(1070.0,839.0){\rule[-0.200pt]{0.400pt}{4.818pt}}
\put(1137.0,82.0){\rule[-0.200pt]{0.400pt}{4.818pt}}
\put(1137,41){\makebox(0,0){ 31}}
\put(1137.0,839.0){\rule[-0.200pt]{0.400pt}{4.818pt}}
\put(1204.0,82.0){\rule[-0.200pt]{0.400pt}{4.818pt}}
\put(1204,41){\makebox(0,0){ 33}}
\put(1204.0,839.0){\rule[-0.200pt]{0.400pt}{4.818pt}}
\put(1271.0,82.0){\rule[-0.200pt]{0.400pt}{4.818pt}}
\put(1271,41){\makebox(0,0){ 35}}
\put(1271.0,839.0){\rule[-0.200pt]{0.400pt}{4.818pt}}
\put(1338.0,82.0){\rule[-0.200pt]{0.400pt}{4.818pt}}
\put(1338,41){\makebox(0,0){ 37}}
\put(1338.0,839.0){\rule[-0.200pt]{0.400pt}{4.818pt}}
\put(1405.0,82.0){\rule[-0.200pt]{0.400pt}{4.818pt}}
\put(1405,41){\makebox(0,0){ 39}}
\put(1405.0,839.0){\rule[-0.200pt]{0.400pt}{4.818pt}}
\put(131.0,82.0){\rule[-0.200pt]{0.400pt}{187.179pt}}
\put(131.0,82.0){\rule[-0.200pt]{315.097pt}{0.400pt}}
\put(1439.0,82.0){\rule[-0.200pt]{0.400pt}{187.179pt}}
\put(131.0,859.0){\rule[-0.200pt]{315.097pt}{0.400pt}}
\put(30,470){\makebox(0,0){\rotatebox{90}{用户贡献}}}
\put(1279,819){\makebox(0,0)[r]{仿真数据}}
\put(1299.0,819.0){\rule[-0.200pt]{24.090pt}{0.400pt}}
\put(1439,547){\usebox{\plotpoint}}
\multiput(1424.47,545.94)(-4.868,-0.468){5}{\rule{3.500pt}{0.113pt}}
\multiput(1431.74,546.17)(-26.736,-4.000){2}{\rule{1.750pt}{0.400pt}}
\multiput(1393.63,541.93)(-3.604,-0.477){7}{\rule{2.740pt}{0.115pt}}
\multiput(1399.31,542.17)(-27.313,-5.000){2}{\rule{1.370pt}{0.400pt}}
\multiput(1357.47,536.94)(-4.868,-0.468){5}{\rule{3.500pt}{0.113pt}}
\multiput(1364.74,537.17)(-26.736,-4.000){2}{\rule{1.750pt}{0.400pt}}
\multiput(1323.89,532.94)(-4.722,-0.468){5}{\rule{3.400pt}{0.113pt}}
\multiput(1330.94,533.17)(-25.943,-4.000){2}{\rule{1.700pt}{0.400pt}}
\multiput(1290.47,528.94)(-4.868,-0.468){5}{\rule{3.500pt}{0.113pt}}
\multiput(1297.74,529.17)(-26.736,-4.000){2}{\rule{1.750pt}{0.400pt}}
\multiput(1256.89,524.94)(-4.722,-0.468){5}{\rule{3.400pt}{0.113pt}}
\multiput(1263.94,525.17)(-25.943,-4.000){2}{\rule{1.700pt}{0.400pt}}
\multiput(1218.77,520.95)(-7.383,-0.447){3}{\rule{4.633pt}{0.108pt}}
\multiput(1228.38,521.17)(-24.383,-3.000){2}{\rule{2.317pt}{0.400pt}}
\multiput(1189.89,517.94)(-4.722,-0.468){5}{\rule{3.400pt}{0.113pt}}
\multiput(1196.94,518.17)(-25.943,-4.000){2}{\rule{1.700pt}{0.400pt}}
\multiput(1151.77,513.95)(-7.383,-0.447){3}{\rule{4.633pt}{0.108pt}}
\multiput(1161.38,514.17)(-24.383,-3.000){2}{\rule{2.317pt}{0.400pt}}
\multiput(1122.89,510.94)(-4.722,-0.468){5}{\rule{3.400pt}{0.113pt}}
\multiput(1129.94,511.17)(-25.943,-4.000){2}{\rule{1.700pt}{0.400pt}}
\multiput(1084.77,506.95)(-7.383,-0.447){3}{\rule{4.633pt}{0.108pt}}
\multiput(1094.38,507.17)(-24.383,-3.000){2}{\rule{2.317pt}{0.400pt}}
\multiput(1051.32,503.95)(-7.160,-0.447){3}{\rule{4.500pt}{0.108pt}}
\multiput(1060.66,504.17)(-23.660,-3.000){2}{\rule{2.250pt}{0.400pt}}
\multiput(1017.77,500.95)(-7.383,-0.447){3}{\rule{4.633pt}{0.108pt}}
\multiput(1027.38,501.17)(-24.383,-3.000){2}{\rule{2.317pt}{0.400pt}}
\multiput(983.77,497.95)(-7.383,-0.447){3}{\rule{4.633pt}{0.108pt}}
\multiput(993.38,498.17)(-24.383,-3.000){2}{\rule{2.317pt}{0.400pt}}
\multiput(950.32,494.95)(-7.160,-0.447){3}{\rule{4.500pt}{0.108pt}}
\multiput(959.66,495.17)(-23.660,-3.000){2}{\rule{2.250pt}{0.400pt}}
\multiput(916.77,491.95)(-7.383,-0.447){3}{\rule{4.633pt}{0.108pt}}
\multiput(926.38,492.17)(-24.383,-3.000){2}{\rule{2.317pt}{0.400pt}}
\multiput(883.32,488.95)(-7.160,-0.447){3}{\rule{4.500pt}{0.108pt}}
\multiput(892.66,489.17)(-23.660,-3.000){2}{\rule{2.250pt}{0.400pt}}
\put(835,485.17){\rule{6.900pt}{0.400pt}}
\multiput(854.68,486.17)(-19.679,-2.000){2}{\rule{3.450pt}{0.400pt}}
\multiput(816.32,483.95)(-7.160,-0.447){3}{\rule{4.500pt}{0.108pt}}
\multiput(825.66,484.17)(-23.660,-3.000){2}{\rule{2.250pt}{0.400pt}}
\put(768,480.17){\rule{6.900pt}{0.400pt}}
\multiput(787.68,481.17)(-19.679,-2.000){2}{\rule{3.450pt}{0.400pt}}
\multiput(749.32,478.95)(-7.160,-0.447){3}{\rule{4.500pt}{0.108pt}}
\multiput(758.66,479.17)(-23.660,-3.000){2}{\rule{2.250pt}{0.400pt}}
\put(701,475.17){\rule{6.900pt}{0.400pt}}
\multiput(720.68,476.17)(-19.679,-2.000){2}{\rule{3.450pt}{0.400pt}}
\put(668,473.17){\rule{6.700pt}{0.400pt}}
\multiput(687.09,474.17)(-19.094,-2.000){2}{\rule{3.350pt}{0.400pt}}
\put(634,471.17){\rule{6.900pt}{0.400pt}}
\multiput(653.68,472.17)(-19.679,-2.000){2}{\rule{3.450pt}{0.400pt}}
\put(601,469.17){\rule{6.700pt}{0.400pt}}
\multiput(620.09,470.17)(-19.094,-2.000){2}{\rule{3.350pt}{0.400pt}}
\put(567,467.67){\rule{8.191pt}{0.400pt}}
\multiput(584.00,468.17)(-17.000,-1.000){2}{\rule{4.095pt}{0.400pt}}
\put(533,466.17){\rule{6.900pt}{0.400pt}}
\multiput(552.68,467.17)(-19.679,-2.000){2}{\rule{3.450pt}{0.400pt}}
\put(500,464.17){\rule{6.700pt}{0.400pt}}
\multiput(519.09,465.17)(-19.094,-2.000){2}{\rule{3.350pt}{0.400pt}}
\put(466,462.67){\rule{8.191pt}{0.400pt}}
\multiput(483.00,463.17)(-17.000,-1.000){2}{\rule{4.095pt}{0.400pt}}
\put(433,461.67){\rule{7.950pt}{0.400pt}}
\multiput(449.50,462.17)(-16.500,-1.000){2}{\rule{3.975pt}{0.400pt}}
\put(399,460.17){\rule{6.900pt}{0.400pt}}
\multiput(418.68,461.17)(-19.679,-2.000){2}{\rule{3.450pt}{0.400pt}}
\put(366,458.67){\rule{7.950pt}{0.400pt}}
\multiput(382.50,459.17)(-16.500,-1.000){2}{\rule{3.975pt}{0.400pt}}
\put(332,457.67){\rule{8.191pt}{0.400pt}}
\multiput(349.00,458.17)(-17.000,-1.000){2}{\rule{4.095pt}{0.400pt}}
\put(299,456.67){\rule{7.950pt}{0.400pt}}
\multiput(315.50,457.17)(-16.500,-1.000){2}{\rule{3.975pt}{0.400pt}}
\put(265,455.67){\rule{8.191pt}{0.400pt}}
\multiput(282.00,456.17)(-17.000,-1.000){2}{\rule{4.095pt}{0.400pt}}
\put(198,454.67){\rule{8.191pt}{0.400pt}}
\multiput(215.00,455.17)(-17.000,-1.000){2}{\rule{4.095pt}{0.400pt}}
\put(232.0,456.0){\rule[-0.200pt]{7.950pt}{0.400pt}}
\put(131,453.67){\rule{8.191pt}{0.400pt}}
\multiput(148.00,454.17)(-17.000,-1.000){2}{\rule{4.095pt}{0.400pt}}
\put(165.0,455.0){\rule[-0.200pt]{7.950pt}{0.400pt}}
\put(1279,768){\makebox(0,0)[r]{实际数据}}
\put(1299.0,768.0){\rule[-0.200pt]{24.090pt}{0.400pt}}
\put(1439,581){\usebox{\plotpoint}}
\multiput(1434.55,581.58)(-1.231,0.494){25}{\rule{1.071pt}{0.119pt}}
\multiput(1436.78,580.17)(-31.776,14.000){2}{\rule{0.536pt}{0.400pt}}
\multiput(1396.76,593.93)(-2.476,-0.485){11}{\rule{1.986pt}{0.117pt}}
\multiput(1400.88,594.17)(-28.879,-7.000){2}{\rule{0.993pt}{0.400pt}}
\multiput(1366.88,586.92)(-1.444,-0.492){21}{\rule{1.233pt}{0.119pt}}
\multiput(1369.44,587.17)(-31.440,-12.000){2}{\rule{0.617pt}{0.400pt}}
\multiput(1319.32,576.61)(-7.160,0.447){3}{\rule{4.500pt}{0.108pt}}
\multiput(1328.66,575.17)(-23.660,3.000){2}{\rule{2.250pt}{0.400pt}}
\multiput(1300.55,577.92)(-1.231,-0.494){25}{\rule{1.071pt}{0.119pt}}
\multiput(1302.78,578.17)(-31.776,-14.000){2}{\rule{0.536pt}{0.400pt}}
\multiput(1269.92,565.00)(-0.497,1.035){63}{\rule{0.120pt}{0.924pt}}
\multiput(1270.17,565.00)(-33.000,66.082){2}{\rule{0.400pt}{0.462pt}}
\multiput(1236.92,633.00)(-0.498,0.856){65}{\rule{0.120pt}{0.782pt}}
\multiput(1237.17,633.00)(-34.000,56.376){2}{\rule{0.400pt}{0.391pt}}
\multiput(1201.39,689.92)(-0.661,-0.497){47}{\rule{0.628pt}{0.120pt}}
\multiput(1202.70,690.17)(-31.697,-25.000){2}{\rule{0.314pt}{0.400pt}}
\multiput(1169.92,658.70)(-0.498,-2.089){65}{\rule{0.120pt}{1.759pt}}
\multiput(1170.17,662.35)(-34.000,-137.349){2}{\rule{0.400pt}{0.879pt}}
\multiput(1135.92,522.47)(-0.497,-0.637){63}{\rule{0.120pt}{0.609pt}}
\multiput(1136.17,523.74)(-33.000,-40.736){2}{\rule{0.400pt}{0.305pt}}
\put(1070,481.67){\rule{8.191pt}{0.400pt}}
\multiput(1087.00,482.17)(-17.000,-1.000){2}{\rule{4.095pt}{0.400pt}}
\multiput(1062.74,482.59)(-2.145,0.488){13}{\rule{1.750pt}{0.117pt}}
\multiput(1066.37,481.17)(-29.368,8.000){2}{\rule{0.875pt}{0.400pt}}
\multiput(1035.92,490.00)(-0.498,0.514){65}{\rule{0.120pt}{0.512pt}}
\multiput(1036.17,490.00)(-34.000,33.938){2}{\rule{0.400pt}{0.256pt}}
\multiput(1000.02,523.92)(-0.775,-0.496){41}{\rule{0.718pt}{0.120pt}}
\multiput(1001.51,524.17)(-32.509,-22.000){2}{\rule{0.359pt}{0.400pt}}
\multiput(967.92,503.00)(-0.497,0.621){63}{\rule{0.120pt}{0.597pt}}
\multiput(968.17,503.00)(-33.000,39.761){2}{\rule{0.400pt}{0.298pt}}
\multiput(934.92,539.82)(-0.498,-1.138){65}{\rule{0.120pt}{1.006pt}}
\multiput(935.17,541.91)(-34.000,-74.912){2}{\rule{0.400pt}{0.503pt}}
\multiput(900.92,467.00)(-0.497,0.637){63}{\rule{0.120pt}{0.609pt}}
\multiput(901.17,467.00)(-33.000,40.736){2}{\rule{0.400pt}{0.305pt}}
\multiput(866.82,507.92)(-0.530,-0.497){61}{\rule{0.525pt}{0.120pt}}
\multiput(867.91,508.17)(-32.910,-32.000){2}{\rule{0.263pt}{0.400pt}}
\multiput(833.92,474.52)(-0.497,-0.621){63}{\rule{0.120pt}{0.597pt}}
\multiput(834.17,475.76)(-33.000,-39.761){2}{\rule{0.400pt}{0.298pt}}
\multiput(800.92,433.05)(-0.498,-0.766){65}{\rule{0.120pt}{0.712pt}}
\multiput(801.17,434.52)(-34.000,-50.523){2}{\rule{0.400pt}{0.356pt}}
\multiput(765.20,382.92)(-0.719,-0.496){43}{\rule{0.674pt}{0.120pt}}
\multiput(766.60,383.17)(-31.601,-23.000){2}{\rule{0.337pt}{0.400pt}}
\multiput(733.92,361.00)(-0.498,1.851){65}{\rule{0.120pt}{1.571pt}}
\multiput(734.17,361.00)(-34.000,121.740){2}{\rule{0.400pt}{0.785pt}}
\put(668,486.17){\rule{6.700pt}{0.400pt}}
\multiput(687.09,485.17)(-19.094,2.000){2}{\rule{3.350pt}{0.400pt}}
\multiput(665.41,488.58)(-0.654,0.497){49}{\rule{0.623pt}{0.120pt}}
\multiput(666.71,487.17)(-32.707,26.000){2}{\rule{0.312pt}{0.400pt}}
\multiput(615.32,512.95)(-7.160,-0.447){3}{\rule{4.500pt}{0.108pt}}
\multiput(624.66,513.17)(-23.660,-3.000){2}{\rule{2.250pt}{0.400pt}}
\multiput(589.29,509.93)(-3.716,-0.477){7}{\rule{2.820pt}{0.115pt}}
\multiput(595.15,510.17)(-28.147,-5.000){2}{\rule{1.410pt}{0.400pt}}
\multiput(565.92,503.39)(-0.498,-0.662){65}{\rule{0.120pt}{0.629pt}}
\multiput(566.17,504.69)(-34.000,-43.694){2}{\rule{0.400pt}{0.315pt}}
\multiput(523.45,461.59)(-2.932,0.482){9}{\rule{2.300pt}{0.116pt}}
\multiput(528.23,460.17)(-28.226,6.000){2}{\rule{1.150pt}{0.400pt}}
\multiput(496.76,467.58)(-0.855,0.496){37}{\rule{0.780pt}{0.119pt}}
\multiput(498.38,466.17)(-32.381,20.000){2}{\rule{0.390pt}{0.400pt}}
\multiput(463.56,485.92)(-0.611,-0.497){51}{\rule{0.589pt}{0.120pt}}
\multiput(464.78,486.17)(-31.778,-27.000){2}{\rule{0.294pt}{0.400pt}}
\multiput(429.61,458.92)(-0.900,-0.495){35}{\rule{0.816pt}{0.119pt}}
\multiput(431.31,459.17)(-32.307,-19.000){2}{\rule{0.408pt}{0.400pt}}
\multiput(380.32,441.61)(-7.160,0.447){3}{\rule{4.500pt}{0.108pt}}
\multiput(389.66,440.17)(-23.660,3.000){2}{\rule{2.250pt}{0.400pt}}
\multiput(317.89,442.94)(-4.722,-0.468){5}{\rule{3.400pt}{0.113pt}}
\multiput(324.94,443.17)(-25.943,-4.000){2}{\rule{1.700pt}{0.400pt}}
\multiput(287.29,438.93)(-3.716,-0.477){7}{\rule{2.820pt}{0.115pt}}
\multiput(293.15,439.17)(-28.147,-5.000){2}{\rule{1.410pt}{0.400pt}}
\multiput(261.98,435.58)(-0.789,0.496){39}{\rule{0.729pt}{0.119pt}}
\multiput(263.49,434.17)(-31.488,21.000){2}{\rule{0.364pt}{0.400pt}}
\multiput(212.77,454.95)(-7.383,-0.447){3}{\rule{4.633pt}{0.108pt}}
\multiput(222.38,455.17)(-24.383,-3.000){2}{\rule{2.317pt}{0.400pt}}
\multiput(196.92,453.00)(-0.497,0.912){63}{\rule{0.120pt}{0.827pt}}
\multiput(197.17,453.00)(-33.000,58.283){2}{\rule{0.400pt}{0.414pt}}
\put(332.0,444.0){\rule[-0.200pt]{8.191pt}{0.400pt}}
\put(1439,581){\makebox(0,0){$+$}}
\put(1405,595){\makebox(0,0){$+$}}
\put(1372,588){\makebox(0,0){$+$}}
\put(1338,576){\makebox(0,0){$+$}}
\put(1305,579){\makebox(0,0){$+$}}
\put(1271,565){\makebox(0,0){$+$}}
\put(1238,633){\makebox(0,0){$+$}}
\put(1204,691){\makebox(0,0){$+$}}
\put(1171,666){\makebox(0,0){$+$}}
\put(1137,525){\makebox(0,0){$+$}}
\put(1104,483){\makebox(0,0){$+$}}
\put(1070,482){\makebox(0,0){$+$}}
\put(1037,490){\makebox(0,0){$+$}}
\put(1003,525){\makebox(0,0){$+$}}
\put(969,503){\makebox(0,0){$+$}}
\put(936,544){\makebox(0,0){$+$}}
\put(902,467){\makebox(0,0){$+$}}
\put(869,509){\makebox(0,0){$+$}}
\put(835,477){\makebox(0,0){$+$}}
\put(802,436){\makebox(0,0){$+$}}
\put(768,384){\makebox(0,0){$+$}}
\put(735,361){\makebox(0,0){$+$}}
\put(701,486){\makebox(0,0){$+$}}
\put(668,488){\makebox(0,0){$+$}}
\put(634,514){\makebox(0,0){$+$}}
\put(601,511){\makebox(0,0){$+$}}
\put(567,506){\makebox(0,0){$+$}}
\put(533,461){\makebox(0,0){$+$}}
\put(500,467){\makebox(0,0){$+$}}
\put(466,487){\makebox(0,0){$+$}}
\put(433,460){\makebox(0,0){$+$}}
\put(399,441){\makebox(0,0){$+$}}
\put(366,444){\makebox(0,0){$+$}}
\put(332,444){\makebox(0,0){$+$}}
\put(299,440){\makebox(0,0){$+$}}
\put(265,435){\makebox(0,0){$+$}}
\put(232,456){\makebox(0,0){$+$}}
\put(198,453){\makebox(0,0){$+$}}
\put(165,513){\makebox(0,0){$+$}}
\put(131,513){\makebox(0,0){$+$}}
\put(1349,768){\makebox(0,0){$+$}}
\put(131.0,513.0){\rule[-0.200pt]{8.191pt}{0.400pt}}
\put(131.0,82.0){\rule[-0.200pt]{0.400pt}{187.179pt}}
\put(131.0,82.0){\rule[-0.200pt]{315.097pt}{0.400pt}}
\put(1439.0,82.0){\rule[-0.200pt]{0.400pt}{187.179pt}}
\put(131.0,859.0){\rule[-0.200pt]{315.097pt}{0.400pt}}
\end{picture}
} 
   \end{frame}

   \begin{frame}{历史检验}
     内容维护者用户
\scalebox{0.8}{% GNUPLOT: LaTeX picture
\setlength{\unitlength}{0.240900pt}
\ifx\plotpoint\undefined\newsavebox{\plotpoint}\fi
\begin{picture}(1500,900)(0,0)
\sbox{\plotpoint}{\rule[-0.200pt]{0.400pt}{0.400pt}}%
\put(131.0,82.0){\rule[-0.200pt]{4.818pt}{0.400pt}}
\put(111,82){\makebox(0,0)[r]{-1}}
\put(1419.0,82.0){\rule[-0.200pt]{4.818pt}{0.400pt}}
\put(131.0,237.0){\rule[-0.200pt]{4.818pt}{0.400pt}}
\put(111,237){\makebox(0,0)[r]{ 0}}
\put(1419.0,237.0){\rule[-0.200pt]{4.818pt}{0.400pt}}
\put(131.0,393.0){\rule[-0.200pt]{4.818pt}{0.400pt}}
\put(111,393){\makebox(0,0)[r]{ 1}}
\put(1419.0,393.0){\rule[-0.200pt]{4.818pt}{0.400pt}}
\put(131.0,548.0){\rule[-0.200pt]{4.818pt}{0.400pt}}
\put(111,548){\makebox(0,0)[r]{ 2}}
\put(1419.0,548.0){\rule[-0.200pt]{4.818pt}{0.400pt}}
\put(131.0,704.0){\rule[-0.200pt]{4.818pt}{0.400pt}}
\put(111,704){\makebox(0,0)[r]{ 3}}
\put(1419.0,704.0){\rule[-0.200pt]{4.818pt}{0.400pt}}
\put(131.0,82.0){\rule[-0.200pt]{0.400pt}{4.818pt}}
\put(131,41){\makebox(0,0){ 1}}
\put(131.0,839.0){\rule[-0.200pt]{0.400pt}{4.818pt}}
\put(198.0,82.0){\rule[-0.200pt]{0.400pt}{4.818pt}}
\put(198,41){\makebox(0,0){ 3}}
\put(198.0,839.0){\rule[-0.200pt]{0.400pt}{4.818pt}}
\put(265.0,82.0){\rule[-0.200pt]{0.400pt}{4.818pt}}
\put(265,41){\makebox(0,0){ 5}}
\put(265.0,839.0){\rule[-0.200pt]{0.400pt}{4.818pt}}
\put(332.0,82.0){\rule[-0.200pt]{0.400pt}{4.818pt}}
\put(332,41){\makebox(0,0){ 7}}
\put(332.0,839.0){\rule[-0.200pt]{0.400pt}{4.818pt}}
\put(399.0,82.0){\rule[-0.200pt]{0.400pt}{4.818pt}}
\put(399,41){\makebox(0,0){ 9}}
\put(399.0,839.0){\rule[-0.200pt]{0.400pt}{4.818pt}}
\put(466.0,82.0){\rule[-0.200pt]{0.400pt}{4.818pt}}
\put(466,41){\makebox(0,0){ 11}}
\put(466.0,839.0){\rule[-0.200pt]{0.400pt}{4.818pt}}
\put(533.0,82.0){\rule[-0.200pt]{0.400pt}{4.818pt}}
\put(533,41){\makebox(0,0){ 13}}
\put(533.0,839.0){\rule[-0.200pt]{0.400pt}{4.818pt}}
\put(601.0,82.0){\rule[-0.200pt]{0.400pt}{4.818pt}}
\put(601,41){\makebox(0,0){ 15}}
\put(601.0,839.0){\rule[-0.200pt]{0.400pt}{4.818pt}}
\put(668.0,82.0){\rule[-0.200pt]{0.400pt}{4.818pt}}
\put(668,41){\makebox(0,0){ 17}}
\put(668.0,839.0){\rule[-0.200pt]{0.400pt}{4.818pt}}
\put(735.0,82.0){\rule[-0.200pt]{0.400pt}{4.818pt}}
\put(735,41){\makebox(0,0){ 19}}
\put(735.0,839.0){\rule[-0.200pt]{0.400pt}{4.818pt}}
\put(802.0,82.0){\rule[-0.200pt]{0.400pt}{4.818pt}}
\put(802,41){\makebox(0,0){ 21}}
\put(802.0,839.0){\rule[-0.200pt]{0.400pt}{4.818pt}}
\put(869.0,82.0){\rule[-0.200pt]{0.400pt}{4.818pt}}
\put(869,41){\makebox(0,0){ 23}}
\put(869.0,839.0){\rule[-0.200pt]{0.400pt}{4.818pt}}
\put(936.0,82.0){\rule[-0.200pt]{0.400pt}{4.818pt}}
\put(936,41){\makebox(0,0){ 25}}
\put(936.0,839.0){\rule[-0.200pt]{0.400pt}{4.818pt}}
\put(1003.0,82.0){\rule[-0.200pt]{0.400pt}{4.818pt}}
\put(1003,41){\makebox(0,0){ 27}}
\put(1003.0,839.0){\rule[-0.200pt]{0.400pt}{4.818pt}}
\put(1070.0,82.0){\rule[-0.200pt]{0.400pt}{4.818pt}}
\put(1070,41){\makebox(0,0){ 29}}
\put(1070.0,839.0){\rule[-0.200pt]{0.400pt}{4.818pt}}
\put(1137.0,82.0){\rule[-0.200pt]{0.400pt}{4.818pt}}
\put(1137,41){\makebox(0,0){ 31}}
\put(1137.0,839.0){\rule[-0.200pt]{0.400pt}{4.818pt}}
\put(1204.0,82.0){\rule[-0.200pt]{0.400pt}{4.818pt}}
\put(1204,41){\makebox(0,0){ 33}}
\put(1204.0,839.0){\rule[-0.200pt]{0.400pt}{4.818pt}}
\put(1271.0,82.0){\rule[-0.200pt]{0.400pt}{4.818pt}}
\put(1271,41){\makebox(0,0){ 35}}
\put(1271.0,839.0){\rule[-0.200pt]{0.400pt}{4.818pt}}
\put(1338.0,82.0){\rule[-0.200pt]{0.400pt}{4.818pt}}
\put(1338,41){\makebox(0,0){ 37}}
\put(1338.0,839.0){\rule[-0.200pt]{0.400pt}{4.818pt}}
\put(1405.0,82.0){\rule[-0.200pt]{0.400pt}{4.818pt}}
\put(1405,41){\makebox(0,0){ 39}}
\put(1405.0,839.0){\rule[-0.200pt]{0.400pt}{4.818pt}}
\put(131.0,82.0){\rule[-0.200pt]{0.400pt}{187.179pt}}
\put(131.0,82.0){\rule[-0.200pt]{315.097pt}{0.400pt}}
\put(1439.0,82.0){\rule[-0.200pt]{0.400pt}{187.179pt}}
\put(131.0,859.0){\rule[-0.200pt]{315.097pt}{0.400pt}}
\put(30,470){\makebox(0,0){\rotatebox{90}{用户贡献}}}
\put(1279,819){\makebox(0,0)[r]{仿真数据}}
\put(785,-10){\makebox(0,0){月度}}
\put(1299.0,819.0){\rule[-0.200pt]{24.090pt}{0.400pt}}
\put(131,646){\usebox{\plotpoint}}
\multiput(131.58,643.05)(0.498,-0.766){65}{\rule{0.120pt}{0.712pt}}
\multiput(130.17,644.52)(34.000,-50.523){2}{\rule{0.400pt}{0.356pt}}
\multiput(165.00,592.92)(0.549,-0.497){57}{\rule{0.540pt}{0.120pt}}
\multiput(165.00,593.17)(31.879,-30.000){2}{\rule{0.270pt}{0.400pt}}
\multiput(198.00,562.92)(0.607,-0.497){53}{\rule{0.586pt}{0.120pt}}
\multiput(198.00,563.17)(32.784,-28.000){2}{\rule{0.293pt}{0.400pt}}
\multiput(232.00,534.92)(0.979,-0.495){31}{\rule{0.876pt}{0.119pt}}
\multiput(232.00,535.17)(31.181,-17.000){2}{\rule{0.438pt}{0.400pt}}
\multiput(265.00,517.92)(1.231,-0.494){25}{\rule{1.071pt}{0.119pt}}
\multiput(265.00,518.17)(31.776,-14.000){2}{\rule{0.536pt}{0.400pt}}
\multiput(299.00,503.92)(0.874,-0.495){35}{\rule{0.795pt}{0.119pt}}
\multiput(299.00,504.17)(31.350,-19.000){2}{\rule{0.397pt}{0.400pt}}
\multiput(332.00,484.93)(2.211,-0.488){13}{\rule{1.800pt}{0.117pt}}
\multiput(332.00,485.17)(30.264,-8.000){2}{\rule{0.900pt}{0.400pt}}
\multiput(366.00,476.93)(3.604,-0.477){7}{\rule{2.740pt}{0.115pt}}
\multiput(366.00,477.17)(27.313,-5.000){2}{\rule{1.370pt}{0.400pt}}
\multiput(399.00,471.92)(1.329,-0.493){23}{\rule{1.146pt}{0.119pt}}
\multiput(399.00,472.17)(31.621,-13.000){2}{\rule{0.573pt}{0.400pt}}
\multiput(433.00,458.92)(1.041,-0.494){29}{\rule{0.925pt}{0.119pt}}
\multiput(433.00,459.17)(31.080,-16.000){2}{\rule{0.463pt}{0.400pt}}
\put(466,443.67){\rule{8.191pt}{0.400pt}}
\multiput(466.00,443.17)(17.000,1.000){2}{\rule{4.095pt}{0.400pt}}
\multiput(500.00,443.93)(3.604,-0.477){7}{\rule{2.740pt}{0.115pt}}
\multiput(500.00,444.17)(27.313,-5.000){2}{\rule{1.370pt}{0.400pt}}
\multiput(533.00,438.93)(2.552,-0.485){11}{\rule{2.043pt}{0.117pt}}
\multiput(533.00,439.17)(29.760,-7.000){2}{\rule{1.021pt}{0.400pt}}
\multiput(567.00,431.93)(3.716,-0.477){7}{\rule{2.820pt}{0.115pt}}
\multiput(567.00,432.17)(28.147,-5.000){2}{\rule{1.410pt}{0.400pt}}
\multiput(601.00,426.93)(3.604,-0.477){7}{\rule{2.740pt}{0.115pt}}
\multiput(601.00,427.17)(27.313,-5.000){2}{\rule{1.370pt}{0.400pt}}
\multiput(634.00,421.93)(3.716,-0.477){7}{\rule{2.820pt}{0.115pt}}
\multiput(634.00,422.17)(28.147,-5.000){2}{\rule{1.410pt}{0.400pt}}
\multiput(668.00,416.93)(1.893,-0.489){15}{\rule{1.567pt}{0.118pt}}
\multiput(668.00,417.17)(29.748,-9.000){2}{\rule{0.783pt}{0.400pt}}
\multiput(701.00,407.94)(4.868,-0.468){5}{\rule{3.500pt}{0.113pt}}
\multiput(701.00,408.17)(26.736,-4.000){2}{\rule{1.750pt}{0.400pt}}
\multiput(735.00,403.93)(2.932,-0.482){9}{\rule{2.300pt}{0.116pt}}
\multiput(735.00,404.17)(28.226,-6.000){2}{\rule{1.150pt}{0.400pt}}
\multiput(768.00,397.93)(2.552,-0.485){11}{\rule{2.043pt}{0.117pt}}
\multiput(768.00,398.17)(29.760,-7.000){2}{\rule{1.021pt}{0.400pt}}
\put(802,391.67){\rule{7.950pt}{0.400pt}}
\multiput(802.00,391.17)(16.500,1.000){2}{\rule{3.975pt}{0.400pt}}
\multiput(835.00,391.93)(2.211,-0.488){13}{\rule{1.800pt}{0.117pt}}
\multiput(835.00,392.17)(30.264,-8.000){2}{\rule{0.900pt}{0.400pt}}
\multiput(869.00,383.93)(2.932,-0.482){9}{\rule{2.300pt}{0.116pt}}
\multiput(869.00,384.17)(28.226,-6.000){2}{\rule{1.150pt}{0.400pt}}
\multiput(902.00,377.95)(7.383,-0.447){3}{\rule{4.633pt}{0.108pt}}
\multiput(902.00,378.17)(24.383,-3.000){2}{\rule{2.317pt}{0.400pt}}
\put(936,376.17){\rule{6.700pt}{0.400pt}}
\multiput(936.00,375.17)(19.094,2.000){2}{\rule{3.350pt}{0.400pt}}
\multiput(969.00,376.94)(4.868,-0.468){5}{\rule{3.500pt}{0.113pt}}
\multiput(969.00,377.17)(26.736,-4.000){2}{\rule{1.750pt}{0.400pt}}
\multiput(1003.00,372.93)(2.211,-0.488){13}{\rule{1.800pt}{0.117pt}}
\multiput(1003.00,373.17)(30.264,-8.000){2}{\rule{0.900pt}{0.400pt}}
\multiput(1037.00,364.93)(2.476,-0.485){11}{\rule{1.986pt}{0.117pt}}
\multiput(1037.00,365.17)(28.879,-7.000){2}{\rule{0.993pt}{0.400pt}}
\put(1070,357.67){\rule{8.191pt}{0.400pt}}
\multiput(1070.00,358.17)(17.000,-1.000){2}{\rule{4.095pt}{0.400pt}}
\multiput(1104.00,356.94)(4.722,-0.468){5}{\rule{3.400pt}{0.113pt}}
\multiput(1104.00,357.17)(25.943,-4.000){2}{\rule{1.700pt}{0.400pt}}
\multiput(1137.00,352.94)(4.868,-0.468){5}{\rule{3.500pt}{0.113pt}}
\multiput(1137.00,353.17)(26.736,-4.000){2}{\rule{1.750pt}{0.400pt}}
\put(1171,348.67){\rule{7.950pt}{0.400pt}}
\multiput(1171.00,349.17)(16.500,-1.000){2}{\rule{3.975pt}{0.400pt}}
\put(1238,347.17){\rule{6.700pt}{0.400pt}}
\multiput(1238.00,348.17)(19.094,-2.000){2}{\rule{3.350pt}{0.400pt}}
\put(1271,345.67){\rule{8.191pt}{0.400pt}}
\multiput(1271.00,346.17)(17.000,-1.000){2}{\rule{4.095pt}{0.400pt}}
\put(1305,345.67){\rule{7.950pt}{0.400pt}}
\multiput(1305.00,345.17)(16.500,1.000){2}{\rule{3.975pt}{0.400pt}}
\put(1204.0,349.0){\rule[-0.200pt]{8.191pt}{0.400pt}}
\multiput(1372.00,345.93)(2.476,-0.485){11}{\rule{1.986pt}{0.117pt}}
\multiput(1372.00,346.17)(28.879,-7.000){2}{\rule{0.993pt}{0.400pt}}
\put(1405,338.17){\rule{6.900pt}{0.400pt}}
\multiput(1405.00,339.17)(19.679,-2.000){2}{\rule{3.450pt}{0.400pt}}
\put(1338.0,347.0){\rule[-0.200pt]{8.191pt}{0.400pt}}
\put(1279,768){\makebox(0,0)[r]{实际数据}}
\put(1299.0,768.0){\rule[-0.200pt]{24.090pt}{0.400pt}}
\put(131,587){\usebox{\plotpoint}}
\multiput(131.58,581.75)(0.498,-1.465){65}{\rule{0.120pt}{1.265pt}}
\multiput(130.17,584.38)(34.000,-96.375){2}{\rule{0.400pt}{0.632pt}}
\multiput(165.58,485.12)(0.497,-0.744){63}{\rule{0.120pt}{0.694pt}}
\multiput(164.17,486.56)(33.000,-47.560){2}{\rule{0.400pt}{0.347pt}}
\multiput(198.58,439.00)(0.498,0.751){65}{\rule{0.120pt}{0.700pt}}
\multiput(197.17,439.00)(34.000,49.547){2}{\rule{0.400pt}{0.350pt}}
\multiput(232.00,490.58)(0.719,0.496){43}{\rule{0.674pt}{0.120pt}}
\multiput(232.00,489.17)(31.601,23.000){2}{\rule{0.337pt}{0.400pt}}
\multiput(265.58,513.00)(0.498,1.197){65}{\rule{0.120pt}{1.053pt}}
\multiput(264.17,513.00)(34.000,78.815){2}{\rule{0.400pt}{0.526pt}}
\multiput(299.58,594.00)(0.497,1.265){63}{\rule{0.120pt}{1.106pt}}
\multiput(298.17,594.00)(33.000,80.704){2}{\rule{0.400pt}{0.553pt}}
\multiput(332.58,673.17)(0.498,-1.034){65}{\rule{0.120pt}{0.924pt}}
\multiput(331.17,675.08)(34.000,-68.083){2}{\rule{0.400pt}{0.462pt}}
\multiput(366.58,600.09)(0.497,-1.969){63}{\rule{0.120pt}{1.664pt}}
\multiput(365.17,603.55)(33.000,-125.547){2}{\rule{0.400pt}{0.832pt}}
\multiput(399.00,478.59)(1.951,0.489){15}{\rule{1.611pt}{0.118pt}}
\multiput(399.00,477.17)(30.656,9.000){2}{\rule{0.806pt}{0.400pt}}
\multiput(433.00,485.92)(0.549,-0.497){57}{\rule{0.540pt}{0.120pt}}
\multiput(433.00,486.17)(31.879,-30.000){2}{\rule{0.270pt}{0.400pt}}
\multiput(466.00,455.92)(0.855,-0.496){37}{\rule{0.780pt}{0.119pt}}
\multiput(466.00,456.17)(32.381,-20.000){2}{\rule{0.390pt}{0.400pt}}
\multiput(500.00,435.93)(2.476,-0.485){11}{\rule{1.986pt}{0.117pt}}
\multiput(500.00,436.17)(28.879,-7.000){2}{\rule{0.993pt}{0.400pt}}
\multiput(533.58,430.00)(0.498,0.514){65}{\rule{0.120pt}{0.512pt}}
\multiput(532.17,430.00)(34.000,33.938){2}{\rule{0.400pt}{0.256pt}}
\multiput(567.00,465.58)(1.009,0.495){31}{\rule{0.900pt}{0.119pt}}
\multiput(567.00,464.17)(32.132,17.000){2}{\rule{0.450pt}{0.400pt}}
\multiput(601.58,482.00)(0.497,0.529){63}{\rule{0.120pt}{0.524pt}}
\multiput(600.17,482.00)(33.000,33.912){2}{\rule{0.400pt}{0.262pt}}
\multiput(634.58,510.68)(0.498,-1.792){65}{\rule{0.120pt}{1.524pt}}
\multiput(633.17,513.84)(34.000,-117.838){2}{\rule{0.400pt}{0.762pt}}
\multiput(668.00,396.58)(1.694,0.491){17}{\rule{1.420pt}{0.118pt}}
\multiput(668.00,395.17)(30.053,10.000){2}{\rule{0.710pt}{0.400pt}}
\multiput(701.58,402.95)(0.498,-0.796){65}{\rule{0.120pt}{0.735pt}}
\multiput(700.17,404.47)(34.000,-52.474){2}{\rule{0.400pt}{0.368pt}}
\multiput(735.00,352.58)(0.635,0.497){49}{\rule{0.608pt}{0.120pt}}
\multiput(735.00,351.17)(31.739,26.000){2}{\rule{0.304pt}{0.400pt}}
\multiput(768.00,376.92)(0.775,-0.496){41}{\rule{0.718pt}{0.120pt}}
\multiput(768.00,377.17)(32.509,-22.000){2}{\rule{0.359pt}{0.400pt}}
\multiput(802.00,356.58)(0.752,0.496){41}{\rule{0.700pt}{0.120pt}}
\multiput(802.00,355.17)(31.547,22.000){2}{\rule{0.350pt}{0.400pt}}
\multiput(835.00,376.92)(0.952,-0.495){33}{\rule{0.856pt}{0.119pt}}
\multiput(835.00,377.17)(32.224,-18.000){2}{\rule{0.428pt}{0.400pt}}
\multiput(869.00,358.92)(1.401,-0.492){21}{\rule{1.200pt}{0.119pt}}
\multiput(869.00,359.17)(30.509,-12.000){2}{\rule{0.600pt}{0.400pt}}
\multiput(902.00,348.59)(2.552,0.485){11}{\rule{2.043pt}{0.117pt}}
\multiput(902.00,347.17)(29.760,7.000){2}{\rule{1.021pt}{0.400pt}}
\put(936,354.67){\rule{7.950pt}{0.400pt}}
\multiput(936.00,354.17)(16.500,1.000){2}{\rule{3.975pt}{0.400pt}}
\multiput(969.58,348.11)(0.498,-2.267){65}{\rule{0.120pt}{1.900pt}}
\multiput(968.17,352.06)(34.000,-149.056){2}{\rule{0.400pt}{0.950pt}}
\multiput(1003.58,203.00)(0.498,2.579){65}{\rule{0.120pt}{2.147pt}}
\multiput(1002.17,203.00)(34.000,169.544){2}{\rule{0.400pt}{1.074pt}}
\multiput(1037.00,375.92)(0.923,-0.495){33}{\rule{0.833pt}{0.119pt}}
\multiput(1037.00,376.17)(31.270,-18.000){2}{\rule{0.417pt}{0.400pt}}
\multiput(1070.00,357.92)(1.329,-0.493){23}{\rule{1.146pt}{0.119pt}}
\multiput(1070.00,358.17)(31.621,-13.000){2}{\rule{0.573pt}{0.400pt}}
\put(1104,345.67){\rule{7.950pt}{0.400pt}}
\multiput(1104.00,345.17)(16.500,1.000){2}{\rule{3.975pt}{0.400pt}}
\multiput(1137.00,345.94)(4.868,-0.468){5}{\rule{3.500pt}{0.113pt}}
\multiput(1137.00,346.17)(26.736,-4.000){2}{\rule{1.750pt}{0.400pt}}
\multiput(1171.00,341.93)(3.604,-0.477){7}{\rule{2.740pt}{0.115pt}}
\multiput(1171.00,342.17)(27.313,-5.000){2}{\rule{1.370pt}{0.400pt}}
\multiput(1204.00,338.59)(3.022,0.482){9}{\rule{2.367pt}{0.116pt}}
\multiput(1204.00,337.17)(29.088,6.000){2}{\rule{1.183pt}{0.400pt}}
\multiput(1238.00,344.61)(7.160,0.447){3}{\rule{4.500pt}{0.108pt}}
\multiput(1238.00,343.17)(23.660,3.000){2}{\rule{2.250pt}{0.400pt}}
\multiput(1271.58,344.63)(0.498,-0.588){65}{\rule{0.120pt}{0.571pt}}
\multiput(1270.17,345.82)(34.000,-38.816){2}{\rule{0.400pt}{0.285pt}}
\multiput(1305.00,307.58)(0.531,0.497){59}{\rule{0.526pt}{0.120pt}}
\multiput(1305.00,306.17)(31.909,31.000){2}{\rule{0.263pt}{0.400pt}}
\multiput(1338.00,336.92)(0.741,-0.496){43}{\rule{0.691pt}{0.120pt}}
\multiput(1338.00,337.17)(32.565,-23.000){2}{\rule{0.346pt}{0.400pt}}
\multiput(1372.00,315.59)(2.932,0.482){9}{\rule{2.300pt}{0.116pt}}
\multiput(1372.00,314.17)(28.226,6.000){2}{\rule{1.150pt}{0.400pt}}
\multiput(1405.00,321.60)(4.868,0.468){5}{\rule{3.500pt}{0.113pt}}
\multiput(1405.00,320.17)(26.736,4.000){2}{\rule{1.750pt}{0.400pt}}
\put(131,587){\makebox(0,0){$+$}}
\put(165,488){\makebox(0,0){$+$}}
\put(198,439){\makebox(0,0){$+$}}
\put(232,490){\makebox(0,0){$+$}}
\put(265,513){\makebox(0,0){$+$}}
\put(299,594){\makebox(0,0){$+$}}
\put(332,677){\makebox(0,0){$+$}}
\put(366,607){\makebox(0,0){$+$}}
\put(399,478){\makebox(0,0){$+$}}
\put(433,487){\makebox(0,0){$+$}}
\put(466,457){\makebox(0,0){$+$}}
\put(500,437){\makebox(0,0){$+$}}
\put(533,430){\makebox(0,0){$+$}}
\put(567,465){\makebox(0,0){$+$}}
\put(601,482){\makebox(0,0){$+$}}
\put(634,517){\makebox(0,0){$+$}}
\put(668,396){\makebox(0,0){$+$}}
\put(701,406){\makebox(0,0){$+$}}
\put(735,352){\makebox(0,0){$+$}}
\put(768,378){\makebox(0,0){$+$}}
\put(802,356){\makebox(0,0){$+$}}
\put(835,378){\makebox(0,0){$+$}}
\put(869,360){\makebox(0,0){$+$}}
\put(902,348){\makebox(0,0){$+$}}
\put(936,355){\makebox(0,0){$+$}}
\put(969,356){\makebox(0,0){$+$}}
\put(1003,203){\makebox(0,0){$+$}}
\put(1037,377){\makebox(0,0){$+$}}
\put(1070,359){\makebox(0,0){$+$}}
\put(1104,346){\makebox(0,0){$+$}}
\put(1137,347){\makebox(0,0){$+$}}
\put(1171,343){\makebox(0,0){$+$}}
\put(1204,338){\makebox(0,0){$+$}}
\put(1238,344){\makebox(0,0){$+$}}
\put(1271,347){\makebox(0,0){$+$}}
\put(1305,307){\makebox(0,0){$+$}}
\put(1338,338){\makebox(0,0){$+$}}
\put(1372,315){\makebox(0,0){$+$}}
\put(1405,321){\makebox(0,0){$+$}}
\put(1439,325){\makebox(0,0){$+$}}
\put(1349,768){\makebox(0,0){$+$}}
\put(131.0,82.0){\rule[-0.200pt]{0.400pt}{187.179pt}}
\put(131.0,82.0){\rule[-0.200pt]{315.097pt}{0.400pt}}
\put(1439.0,82.0){\rule[-0.200pt]{0.400pt}{187.179pt}}
\put(131.0,859.0){\rule[-0.200pt]{315.097pt}{0.400pt}}
\end{picture}
} 
   \end{frame}

   \begin{frame}{历史检验}
     边缘用户
\scalebox{0.8}{% GNUPLOT: LaTeX picture
\setlength{\unitlength}{0.240900pt}
\ifx\plotpoint\undefined\newsavebox{\plotpoint}\fi
\begin{picture}(1500,900)(0,0)
\sbox{\plotpoint}{\rule[-0.200pt]{0.400pt}{0.400pt}}%
\put(211.0,82.0){\rule[-0.200pt]{4.818pt}{0.400pt}}
\put(191,82){\makebox(0,0)[r]{-0.02}}
\put(1419.0,82.0){\rule[-0.200pt]{4.818pt}{0.400pt}}
\put(211.0,143.0){\rule[-0.200pt]{4.818pt}{0.400pt}}
\put(191,143){\makebox(0,0)[r]{-0.015}}
\put(1419.0,143.0){\rule[-0.200pt]{4.818pt}{0.400pt}}
\put(211.0,203.0){\rule[-0.200pt]{4.818pt}{0.400pt}}
\put(191,203){\makebox(0,0)[r]{-0.01}}
\put(1419.0,203.0){\rule[-0.200pt]{4.818pt}{0.400pt}}
\put(211.0,264.0){\rule[-0.200pt]{4.818pt}{0.400pt}}
\put(191,264){\makebox(0,0)[r]{-0.005}}
\put(1419.0,264.0){\rule[-0.200pt]{4.818pt}{0.400pt}}
\put(211.0,325.0){\rule[-0.200pt]{4.818pt}{0.400pt}}
\put(191,325){\makebox(0,0)[r]{ 0}}
\put(1419.0,325.0){\rule[-0.200pt]{4.818pt}{0.400pt}}
\put(211.0,386.0){\rule[-0.200pt]{4.818pt}{0.400pt}}
\put(191,386){\makebox(0,0)[r]{ 0.005}}
\put(1419.0,386.0){\rule[-0.200pt]{4.818pt}{0.400pt}}
\put(211.0,446.0){\rule[-0.200pt]{4.818pt}{0.400pt}}
\put(191,446){\makebox(0,0)[r]{ 0.01}}
\put(1419.0,446.0){\rule[-0.200pt]{4.818pt}{0.400pt}}
\put(211.0,507.0){\rule[-0.200pt]{4.818pt}{0.400pt}}
\put(191,507){\makebox(0,0)[r]{ 0.015}}
\put(1419.0,507.0){\rule[-0.200pt]{4.818pt}{0.400pt}}
\put(211.0,568.0){\rule[-0.200pt]{4.818pt}{0.400pt}}
\put(191,568){\makebox(0,0)[r]{ 0.02}}
\put(1419.0,568.0){\rule[-0.200pt]{4.818pt}{0.400pt}}
\put(211.0,628.0){\rule[-0.200pt]{4.818pt}{0.400pt}}
\put(191,628){\makebox(0,0)[r]{ 0.025}}
\put(1419.0,628.0){\rule[-0.200pt]{4.818pt}{0.400pt}}
\put(211.0,689.0){\rule[-0.200pt]{4.818pt}{0.400pt}}
\put(191,689){\makebox(0,0)[r]{ 0.03}}
\put(1419.0,689.0){\rule[-0.200pt]{4.818pt}{0.400pt}}
\put(211.0,82.0){\rule[-0.200pt]{0.400pt}{4.818pt}}
\put(211,41){\makebox(0,0){ 1}}
\put(211.0,839.0){\rule[-0.200pt]{0.400pt}{4.818pt}}
\put(274.0,82.0){\rule[-0.200pt]{0.400pt}{4.818pt}}
\put(274,41){\makebox(0,0){ 3}}
\put(274.0,839.0){\rule[-0.200pt]{0.400pt}{4.818pt}}
\put(337.0,82.0){\rule[-0.200pt]{0.400pt}{4.818pt}}
\put(337,41){\makebox(0,0){ 5}}
\put(337.0,839.0){\rule[-0.200pt]{0.400pt}{4.818pt}}
\put(400.0,82.0){\rule[-0.200pt]{0.400pt}{4.818pt}}
\put(400,41){\makebox(0,0){ 7}}
\put(400.0,839.0){\rule[-0.200pt]{0.400pt}{4.818pt}}
\put(463.0,82.0){\rule[-0.200pt]{0.400pt}{4.818pt}}
\put(463,41){\makebox(0,0){ 9}}
\put(463.0,839.0){\rule[-0.200pt]{0.400pt}{4.818pt}}
\put(526.0,82.0){\rule[-0.200pt]{0.400pt}{4.818pt}}
\put(526,41){\makebox(0,0){ 11}}
\put(526.0,839.0){\rule[-0.200pt]{0.400pt}{4.818pt}}
\put(589.0,82.0){\rule[-0.200pt]{0.400pt}{4.818pt}}
\put(589,41){\makebox(0,0){ 13}}
\put(589.0,839.0){\rule[-0.200pt]{0.400pt}{4.818pt}}
\put(652.0,82.0){\rule[-0.200pt]{0.400pt}{4.818pt}}
\put(652,41){\makebox(0,0){ 15}}
\put(652.0,839.0){\rule[-0.200pt]{0.400pt}{4.818pt}}
\put(715.0,82.0){\rule[-0.200pt]{0.400pt}{4.818pt}}
\put(715,41){\makebox(0,0){ 17}}
\put(715.0,839.0){\rule[-0.200pt]{0.400pt}{4.818pt}}
\put(778.0,82.0){\rule[-0.200pt]{0.400pt}{4.818pt}}
\put(778,41){\makebox(0,0){ 19}}
\put(778.0,839.0){\rule[-0.200pt]{0.400pt}{4.818pt}}
\put(841.0,82.0){\rule[-0.200pt]{0.400pt}{4.818pt}}
\put(841,41){\makebox(0,0){ 21}}
\put(841.0,839.0){\rule[-0.200pt]{0.400pt}{4.818pt}}
\put(904.0,82.0){\rule[-0.200pt]{0.400pt}{4.818pt}}
\put(904,41){\makebox(0,0){ 23}}
\put(904.0,839.0){\rule[-0.200pt]{0.400pt}{4.818pt}}
\put(967.0,82.0){\rule[-0.200pt]{0.400pt}{4.818pt}}
\put(967,41){\makebox(0,0){ 25}}
\put(967.0,839.0){\rule[-0.200pt]{0.400pt}{4.818pt}}
\put(1030.0,82.0){\rule[-0.200pt]{0.400pt}{4.818pt}}
\put(1030,41){\makebox(0,0){ 27}}
\put(1030.0,839.0){\rule[-0.200pt]{0.400pt}{4.818pt}}
\put(1093.0,82.0){\rule[-0.200pt]{0.400pt}{4.818pt}}
\put(1093,41){\makebox(0,0){ 29}}
\put(1093.0,839.0){\rule[-0.200pt]{0.400pt}{4.818pt}}
\put(1156.0,82.0){\rule[-0.200pt]{0.400pt}{4.818pt}}
\put(1156,41){\makebox(0,0){ 31}}
\put(1156.0,839.0){\rule[-0.200pt]{0.400pt}{4.818pt}}
\put(1219.0,82.0){\rule[-0.200pt]{0.400pt}{4.818pt}}
\put(1219,41){\makebox(0,0){ 33}}
\put(1219.0,839.0){\rule[-0.200pt]{0.400pt}{4.818pt}}
\put(1282.0,82.0){\rule[-0.200pt]{0.400pt}{4.818pt}}
\put(1282,41){\makebox(0,0){ 35}}
\put(1282.0,839.0){\rule[-0.200pt]{0.400pt}{4.818pt}}
\put(1345.0,82.0){\rule[-0.200pt]{0.400pt}{4.818pt}}
\put(1345,41){\makebox(0,0){ 37}}
\put(1345.0,839.0){\rule[-0.200pt]{0.400pt}{4.818pt}}
\put(1408.0,82.0){\rule[-0.200pt]{0.400pt}{4.818pt}}
\put(1408,41){\makebox(0,0){ 39}}
\put(1408.0,839.0){\rule[-0.200pt]{0.400pt}{4.818pt}}
\put(211.0,82.0){\rule[-0.200pt]{0.400pt}{187.179pt}}
\put(211.0,82.0){\rule[-0.200pt]{295.825pt}{0.400pt}}
\put(1439.0,82.0){\rule[-0.200pt]{0.400pt}{187.179pt}}
\put(211.0,859.0){\rule[-0.200pt]{295.825pt}{0.400pt}}
\put(30,470){\makebox(0,0){\rotatebox{90}{用户贡献}}}
\put(785,-10){\makebox(0,0){月度}}
\put(1279,819){\makebox(0,0)[r]{仿真数据}}
\put(1299.0,819.0){\rule[-0.200pt]{24.090pt}{0.400pt}}
\put(211,423){\usebox{\plotpoint}}
\put(211,421.17){\rule{6.300pt}{0.400pt}}
\multiput(211.00,422.17)(17.924,-2.000){2}{\rule{3.150pt}{0.400pt}}
\multiput(242.00,419.94)(4.575,-0.468){5}{\rule{3.300pt}{0.113pt}}
\multiput(242.00,420.17)(25.151,-4.000){2}{\rule{1.650pt}{0.400pt}}
\multiput(305.00,415.94)(4.575,-0.468){5}{\rule{3.300pt}{0.113pt}}
\multiput(305.00,416.17)(25.151,-4.000){2}{\rule{1.650pt}{0.400pt}}
\multiput(337.00,413.60)(4.429,0.468){5}{\rule{3.200pt}{0.113pt}}
\multiput(337.00,412.17)(24.358,4.000){2}{\rule{1.600pt}{0.400pt}}
\put(274.0,417.0){\rule[-0.200pt]{7.468pt}{0.400pt}}
\multiput(400.00,415.93)(2.013,-0.488){13}{\rule{1.650pt}{0.117pt}}
\multiput(400.00,416.17)(27.575,-8.000){2}{\rule{0.825pt}{0.400pt}}
\multiput(431.00,407.94)(4.575,-0.468){5}{\rule{3.300pt}{0.113pt}}
\multiput(431.00,408.17)(25.151,-4.000){2}{\rule{1.650pt}{0.400pt}}
\multiput(463.00,403.93)(3.382,-0.477){7}{\rule{2.580pt}{0.115pt}}
\multiput(463.00,404.17)(25.645,-5.000){2}{\rule{1.290pt}{0.400pt}}
\multiput(494.00,400.59)(3.493,0.477){7}{\rule{2.660pt}{0.115pt}}
\multiput(494.00,399.17)(26.479,5.000){2}{\rule{1.330pt}{0.400pt}}
\multiput(526.00,403.93)(3.382,-0.477){7}{\rule{2.580pt}{0.115pt}}
\multiput(526.00,404.17)(25.645,-5.000){2}{\rule{1.290pt}{0.400pt}}
\multiput(557.00,398.93)(2.079,-0.488){13}{\rule{1.700pt}{0.117pt}}
\multiput(557.00,399.17)(28.472,-8.000){2}{\rule{0.850pt}{0.400pt}}
\multiput(589.00,390.93)(3.382,-0.477){7}{\rule{2.580pt}{0.115pt}}
\multiput(589.00,391.17)(25.645,-5.000){2}{\rule{1.290pt}{0.400pt}}
\multiput(620.00,385.94)(4.575,-0.468){5}{\rule{3.300pt}{0.113pt}}
\multiput(620.00,386.17)(25.151,-4.000){2}{\rule{1.650pt}{0.400pt}}
\multiput(652.00,381.93)(2.013,-0.488){13}{\rule{1.650pt}{0.117pt}}
\multiput(652.00,382.17)(27.575,-8.000){2}{\rule{0.825pt}{0.400pt}}
\multiput(683.00,373.95)(6.937,-0.447){3}{\rule{4.367pt}{0.108pt}}
\multiput(683.00,374.17)(22.937,-3.000){2}{\rule{2.183pt}{0.400pt}}
\multiput(715.00,370.93)(3.382,-0.477){7}{\rule{2.580pt}{0.115pt}}
\multiput(715.00,371.17)(25.645,-5.000){2}{\rule{1.290pt}{0.400pt}}
\multiput(746.00,365.93)(1.834,-0.489){15}{\rule{1.522pt}{0.118pt}}
\multiput(746.00,366.17)(28.841,-9.000){2}{\rule{0.761pt}{0.400pt}}
\multiput(778.00,358.60)(4.429,0.468){5}{\rule{3.200pt}{0.113pt}}
\multiput(778.00,357.17)(24.358,4.000){2}{\rule{1.600pt}{0.400pt}}
\multiput(809.00,360.92)(1.250,-0.493){23}{\rule{1.085pt}{0.119pt}}
\multiput(809.00,361.17)(29.749,-13.000){2}{\rule{0.542pt}{0.400pt}}
\multiput(841.00,349.60)(4.429,0.468){5}{\rule{3.200pt}{0.113pt}}
\multiput(841.00,348.17)(24.358,4.000){2}{\rule{1.600pt}{0.400pt}}
\multiput(872.00,351.92)(1.009,-0.494){29}{\rule{0.900pt}{0.119pt}}
\multiput(872.00,352.17)(30.132,-16.000){2}{\rule{0.450pt}{0.400pt}}
\put(904,335.67){\rule{7.468pt}{0.400pt}}
\multiput(904.00,336.17)(15.500,-1.000){2}{\rule{3.734pt}{0.400pt}}
\multiput(935.00,334.93)(2.841,-0.482){9}{\rule{2.233pt}{0.116pt}}
\multiput(935.00,335.17)(27.365,-6.000){2}{\rule{1.117pt}{0.400pt}}
\multiput(967.00,328.92)(1.590,-0.491){17}{\rule{1.340pt}{0.118pt}}
\multiput(967.00,329.17)(28.219,-10.000){2}{\rule{0.670pt}{0.400pt}}
\put(998,318.17){\rule{6.500pt}{0.400pt}}
\multiput(998.00,319.17)(18.509,-2.000){2}{\rule{3.250pt}{0.400pt}}
\multiput(1030.00,316.92)(0.977,-0.494){29}{\rule{0.875pt}{0.119pt}}
\multiput(1030.00,317.17)(29.184,-16.000){2}{\rule{0.438pt}{0.400pt}}
\multiput(1061.00,300.93)(2.079,-0.488){13}{\rule{1.700pt}{0.117pt}}
\multiput(1061.00,301.17)(28.472,-8.000){2}{\rule{0.850pt}{0.400pt}}
\multiput(1093.00,292.93)(2.013,-0.488){13}{\rule{1.650pt}{0.117pt}}
\multiput(1093.00,293.17)(27.575,-8.000){2}{\rule{0.825pt}{0.400pt}}
\put(1124,284.67){\rule{7.709pt}{0.400pt}}
\multiput(1124.00,285.17)(16.000,-1.000){2}{\rule{3.854pt}{0.400pt}}
\multiput(1156.00,283.92)(1.590,-0.491){17}{\rule{1.340pt}{0.118pt}}
\multiput(1156.00,284.17)(28.219,-10.000){2}{\rule{0.670pt}{0.400pt}}
\multiput(1187.00,273.92)(1.642,-0.491){17}{\rule{1.380pt}{0.118pt}}
\multiput(1187.00,274.17)(29.136,-10.000){2}{\rule{0.690pt}{0.400pt}}
\multiput(1219.00,263.95)(6.714,-0.447){3}{\rule{4.233pt}{0.108pt}}
\multiput(1219.00,264.17)(22.214,-3.000){2}{\rule{2.117pt}{0.400pt}}
\multiput(1250.00,260.92)(1.642,-0.491){17}{\rule{1.380pt}{0.118pt}}
\multiput(1250.00,261.17)(29.136,-10.000){2}{\rule{0.690pt}{0.400pt}}
\multiput(1282.00,250.92)(1.439,-0.492){19}{\rule{1.227pt}{0.118pt}}
\multiput(1282.00,251.17)(28.453,-11.000){2}{\rule{0.614pt}{0.400pt}}
\multiput(1313.00,239.92)(1.642,-0.491){17}{\rule{1.380pt}{0.118pt}}
\multiput(1313.00,240.17)(29.136,-10.000){2}{\rule{0.690pt}{0.400pt}}
\multiput(1345.00,229.93)(2.013,-0.488){13}{\rule{1.650pt}{0.117pt}}
\multiput(1345.00,230.17)(27.575,-8.000){2}{\rule{0.825pt}{0.400pt}}
\multiput(1376.00,221.93)(3.493,-0.477){7}{\rule{2.660pt}{0.115pt}}
\multiput(1376.00,222.17)(26.479,-5.000){2}{\rule{1.330pt}{0.400pt}}
\multiput(1408.00,216.92)(1.315,-0.492){21}{\rule{1.133pt}{0.119pt}}
\multiput(1408.00,217.17)(28.648,-12.000){2}{\rule{0.567pt}{0.400pt}}
\put(368.0,417.0){\rule[-0.200pt]{7.709pt}{0.400pt}}
\put(1279,768){\makebox(0,0)[r]{实际数据}}
\put(1299.0,768.0){\rule[-0.200pt]{24.090pt}{0.400pt}}
\put(211,512){\usebox{\plotpoint}}
\multiput(211.58,502.80)(0.497,-2.669){59}{\rule{0.120pt}{2.216pt}}
\multiput(210.17,507.40)(31.000,-159.400){2}{\rule{0.400pt}{1.108pt}}
\multiput(274.58,348.00)(0.497,2.800){59}{\rule{0.120pt}{2.319pt}}
\multiput(273.17,348.00)(31.000,167.186){2}{\rule{0.400pt}{1.160pt}}
\multiput(305.58,520.00)(0.497,1.968){61}{\rule{0.120pt}{1.663pt}}
\multiput(304.17,520.00)(32.000,121.549){2}{\rule{0.400pt}{0.831pt}}
\multiput(337.58,639.44)(0.497,-1.559){59}{\rule{0.120pt}{1.339pt}}
\multiput(336.17,642.22)(31.000,-93.221){2}{\rule{0.400pt}{0.669pt}}
\multiput(368.58,543.24)(0.497,-1.621){61}{\rule{0.120pt}{1.388pt}}
\multiput(367.17,546.12)(32.000,-100.120){2}{\rule{0.400pt}{0.694pt}}
\multiput(400.58,442.10)(0.497,-1.054){59}{\rule{0.120pt}{0.939pt}}
\multiput(399.17,444.05)(31.000,-63.052){2}{\rule{0.400pt}{0.469pt}}
\multiput(431.00,381.58)(0.729,0.496){41}{\rule{0.682pt}{0.120pt}}
\multiput(431.00,380.17)(30.585,22.000){2}{\rule{0.341pt}{0.400pt}}
\multiput(463.58,403.00)(0.497,0.841){59}{\rule{0.120pt}{0.771pt}}
\multiput(462.17,403.00)(31.000,50.400){2}{\rule{0.400pt}{0.385pt}}
\multiput(494.58,449.03)(0.497,-1.684){61}{\rule{0.120pt}{1.438pt}}
\multiput(493.17,452.02)(32.000,-104.016){2}{\rule{0.400pt}{0.719pt}}
\multiput(526.00,348.59)(2.751,0.482){9}{\rule{2.167pt}{0.116pt}}
\multiput(526.00,347.17)(26.503,6.000){2}{\rule{1.083pt}{0.400pt}}
\multiput(557.00,354.58)(1.642,0.491){17}{\rule{1.380pt}{0.118pt}}
\multiput(557.00,353.17)(29.136,10.000){2}{\rule{0.690pt}{0.400pt}}
\multiput(589.00,364.58)(0.778,0.496){37}{\rule{0.720pt}{0.119pt}}
\multiput(589.00,363.17)(29.506,20.000){2}{\rule{0.360pt}{0.400pt}}
\multiput(620.58,381.77)(0.497,-0.546){61}{\rule{0.120pt}{0.538pt}}
\multiput(619.17,382.88)(32.000,-33.884){2}{\rule{0.400pt}{0.269pt}}
\multiput(652.58,349.00)(0.497,0.874){59}{\rule{0.120pt}{0.797pt}}
\multiput(651.17,349.00)(31.000,52.346){2}{\rule{0.400pt}{0.398pt}}
\multiput(683.58,403.00)(0.497,0.910){61}{\rule{0.120pt}{0.825pt}}
\multiput(682.17,403.00)(32.000,56.288){2}{\rule{0.400pt}{0.413pt}}
\multiput(715.58,451.48)(0.497,-2.767){59}{\rule{0.120pt}{2.294pt}}
\multiput(714.17,456.24)(31.000,-165.240){2}{\rule{0.400pt}{1.147pt}}
\multiput(746.58,291.00)(0.497,1.115){61}{\rule{0.120pt}{0.988pt}}
\multiput(745.17,291.00)(32.000,68.950){2}{\rule{0.400pt}{0.494pt}}
\multiput(778.58,358.05)(0.497,-1.070){59}{\rule{0.120pt}{0.952pt}}
\multiput(777.17,360.02)(31.000,-64.025){2}{\rule{0.400pt}{0.476pt}}
\multiput(809.58,296.00)(0.497,1.747){61}{\rule{0.120pt}{1.488pt}}
\multiput(808.17,296.00)(32.000,107.913){2}{\rule{0.400pt}{0.744pt}}
\multiput(841.00,405.92)(1.044,-0.494){27}{\rule{0.927pt}{0.119pt}}
\multiput(841.00,406.17)(29.077,-15.000){2}{\rule{0.463pt}{0.400pt}}
\multiput(872.00,392.58)(0.551,0.497){55}{\rule{0.541pt}{0.120pt}}
\multiput(872.00,391.17)(30.876,29.000){2}{\rule{0.271pt}{0.400pt}}
\multiput(904.58,414.21)(0.497,-1.935){59}{\rule{0.120pt}{1.635pt}}
\multiput(903.17,417.61)(31.000,-115.605){2}{\rule{0.400pt}{0.818pt}}
\multiput(935.58,299.35)(0.497,-0.673){61}{\rule{0.120pt}{0.637pt}}
\multiput(934.17,300.68)(32.000,-41.677){2}{\rule{0.400pt}{0.319pt}}
\multiput(967.58,259.00)(0.497,2.555){59}{\rule{0.120pt}{2.126pt}}
\multiput(966.17,259.00)(31.000,152.588){2}{\rule{0.400pt}{1.063pt}}
\multiput(998.58,410.71)(0.497,-1.478){61}{\rule{0.120pt}{1.275pt}}
\multiput(997.17,413.35)(32.000,-91.354){2}{\rule{0.400pt}{0.638pt}}
\multiput(1030.58,322.00)(0.497,1.543){59}{\rule{0.120pt}{1.326pt}}
\multiput(1029.17,322.00)(31.000,92.248){2}{\rule{0.400pt}{0.663pt}}
\multiput(1061.58,414.61)(0.497,-0.594){61}{\rule{0.120pt}{0.575pt}}
\multiput(1060.17,415.81)(32.000,-36.807){2}{\rule{0.400pt}{0.288pt}}
\multiput(1093.58,373.92)(0.497,-1.412){59}{\rule{0.120pt}{1.223pt}}
\multiput(1092.17,376.46)(31.000,-84.462){2}{\rule{0.400pt}{0.611pt}}
\multiput(1124.00,292.58)(0.668,0.496){45}{\rule{0.633pt}{0.120pt}}
\multiput(1124.00,291.17)(30.685,24.000){2}{\rule{0.317pt}{0.400pt}}
\multiput(1156.58,310.23)(0.497,-1.625){59}{\rule{0.120pt}{1.390pt}}
\multiput(1155.17,313.11)(31.000,-97.114){2}{\rule{0.400pt}{0.695pt}}
\multiput(1187.00,214.92)(1.079,-0.494){27}{\rule{0.953pt}{0.119pt}}
\multiput(1187.00,215.17)(30.021,-15.000){2}{\rule{0.477pt}{0.400pt}}
\multiput(1219.00,201.59)(2.013,0.488){13}{\rule{1.650pt}{0.117pt}}
\multiput(1219.00,200.17)(27.575,8.000){2}{\rule{0.825pt}{0.400pt}}
\multiput(1250.58,209.00)(0.497,1.178){61}{\rule{0.120pt}{1.038pt}}
\multiput(1249.17,209.00)(32.000,72.847){2}{\rule{0.400pt}{0.519pt}}
\multiput(1282.58,280.85)(0.497,-0.825){59}{\rule{0.120pt}{0.758pt}}
\multiput(1281.17,282.43)(31.000,-49.427){2}{\rule{0.400pt}{0.379pt}}
\multiput(1313.58,230.09)(0.497,-0.752){61}{\rule{0.120pt}{0.700pt}}
\multiput(1312.17,231.55)(32.000,-46.547){2}{\rule{0.400pt}{0.350pt}}
\multiput(1345.58,181.85)(0.497,-0.825){59}{\rule{0.120pt}{0.758pt}}
\multiput(1344.17,183.43)(31.000,-49.427){2}{\rule{0.400pt}{0.379pt}}
\multiput(1376.00,132.92)(0.668,-0.496){45}{\rule{0.633pt}{0.120pt}}
\multiput(1376.00,133.17)(30.685,-24.000){2}{\rule{0.317pt}{0.400pt}}
\multiput(1408.58,110.00)(0.497,0.646){59}{\rule{0.120pt}{0.616pt}}
\multiput(1407.17,110.00)(31.000,38.721){2}{\rule{0.400pt}{0.308pt}}
\put(211,512){\makebox(0,0){$+$}}
\put(242,348){\makebox(0,0){$+$}}
\put(274,348){\makebox(0,0){$+$}}
\put(305,520){\makebox(0,0){$+$}}
\put(337,645){\makebox(0,0){$+$}}
\put(368,549){\makebox(0,0){$+$}}
\put(400,446){\makebox(0,0){$+$}}
\put(431,381){\makebox(0,0){$+$}}
\put(463,403){\makebox(0,0){$+$}}
\put(494,455){\makebox(0,0){$+$}}
\put(526,348){\makebox(0,0){$+$}}
\put(557,354){\makebox(0,0){$+$}}
\put(589,364){\makebox(0,0){$+$}}
\put(620,384){\makebox(0,0){$+$}}
\put(652,349){\makebox(0,0){$+$}}
\put(683,403){\makebox(0,0){$+$}}
\put(715,461){\makebox(0,0){$+$}}
\put(746,291){\makebox(0,0){$+$}}
\put(778,362){\makebox(0,0){$+$}}
\put(809,296){\makebox(0,0){$+$}}
\put(841,407){\makebox(0,0){$+$}}
\put(872,392){\makebox(0,0){$+$}}
\put(904,421){\makebox(0,0){$+$}}
\put(935,302){\makebox(0,0){$+$}}
\put(967,259){\makebox(0,0){$+$}}
\put(998,416){\makebox(0,0){$+$}}
\put(1030,322){\makebox(0,0){$+$}}
\put(1061,417){\makebox(0,0){$+$}}
\put(1093,379){\makebox(0,0){$+$}}
\put(1124,292){\makebox(0,0){$+$}}
\put(1156,316){\makebox(0,0){$+$}}
\put(1187,216){\makebox(0,0){$+$}}
\put(1219,201){\makebox(0,0){$+$}}
\put(1250,209){\makebox(0,0){$+$}}
\put(1282,284){\makebox(0,0){$+$}}
\put(1313,233){\makebox(0,0){$+$}}
\put(1345,185){\makebox(0,0){$+$}}
\put(1376,134){\makebox(0,0){$+$}}
\put(1408,110){\makebox(0,0){$+$}}
\put(1439,150){\makebox(0,0){$+$}}
\put(1349,768){\makebox(0,0){$+$}}
\put(242.0,348.0){\rule[-0.200pt]{7.709pt}{0.400pt}}
\put(211.0,82.0){\rule[-0.200pt]{0.400pt}{187.179pt}}
\put(211.0,82.0){\rule[-0.200pt]{295.825pt}{0.400pt}}
\put(1439.0,82.0){\rule[-0.200pt]{0.400pt}{187.179pt}}
\put(211.0,859.0){\rule[-0.200pt]{295.825pt}{0.400pt}}
\end{picture}
} 
   \end{frame}
   \begin{frame}{敏感性分析}
     \begin{block}{两个主要目的}
1)加深对模型的理解,促进模型描述从定性到定量;2)进行政策调控仿
真,分析参数变化对系统行为的影响。
     \end{block} \vfill
     \begin{block}{三种类别}
       数值敏感性、行为敏感性以及政策敏感性。数值敏感性关注于参
数和模型输出结果的精确性;行为敏感性关注参数值的改变是否会改变系统行
为;政策敏感性则关注当模型假设发生改变时最优策略的变化情况。
     \end{block}
   \end{frame}

   \begin{frame}{参数选择}
     由于系统对于大多数参数是不敏感的,因此对所有的参数进行敏感性分析是不必
要的。针对不同参数的敏感程度排序,快速筛选出那些最敏感、可能改变系统行
为的变量是非常重要的。Ford和Flynn提出使用Pearson相关分析进行进行快速筛
选。通过计算模型参数与输出结果之间的Pearson相关系数,选择系数较大的参
数进行敏感性分析。相关系数越大,说明参数对于输出结果的影响越大,其变化
越有可能改变系统行为。
   \end{frame}

   \begin{frame}{测试参数}
     \begin{enumerate}
\item 领导者用户。没有符合条件的参数,既模型对于所有动机因素的变化均不
  敏感。
\item 领域专家用户。成就需求是协同行为的主要影响因素,将分析其变化对模
  型的影响。
\item 内容贡献者。没有符合条件的参数,既模型对于所有动机因素的变化均不
  敏感。
\item 内容维护者。自我效能和认知失调是协同行为的主要影响因素,将分析其变化对模
  型的影响。
\item 边缘用户。自我效能和归属感是协同行为的主要影响因素,将分析其变化对模
  型的影响。
\end{enumerate}
   \end{frame}

   \begin{frame}
     提升领导者用户初始动机的仿真结果
\scalebox{0.8}{% GNUPLOT: LaTeX picture
\setlength{\unitlength}{0.240900pt}
\ifx\plotpoint\undefined\newsavebox{\plotpoint}\fi
\sbox{\plotpoint}{\rule[-0.200pt]{0.400pt}{0.400pt}}%
\begin{picture}(1500,900)(0,0)
\sbox{\plotpoint}{\rule[-0.200pt]{0.400pt}{0.400pt}}%
\put(131.0,82.0){\rule[-0.200pt]{4.818pt}{0.400pt}}
\put(111,82){\makebox(0,0)[r]{10}}
\put(1419.0,82.0){\rule[-0.200pt]{4.818pt}{0.400pt}}
\put(131.0,211.0){\rule[-0.200pt]{4.818pt}{0.400pt}}
\put(111,211){\makebox(0,0)[r]{15}}
\put(1419.0,211.0){\rule[-0.200pt]{4.818pt}{0.400pt}}
\put(131.0,341.0){\rule[-0.200pt]{4.818pt}{0.400pt}}
\put(111,341){\makebox(0,0)[r]{20}}
\put(1419.0,341.0){\rule[-0.200pt]{4.818pt}{0.400pt}}
\put(131.0,470.0){\rule[-0.200pt]{4.818pt}{0.400pt}}
\put(111,470){\makebox(0,0)[r]{25}}
\put(1419.0,470.0){\rule[-0.200pt]{4.818pt}{0.400pt}}
\put(131.0,600.0){\rule[-0.200pt]{4.818pt}{0.400pt}}
\put(111,600){\makebox(0,0)[r]{30}}
\put(1419.0,600.0){\rule[-0.200pt]{4.818pt}{0.400pt}}
\put(131.0,729.0){\rule[-0.200pt]{4.818pt}{0.400pt}}
\put(111,729){\makebox(0,0)[r]{35}}
\put(1419.0,729.0){\rule[-0.200pt]{4.818pt}{0.400pt}}
\put(131.0,859.0){\rule[-0.200pt]{4.818pt}{0.400pt}}
\put(111,859){\makebox(0,0)[r]{40}}
\put(1419.0,859.0){\rule[-0.200pt]{4.818pt}{0.400pt}}
\put(131.0,82.0){\rule[-0.200pt]{0.400pt}{4.818pt}}
\put(131,41){\makebox(0,0){ 1}}
\put(131.0,839.0){\rule[-0.200pt]{0.400pt}{4.818pt}}
\put(198.0,82.0){\rule[-0.200pt]{0.400pt}{4.818pt}}
\put(198,41){\makebox(0,0){ 3}}
\put(198.0,839.0){\rule[-0.200pt]{0.400pt}{4.818pt}}
\put(265.0,82.0){\rule[-0.200pt]{0.400pt}{4.818pt}}
\put(265,41){\makebox(0,0){ 5}}
\put(265.0,839.0){\rule[-0.200pt]{0.400pt}{4.818pt}}
\put(332.0,82.0){\rule[-0.200pt]{0.400pt}{4.818pt}}
\put(332,41){\makebox(0,0){ 7}}
\put(332.0,839.0){\rule[-0.200pt]{0.400pt}{4.818pt}}
\put(399.0,82.0){\rule[-0.200pt]{0.400pt}{4.818pt}}
\put(399,41){\makebox(0,0){ 9}}
\put(399.0,839.0){\rule[-0.200pt]{0.400pt}{4.818pt}}
\put(466.0,82.0){\rule[-0.200pt]{0.400pt}{4.818pt}}
\put(466,41){\makebox(0,0){ 11}}
\put(466.0,839.0){\rule[-0.200pt]{0.400pt}{4.818pt}}
\put(533.0,82.0){\rule[-0.200pt]{0.400pt}{4.818pt}}
\put(533,41){\makebox(0,0){ 13}}
\put(533.0,839.0){\rule[-0.200pt]{0.400pt}{4.818pt}}
\put(601.0,82.0){\rule[-0.200pt]{0.400pt}{4.818pt}}
\put(601,41){\makebox(0,0){ 15}}
\put(601.0,839.0){\rule[-0.200pt]{0.400pt}{4.818pt}}
\put(668.0,82.0){\rule[-0.200pt]{0.400pt}{4.818pt}}
\put(668,41){\makebox(0,0){ 17}}
\put(668.0,839.0){\rule[-0.200pt]{0.400pt}{4.818pt}}
\put(735.0,82.0){\rule[-0.200pt]{0.400pt}{4.818pt}}
\put(735,41){\makebox(0,0){ 19}}
\put(735.0,839.0){\rule[-0.200pt]{0.400pt}{4.818pt}}
\put(802.0,82.0){\rule[-0.200pt]{0.400pt}{4.818pt}}
\put(802,41){\makebox(0,0){ 21}}
\put(802.0,839.0){\rule[-0.200pt]{0.400pt}{4.818pt}}
\put(869.0,82.0){\rule[-0.200pt]{0.400pt}{4.818pt}}
\put(869,41){\makebox(0,0){ 23}}
\put(869.0,839.0){\rule[-0.200pt]{0.400pt}{4.818pt}}
\put(936.0,82.0){\rule[-0.200pt]{0.400pt}{4.818pt}}
\put(936,41){\makebox(0,0){ 25}}
\put(936.0,839.0){\rule[-0.200pt]{0.400pt}{4.818pt}}
\put(1003.0,82.0){\rule[-0.200pt]{0.400pt}{4.818pt}}
\put(1003,41){\makebox(0,0){ 27}}
\put(1003.0,839.0){\rule[-0.200pt]{0.400pt}{4.818pt}}
\put(1070.0,82.0){\rule[-0.200pt]{0.400pt}{4.818pt}}
\put(1070,41){\makebox(0,0){ 29}}
\put(1070.0,839.0){\rule[-0.200pt]{0.400pt}{4.818pt}}
\put(1137.0,82.0){\rule[-0.200pt]{0.400pt}{4.818pt}}
\put(1137,41){\makebox(0,0){ 31}}
\put(1137.0,839.0){\rule[-0.200pt]{0.400pt}{4.818pt}}
\put(1204.0,82.0){\rule[-0.200pt]{0.400pt}{4.818pt}}
\put(1204,41){\makebox(0,0){ 33}}
\put(1204.0,839.0){\rule[-0.200pt]{0.400pt}{4.818pt}}
\put(1271.0,82.0){\rule[-0.200pt]{0.400pt}{4.818pt}}
\put(1271,41){\makebox(0,0){ 35}}
\put(1271.0,839.0){\rule[-0.200pt]{0.400pt}{4.818pt}}
\put(1338.0,82.0){\rule[-0.200pt]{0.400pt}{4.818pt}}
\put(1338,41){\makebox(0,0){ 37}}
\put(1338.0,839.0){\rule[-0.200pt]{0.400pt}{4.818pt}}
\put(1405.0,82.0){\rule[-0.200pt]{0.400pt}{4.818pt}}
\put(1405,41){\makebox(0,0){ 39}}
\put(1405.0,839.0){\rule[-0.200pt]{0.400pt}{4.818pt}}
\put(131.0,82.0){\rule[-0.200pt]{0.400pt}{187.179pt}}
\put(131.0,82.0){\rule[-0.200pt]{315.097pt}{0.400pt}}
\put(1439.0,82.0){\rule[-0.200pt]{0.400pt}{187.179pt}}
\put(131.0,859.0){\rule[-0.200pt]{315.097pt}{0.400pt}}
\put(30,470){\makebox(0,0){\rotatebox{90}{用户贡献}}}
\put(1279,819){\makebox(0,0)[r]{仿真数据}}
\put(1299.0,819.0){\rule[-0.200pt]{24.090pt}{0.400pt}}
\put(131,139){\usebox{\plotpoint}}
\multiput(131.58,139.00)(0.498,1.316){65}{\rule{0.120pt}{1.147pt}}
\multiput(130.17,139.00)(34.000,86.619){2}{\rule{0.400pt}{0.574pt}}
\multiput(165.58,228.00)(0.497,0.790){63}{\rule{0.120pt}{0.730pt}}
\multiput(164.17,228.00)(33.000,50.484){2}{\rule{0.400pt}{0.365pt}}
\multiput(198.58,280.00)(0.498,0.543){65}{\rule{0.120pt}{0.535pt}}
\multiput(197.17,280.00)(34.000,35.889){2}{\rule{0.400pt}{0.268pt}}
\multiput(232.00,317.58)(0.568,0.497){55}{\rule{0.555pt}{0.120pt}}
\multiput(232.00,316.17)(31.848,29.000){2}{\rule{0.278pt}{0.400pt}}
\multiput(265.00,346.58)(0.710,0.496){45}{\rule{0.667pt}{0.120pt}}
\multiput(265.00,345.17)(32.616,24.000){2}{\rule{0.333pt}{0.400pt}}
\multiput(299.00,370.58)(0.874,0.495){35}{\rule{0.795pt}{0.119pt}}
\multiput(299.00,369.17)(31.350,19.000){2}{\rule{0.397pt}{0.400pt}}
\multiput(332.00,389.58)(0.952,0.495){33}{\rule{0.856pt}{0.119pt}}
\multiput(332.00,388.17)(32.224,18.000){2}{\rule{0.428pt}{0.400pt}}
\multiput(366.00,407.58)(1.113,0.494){27}{\rule{0.980pt}{0.119pt}}
\multiput(366.00,406.17)(30.966,15.000){2}{\rule{0.490pt}{0.400pt}}
\multiput(399.00,422.58)(1.329,0.493){23}{\rule{1.146pt}{0.119pt}}
\multiput(399.00,421.17)(31.621,13.000){2}{\rule{0.573pt}{0.400pt}}
\multiput(433.00,435.58)(1.290,0.493){23}{\rule{1.115pt}{0.119pt}}
\multiput(433.00,434.17)(30.685,13.000){2}{\rule{0.558pt}{0.400pt}}
\multiput(466.00,448.58)(1.581,0.492){19}{\rule{1.336pt}{0.118pt}}
\multiput(466.00,447.17)(31.226,11.000){2}{\rule{0.668pt}{0.400pt}}
\multiput(500.00,459.58)(1.694,0.491){17}{\rule{1.420pt}{0.118pt}}
\multiput(500.00,458.17)(30.053,10.000){2}{\rule{0.710pt}{0.400pt}}
\multiput(533.00,469.58)(1.746,0.491){17}{\rule{1.460pt}{0.118pt}}
\multiput(533.00,468.17)(30.970,10.000){2}{\rule{0.730pt}{0.400pt}}
\multiput(567.00,479.59)(1.951,0.489){15}{\rule{1.611pt}{0.118pt}}
\multiput(567.00,478.17)(30.656,9.000){2}{\rule{0.806pt}{0.400pt}}
\multiput(601.00,488.59)(2.145,0.488){13}{\rule{1.750pt}{0.117pt}}
\multiput(601.00,487.17)(29.368,8.000){2}{\rule{0.875pt}{0.400pt}}
\multiput(634.00,496.59)(2.211,0.488){13}{\rule{1.800pt}{0.117pt}}
\multiput(634.00,495.17)(30.264,8.000){2}{\rule{0.900pt}{0.400pt}}
\multiput(668.00,504.59)(2.476,0.485){11}{\rule{1.986pt}{0.117pt}}
\multiput(668.00,503.17)(28.879,7.000){2}{\rule{0.993pt}{0.400pt}}
\multiput(701.00,511.59)(2.552,0.485){11}{\rule{2.043pt}{0.117pt}}
\multiput(701.00,510.17)(29.760,7.000){2}{\rule{1.021pt}{0.400pt}}
\multiput(735.00,518.59)(2.476,0.485){11}{\rule{1.986pt}{0.117pt}}
\multiput(735.00,517.17)(28.879,7.000){2}{\rule{0.993pt}{0.400pt}}
\multiput(768.00,525.59)(3.022,0.482){9}{\rule{2.367pt}{0.116pt}}
\multiput(768.00,524.17)(29.088,6.000){2}{\rule{1.183pt}{0.400pt}}
\multiput(802.00,531.59)(2.932,0.482){9}{\rule{2.300pt}{0.116pt}}
\multiput(802.00,530.17)(28.226,6.000){2}{\rule{1.150pt}{0.400pt}}
\multiput(835.00,537.59)(3.022,0.482){9}{\rule{2.367pt}{0.116pt}}
\multiput(835.00,536.17)(29.088,6.000){2}{\rule{1.183pt}{0.400pt}}
\multiput(869.00,543.59)(3.604,0.477){7}{\rule{2.740pt}{0.115pt}}
\multiput(869.00,542.17)(27.313,5.000){2}{\rule{1.370pt}{0.400pt}}
\multiput(902.00,548.59)(3.716,0.477){7}{\rule{2.820pt}{0.115pt}}
\multiput(902.00,547.17)(28.147,5.000){2}{\rule{1.410pt}{0.400pt}}
\multiput(936.00,553.59)(2.932,0.482){9}{\rule{2.300pt}{0.116pt}}
\multiput(936.00,552.17)(28.226,6.000){2}{\rule{1.150pt}{0.400pt}}
\multiput(969.00,559.60)(4.868,0.468){5}{\rule{3.500pt}{0.113pt}}
\multiput(969.00,558.17)(26.736,4.000){2}{\rule{1.750pt}{0.400pt}}
\multiput(1003.00,563.59)(3.716,0.477){7}{\rule{2.820pt}{0.115pt}}
\multiput(1003.00,562.17)(28.147,5.000){2}{\rule{1.410pt}{0.400pt}}
\multiput(1037.00,568.59)(3.604,0.477){7}{\rule{2.740pt}{0.115pt}}
\multiput(1037.00,567.17)(27.313,5.000){2}{\rule{1.370pt}{0.400pt}}
\multiput(1070.00,573.60)(4.868,0.468){5}{\rule{3.500pt}{0.113pt}}
\multiput(1070.00,572.17)(26.736,4.000){2}{\rule{1.750pt}{0.400pt}}
\multiput(1104.00,577.60)(4.722,0.468){5}{\rule{3.400pt}{0.113pt}}
\multiput(1104.00,576.17)(25.943,4.000){2}{\rule{1.700pt}{0.400pt}}
\multiput(1137.00,581.60)(4.868,0.468){5}{\rule{3.500pt}{0.113pt}}
\multiput(1137.00,580.17)(26.736,4.000){2}{\rule{1.750pt}{0.400pt}}
\multiput(1171.00,585.60)(4.722,0.468){5}{\rule{3.400pt}{0.113pt}}
\multiput(1171.00,584.17)(25.943,4.000){2}{\rule{1.700pt}{0.400pt}}
\multiput(1204.00,589.60)(4.868,0.468){5}{\rule{3.500pt}{0.113pt}}
\multiput(1204.00,588.17)(26.736,4.000){2}{\rule{1.750pt}{0.400pt}}
\multiput(1238.00,593.60)(4.722,0.468){5}{\rule{3.400pt}{0.113pt}}
\multiput(1238.00,592.17)(25.943,4.000){2}{\rule{1.700pt}{0.400pt}}
\multiput(1271.00,597.61)(7.383,0.447){3}{\rule{4.633pt}{0.108pt}}
\multiput(1271.00,596.17)(24.383,3.000){2}{\rule{2.317pt}{0.400pt}}
\multiput(1305.00,600.60)(4.722,0.468){5}{\rule{3.400pt}{0.113pt}}
\multiput(1305.00,599.17)(25.943,4.000){2}{\rule{1.700pt}{0.400pt}}
\multiput(1338.00,604.61)(7.383,0.447){3}{\rule{4.633pt}{0.108pt}}
\multiput(1338.00,603.17)(24.383,3.000){2}{\rule{2.317pt}{0.400pt}}
\multiput(1372.00,607.60)(4.722,0.468){5}{\rule{3.400pt}{0.113pt}}
\multiput(1372.00,606.17)(25.943,4.000){2}{\rule{1.700pt}{0.400pt}}
\multiput(1405.00,611.61)(7.383,0.447){3}{\rule{4.633pt}{0.108pt}}
\multiput(1405.00,610.17)(24.383,3.000){2}{\rule{2.317pt}{0.400pt}}
\put(1279,768){\makebox(0,0)[r]{改进数据}}
\put(1299.0,768.0){\rule[-0.200pt]{24.090pt}{0.400pt}}
\put(131,150){\usebox{\plotpoint}}
\multiput(131.58,150.00)(0.498,0.960){65}{\rule{0.120pt}{0.865pt}}
\multiput(130.17,150.00)(34.000,63.205){2}{\rule{0.400pt}{0.432pt}}
\multiput(165.58,215.00)(0.497,0.774){63}{\rule{0.120pt}{0.718pt}}
\multiput(164.17,215.00)(33.000,49.509){2}{\rule{0.400pt}{0.359pt}}
\multiput(198.58,266.00)(0.498,0.930){65}{\rule{0.120pt}{0.841pt}}
\multiput(197.17,266.00)(34.000,61.254){2}{\rule{0.400pt}{0.421pt}}
\multiput(232.00,329.58)(0.661,0.497){47}{\rule{0.628pt}{0.120pt}}
\multiput(232.00,328.17)(31.697,25.000){2}{\rule{0.314pt}{0.400pt}}
\put(265,352.67){\rule{8.191pt}{0.400pt}}
\multiput(265.00,353.17)(17.000,-1.000){2}{\rule{4.095pt}{0.400pt}}
\multiput(299.00,353.58)(0.752,0.496){41}{\rule{0.700pt}{0.120pt}}
\multiput(299.00,352.17)(31.547,22.000){2}{\rule{0.350pt}{0.400pt}}
\multiput(332.58,375.00)(0.498,0.633){65}{\rule{0.120pt}{0.606pt}}
\multiput(331.17,375.00)(34.000,41.742){2}{\rule{0.400pt}{0.303pt}}
\multiput(366.00,416.93)(2.932,-0.482){9}{\rule{2.300pt}{0.116pt}}
\multiput(366.00,417.17)(28.226,-6.000){2}{\rule{1.150pt}{0.400pt}}
\multiput(399.00,412.58)(0.630,0.497){51}{\rule{0.604pt}{0.120pt}}
\multiput(399.00,411.17)(32.747,27.000){2}{\rule{0.302pt}{0.400pt}}
\multiput(433.00,439.58)(0.752,0.496){41}{\rule{0.700pt}{0.120pt}}
\multiput(433.00,438.17)(31.547,22.000){2}{\rule{0.350pt}{0.400pt}}
\multiput(466.00,461.59)(2.552,0.485){11}{\rule{2.043pt}{0.117pt}}
\multiput(466.00,460.17)(29.760,7.000){2}{\rule{1.021pt}{0.400pt}}
\multiput(500.00,468.58)(0.923,0.495){33}{\rule{0.833pt}{0.119pt}}
\multiput(500.00,467.17)(31.270,18.000){2}{\rule{0.417pt}{0.400pt}}
\put(533,484.17){\rule{6.900pt}{0.400pt}}
\multiput(533.00,485.17)(19.679,-2.000){2}{\rule{3.450pt}{0.400pt}}
\multiput(567.00,484.58)(1.746,0.491){17}{\rule{1.460pt}{0.118pt}}
\multiput(567.00,483.17)(30.970,10.000){2}{\rule{0.730pt}{0.400pt}}
\multiput(601.00,492.92)(0.874,-0.495){35}{\rule{0.795pt}{0.119pt}}
\multiput(601.00,493.17)(31.350,-19.000){2}{\rule{0.397pt}{0.400pt}}
\multiput(634.58,475.00)(0.498,0.558){65}{\rule{0.120pt}{0.547pt}}
\multiput(633.17,475.00)(34.000,36.865){2}{\rule{0.400pt}{0.274pt}}
\put(668,511.67){\rule{7.950pt}{0.400pt}}
\multiput(668.00,512.17)(16.500,-1.000){2}{\rule{3.975pt}{0.400pt}}
\multiput(701.00,512.59)(2.552,0.485){11}{\rule{2.043pt}{0.117pt}}
\multiput(701.00,511.17)(29.760,7.000){2}{\rule{1.021pt}{0.400pt}}
\put(735,518.67){\rule{7.950pt}{0.400pt}}
\multiput(735.00,518.17)(16.500,1.000){2}{\rule{3.975pt}{0.400pt}}
\multiput(768.00,520.60)(4.868,0.468){5}{\rule{3.500pt}{0.113pt}}
\multiput(768.00,519.17)(26.736,4.000){2}{\rule{1.750pt}{0.400pt}}
\multiput(802.00,524.58)(0.568,0.497){55}{\rule{0.555pt}{0.120pt}}
\multiput(802.00,523.17)(31.848,29.000){2}{\rule{0.278pt}{0.400pt}}
\multiput(835.00,553.60)(4.868,0.468){5}{\rule{3.500pt}{0.113pt}}
\multiput(835.00,552.17)(26.736,4.000){2}{\rule{1.750pt}{0.400pt}}
\multiput(869.00,557.58)(1.401,0.492){21}{\rule{1.200pt}{0.119pt}}
\multiput(869.00,556.17)(30.509,12.000){2}{\rule{0.600pt}{0.400pt}}
\multiput(902.00,567.93)(1.951,-0.489){15}{\rule{1.611pt}{0.118pt}}
\multiput(902.00,568.17)(30.656,-9.000){2}{\rule{0.806pt}{0.400pt}}
\multiput(936.00,560.58)(1.534,0.492){19}{\rule{1.300pt}{0.118pt}}
\multiput(936.00,559.17)(30.302,11.000){2}{\rule{0.650pt}{0.400pt}}
\multiput(969.00,569.93)(2.211,-0.488){13}{\rule{1.800pt}{0.117pt}}
\multiput(969.00,570.17)(30.264,-8.000){2}{\rule{0.900pt}{0.400pt}}
\multiput(1003.00,563.58)(1.073,0.494){29}{\rule{0.950pt}{0.119pt}}
\multiput(1003.00,562.17)(32.028,16.000){2}{\rule{0.475pt}{0.400pt}}
\multiput(1037.00,579.58)(0.979,0.495){31}{\rule{0.876pt}{0.119pt}}
\multiput(1037.00,578.17)(31.181,17.000){2}{\rule{0.438pt}{0.400pt}}
\multiput(1070.00,596.61)(7.383,0.447){3}{\rule{4.633pt}{0.108pt}}
\multiput(1070.00,595.17)(24.383,3.000){2}{\rule{2.317pt}{0.400pt}}
\multiput(1104.00,597.92)(0.789,-0.496){39}{\rule{0.729pt}{0.119pt}}
\multiput(1104.00,598.17)(31.488,-21.000){2}{\rule{0.364pt}{0.400pt}}
\multiput(1137.00,578.58)(1.746,0.491){17}{\rule{1.460pt}{0.118pt}}
\multiput(1137.00,577.17)(30.970,10.000){2}{\rule{0.730pt}{0.400pt}}
\multiput(1171.00,586.92)(0.979,-0.495){31}{\rule{0.876pt}{0.119pt}}
\multiput(1171.00,587.17)(31.181,-17.000){2}{\rule{0.438pt}{0.400pt}}
\multiput(1204.00,571.58)(0.499,0.498){65}{\rule{0.500pt}{0.120pt}}
\multiput(1204.00,570.17)(32.962,34.000){2}{\rule{0.250pt}{0.400pt}}
\multiput(1238.00,605.58)(1.290,0.493){23}{\rule{1.115pt}{0.119pt}}
\multiput(1238.00,604.17)(30.685,13.000){2}{\rule{0.558pt}{0.400pt}}
\multiput(1271.00,616.92)(1.444,-0.492){21}{\rule{1.233pt}{0.119pt}}
\multiput(1271.00,617.17)(31.440,-12.000){2}{\rule{0.617pt}{0.400pt}}
\put(1305,604.17){\rule{6.700pt}{0.400pt}}
\multiput(1305.00,605.17)(19.094,-2.000){2}{\rule{3.350pt}{0.400pt}}
\multiput(1338.00,604.59)(1.951,0.489){15}{\rule{1.611pt}{0.118pt}}
\multiput(1338.00,603.17)(30.656,9.000){2}{\rule{0.806pt}{0.400pt}}
\multiput(1372.00,613.59)(2.932,0.482){9}{\rule{2.300pt}{0.116pt}}
\multiput(1372.00,612.17)(28.226,6.000){2}{\rule{1.150pt}{0.400pt}}
\put(1405,617.17){\rule{6.900pt}{0.400pt}}
\multiput(1405.00,618.17)(19.679,-2.000){2}{\rule{3.450pt}{0.400pt}}
\put(131,150){\makebox(0,0){$+$}}
\put(165,215){\makebox(0,0){$+$}}
\put(198,266){\makebox(0,0){$+$}}
\put(232,329){\makebox(0,0){$+$}}
\put(265,354){\makebox(0,0){$+$}}
\put(299,353){\makebox(0,0){$+$}}
\put(332,375){\makebox(0,0){$+$}}
\put(366,418){\makebox(0,0){$+$}}
\put(399,412){\makebox(0,0){$+$}}
\put(433,439){\makebox(0,0){$+$}}
\put(466,461){\makebox(0,0){$+$}}
\put(500,468){\makebox(0,0){$+$}}
\put(533,486){\makebox(0,0){$+$}}
\put(567,484){\makebox(0,0){$+$}}
\put(601,494){\makebox(0,0){$+$}}
\put(634,475){\makebox(0,0){$+$}}
\put(668,513){\makebox(0,0){$+$}}
\put(701,512){\makebox(0,0){$+$}}
\put(735,519){\makebox(0,0){$+$}}
\put(768,520){\makebox(0,0){$+$}}
\put(802,524){\makebox(0,0){$+$}}
\put(835,553){\makebox(0,0){$+$}}
\put(869,557){\makebox(0,0){$+$}}
\put(902,569){\makebox(0,0){$+$}}
\put(936,560){\makebox(0,0){$+$}}
\put(969,571){\makebox(0,0){$+$}}
\put(1003,563){\makebox(0,0){$+$}}
\put(1037,579){\makebox(0,0){$+$}}
\put(1070,596){\makebox(0,0){$+$}}
\put(1104,599){\makebox(0,0){$+$}}
\put(1137,578){\makebox(0,0){$+$}}
\put(1171,588){\makebox(0,0){$+$}}
\put(1204,571){\makebox(0,0){$+$}}
\put(1238,605){\makebox(0,0){$+$}}
\put(1271,618){\makebox(0,0){$+$}}
\put(1305,606){\makebox(0,0){$+$}}
\put(1338,604){\makebox(0,0){$+$}}
\put(1372,613){\makebox(0,0){$+$}}
\put(1405,619){\makebox(0,0){$+$}}
\put(1439,617){\makebox(0,0){$+$}}
\put(1349,778){\makebox(0,0){$+$}}
\put(131.0,82.0){\rule[-0.200pt]{0.400pt}{187.179pt}}
\put(131.0,82.0){\rule[-0.200pt]{315.097pt}{0.400pt}}
\put(1439.0,82.0){\rule[-0.200pt]{0.400pt}{187.179pt}}
\put(131.0,859.0){\rule[-0.200pt]{315.097pt}{0.400pt}}
\end{picture}
}
   \end{frame}

   \begin{frame}
     提升领导者用户最大工作时间的仿真结果
\scalebox{0.8}{% GNUPLOT: LaTeX picture
\setlength{\unitlength}{0.240900pt}
\ifx\plotpoint\undefined\newsavebox{\plotpoint}\fi
\begin{picture}(1500,900)(0,0)
\sbox{\plotpoint}{\rule[-0.200pt]{0.400pt}{0.400pt}}%
\put(131.0,131.0){\rule[-0.200pt]{4.818pt}{0.400pt}}
\put(111,131){\makebox(0,0)[r]{10}}
\put(1419.0,131.0){\rule[-0.200pt]{4.818pt}{0.400pt}}
\put(131.0,252.0){\rule[-0.200pt]{4.818pt}{0.400pt}}
\put(111,252){\makebox(0,0)[r]{15}}
\put(1419.0,252.0){\rule[-0.200pt]{4.818pt}{0.400pt}}
\put(131.0,374.0){\rule[-0.200pt]{4.818pt}{0.400pt}}
\put(111,374){\makebox(0,0)[r]{20}}
\put(1419.0,374.0){\rule[-0.200pt]{4.818pt}{0.400pt}}
\put(131.0,495.0){\rule[-0.200pt]{4.818pt}{0.400pt}}
\put(111,495){\makebox(0,0)[r]{25}}
\put(1419.0,495.0){\rule[-0.200pt]{4.818pt}{0.400pt}}
\put(131.0,616.0){\rule[-0.200pt]{4.818pt}{0.400pt}}
\put(111,616){\makebox(0,0)[r]{30}}
\put(1419.0,616.0){\rule[-0.200pt]{4.818pt}{0.400pt}}
\put(131.0,738.0){\rule[-0.200pt]{4.818pt}{0.400pt}}
\put(111,738){\makebox(0,0)[r]{35}}
\put(1419.0,738.0){\rule[-0.200pt]{4.818pt}{0.400pt}}
\put(131.0,859.0){\rule[-0.200pt]{4.818pt}{0.400pt}}
\put(111,859){\makebox(0,0)[r]{40}}
\put(1419.0,859.0){\rule[-0.200pt]{4.818pt}{0.400pt}}
\put(131.0,131.0){\rule[-0.200pt]{0.400pt}{4.818pt}}
\put(131,90){\makebox(0,0){ 1}}
\put(131.0,839.0){\rule[-0.200pt]{0.400pt}{4.818pt}}
\put(198.0,131.0){\rule[-0.200pt]{0.400pt}{4.818pt}}
\put(198,90){\makebox(0,0){ 3}}
\put(198.0,839.0){\rule[-0.200pt]{0.400pt}{4.818pt}}
\put(265.0,131.0){\rule[-0.200pt]{0.400pt}{4.818pt}}
\put(265,90){\makebox(0,0){ 5}}
\put(265.0,839.0){\rule[-0.200pt]{0.400pt}{4.818pt}}
\put(332.0,131.0){\rule[-0.200pt]{0.400pt}{4.818pt}}
\put(332,90){\makebox(0,0){ 7}}
\put(332.0,839.0){\rule[-0.200pt]{0.400pt}{4.818pt}}
\put(399.0,131.0){\rule[-0.200pt]{0.400pt}{4.818pt}}
\put(399,90){\makebox(0,0){ 9}}
\put(399.0,839.0){\rule[-0.200pt]{0.400pt}{4.818pt}}
\put(466.0,131.0){\rule[-0.200pt]{0.400pt}{4.818pt}}
\put(466,90){\makebox(0,0){ 11}}
\put(466.0,839.0){\rule[-0.200pt]{0.400pt}{4.818pt}}
\put(533.0,131.0){\rule[-0.200pt]{0.400pt}{4.818pt}}
\put(533,90){\makebox(0,0){ 13}}
\put(533.0,839.0){\rule[-0.200pt]{0.400pt}{4.818pt}}
\put(601.0,131.0){\rule[-0.200pt]{0.400pt}{4.818pt}}
\put(601,90){\makebox(0,0){ 15}}
\put(601.0,839.0){\rule[-0.200pt]{0.400pt}{4.818pt}}
\put(668.0,131.0){\rule[-0.200pt]{0.400pt}{4.818pt}}
\put(668,90){\makebox(0,0){ 17}}
\put(668.0,839.0){\rule[-0.200pt]{0.400pt}{4.818pt}}
\put(735.0,131.0){\rule[-0.200pt]{0.400pt}{4.818pt}}
\put(735,90){\makebox(0,0){ 19}}
\put(735.0,839.0){\rule[-0.200pt]{0.400pt}{4.818pt}}
\put(802.0,131.0){\rule[-0.200pt]{0.400pt}{4.818pt}}
\put(802,90){\makebox(0,0){ 21}}
\put(802.0,839.0){\rule[-0.200pt]{0.400pt}{4.818pt}}
\put(869.0,131.0){\rule[-0.200pt]{0.400pt}{4.818pt}}
\put(869,90){\makebox(0,0){ 23}}
\put(869.0,839.0){\rule[-0.200pt]{0.400pt}{4.818pt}}
\put(936.0,131.0){\rule[-0.200pt]{0.400pt}{4.818pt}}
\put(936,90){\makebox(0,0){ 25}}
\put(936.0,839.0){\rule[-0.200pt]{0.400pt}{4.818pt}}
\put(1003.0,131.0){\rule[-0.200pt]{0.400pt}{4.818pt}}
\put(1003,90){\makebox(0,0){ 27}}
\put(1003.0,839.0){\rule[-0.200pt]{0.400pt}{4.818pt}}
\put(1070.0,131.0){\rule[-0.200pt]{0.400pt}{4.818pt}}
\put(1070,90){\makebox(0,0){ 29}}
\put(1070.0,839.0){\rule[-0.200pt]{0.400pt}{4.818pt}}
\put(1137.0,131.0){\rule[-0.200pt]{0.400pt}{4.818pt}}
\put(1137,90){\makebox(0,0){ 31}}
\put(1137.0,839.0){\rule[-0.200pt]{0.400pt}{4.818pt}}
\put(1204.0,131.0){\rule[-0.200pt]{0.400pt}{4.818pt}}
\put(1204,90){\makebox(0,0){ 33}}
\put(1204.0,839.0){\rule[-0.200pt]{0.400pt}{4.818pt}}
\put(1271.0,131.0){\rule[-0.200pt]{0.400pt}{4.818pt}}
\put(1271,90){\makebox(0,0){ 35}}
\put(1271.0,839.0){\rule[-0.200pt]{0.400pt}{4.818pt}}
\put(1338.0,131.0){\rule[-0.200pt]{0.400pt}{4.818pt}}
\put(1338,90){\makebox(0,0){ 37}}
\put(1338.0,839.0){\rule[-0.200pt]{0.400pt}{4.818pt}}
\put(1405.0,131.0){\rule[-0.200pt]{0.400pt}{4.818pt}}
\put(1405,90){\makebox(0,0){ 39}}
\put(1405.0,839.0){\rule[-0.200pt]{0.400pt}{4.818pt}}
\put(131.0,131.0){\rule[-0.200pt]{0.400pt}{175.375pt}}
\put(131.0,131.0){\rule[-0.200pt]{315.097pt}{0.400pt}}
\put(1439.0,131.0){\rule[-0.200pt]{0.400pt}{175.375pt}}
\put(131.0,859.0){\rule[-0.200pt]{315.097pt}{0.400pt}}
\put(30,495){\makebox(0,0){\rotatebox{90}{用户贡献}}}
\put(785,29){\makebox(0,0){}}
\put(1279,819){\makebox(0,0)[r]{仿真数据}}
\put(1299.0,819.0){\rule[-0.200pt]{24.090pt}{0.400pt}}
\put(131,184){\usebox{\plotpoint}}
\multiput(131.58,184.00)(0.498,1.242){65}{\rule{0.120pt}{1.088pt}}
\multiput(130.17,184.00)(34.000,81.741){2}{\rule{0.400pt}{0.544pt}}
\multiput(165.58,268.00)(0.497,0.744){63}{\rule{0.120pt}{0.694pt}}
\multiput(164.17,268.00)(33.000,47.560){2}{\rule{0.400pt}{0.347pt}}
\multiput(198.58,317.00)(0.498,0.514){65}{\rule{0.120pt}{0.512pt}}
\multiput(197.17,317.00)(34.000,33.938){2}{\rule{0.400pt}{0.256pt}}
\multiput(232.00,352.58)(0.635,0.497){49}{\rule{0.608pt}{0.120pt}}
\multiput(232.00,351.17)(31.739,26.000){2}{\rule{0.304pt}{0.400pt}}
\multiput(265.00,378.58)(0.775,0.496){41}{\rule{0.718pt}{0.120pt}}
\multiput(265.00,377.17)(32.509,22.000){2}{\rule{0.359pt}{0.400pt}}
\multiput(299.00,400.58)(0.874,0.495){35}{\rule{0.795pt}{0.119pt}}
\multiput(299.00,399.17)(31.350,19.000){2}{\rule{0.397pt}{0.400pt}}
\multiput(332.00,419.58)(1.073,0.494){29}{\rule{0.950pt}{0.119pt}}
\multiput(332.00,418.17)(32.028,16.000){2}{\rule{0.475pt}{0.400pt}}
\multiput(366.00,435.58)(1.195,0.494){25}{\rule{1.043pt}{0.119pt}}
\multiput(366.00,434.17)(30.835,14.000){2}{\rule{0.521pt}{0.400pt}}
\multiput(399.00,449.58)(1.329,0.493){23}{\rule{1.146pt}{0.119pt}}
\multiput(399.00,448.17)(31.621,13.000){2}{\rule{0.573pt}{0.400pt}}
\multiput(433.00,462.58)(1.401,0.492){21}{\rule{1.200pt}{0.119pt}}
\multiput(433.00,461.17)(30.509,12.000){2}{\rule{0.600pt}{0.400pt}}
\multiput(466.00,474.58)(1.746,0.491){17}{\rule{1.460pt}{0.118pt}}
\multiput(466.00,473.17)(30.970,10.000){2}{\rule{0.730pt}{0.400pt}}
\multiput(500.00,484.58)(1.694,0.491){17}{\rule{1.420pt}{0.118pt}}
\multiput(500.00,483.17)(30.053,10.000){2}{\rule{0.710pt}{0.400pt}}
\multiput(533.00,494.59)(1.951,0.489){15}{\rule{1.611pt}{0.118pt}}
\multiput(533.00,493.17)(30.656,9.000){2}{\rule{0.806pt}{0.400pt}}
\multiput(567.00,503.59)(2.211,0.488){13}{\rule{1.800pt}{0.117pt}}
\multiput(567.00,502.17)(30.264,8.000){2}{\rule{0.900pt}{0.400pt}}
\multiput(601.00,511.59)(2.145,0.488){13}{\rule{1.750pt}{0.117pt}}
\multiput(601.00,510.17)(29.368,8.000){2}{\rule{0.875pt}{0.400pt}}
\multiput(634.00,519.59)(2.552,0.485){11}{\rule{2.043pt}{0.117pt}}
\multiput(634.00,518.17)(29.760,7.000){2}{\rule{1.021pt}{0.400pt}}
\multiput(668.00,526.59)(2.476,0.485){11}{\rule{1.986pt}{0.117pt}}
\multiput(668.00,525.17)(28.879,7.000){2}{\rule{0.993pt}{0.400pt}}
\multiput(701.00,533.59)(2.552,0.485){11}{\rule{2.043pt}{0.117pt}}
\multiput(701.00,532.17)(29.760,7.000){2}{\rule{1.021pt}{0.400pt}}
\multiput(735.00,540.59)(2.932,0.482){9}{\rule{2.300pt}{0.116pt}}
\multiput(735.00,539.17)(28.226,6.000){2}{\rule{1.150pt}{0.400pt}}
\multiput(768.00,546.59)(3.022,0.482){9}{\rule{2.367pt}{0.116pt}}
\multiput(768.00,545.17)(29.088,6.000){2}{\rule{1.183pt}{0.400pt}}
\multiput(802.00,552.59)(3.604,0.477){7}{\rule{2.740pt}{0.115pt}}
\multiput(802.00,551.17)(27.313,5.000){2}{\rule{1.370pt}{0.400pt}}
\multiput(835.00,557.59)(3.022,0.482){9}{\rule{2.367pt}{0.116pt}}
\multiput(835.00,556.17)(29.088,6.000){2}{\rule{1.183pt}{0.400pt}}
\multiput(869.00,563.59)(3.604,0.477){7}{\rule{2.740pt}{0.115pt}}
\multiput(869.00,562.17)(27.313,5.000){2}{\rule{1.370pt}{0.400pt}}
\multiput(902.00,568.59)(3.716,0.477){7}{\rule{2.820pt}{0.115pt}}
\multiput(902.00,567.17)(28.147,5.000){2}{\rule{1.410pt}{0.400pt}}
\multiput(936.00,573.60)(4.722,0.468){5}{\rule{3.400pt}{0.113pt}}
\multiput(936.00,572.17)(25.943,4.000){2}{\rule{1.700pt}{0.400pt}}
\multiput(969.00,577.59)(3.716,0.477){7}{\rule{2.820pt}{0.115pt}}
\multiput(969.00,576.17)(28.147,5.000){2}{\rule{1.410pt}{0.400pt}}
\multiput(1003.00,582.60)(4.868,0.468){5}{\rule{3.500pt}{0.113pt}}
\multiput(1003.00,581.17)(26.736,4.000){2}{\rule{1.750pt}{0.400pt}}
\multiput(1037.00,586.59)(3.604,0.477){7}{\rule{2.740pt}{0.115pt}}
\multiput(1037.00,585.17)(27.313,5.000){2}{\rule{1.370pt}{0.400pt}}
\multiput(1070.00,591.60)(4.868,0.468){5}{\rule{3.500pt}{0.113pt}}
\multiput(1070.00,590.17)(26.736,4.000){2}{\rule{1.750pt}{0.400pt}}
\multiput(1104.00,595.60)(4.722,0.468){5}{\rule{3.400pt}{0.113pt}}
\multiput(1104.00,594.17)(25.943,4.000){2}{\rule{1.700pt}{0.400pt}}
\multiput(1137.00,599.60)(4.868,0.468){5}{\rule{3.500pt}{0.113pt}}
\multiput(1137.00,598.17)(26.736,4.000){2}{\rule{1.750pt}{0.400pt}}
\multiput(1171.00,603.61)(7.160,0.447){3}{\rule{4.500pt}{0.108pt}}
\multiput(1171.00,602.17)(23.660,3.000){2}{\rule{2.250pt}{0.400pt}}
\multiput(1204.00,606.60)(4.868,0.468){5}{\rule{3.500pt}{0.113pt}}
\multiput(1204.00,605.17)(26.736,4.000){2}{\rule{1.750pt}{0.400pt}}
\multiput(1238.00,610.61)(7.160,0.447){3}{\rule{4.500pt}{0.108pt}}
\multiput(1238.00,609.17)(23.660,3.000){2}{\rule{2.250pt}{0.400pt}}
\multiput(1271.00,613.60)(4.868,0.468){5}{\rule{3.500pt}{0.113pt}}
\multiput(1271.00,612.17)(26.736,4.000){2}{\rule{1.750pt}{0.400pt}}
\multiput(1305.00,617.61)(7.160,0.447){3}{\rule{4.500pt}{0.108pt}}
\multiput(1305.00,616.17)(23.660,3.000){2}{\rule{2.250pt}{0.400pt}}
\multiput(1338.00,620.61)(7.383,0.447){3}{\rule{4.633pt}{0.108pt}}
\multiput(1338.00,619.17)(24.383,3.000){2}{\rule{2.317pt}{0.400pt}}
\multiput(1372.00,623.61)(7.160,0.447){3}{\rule{4.500pt}{0.108pt}}
\multiput(1372.00,622.17)(23.660,3.000){2}{\rule{2.250pt}{0.400pt}}
\multiput(1405.00,626.61)(7.383,0.447){3}{\rule{4.633pt}{0.108pt}}
\multiput(1405.00,625.17)(24.383,3.000){2}{\rule{2.317pt}{0.400pt}}
\put(131,184){\makebox(0,0){$+$}}
\put(165,268){\makebox(0,0){$+$}}
\put(198,317){\makebox(0,0){$+$}}
\put(232,352){\makebox(0,0){$+$}}
\put(265,378){\makebox(0,0){$+$}}
\put(299,400){\makebox(0,0){$+$}}
\put(332,419){\makebox(0,0){$+$}}
\put(366,435){\makebox(0,0){$+$}}
\put(399,449){\makebox(0,0){$+$}}
\put(433,462){\makebox(0,0){$+$}}
\put(466,474){\makebox(0,0){$+$}}
\put(500,484){\makebox(0,0){$+$}}
\put(533,494){\makebox(0,0){$+$}}
\put(567,503){\makebox(0,0){$+$}}
\put(601,511){\makebox(0,0){$+$}}
\put(634,519){\makebox(0,0){$+$}}
\put(668,526){\makebox(0,0){$+$}}
\put(701,533){\makebox(0,0){$+$}}
\put(735,540){\makebox(0,0){$+$}}
\put(768,546){\makebox(0,0){$+$}}
\put(802,552){\makebox(0,0){$+$}}
\put(835,557){\makebox(0,0){$+$}}
\put(869,563){\makebox(0,0){$+$}}
\put(902,568){\makebox(0,0){$+$}}
\put(936,573){\makebox(0,0){$+$}}
\put(969,577){\makebox(0,0){$+$}}
\put(1003,582){\makebox(0,0){$+$}}
\put(1037,586){\makebox(0,0){$+$}}
\put(1070,591){\makebox(0,0){$+$}}
\put(1104,595){\makebox(0,0){$+$}}
\put(1137,599){\makebox(0,0){$+$}}
\put(1171,603){\makebox(0,0){$+$}}
\put(1204,606){\makebox(0,0){$+$}}
\put(1238,610){\makebox(0,0){$+$}}
\put(1271,613){\makebox(0,0){$+$}}
\put(1305,617){\makebox(0,0){$+$}}
\put(1338,620){\makebox(0,0){$+$}}
\put(1372,623){\makebox(0,0){$+$}}
\put(1405,626){\makebox(0,0){$+$}}
\put(1439,629){\makebox(0,0){$+$}}
\put(1349,819){\makebox(0,0){$+$}}
\put(1279,768){\makebox(0,0)[r]{改进数据}}
\put(1299.0,768.0){\rule[-0.200pt]{24.090pt}{0.400pt}}
\put(131,333){\usebox{\plotpoint}}
\multiput(131.00,333.59)(2.211,0.488){13}{\rule{1.800pt}{0.117pt}}
\multiput(131.00,332.17)(30.264,8.000){2}{\rule{0.900pt}{0.400pt}}
\multiput(165.00,341.59)(2.145,0.488){13}{\rule{1.750pt}{0.117pt}}
\multiput(165.00,340.17)(29.368,8.000){2}{\rule{0.875pt}{0.400pt}}
\multiput(198.00,349.59)(1.951,0.489){15}{\rule{1.611pt}{0.118pt}}
\multiput(198.00,348.17)(30.656,9.000){2}{\rule{0.806pt}{0.400pt}}
\multiput(232.00,358.59)(2.145,0.488){13}{\rule{1.750pt}{0.117pt}}
\multiput(232.00,357.17)(29.368,8.000){2}{\rule{0.875pt}{0.400pt}}
\multiput(265.00,366.59)(1.951,0.489){15}{\rule{1.611pt}{0.118pt}}
\multiput(265.00,365.17)(30.656,9.000){2}{\rule{0.806pt}{0.400pt}}
\multiput(299.00,375.59)(1.893,0.489){15}{\rule{1.567pt}{0.118pt}}
\multiput(299.00,374.17)(29.748,9.000){2}{\rule{0.783pt}{0.400pt}}
\multiput(332.00,384.59)(1.951,0.489){15}{\rule{1.611pt}{0.118pt}}
\multiput(332.00,383.17)(30.656,9.000){2}{\rule{0.806pt}{0.400pt}}
\multiput(366.00,393.59)(1.893,0.489){15}{\rule{1.567pt}{0.118pt}}
\multiput(366.00,392.17)(29.748,9.000){2}{\rule{0.783pt}{0.400pt}}
\multiput(399.00,402.59)(1.951,0.489){15}{\rule{1.611pt}{0.118pt}}
\multiput(399.00,401.17)(30.656,9.000){2}{\rule{0.806pt}{0.400pt}}
\multiput(433.00,411.58)(1.694,0.491){17}{\rule{1.420pt}{0.118pt}}
\multiput(433.00,410.17)(30.053,10.000){2}{\rule{0.710pt}{0.400pt}}
\multiput(466.00,421.59)(1.951,0.489){15}{\rule{1.611pt}{0.118pt}}
\multiput(466.00,420.17)(30.656,9.000){2}{\rule{0.806pt}{0.400pt}}
\multiput(500.00,430.58)(1.694,0.491){17}{\rule{1.420pt}{0.118pt}}
\multiput(500.00,429.17)(30.053,10.000){2}{\rule{0.710pt}{0.400pt}}
\multiput(533.00,440.58)(1.746,0.491){17}{\rule{1.460pt}{0.118pt}}
\multiput(533.00,439.17)(30.970,10.000){2}{\rule{0.730pt}{0.400pt}}
\multiput(567.00,450.58)(1.581,0.492){19}{\rule{1.336pt}{0.118pt}}
\multiput(567.00,449.17)(31.226,11.000){2}{\rule{0.668pt}{0.400pt}}
\multiput(601.00,461.58)(1.694,0.491){17}{\rule{1.420pt}{0.118pt}}
\multiput(601.00,460.17)(30.053,10.000){2}{\rule{0.710pt}{0.400pt}}
\multiput(634.00,471.58)(1.746,0.491){17}{\rule{1.460pt}{0.118pt}}
\multiput(634.00,470.17)(30.970,10.000){2}{\rule{0.730pt}{0.400pt}}
\multiput(668.00,481.58)(1.534,0.492){19}{\rule{1.300pt}{0.118pt}}
\multiput(668.00,480.17)(30.302,11.000){2}{\rule{0.650pt}{0.400pt}}
\multiput(701.00,492.58)(1.581,0.492){19}{\rule{1.336pt}{0.118pt}}
\multiput(701.00,491.17)(31.226,11.000){2}{\rule{0.668pt}{0.400pt}}
\multiput(735.00,503.58)(1.534,0.492){19}{\rule{1.300pt}{0.118pt}}
\multiput(735.00,502.17)(30.302,11.000){2}{\rule{0.650pt}{0.400pt}}
\multiput(768.00,514.58)(1.444,0.492){21}{\rule{1.233pt}{0.119pt}}
\multiput(768.00,513.17)(31.440,12.000){2}{\rule{0.617pt}{0.400pt}}
\multiput(802.00,526.58)(1.534,0.492){19}{\rule{1.300pt}{0.118pt}}
\multiput(802.00,525.17)(30.302,11.000){2}{\rule{0.650pt}{0.400pt}}
\multiput(835.00,537.58)(1.444,0.492){21}{\rule{1.233pt}{0.119pt}}
\multiput(835.00,536.17)(31.440,12.000){2}{\rule{0.617pt}{0.400pt}}
\multiput(869.00,549.58)(1.401,0.492){21}{\rule{1.200pt}{0.119pt}}
\multiput(869.00,548.17)(30.509,12.000){2}{\rule{0.600pt}{0.400pt}}
\multiput(902.00,561.58)(1.444,0.492){21}{\rule{1.233pt}{0.119pt}}
\multiput(902.00,560.17)(31.440,12.000){2}{\rule{0.617pt}{0.400pt}}
\multiput(936.00,573.58)(1.290,0.493){23}{\rule{1.115pt}{0.119pt}}
\multiput(936.00,572.17)(30.685,13.000){2}{\rule{0.558pt}{0.400pt}}
\multiput(969.00,586.58)(1.444,0.492){21}{\rule{1.233pt}{0.119pt}}
\multiput(969.00,585.17)(31.440,12.000){2}{\rule{0.617pt}{0.400pt}}
\multiput(1003.00,598.58)(1.329,0.493){23}{\rule{1.146pt}{0.119pt}}
\multiput(1003.00,597.17)(31.621,13.000){2}{\rule{0.573pt}{0.400pt}}
\multiput(1037.00,611.58)(1.195,0.494){25}{\rule{1.043pt}{0.119pt}}
\multiput(1037.00,610.17)(30.835,14.000){2}{\rule{0.521pt}{0.400pt}}
\multiput(1070.00,625.58)(1.329,0.493){23}{\rule{1.146pt}{0.119pt}}
\multiput(1070.00,624.17)(31.621,13.000){2}{\rule{0.573pt}{0.400pt}}
\multiput(1104.00,638.58)(1.195,0.494){25}{\rule{1.043pt}{0.119pt}}
\multiput(1104.00,637.17)(30.835,14.000){2}{\rule{0.521pt}{0.400pt}}
\multiput(1137.00,652.58)(1.329,0.493){23}{\rule{1.146pt}{0.119pt}}
\multiput(1137.00,651.17)(31.621,13.000){2}{\rule{0.573pt}{0.400pt}}
\multiput(1171.00,665.58)(1.195,0.494){25}{\rule{1.043pt}{0.119pt}}
\multiput(1171.00,664.17)(30.835,14.000){2}{\rule{0.521pt}{0.400pt}}
\multiput(1204.00,679.58)(1.147,0.494){27}{\rule{1.007pt}{0.119pt}}
\multiput(1204.00,678.17)(31.911,15.000){2}{\rule{0.503pt}{0.400pt}}
\multiput(1238.00,694.58)(1.113,0.494){27}{\rule{0.980pt}{0.119pt}}
\multiput(1238.00,693.17)(30.966,15.000){2}{\rule{0.490pt}{0.400pt}}
\multiput(1271.00,709.58)(1.231,0.494){25}{\rule{1.071pt}{0.119pt}}
\multiput(1271.00,708.17)(31.776,14.000){2}{\rule{0.536pt}{0.400pt}}
\multiput(1305.00,723.58)(1.041,0.494){29}{\rule{0.925pt}{0.119pt}}
\multiput(1305.00,722.17)(31.080,16.000){2}{\rule{0.463pt}{0.400pt}}
\multiput(1338.00,739.58)(1.147,0.494){27}{\rule{1.007pt}{0.119pt}}
\multiput(1338.00,738.17)(31.911,15.000){2}{\rule{0.503pt}{0.400pt}}
\multiput(1372.00,754.58)(1.041,0.494){29}{\rule{0.925pt}{0.119pt}}
\multiput(1372.00,753.17)(31.080,16.000){2}{\rule{0.463pt}{0.400pt}}
\multiput(1405.00,770.58)(1.073,0.494){29}{\rule{0.950pt}{0.119pt}}
\multiput(1405.00,769.17)(32.028,16.000){2}{\rule{0.475pt}{0.400pt}}
\put(131,333){\makebox(0,0){$\times$}}
\put(165,341){\makebox(0,0){$\times$}}
\put(198,349){\makebox(0,0){$\times$}}
\put(232,358){\makebox(0,0){$\times$}}
\put(265,366){\makebox(0,0){$\times$}}
\put(299,375){\makebox(0,0){$\times$}}
\put(332,384){\makebox(0,0){$\times$}}
\put(366,393){\makebox(0,0){$\times$}}
\put(399,402){\makebox(0,0){$\times$}}
\put(433,411){\makebox(0,0){$\times$}}
\put(466,421){\makebox(0,0){$\times$}}
\put(500,430){\makebox(0,0){$\times$}}
\put(533,440){\makebox(0,0){$\times$}}
\put(567,450){\makebox(0,0){$\times$}}
\put(601,461){\makebox(0,0){$\times$}}
\put(634,471){\makebox(0,0){$\times$}}
\put(668,481){\makebox(0,0){$\times$}}
\put(701,492){\makebox(0,0){$\times$}}
\put(735,503){\makebox(0,0){$\times$}}
\put(768,514){\makebox(0,0){$\times$}}
\put(802,526){\makebox(0,0){$\times$}}
\put(835,537){\makebox(0,0){$\times$}}
\put(869,549){\makebox(0,0){$\times$}}
\put(902,561){\makebox(0,0){$\times$}}
\put(936,573){\makebox(0,0){$\times$}}
\put(969,586){\makebox(0,0){$\times$}}
\put(1003,598){\makebox(0,0){$\times$}}
\put(1037,611){\makebox(0,0){$\times$}}
\put(1070,625){\makebox(0,0){$\times$}}
\put(1104,638){\makebox(0,0){$\times$}}
\put(1137,652){\makebox(0,0){$\times$}}
\put(1171,665){\makebox(0,0){$\times$}}
\put(1204,679){\makebox(0,0){$\times$}}
\put(1238,694){\makebox(0,0){$\times$}}
\put(1271,709){\makebox(0,0){$\times$}}
\put(1305,723){\makebox(0,0){$\times$}}
\put(1338,739){\makebox(0,0){$\times$}}
\put(1372,754){\makebox(0,0){$\times$}}
\put(1405,770){\makebox(0,0){$\times$}}
\put(1439,786){\makebox(0,0){$\times$}}
\put(1349,778){\makebox(0,0){$\times$}}
\put(131.0,131.0){\rule[-0.200pt]{0.400pt}{175.375pt}}
\put(131.0,131.0){\rule[-0.200pt]{315.097pt}{0.400pt}}
\put(1439.0,131.0){\rule[-0.200pt]{0.400pt}{175.375pt}}
\put(131.0,859.0){\rule[-0.200pt]{315.097pt}{0.400pt}}
\end{picture}
}
   \end{frame}

  \begin{frame}
提升领域专家所有非敏感动机的仿真结果
\scalebox{0.8}{% GNUPLOT: LaTeX picture
\setlength{\unitlength}{0.240900pt}
\ifx\plotpoint\undefined\newsavebox{\plotpoint}\fi
\begin{picture}(1500,900)(0,0)
\sbox{\plotpoint}{\rule[-0.200pt]{0.400pt}{0.400pt}}%
\put(131.0,82.0){\rule[-0.200pt]{4.818pt}{0.400pt}}
\put(111,82){\makebox(0,0)[r]{ 0}}
\put(1419.0,82.0){\rule[-0.200pt]{4.818pt}{0.400pt}}
\put(131.0,276.0){\rule[-0.200pt]{4.818pt}{0.400pt}}
\put(111,276){\makebox(0,0)[r]{ 1}}
\put(1419.0,276.0){\rule[-0.200pt]{4.818pt}{0.400pt}}
\put(131.0,471.0){\rule[-0.200pt]{4.818pt}{0.400pt}}
\put(111,471){\makebox(0,0)[r]{ 2}}
\put(1419.0,471.0){\rule[-0.200pt]{4.818pt}{0.400pt}}
\put(131.0,665.0){\rule[-0.200pt]{4.818pt}{0.400pt}}
\put(111,665){\makebox(0,0)[r]{ 3}}
\put(1419.0,665.0){\rule[-0.200pt]{4.818pt}{0.400pt}}
\put(131.0,82.0){\rule[-0.200pt]{0.400pt}{4.818pt}}
\put(131,41){\makebox(0,0){ 1}}
\put(131.0,839.0){\rule[-0.200pt]{0.400pt}{4.818pt}}
\put(198.0,82.0){\rule[-0.200pt]{0.400pt}{4.818pt}}
\put(198,41){\makebox(0,0){ 3}}
\put(198.0,839.0){\rule[-0.200pt]{0.400pt}{4.818pt}}
\put(265.0,82.0){\rule[-0.200pt]{0.400pt}{4.818pt}}
\put(265,41){\makebox(0,0){ 5}}
\put(265.0,839.0){\rule[-0.200pt]{0.400pt}{4.818pt}}
\put(332.0,82.0){\rule[-0.200pt]{0.400pt}{4.818pt}}
\put(332,41){\makebox(0,0){ 7}}
\put(332.0,839.0){\rule[-0.200pt]{0.400pt}{4.818pt}}
\put(399.0,82.0){\rule[-0.200pt]{0.400pt}{4.818pt}}
\put(399,41){\makebox(0,0){ 9}}
\put(399.0,839.0){\rule[-0.200pt]{0.400pt}{4.818pt}}
\put(466.0,82.0){\rule[-0.200pt]{0.400pt}{4.818pt}}
\put(466,41){\makebox(0,0){ 11}}
\put(466.0,839.0){\rule[-0.200pt]{0.400pt}{4.818pt}}
\put(533.0,82.0){\rule[-0.200pt]{0.400pt}{4.818pt}}
\put(533,41){\makebox(0,0){ 13}}
\put(533.0,839.0){\rule[-0.200pt]{0.400pt}{4.818pt}}
\put(601.0,82.0){\rule[-0.200pt]{0.400pt}{4.818pt}}
\put(601,41){\makebox(0,0){ 15}}
\put(601.0,839.0){\rule[-0.200pt]{0.400pt}{4.818pt}}
\put(668.0,82.0){\rule[-0.200pt]{0.400pt}{4.818pt}}
\put(668,41){\makebox(0,0){ 17}}
\put(668.0,839.0){\rule[-0.200pt]{0.400pt}{4.818pt}}
\put(735.0,82.0){\rule[-0.200pt]{0.400pt}{4.818pt}}
\put(735,41){\makebox(0,0){ 19}}
\put(735.0,839.0){\rule[-0.200pt]{0.400pt}{4.818pt}}
\put(802.0,82.0){\rule[-0.200pt]{0.400pt}{4.818pt}}
\put(802,41){\makebox(0,0){ 21}}
\put(802.0,839.0){\rule[-0.200pt]{0.400pt}{4.818pt}}
\put(869.0,82.0){\rule[-0.200pt]{0.400pt}{4.818pt}}
\put(869,41){\makebox(0,0){ 23}}
\put(869.0,839.0){\rule[-0.200pt]{0.400pt}{4.818pt}}
\put(936.0,82.0){\rule[-0.200pt]{0.400pt}{4.818pt}}
\put(936,41){\makebox(0,0){ 25}}
\put(936.0,839.0){\rule[-0.200pt]{0.400pt}{4.818pt}}
\put(1003.0,82.0){\rule[-0.200pt]{0.400pt}{4.818pt}}
\put(1003,41){\makebox(0,0){ 27}}
\put(1003.0,839.0){\rule[-0.200pt]{0.400pt}{4.818pt}}
\put(1070.0,82.0){\rule[-0.200pt]{0.400pt}{4.818pt}}
\put(1070,41){\makebox(0,0){ 29}}
\put(1070.0,839.0){\rule[-0.200pt]{0.400pt}{4.818pt}}
\put(1137.0,82.0){\rule[-0.200pt]{0.400pt}{4.818pt}}
\put(1137,41){\makebox(0,0){ 31}}
\put(1137.0,839.0){\rule[-0.200pt]{0.400pt}{4.818pt}}
\put(1204.0,82.0){\rule[-0.200pt]{0.400pt}{4.818pt}}
\put(1204,41){\makebox(0,0){ 33}}
\put(1204.0,839.0){\rule[-0.200pt]{0.400pt}{4.818pt}}
\put(1271.0,82.0){\rule[-0.200pt]{0.400pt}{4.818pt}}
\put(1271,41){\makebox(0,0){ 35}}
\put(1271.0,839.0){\rule[-0.200pt]{0.400pt}{4.818pt}}
\put(1338.0,82.0){\rule[-0.200pt]{0.400pt}{4.818pt}}
\put(1338,41){\makebox(0,0){ 37}}
\put(1338.0,839.0){\rule[-0.200pt]{0.400pt}{4.818pt}}
\put(1405.0,82.0){\rule[-0.200pt]{0.400pt}{4.818pt}}
\put(1405,41){\makebox(0,0){ 39}}
\put(1405.0,839.0){\rule[-0.200pt]{0.400pt}{4.818pt}}
\put(131.0,82.0){\rule[-0.200pt]{0.400pt}{187.179pt}}
\put(131.0,82.0){\rule[-0.200pt]{315.097pt}{0.400pt}}
\put(1439.0,82.0){\rule[-0.200pt]{0.400pt}{187.179pt}}
\put(131.0,859.0){\rule[-0.200pt]{315.097pt}{0.400pt}}
\put(30,470){\makebox(0,0){\rotatebox{90}{用户贡献}}}
\put(1279,819){\makebox(0,0)[r]{仿真数据}}
\put(1299.0,819.0){\rule[-0.200pt]{24.090pt}{0.400pt}}
\put(131,366){\usebox{\plotpoint}}
\multiput(131.00,364.92)(0.952,-0.495){33}{\rule{0.856pt}{0.119pt}}
\multiput(131.00,365.17)(32.224,-18.000){2}{\rule{0.428pt}{0.400pt}}
\multiput(165.00,346.92)(1.401,-0.492){21}{\rule{1.200pt}{0.119pt}}
\multiput(165.00,347.17)(30.509,-12.000){2}{\rule{0.600pt}{0.400pt}}
\multiput(198.00,334.93)(3.716,-0.477){7}{\rule{2.820pt}{0.115pt}}
\multiput(198.00,335.17)(28.147,-5.000){2}{\rule{1.410pt}{0.400pt}}
\multiput(232.00,329.93)(2.932,-0.482){9}{\rule{2.300pt}{0.116pt}}
\multiput(232.00,330.17)(28.226,-6.000){2}{\rule{1.150pt}{0.400pt}}
\multiput(265.00,323.93)(3.022,-0.482){9}{\rule{2.367pt}{0.116pt}}
\multiput(265.00,324.17)(29.088,-6.000){2}{\rule{1.183pt}{0.400pt}}
\multiput(299.00,317.94)(4.722,-0.468){5}{\rule{3.400pt}{0.113pt}}
\multiput(299.00,318.17)(25.943,-4.000){2}{\rule{1.700pt}{0.400pt}}
\put(332,313.17){\rule{6.900pt}{0.400pt}}
\multiput(332.00,314.17)(19.679,-2.000){2}{\rule{3.450pt}{0.400pt}}
\multiput(366.00,311.94)(4.722,-0.468){5}{\rule{3.400pt}{0.113pt}}
\multiput(366.00,312.17)(25.943,-4.000){2}{\rule{1.700pt}{0.400pt}}
\put(399,307.17){\rule{6.900pt}{0.400pt}}
\multiput(399.00,308.17)(19.679,-2.000){2}{\rule{3.450pt}{0.400pt}}
\put(433,305.17){\rule{6.700pt}{0.400pt}}
\multiput(433.00,306.17)(19.094,-2.000){2}{\rule{3.350pt}{0.400pt}}
\put(466,303.17){\rule{6.900pt}{0.400pt}}
\multiput(466.00,304.17)(19.679,-2.000){2}{\rule{3.450pt}{0.400pt}}
\put(500,301.67){\rule{7.950pt}{0.400pt}}
\multiput(500.00,302.17)(16.500,-1.000){2}{\rule{3.975pt}{0.400pt}}
\put(533,300.17){\rule{6.900pt}{0.400pt}}
\multiput(533.00,301.17)(19.679,-2.000){2}{\rule{3.450pt}{0.400pt}}
\put(567,298.17){\rule{6.900pt}{0.400pt}}
\multiput(567.00,299.17)(19.679,-2.000){2}{\rule{3.450pt}{0.400pt}}
\put(601,296.17){\rule{6.700pt}{0.400pt}}
\multiput(601.00,297.17)(19.094,-2.000){2}{\rule{3.350pt}{0.400pt}}
\put(634,294.17){\rule{6.900pt}{0.400pt}}
\multiput(634.00,295.17)(19.679,-2.000){2}{\rule{3.450pt}{0.400pt}}
\put(668,292.17){\rule{6.700pt}{0.400pt}}
\multiput(668.00,293.17)(19.094,-2.000){2}{\rule{3.350pt}{0.400pt}}
\put(735,290.17){\rule{6.700pt}{0.400pt}}
\multiput(735.00,291.17)(19.094,-2.000){2}{\rule{3.350pt}{0.400pt}}
\put(768,288.17){\rule{6.900pt}{0.400pt}}
\multiput(768.00,289.17)(19.679,-2.000){2}{\rule{3.450pt}{0.400pt}}
\put(701.0,292.0){\rule[-0.200pt]{8.191pt}{0.400pt}}
\put(835,286.17){\rule{6.900pt}{0.400pt}}
\multiput(835.00,287.17)(19.679,-2.000){2}{\rule{3.450pt}{0.400pt}}
\put(802.0,288.0){\rule[-0.200pt]{7.950pt}{0.400pt}}
\put(902,284.17){\rule{6.900pt}{0.400pt}}
\multiput(902.00,285.17)(19.679,-2.000){2}{\rule{3.450pt}{0.400pt}}
\put(869.0,286.0){\rule[-0.200pt]{7.950pt}{0.400pt}}
\put(969,282.17){\rule{6.900pt}{0.400pt}}
\multiput(969.00,283.17)(19.679,-2.000){2}{\rule{3.450pt}{0.400pt}}
\put(936.0,284.0){\rule[-0.200pt]{7.950pt}{0.400pt}}
\put(1037,280.17){\rule{6.700pt}{0.400pt}}
\multiput(1037.00,281.17)(19.094,-2.000){2}{\rule{3.350pt}{0.400pt}}
\put(1003.0,282.0){\rule[-0.200pt]{8.191pt}{0.400pt}}
\put(1104,278.17){\rule{6.700pt}{0.400pt}}
\multiput(1104.00,279.17)(19.094,-2.000){2}{\rule{3.350pt}{0.400pt}}
\put(1070.0,280.0){\rule[-0.200pt]{8.191pt}{0.400pt}}
\put(1171,276.17){\rule{6.700pt}{0.400pt}}
\multiput(1171.00,277.17)(19.094,-2.000){2}{\rule{3.350pt}{0.400pt}}
\put(1137.0,278.0){\rule[-0.200pt]{8.191pt}{0.400pt}}
\put(1271,274.17){\rule{6.900pt}{0.400pt}}
\multiput(1271.00,275.17)(19.679,-2.000){2}{\rule{3.450pt}{0.400pt}}
\put(1204.0,276.0){\rule[-0.200pt]{16.140pt}{0.400pt}}
\put(1372,272.17){\rule{6.700pt}{0.400pt}}
\multiput(1372.00,273.17)(19.094,-2.000){2}{\rule{3.350pt}{0.400pt}}
\put(1305.0,274.0){\rule[-0.200pt]{16.140pt}{0.400pt}}
\put(1405.0,272.0){\rule[-0.200pt]{8.191pt}{0.400pt}}
\put(1279,768){\makebox(0,0)[r]{改进数据}}
\put(1299.0,768.0){\rule[-0.200pt]{24.090pt}{0.400pt}}
\put(131,587){\usebox{\plotpoint}}
\multiput(131.00,585.92)(0.514,-0.497){63}{\rule{0.512pt}{0.120pt}}
\multiput(131.00,586.17)(32.937,-33.000){2}{\rule{0.256pt}{0.400pt}}
\multiput(165.00,552.93)(2.145,-0.488){13}{\rule{1.750pt}{0.117pt}}
\multiput(165.00,553.17)(29.368,-8.000){2}{\rule{0.875pt}{0.400pt}}
\multiput(198.00,544.92)(1.581,-0.492){19}{\rule{1.336pt}{0.118pt}}
\multiput(198.00,545.17)(31.226,-11.000){2}{\rule{0.668pt}{0.400pt}}
\multiput(232.00,533.93)(2.145,-0.488){13}{\rule{1.750pt}{0.117pt}}
\multiput(232.00,534.17)(29.368,-8.000){2}{\rule{0.875pt}{0.400pt}}
\put(265,527.17){\rule{6.900pt}{0.400pt}}
\multiput(265.00,526.17)(19.679,2.000){2}{\rule{3.450pt}{0.400pt}}
\multiput(299.00,527.93)(2.932,-0.482){9}{\rule{2.300pt}{0.116pt}}
\multiput(299.00,528.17)(28.226,-6.000){2}{\rule{1.150pt}{0.400pt}}
\put(332,523.17){\rule{6.900pt}{0.400pt}}
\multiput(332.00,522.17)(19.679,2.000){2}{\rule{3.450pt}{0.400pt}}
\multiput(366.00,523.92)(1.694,-0.491){17}{\rule{1.420pt}{0.118pt}}
\multiput(366.00,524.17)(30.053,-10.000){2}{\rule{0.710pt}{0.400pt}}
\put(399,515.17){\rule{6.900pt}{0.400pt}}
\multiput(399.00,514.17)(19.679,2.000){2}{\rule{3.450pt}{0.400pt}}
\multiput(433.00,515.93)(2.145,-0.488){13}{\rule{1.750pt}{0.117pt}}
\multiput(433.00,516.17)(29.368,-8.000){2}{\rule{0.875pt}{0.400pt}}
\put(466,507.17){\rule{6.900pt}{0.400pt}}
\multiput(466.00,508.17)(19.679,-2.000){2}{\rule{3.450pt}{0.400pt}}
\multiput(500.00,507.60)(4.722,0.468){5}{\rule{3.400pt}{0.113pt}}
\multiput(500.00,506.17)(25.943,4.000){2}{\rule{1.700pt}{0.400pt}}
\put(533,509.17){\rule{6.900pt}{0.400pt}}
\multiput(533.00,510.17)(19.679,-2.000){2}{\rule{3.450pt}{0.400pt}}
\multiput(567.00,507.94)(4.868,-0.468){5}{\rule{3.500pt}{0.113pt}}
\multiput(567.00,508.17)(26.736,-4.000){2}{\rule{1.750pt}{0.400pt}}
\multiput(601.00,503.95)(7.160,-0.447){3}{\rule{4.500pt}{0.108pt}}
\multiput(601.00,504.17)(23.660,-3.000){2}{\rule{2.250pt}{0.400pt}}
\put(634,500.17){\rule{6.900pt}{0.400pt}}
\multiput(634.00,501.17)(19.679,-2.000){2}{\rule{3.450pt}{0.400pt}}
\multiput(668.00,500.60)(4.722,0.468){5}{\rule{3.400pt}{0.113pt}}
\multiput(668.00,499.17)(25.943,4.000){2}{\rule{1.700pt}{0.400pt}}
\multiput(701.00,502.94)(4.868,-0.468){5}{\rule{3.500pt}{0.113pt}}
\multiput(701.00,503.17)(26.736,-4.000){2}{\rule{1.750pt}{0.400pt}}
\multiput(735.00,500.60)(4.722,0.468){5}{\rule{3.400pt}{0.113pt}}
\multiput(735.00,499.17)(25.943,4.000){2}{\rule{1.700pt}{0.400pt}}
\multiput(768.00,502.93)(2.211,-0.488){13}{\rule{1.800pt}{0.117pt}}
\multiput(768.00,503.17)(30.264,-8.000){2}{\rule{0.900pt}{0.400pt}}
\put(802,494.17){\rule{6.700pt}{0.400pt}}
\multiput(802.00,495.17)(19.094,-2.000){2}{\rule{3.350pt}{0.400pt}}
\multiput(835.00,492.93)(3.022,-0.482){9}{\rule{2.367pt}{0.116pt}}
\multiput(835.00,493.17)(29.088,-6.000){2}{\rule{1.183pt}{0.400pt}}
\multiput(869.00,488.59)(2.932,0.482){9}{\rule{2.300pt}{0.116pt}}
\multiput(869.00,487.17)(28.226,6.000){2}{\rule{1.150pt}{0.400pt}}
\put(902,492.17){\rule{6.900pt}{0.400pt}}
\multiput(902.00,493.17)(19.679,-2.000){2}{\rule{3.450pt}{0.400pt}}
\multiput(969.00,490.94)(4.868,-0.468){5}{\rule{3.500pt}{0.113pt}}
\multiput(969.00,491.17)(26.736,-4.000){2}{\rule{1.750pt}{0.400pt}}
\put(1003,488.17){\rule{6.900pt}{0.400pt}}
\multiput(1003.00,487.17)(19.679,2.000){2}{\rule{3.450pt}{0.400pt}}
\multiput(1037.00,490.60)(4.722,0.468){5}{\rule{3.400pt}{0.113pt}}
\multiput(1037.00,489.17)(25.943,4.000){2}{\rule{1.700pt}{0.400pt}}
\multiput(1070.00,492.93)(2.211,-0.488){13}{\rule{1.800pt}{0.117pt}}
\multiput(1070.00,493.17)(30.264,-8.000){2}{\rule{0.900pt}{0.400pt}}
\multiput(1104.00,484.93)(2.932,-0.482){9}{\rule{2.300pt}{0.116pt}}
\multiput(1104.00,485.17)(28.226,-6.000){2}{\rule{1.150pt}{0.400pt}}
\put(1137,478.17){\rule{6.900pt}{0.400pt}}
\multiput(1137.00,479.17)(19.679,-2.000){2}{\rule{3.450pt}{0.400pt}}
\put(1171,476.17){\rule{6.700pt}{0.400pt}}
\multiput(1171.00,477.17)(19.094,-2.000){2}{\rule{3.350pt}{0.400pt}}
\multiput(1204.00,476.59)(2.211,0.488){13}{\rule{1.800pt}{0.117pt}}
\multiput(1204.00,475.17)(30.264,8.000){2}{\rule{0.900pt}{0.400pt}}
\multiput(1238.00,482.92)(1.694,-0.491){17}{\rule{1.420pt}{0.118pt}}
\multiput(1238.00,483.17)(30.053,-10.000){2}{\rule{0.710pt}{0.400pt}}
\put(1271,472.17){\rule{6.900pt}{0.400pt}}
\multiput(1271.00,473.17)(19.679,-2.000){2}{\rule{3.450pt}{0.400pt}}
\multiput(1305.00,472.58)(1.694,0.491){17}{\rule{1.420pt}{0.118pt}}
\multiput(1305.00,471.17)(30.053,10.000){2}{\rule{0.710pt}{0.400pt}}
\multiput(1338.00,480.92)(1.581,-0.492){19}{\rule{1.336pt}{0.118pt}}
\multiput(1338.00,481.17)(31.226,-11.000){2}{\rule{0.668pt}{0.400pt}}
\multiput(1372.00,471.59)(3.604,0.477){7}{\rule{2.740pt}{0.115pt}}
\multiput(1372.00,470.17)(27.313,5.000){2}{\rule{1.370pt}{0.400pt}}
\multiput(1405.00,476.59)(3.022,0.482){9}{\rule{2.367pt}{0.116pt}}
\multiput(1405.00,475.17)(29.088,6.000){2}{\rule{1.183pt}{0.400pt}}
\put(131,587){\makebox(0,0){$+$}}
\put(165,554){\makebox(0,0){$+$}}
\put(198,546){\makebox(0,0){$+$}}
\put(232,535){\makebox(0,0){$+$}}
\put(265,527){\makebox(0,0){$+$}}
\put(299,529){\makebox(0,0){$+$}}
\put(332,523){\makebox(0,0){$+$}}
\put(366,525){\makebox(0,0){$+$}}
\put(399,515){\makebox(0,0){$+$}}
\put(433,517){\makebox(0,0){$+$}}
\put(466,509){\makebox(0,0){$+$}}
\put(500,507){\makebox(0,0){$+$}}
\put(533,511){\makebox(0,0){$+$}}
\put(567,509){\makebox(0,0){$+$}}
\put(601,505){\makebox(0,0){$+$}}
\put(634,502){\makebox(0,0){$+$}}
\put(668,500){\makebox(0,0){$+$}}
\put(701,504){\makebox(0,0){$+$}}
\put(735,500){\makebox(0,0){$+$}}
\put(768,504){\makebox(0,0){$+$}}
\put(802,496){\makebox(0,0){$+$}}
\put(835,494){\makebox(0,0){$+$}}
\put(869,488){\makebox(0,0){$+$}}
\put(902,494){\makebox(0,0){$+$}}
\put(936,492){\makebox(0,0){$+$}}
\put(969,492){\makebox(0,0){$+$}}
\put(1003,488){\makebox(0,0){$+$}}
\put(1037,490){\makebox(0,0){$+$}}
\put(1070,494){\makebox(0,0){$+$}}
\put(1104,486){\makebox(0,0){$+$}}
\put(1137,480){\makebox(0,0){$+$}}
\put(1171,478){\makebox(0,0){$+$}}
\put(1204,476){\makebox(0,0){$+$}}
\put(1238,484){\makebox(0,0){$+$}}
\put(1271,474){\makebox(0,0){$+$}}
\put(1305,472){\makebox(0,0){$+$}}
\put(1338,482){\makebox(0,0){$+$}}
\put(1372,471){\makebox(0,0){$+$}}
\put(1405,476){\makebox(0,0){$+$}}
\put(1439,482){\makebox(0,0){$+$}}
\put(1349,768){\makebox(0,0){$+$}}
\put(936.0,492.0){\rule[-0.200pt]{7.950pt}{0.400pt}}
\put(131.0,82.0){\rule[-0.200pt]{0.400pt}{187.179pt}}
\put(131.0,82.0){\rule[-0.200pt]{315.097pt}{0.400pt}}
\put(1439.0,82.0){\rule[-0.200pt]{0.400pt}{187.179pt}}
\put(131.0,859.0){\rule[-0.200pt]{315.097pt}{0.400pt}}
\end{picture}
}
   \end{frame}

  \begin{frame}
提升领域专家成就需求的仿真结果
\scalebox{0.8}{% GNUPLOT: LaTeX picture
\setlength{\unitlength}{0.240900pt}
\ifx\plotpoint\undefined\newsavebox{\plotpoint}\fi
\begin{picture}(1500,900)(0,0)
\sbox{\plotpoint}{\rule[-0.200pt]{0.400pt}{0.400pt}}%
\put(131.0,82.0){\rule[-0.200pt]{4.818pt}{0.400pt}}
\put(111,82){\makebox(0,0)[r]{ 0}}
\put(1419.0,82.0){\rule[-0.200pt]{4.818pt}{0.400pt}}
\put(131.0,276.0){\rule[-0.200pt]{4.818pt}{0.400pt}}
\put(111,276){\makebox(0,0)[r]{ 1}}
\put(1419.0,276.0){\rule[-0.200pt]{4.818pt}{0.400pt}}
\put(131.0,471.0){\rule[-0.200pt]{4.818pt}{0.400pt}}
\put(111,471){\makebox(0,0)[r]{ 2}}
\put(1419.0,471.0){\rule[-0.200pt]{4.818pt}{0.400pt}}
\put(131.0,665.0){\rule[-0.200pt]{4.818pt}{0.400pt}}
\put(111,665){\makebox(0,0)[r]{ 3}}
\put(1419.0,665.0){\rule[-0.200pt]{4.818pt}{0.400pt}}
\put(131.0,82.0){\rule[-0.200pt]{0.400pt}{4.818pt}}
\put(131,41){\makebox(0,0){ 1}}
\put(131.0,839.0){\rule[-0.200pt]{0.400pt}{4.818pt}}
\put(198.0,82.0){\rule[-0.200pt]{0.400pt}{4.818pt}}
\put(198,41){\makebox(0,0){ 3}}
\put(198.0,839.0){\rule[-0.200pt]{0.400pt}{4.818pt}}
\put(265.0,82.0){\rule[-0.200pt]{0.400pt}{4.818pt}}
\put(265,41){\makebox(0,0){ 5}}
\put(265.0,839.0){\rule[-0.200pt]{0.400pt}{4.818pt}}
\put(332.0,82.0){\rule[-0.200pt]{0.400pt}{4.818pt}}
\put(332,41){\makebox(0,0){ 7}}
\put(332.0,839.0){\rule[-0.200pt]{0.400pt}{4.818pt}}
\put(399.0,82.0){\rule[-0.200pt]{0.400pt}{4.818pt}}
\put(399,41){\makebox(0,0){ 9}}
\put(399.0,839.0){\rule[-0.200pt]{0.400pt}{4.818pt}}
\put(466.0,82.0){\rule[-0.200pt]{0.400pt}{4.818pt}}
\put(466,41){\makebox(0,0){ 11}}
\put(466.0,839.0){\rule[-0.200pt]{0.400pt}{4.818pt}}
\put(533.0,82.0){\rule[-0.200pt]{0.400pt}{4.818pt}}
\put(533,41){\makebox(0,0){ 13}}
\put(533.0,839.0){\rule[-0.200pt]{0.400pt}{4.818pt}}
\put(601.0,82.0){\rule[-0.200pt]{0.400pt}{4.818pt}}
\put(601,41){\makebox(0,0){ 15}}
\put(601.0,839.0){\rule[-0.200pt]{0.400pt}{4.818pt}}
\put(668.0,82.0){\rule[-0.200pt]{0.400pt}{4.818pt}}
\put(668,41){\makebox(0,0){ 17}}
\put(668.0,839.0){\rule[-0.200pt]{0.400pt}{4.818pt}}
\put(735.0,82.0){\rule[-0.200pt]{0.400pt}{4.818pt}}
\put(735,41){\makebox(0,0){ 19}}
\put(735.0,839.0){\rule[-0.200pt]{0.400pt}{4.818pt}}
\put(802.0,82.0){\rule[-0.200pt]{0.400pt}{4.818pt}}
\put(802,41){\makebox(0,0){ 21}}
\put(802.0,839.0){\rule[-0.200pt]{0.400pt}{4.818pt}}
\put(869.0,82.0){\rule[-0.200pt]{0.400pt}{4.818pt}}
\put(869,41){\makebox(0,0){ 23}}
\put(869.0,839.0){\rule[-0.200pt]{0.400pt}{4.818pt}}
\put(936.0,82.0){\rule[-0.200pt]{0.400pt}{4.818pt}}
\put(936,41){\makebox(0,0){ 25}}
\put(936.0,839.0){\rule[-0.200pt]{0.400pt}{4.818pt}}
\put(1003.0,82.0){\rule[-0.200pt]{0.400pt}{4.818pt}}
\put(1003,41){\makebox(0,0){ 27}}
\put(1003.0,839.0){\rule[-0.200pt]{0.400pt}{4.818pt}}
\put(1070.0,82.0){\rule[-0.200pt]{0.400pt}{4.818pt}}
\put(1070,41){\makebox(0,0){ 29}}
\put(1070.0,839.0){\rule[-0.200pt]{0.400pt}{4.818pt}}
\put(1137.0,82.0){\rule[-0.200pt]{0.400pt}{4.818pt}}
\put(1137,41){\makebox(0,0){ 31}}
\put(1137.0,839.0){\rule[-0.200pt]{0.400pt}{4.818pt}}
\put(1204.0,82.0){\rule[-0.200pt]{0.400pt}{4.818pt}}
\put(1204,41){\makebox(0,0){ 33}}
\put(1204.0,839.0){\rule[-0.200pt]{0.400pt}{4.818pt}}
\put(1271.0,82.0){\rule[-0.200pt]{0.400pt}{4.818pt}}
\put(1271,41){\makebox(0,0){ 35}}
\put(1271.0,839.0){\rule[-0.200pt]{0.400pt}{4.818pt}}
\put(1338.0,82.0){\rule[-0.200pt]{0.400pt}{4.818pt}}
\put(1338,41){\makebox(0,0){ 37}}
\put(1338.0,839.0){\rule[-0.200pt]{0.400pt}{4.818pt}}
\put(1405.0,82.0){\rule[-0.200pt]{0.400pt}{4.818pt}}
\put(1405,41){\makebox(0,0){ 39}}
\put(1405.0,839.0){\rule[-0.200pt]{0.400pt}{4.818pt}}
\put(131.0,82.0){\rule[-0.200pt]{0.400pt}{187.179pt}}
\put(131.0,82.0){\rule[-0.200pt]{315.097pt}{0.400pt}}
\put(1439.0,82.0){\rule[-0.200pt]{0.400pt}{187.179pt}}
\put(131.0,859.0){\rule[-0.200pt]{315.097pt}{0.400pt}}
\put(30,470){\makebox(0,0){\rotatebox{90}{用户贡献}}}
\put(1279,819){\makebox(0,0)[r]{仿真数据}}
\put(1299.0,819.0){\rule[-0.200pt]{24.090pt}{0.400pt}}
\put(131,366){\usebox{\plotpoint}}
\multiput(131.00,364.92)(0.952,-0.495){33}{\rule{0.856pt}{0.119pt}}
\multiput(131.00,365.17)(32.224,-18.000){2}{\rule{0.428pt}{0.400pt}}
\multiput(165.00,346.92)(1.401,-0.492){21}{\rule{1.200pt}{0.119pt}}
\multiput(165.00,347.17)(30.509,-12.000){2}{\rule{0.600pt}{0.400pt}}
\multiput(198.00,334.93)(3.716,-0.477){7}{\rule{2.820pt}{0.115pt}}
\multiput(198.00,335.17)(28.147,-5.000){2}{\rule{1.410pt}{0.400pt}}
\multiput(232.00,329.93)(2.932,-0.482){9}{\rule{2.300pt}{0.116pt}}
\multiput(232.00,330.17)(28.226,-6.000){2}{\rule{1.150pt}{0.400pt}}
\multiput(265.00,323.93)(3.022,-0.482){9}{\rule{2.367pt}{0.116pt}}
\multiput(265.00,324.17)(29.088,-6.000){2}{\rule{1.183pt}{0.400pt}}
\multiput(299.00,317.94)(4.722,-0.468){5}{\rule{3.400pt}{0.113pt}}
\multiput(299.00,318.17)(25.943,-4.000){2}{\rule{1.700pt}{0.400pt}}
\put(332,313.17){\rule{6.900pt}{0.400pt}}
\multiput(332.00,314.17)(19.679,-2.000){2}{\rule{3.450pt}{0.400pt}}
\multiput(366.00,311.94)(4.722,-0.468){5}{\rule{3.400pt}{0.113pt}}
\multiput(366.00,312.17)(25.943,-4.000){2}{\rule{1.700pt}{0.400pt}}
\put(399,307.17){\rule{6.900pt}{0.400pt}}
\multiput(399.00,308.17)(19.679,-2.000){2}{\rule{3.450pt}{0.400pt}}
\put(433,305.17){\rule{6.700pt}{0.400pt}}
\multiput(433.00,306.17)(19.094,-2.000){2}{\rule{3.350pt}{0.400pt}}
\put(466,303.17){\rule{6.900pt}{0.400pt}}
\multiput(466.00,304.17)(19.679,-2.000){2}{\rule{3.450pt}{0.400pt}}
\put(500,301.67){\rule{7.950pt}{0.400pt}}
\multiput(500.00,302.17)(16.500,-1.000){2}{\rule{3.975pt}{0.400pt}}
\put(533,300.17){\rule{6.900pt}{0.400pt}}
\multiput(533.00,301.17)(19.679,-2.000){2}{\rule{3.450pt}{0.400pt}}
\put(567,298.17){\rule{6.900pt}{0.400pt}}
\multiput(567.00,299.17)(19.679,-2.000){2}{\rule{3.450pt}{0.400pt}}
\put(601,296.17){\rule{6.700pt}{0.400pt}}
\multiput(601.00,297.17)(19.094,-2.000){2}{\rule{3.350pt}{0.400pt}}
\put(634,294.17){\rule{6.900pt}{0.400pt}}
\multiput(634.00,295.17)(19.679,-2.000){2}{\rule{3.450pt}{0.400pt}}
\put(668,292.17){\rule{6.700pt}{0.400pt}}
\multiput(668.00,293.17)(19.094,-2.000){2}{\rule{3.350pt}{0.400pt}}
\put(735,290.17){\rule{6.700pt}{0.400pt}}
\multiput(735.00,291.17)(19.094,-2.000){2}{\rule{3.350pt}{0.400pt}}
\put(768,288.17){\rule{6.900pt}{0.400pt}}
\multiput(768.00,289.17)(19.679,-2.000){2}{\rule{3.450pt}{0.400pt}}
\put(701.0,292.0){\rule[-0.200pt]{8.191pt}{0.400pt}}
\put(835,286.17){\rule{6.900pt}{0.400pt}}
\multiput(835.00,287.17)(19.679,-2.000){2}{\rule{3.450pt}{0.400pt}}
\put(802.0,288.0){\rule[-0.200pt]{7.950pt}{0.400pt}}
\put(902,284.17){\rule{6.900pt}{0.400pt}}
\multiput(902.00,285.17)(19.679,-2.000){2}{\rule{3.450pt}{0.400pt}}
\put(869.0,286.0){\rule[-0.200pt]{7.950pt}{0.400pt}}
\put(969,282.17){\rule{6.900pt}{0.400pt}}
\multiput(969.00,283.17)(19.679,-2.000){2}{\rule{3.450pt}{0.400pt}}
\put(936.0,284.0){\rule[-0.200pt]{7.950pt}{0.400pt}}
\put(1037,280.17){\rule{6.700pt}{0.400pt}}
\multiput(1037.00,281.17)(19.094,-2.000){2}{\rule{3.350pt}{0.400pt}}
\put(1003.0,282.0){\rule[-0.200pt]{8.191pt}{0.400pt}}
\put(1104,278.17){\rule{6.700pt}{0.400pt}}
\multiput(1104.00,279.17)(19.094,-2.000){2}{\rule{3.350pt}{0.400pt}}
\put(1070.0,280.0){\rule[-0.200pt]{8.191pt}{0.400pt}}
\put(1171,276.17){\rule{6.700pt}{0.400pt}}
\multiput(1171.00,277.17)(19.094,-2.000){2}{\rule{3.350pt}{0.400pt}}
\put(1137.0,278.0){\rule[-0.200pt]{8.191pt}{0.400pt}}
\put(1271,274.17){\rule{6.900pt}{0.400pt}}
\multiput(1271.00,275.17)(19.679,-2.000){2}{\rule{3.450pt}{0.400pt}}
\put(1204.0,276.0){\rule[-0.200pt]{16.140pt}{0.400pt}}
\put(1372,272.17){\rule{6.700pt}{0.400pt}}
\multiput(1372.00,273.17)(19.094,-2.000){2}{\rule{3.350pt}{0.400pt}}
\put(1305.0,274.0){\rule[-0.200pt]{16.140pt}{0.400pt}}
\put(1405.0,272.0){\rule[-0.200pt]{8.191pt}{0.400pt}}
\put(1279,768){\makebox(0,0)[r]{改进数据}}
\put(1299.0,768.0){\rule[-0.200pt]{24.090pt}{0.400pt}}
\put(131,379){\usebox{\plotpoint}}
\multiput(131.00,379.59)(3.022,0.482){9}{\rule{2.367pt}{0.116pt}}
\multiput(131.00,378.17)(29.088,6.000){2}{\rule{1.183pt}{0.400pt}}
\multiput(165.00,385.59)(2.932,0.482){9}{\rule{2.300pt}{0.116pt}}
\multiput(165.00,384.17)(28.226,6.000){2}{\rule{1.150pt}{0.400pt}}
\multiput(198.00,391.59)(3.022,0.482){9}{\rule{2.367pt}{0.116pt}}
\multiput(198.00,390.17)(29.088,6.000){2}{\rule{1.183pt}{0.400pt}}
\multiput(265.00,397.61)(7.383,0.447){3}{\rule{4.633pt}{0.108pt}}
\multiput(265.00,396.17)(24.383,3.000){2}{\rule{2.317pt}{0.400pt}}
\multiput(299.00,400.59)(2.932,0.482){9}{\rule{2.300pt}{0.116pt}}
\multiput(299.00,399.17)(28.226,6.000){2}{\rule{1.150pt}{0.400pt}}
\multiput(332.00,406.59)(2.211,0.488){13}{\rule{1.800pt}{0.117pt}}
\multiput(332.00,405.17)(30.264,8.000){2}{\rule{0.900pt}{0.400pt}}
\multiput(366.00,414.61)(7.160,0.447){3}{\rule{4.500pt}{0.108pt}}
\multiput(366.00,413.17)(23.660,3.000){2}{\rule{2.250pt}{0.400pt}}
\put(399,417.17){\rule{6.900pt}{0.400pt}}
\multiput(399.00,416.17)(19.679,2.000){2}{\rule{3.450pt}{0.400pt}}
\multiput(433.00,419.61)(7.160,0.447){3}{\rule{4.500pt}{0.108pt}}
\multiput(433.00,418.17)(23.660,3.000){2}{\rule{2.250pt}{0.400pt}}
\multiput(466.00,422.59)(3.022,0.482){9}{\rule{2.367pt}{0.116pt}}
\multiput(466.00,421.17)(29.088,6.000){2}{\rule{1.183pt}{0.400pt}}
\put(500,427.67){\rule{7.950pt}{0.400pt}}
\multiput(500.00,427.17)(16.500,1.000){2}{\rule{3.975pt}{0.400pt}}
\multiput(533.00,429.58)(1.329,0.493){23}{\rule{1.146pt}{0.119pt}}
\multiput(533.00,428.17)(31.621,13.000){2}{\rule{0.573pt}{0.400pt}}
\multiput(567.00,442.59)(3.022,0.482){9}{\rule{2.367pt}{0.116pt}}
\multiput(567.00,441.17)(29.088,6.000){2}{\rule{1.183pt}{0.400pt}}
\multiput(601.00,448.59)(2.476,0.485){11}{\rule{1.986pt}{0.117pt}}
\multiput(601.00,447.17)(28.879,7.000){2}{\rule{0.993pt}{0.400pt}}
\multiput(634.00,455.58)(1.009,0.495){31}{\rule{0.900pt}{0.119pt}}
\multiput(634.00,454.17)(32.132,17.000){2}{\rule{0.450pt}{0.400pt}}
\multiput(668.00,472.59)(1.893,0.489){15}{\rule{1.567pt}{0.118pt}}
\multiput(668.00,471.17)(29.748,9.000){2}{\rule{0.783pt}{0.400pt}}
\multiput(701.00,481.58)(1.746,0.491){17}{\rule{1.460pt}{0.118pt}}
\multiput(701.00,480.17)(30.970,10.000){2}{\rule{0.730pt}{0.400pt}}
\multiput(735.00,491.60)(4.722,0.468){5}{\rule{3.400pt}{0.113pt}}
\multiput(735.00,490.17)(25.943,4.000){2}{\rule{1.700pt}{0.400pt}}
\multiput(768.00,495.61)(7.383,0.447){3}{\rule{4.633pt}{0.108pt}}
\multiput(768.00,494.17)(24.383,3.000){2}{\rule{2.317pt}{0.400pt}}
\multiput(802.00,498.59)(2.145,0.488){13}{\rule{1.750pt}{0.117pt}}
\multiput(802.00,497.17)(29.368,8.000){2}{\rule{0.875pt}{0.400pt}}
\multiput(835.00,506.59)(2.211,0.488){13}{\rule{1.800pt}{0.117pt}}
\multiput(835.00,505.17)(30.264,8.000){2}{\rule{0.900pt}{0.400pt}}
\multiput(869.00,514.58)(1.041,0.494){29}{\rule{0.925pt}{0.119pt}}
\multiput(869.00,513.17)(31.080,16.000){2}{\rule{0.463pt}{0.400pt}}
\multiput(902.00,530.59)(3.716,0.477){7}{\rule{2.820pt}{0.115pt}}
\multiput(902.00,529.17)(28.147,5.000){2}{\rule{1.410pt}{0.400pt}}
\multiput(936.00,535.59)(2.145,0.488){13}{\rule{1.750pt}{0.117pt}}
\multiput(936.00,534.17)(29.368,8.000){2}{\rule{0.875pt}{0.400pt}}
\multiput(969.00,543.58)(1.581,0.492){19}{\rule{1.336pt}{0.118pt}}
\multiput(969.00,542.17)(31.226,11.000){2}{\rule{0.668pt}{0.400pt}}
\multiput(1003.00,554.59)(2.211,0.488){13}{\rule{1.800pt}{0.117pt}}
\multiput(1003.00,553.17)(30.264,8.000){2}{\rule{0.900pt}{0.400pt}}
\multiput(1037.00,562.59)(2.476,0.485){11}{\rule{1.986pt}{0.117pt}}
\multiput(1037.00,561.17)(28.879,7.000){2}{\rule{0.993pt}{0.400pt}}
\multiput(1070.00,569.59)(2.211,0.488){13}{\rule{1.800pt}{0.117pt}}
\multiput(1070.00,568.17)(30.264,8.000){2}{\rule{0.900pt}{0.400pt}}
\multiput(1104.00,577.59)(2.145,0.488){13}{\rule{1.750pt}{0.117pt}}
\multiput(1104.00,576.17)(29.368,8.000){2}{\rule{0.875pt}{0.400pt}}
\multiput(1137.00,585.59)(2.211,0.488){13}{\rule{1.800pt}{0.117pt}}
\multiput(1137.00,584.17)(30.264,8.000){2}{\rule{0.900pt}{0.400pt}}
\multiput(1171.00,593.59)(2.145,0.488){13}{\rule{1.750pt}{0.117pt}}
\multiput(1171.00,592.17)(29.368,8.000){2}{\rule{0.875pt}{0.400pt}}
\multiput(1204.00,601.59)(2.211,0.488){13}{\rule{1.800pt}{0.117pt}}
\multiput(1204.00,600.17)(30.264,8.000){2}{\rule{0.900pt}{0.400pt}}
\multiput(1238.00,609.59)(1.893,0.489){15}{\rule{1.567pt}{0.118pt}}
\multiput(1238.00,608.17)(29.748,9.000){2}{\rule{0.783pt}{0.400pt}}
\multiput(1271.00,618.59)(2.211,0.488){13}{\rule{1.800pt}{0.117pt}}
\multiput(1271.00,617.17)(30.264,8.000){2}{\rule{0.900pt}{0.400pt}}
\multiput(1305.00,626.59)(2.145,0.488){13}{\rule{1.750pt}{0.117pt}}
\multiput(1305.00,625.17)(29.368,8.000){2}{\rule{0.875pt}{0.400pt}}
\multiput(1338.00,634.59)(1.951,0.489){15}{\rule{1.611pt}{0.118pt}}
\multiput(1338.00,633.17)(30.656,9.000){2}{\rule{0.806pt}{0.400pt}}
\multiput(1372.00,643.59)(2.145,0.488){13}{\rule{1.750pt}{0.117pt}}
\multiput(1372.00,642.17)(29.368,8.000){2}{\rule{0.875pt}{0.400pt}}
\multiput(1405.00,651.59)(1.951,0.489){15}{\rule{1.611pt}{0.118pt}}
\multiput(1405.00,650.17)(30.656,9.000){2}{\rule{0.806pt}{0.400pt}}
\put(131,379){\makebox(0,0){$+$}}
\put(165,385){\makebox(0,0){$+$}}
\put(198,391){\makebox(0,0){$+$}}
\put(232,397){\makebox(0,0){$+$}}
\put(265,397){\makebox(0,0){$+$}}
\put(299,400){\makebox(0,0){$+$}}
\put(332,406){\makebox(0,0){$+$}}
\put(366,414){\makebox(0,0){$+$}}
\put(399,417){\makebox(0,0){$+$}}
\put(433,419){\makebox(0,0){$+$}}
\put(466,422){\makebox(0,0){$+$}}
\put(500,428){\makebox(0,0){$+$}}
\put(533,429){\makebox(0,0){$+$}}
\put(567,442){\makebox(0,0){$+$}}
\put(601,448){\makebox(0,0){$+$}}
\put(634,455){\makebox(0,0){$+$}}
\put(668,472){\makebox(0,0){$+$}}
\put(701,481){\makebox(0,0){$+$}}
\put(735,491){\makebox(0,0){$+$}}
\put(768,495){\makebox(0,0){$+$}}
\put(802,498){\makebox(0,0){$+$}}
\put(835,506){\makebox(0,0){$+$}}
\put(869,514){\makebox(0,0){$+$}}
\put(902,530){\makebox(0,0){$+$}}
\put(936,535){\makebox(0,0){$+$}}
\put(969,543){\makebox(0,0){$+$}}
\put(1003,554){\makebox(0,0){$+$}}
\put(1037,562){\makebox(0,0){$+$}}
\put(1070,569){\makebox(0,0){$+$}}
\put(1104,577){\makebox(0,0){$+$}}
\put(1137,585){\makebox(0,0){$+$}}
\put(1171,593){\makebox(0,0){$+$}}
\put(1204,601){\makebox(0,0){$+$}}
\put(1238,609){\makebox(0,0){$+$}}
\put(1271,618){\makebox(0,0){$+$}}
\put(1305,626){\makebox(0,0){$+$}}
\put(1338,634){\makebox(0,0){$+$}}
\put(1372,643){\makebox(0,0){$+$}}
\put(1405,651){\makebox(0,0){$+$}}
\put(1439,660){\makebox(0,0){$+$}}
\put(1349,768){\makebox(0,0){$+$}}
\put(232.0,397.0){\rule[-0.200pt]{7.950pt}{0.400pt}}
\put(131.0,82.0){\rule[-0.200pt]{0.400pt}{187.179pt}}
\put(131.0,82.0){\rule[-0.200pt]{315.097pt}{0.400pt}}
\put(1439.0,82.0){\rule[-0.200pt]{0.400pt}{187.179pt}}
\put(131.0,859.0){\rule[-0.200pt]{315.097pt}{0.400pt}}
\end{picture}
}
   \end{frame}

  \begin{frame}
提升内容贡献者个体动机的仿真结果
\scalebox{0.8}{% GNUPLOT: LaTeX picture
\setlength{\unitlength}{0.240900pt}
\ifx\plotpoint\undefined\newsavebox{\plotpoint}\fi
\begin{picture}(1500,900)(0,0)
\sbox{\plotpoint}{\rule[-0.200pt]{0.400pt}{0.400pt}}%
\put(131.0,82.0){\rule[-0.200pt]{4.818pt}{0.400pt}}
\put(111,82){\makebox(0,0)[r]{ 0}}
\put(1419.0,82.0){\rule[-0.200pt]{4.818pt}{0.400pt}}
\put(131.0,237.0){\rule[-0.200pt]{4.818pt}{0.400pt}}
\put(111,237){\makebox(0,0)[r]{ 1}}
\put(1419.0,237.0){\rule[-0.200pt]{4.818pt}{0.400pt}}
\put(131.0,393.0){\rule[-0.200pt]{4.818pt}{0.400pt}}
\put(111,393){\makebox(0,0)[r]{ 2}}
\put(1419.0,393.0){\rule[-0.200pt]{4.818pt}{0.400pt}}
\put(131.0,548.0){\rule[-0.200pt]{4.818pt}{0.400pt}}
\put(111,548){\makebox(0,0)[r]{ 3}}
\put(1419.0,548.0){\rule[-0.200pt]{4.818pt}{0.400pt}}
\put(131.0,704.0){\rule[-0.200pt]{4.818pt}{0.400pt}}
\put(111,704){\makebox(0,0)[r]{ 4}}
\put(1419.0,704.0){\rule[-0.200pt]{4.818pt}{0.400pt}}
\put(131.0,82.0){\rule[-0.200pt]{0.400pt}{4.818pt}}
\put(131,41){\makebox(0,0){ 1}}
\put(131.0,839.0){\rule[-0.200pt]{0.400pt}{4.818pt}}
\put(198.0,82.0){\rule[-0.200pt]{0.400pt}{4.818pt}}
\put(198,41){\makebox(0,0){ 3}}
\put(198.0,839.0){\rule[-0.200pt]{0.400pt}{4.818pt}}
\put(265.0,82.0){\rule[-0.200pt]{0.400pt}{4.818pt}}
\put(265,41){\makebox(0,0){ 5}}
\put(265.0,839.0){\rule[-0.200pt]{0.400pt}{4.818pt}}
\put(332.0,82.0){\rule[-0.200pt]{0.400pt}{4.818pt}}
\put(332,41){\makebox(0,0){ 7}}
\put(332.0,839.0){\rule[-0.200pt]{0.400pt}{4.818pt}}
\put(399.0,82.0){\rule[-0.200pt]{0.400pt}{4.818pt}}
\put(399,41){\makebox(0,0){ 9}}
\put(399.0,839.0){\rule[-0.200pt]{0.400pt}{4.818pt}}
\put(466.0,82.0){\rule[-0.200pt]{0.400pt}{4.818pt}}
\put(466,41){\makebox(0,0){ 11}}
\put(466.0,839.0){\rule[-0.200pt]{0.400pt}{4.818pt}}
\put(533.0,82.0){\rule[-0.200pt]{0.400pt}{4.818pt}}
\put(533,41){\makebox(0,0){ 13}}
\put(533.0,839.0){\rule[-0.200pt]{0.400pt}{4.818pt}}
\put(601.0,82.0){\rule[-0.200pt]{0.400pt}{4.818pt}}
\put(601,41){\makebox(0,0){ 15}}
\put(601.0,839.0){\rule[-0.200pt]{0.400pt}{4.818pt}}
\put(668.0,82.0){\rule[-0.200pt]{0.400pt}{4.818pt}}
\put(668,41){\makebox(0,0){ 17}}
\put(668.0,839.0){\rule[-0.200pt]{0.400pt}{4.818pt}}
\put(735.0,82.0){\rule[-0.200pt]{0.400pt}{4.818pt}}
\put(735,41){\makebox(0,0){ 19}}
\put(735.0,839.0){\rule[-0.200pt]{0.400pt}{4.818pt}}
\put(802.0,82.0){\rule[-0.200pt]{0.400pt}{4.818pt}}
\put(802,41){\makebox(0,0){ 21}}
\put(802.0,839.0){\rule[-0.200pt]{0.400pt}{4.818pt}}
\put(869.0,82.0){\rule[-0.200pt]{0.400pt}{4.818pt}}
\put(869,41){\makebox(0,0){ 23}}
\put(869.0,839.0){\rule[-0.200pt]{0.400pt}{4.818pt}}
\put(936.0,82.0){\rule[-0.200pt]{0.400pt}{4.818pt}}
\put(936,41){\makebox(0,0){ 25}}
\put(936.0,839.0){\rule[-0.200pt]{0.400pt}{4.818pt}}
\put(1003.0,82.0){\rule[-0.200pt]{0.400pt}{4.818pt}}
\put(1003,41){\makebox(0,0){ 27}}
\put(1003.0,839.0){\rule[-0.200pt]{0.400pt}{4.818pt}}
\put(1070.0,82.0){\rule[-0.200pt]{0.400pt}{4.818pt}}
\put(1070,41){\makebox(0,0){ 29}}
\put(1070.0,839.0){\rule[-0.200pt]{0.400pt}{4.818pt}}
\put(1137.0,82.0){\rule[-0.200pt]{0.400pt}{4.818pt}}
\put(1137,41){\makebox(0,0){ 31}}
\put(1137.0,839.0){\rule[-0.200pt]{0.400pt}{4.818pt}}
\put(1204.0,82.0){\rule[-0.200pt]{0.400pt}{4.818pt}}
\put(1204,41){\makebox(0,0){ 33}}
\put(1204.0,839.0){\rule[-0.200pt]{0.400pt}{4.818pt}}
\put(1271.0,82.0){\rule[-0.200pt]{0.400pt}{4.818pt}}
\put(1271,41){\makebox(0,0){ 35}}
\put(1271.0,839.0){\rule[-0.200pt]{0.400pt}{4.818pt}}
\put(1338.0,82.0){\rule[-0.200pt]{0.400pt}{4.818pt}}
\put(1338,41){\makebox(0,0){ 37}}
\put(1338.0,839.0){\rule[-0.200pt]{0.400pt}{4.818pt}}
\put(1405.0,82.0){\rule[-0.200pt]{0.400pt}{4.818pt}}
\put(1405,41){\makebox(0,0){ 39}}
\put(1405.0,839.0){\rule[-0.200pt]{0.400pt}{4.818pt}}
\put(131.0,82.0){\rule[-0.200pt]{0.400pt}{187.179pt}}
\put(131.0,82.0){\rule[-0.200pt]{315.097pt}{0.400pt}}
\put(1439.0,82.0){\rule[-0.200pt]{0.400pt}{187.179pt}}
\put(131.0,859.0){\rule[-0.200pt]{315.097pt}{0.400pt}}
\put(30,470){\makebox(0,0){\rotatebox{90}{用户贡献}}}
\put(1279,819){\makebox(0,0)[r]{仿真数据}}
\put(1299.0,819.0){\rule[-0.200pt]{24.090pt}{0.400pt}}
\put(1439,640){\usebox{\plotpoint}}
\multiput(1427.29,638.93)(-3.716,-0.477){7}{\rule{2.820pt}{0.115pt}}
\multiput(1433.15,639.17)(-28.147,-5.000){2}{\rule{1.410pt}{0.400pt}}
\multiput(1393.63,633.93)(-3.604,-0.477){7}{\rule{2.740pt}{0.115pt}}
\multiput(1399.31,634.17)(-27.313,-5.000){2}{\rule{1.370pt}{0.400pt}}
\multiput(1360.29,628.93)(-3.716,-0.477){7}{\rule{2.820pt}{0.115pt}}
\multiput(1366.15,629.17)(-28.147,-5.000){2}{\rule{1.410pt}{0.400pt}}
\multiput(1326.63,623.93)(-3.604,-0.477){7}{\rule{2.740pt}{0.115pt}}
\multiput(1332.31,624.17)(-27.313,-5.000){2}{\rule{1.370pt}{0.400pt}}
\multiput(1293.29,618.93)(-3.716,-0.477){7}{\rule{2.820pt}{0.115pt}}
\multiput(1299.15,619.17)(-28.147,-5.000){2}{\rule{1.410pt}{0.400pt}}
\multiput(1256.89,613.94)(-4.722,-0.468){5}{\rule{3.400pt}{0.113pt}}
\multiput(1263.94,614.17)(-25.943,-4.000){2}{\rule{1.700pt}{0.400pt}}
\multiput(1226.29,609.93)(-3.716,-0.477){7}{\rule{2.820pt}{0.115pt}}
\multiput(1232.15,610.17)(-28.147,-5.000){2}{\rule{1.410pt}{0.400pt}}
\multiput(1189.89,604.94)(-4.722,-0.468){5}{\rule{3.400pt}{0.113pt}}
\multiput(1196.94,605.17)(-25.943,-4.000){2}{\rule{1.700pt}{0.400pt}}
\multiput(1156.47,600.94)(-4.868,-0.468){5}{\rule{3.500pt}{0.113pt}}
\multiput(1163.74,601.17)(-26.736,-4.000){2}{\rule{1.750pt}{0.400pt}}
\multiput(1125.63,596.93)(-3.604,-0.477){7}{\rule{2.740pt}{0.115pt}}
\multiput(1131.31,597.17)(-27.313,-5.000){2}{\rule{1.370pt}{0.400pt}}
\multiput(1089.47,591.94)(-4.868,-0.468){5}{\rule{3.500pt}{0.113pt}}
\multiput(1096.74,592.17)(-26.736,-4.000){2}{\rule{1.750pt}{0.400pt}}
\multiput(1051.32,587.95)(-7.160,-0.447){3}{\rule{4.500pt}{0.108pt}}
\multiput(1060.66,588.17)(-23.660,-3.000){2}{\rule{2.250pt}{0.400pt}}
\multiput(1022.47,584.94)(-4.868,-0.468){5}{\rule{3.500pt}{0.113pt}}
\multiput(1029.74,585.17)(-26.736,-4.000){2}{\rule{1.750pt}{0.400pt}}
\multiput(988.47,580.94)(-4.868,-0.468){5}{\rule{3.500pt}{0.113pt}}
\multiput(995.74,581.17)(-26.736,-4.000){2}{\rule{1.750pt}{0.400pt}}
\multiput(950.32,576.95)(-7.160,-0.447){3}{\rule{4.500pt}{0.108pt}}
\multiput(959.66,577.17)(-23.660,-3.000){2}{\rule{2.250pt}{0.400pt}}
\multiput(921.47,573.94)(-4.868,-0.468){5}{\rule{3.500pt}{0.113pt}}
\multiput(928.74,574.17)(-26.736,-4.000){2}{\rule{1.750pt}{0.400pt}}
\multiput(883.32,569.95)(-7.160,-0.447){3}{\rule{4.500pt}{0.108pt}}
\multiput(892.66,570.17)(-23.660,-3.000){2}{\rule{2.250pt}{0.400pt}}
\multiput(849.77,566.95)(-7.383,-0.447){3}{\rule{4.633pt}{0.108pt}}
\multiput(859.38,567.17)(-24.383,-3.000){2}{\rule{2.317pt}{0.400pt}}
\multiput(816.32,563.95)(-7.160,-0.447){3}{\rule{4.500pt}{0.108pt}}
\multiput(825.66,564.17)(-23.660,-3.000){2}{\rule{2.250pt}{0.400pt}}
\multiput(782.77,560.95)(-7.383,-0.447){3}{\rule{4.633pt}{0.108pt}}
\multiput(792.38,561.17)(-24.383,-3.000){2}{\rule{2.317pt}{0.400pt}}
\put(735,557.17){\rule{6.700pt}{0.400pt}}
\multiput(754.09,558.17)(-19.094,-2.000){2}{\rule{3.350pt}{0.400pt}}
\multiput(715.77,555.95)(-7.383,-0.447){3}{\rule{4.633pt}{0.108pt}}
\multiput(725.38,556.17)(-24.383,-3.000){2}{\rule{2.317pt}{0.400pt}}
\multiput(682.32,552.95)(-7.160,-0.447){3}{\rule{4.500pt}{0.108pt}}
\multiput(691.66,553.17)(-23.660,-3.000){2}{\rule{2.250pt}{0.400pt}}
\put(634,549.17){\rule{6.900pt}{0.400pt}}
\multiput(653.68,550.17)(-19.679,-2.000){2}{\rule{3.450pt}{0.400pt}}
\put(601,547.17){\rule{6.700pt}{0.400pt}}
\multiput(620.09,548.17)(-19.094,-2.000){2}{\rule{3.350pt}{0.400pt}}
\put(567,545.17){\rule{6.900pt}{0.400pt}}
\multiput(586.68,546.17)(-19.679,-2.000){2}{\rule{3.450pt}{0.400pt}}
\put(533,543.17){\rule{6.900pt}{0.400pt}}
\multiput(552.68,544.17)(-19.679,-2.000){2}{\rule{3.450pt}{0.400pt}}
\put(500,541.17){\rule{6.700pt}{0.400pt}}
\multiput(519.09,542.17)(-19.094,-2.000){2}{\rule{3.350pt}{0.400pt}}
\put(466,539.17){\rule{6.900pt}{0.400pt}}
\multiput(485.68,540.17)(-19.679,-2.000){2}{\rule{3.450pt}{0.400pt}}
\put(433,537.17){\rule{6.700pt}{0.400pt}}
\multiput(452.09,538.17)(-19.094,-2.000){2}{\rule{3.350pt}{0.400pt}}
\put(399,535.67){\rule{8.191pt}{0.400pt}}
\multiput(416.00,536.17)(-17.000,-1.000){2}{\rule{4.095pt}{0.400pt}}
\put(366,534.67){\rule{7.950pt}{0.400pt}}
\multiput(382.50,535.17)(-16.500,-1.000){2}{\rule{3.975pt}{0.400pt}}
\put(332,533.17){\rule{6.900pt}{0.400pt}}
\multiput(351.68,534.17)(-19.679,-2.000){2}{\rule{3.450pt}{0.400pt}}
\put(299,531.67){\rule{7.950pt}{0.400pt}}
\multiput(315.50,532.17)(-16.500,-1.000){2}{\rule{3.975pt}{0.400pt}}
\put(265,530.67){\rule{8.191pt}{0.400pt}}
\multiput(282.00,531.17)(-17.000,-1.000){2}{\rule{4.095pt}{0.400pt}}
\put(232,529.67){\rule{7.950pt}{0.400pt}}
\multiput(248.50,530.17)(-16.500,-1.000){2}{\rule{3.975pt}{0.400pt}}
\put(165,528.67){\rule{7.950pt}{0.400pt}}
\multiput(181.50,529.17)(-16.500,-1.000){2}{\rule{3.975pt}{0.400pt}}
\put(198.0,530.0){\rule[-0.200pt]{8.191pt}{0.400pt}}
\put(131.0,529.0){\rule[-0.200pt]{8.191pt}{0.400pt}}
\put(1279,768){\makebox(0,0)[r]{改进数据}}
\put(1299.0,768.0){\rule[-0.200pt]{24.090pt}{0.400pt}}
\put(1439,659){\usebox{\plotpoint}}
\multiput(1432.31,657.93)(-1.951,-0.489){15}{\rule{1.611pt}{0.118pt}}
\multiput(1435.66,658.17)(-30.656,-9.000){2}{\rule{0.806pt}{0.400pt}}
\put(1372,648.17){\rule{6.700pt}{0.400pt}}
\multiput(1391.09,649.17)(-19.094,-2.000){2}{\rule{3.350pt}{0.400pt}}
\multiput(1364.53,646.93)(-2.211,-0.488){13}{\rule{1.800pt}{0.117pt}}
\multiput(1368.26,647.17)(-30.264,-8.000){2}{\rule{0.900pt}{0.400pt}}
\multiput(1319.32,638.95)(-7.160,-0.447){3}{\rule{4.500pt}{0.108pt}}
\multiput(1328.66,639.17)(-23.660,-3.000){2}{\rule{2.250pt}{0.400pt}}
\put(1271,635.67){\rule{8.191pt}{0.400pt}}
\multiput(1288.00,636.17)(-17.000,-1.000){2}{\rule{4.095pt}{0.400pt}}
\multiput(1256.89,634.94)(-4.722,-0.468){5}{\rule{3.400pt}{0.113pt}}
\multiput(1263.94,635.17)(-25.943,-4.000){2}{\rule{1.700pt}{0.400pt}}
\put(1204,630.17){\rule{6.900pt}{0.400pt}}
\multiput(1223.68,631.17)(-19.679,-2.000){2}{\rule{3.450pt}{0.400pt}}
\multiput(1198.60,628.92)(-1.534,-0.492){19}{\rule{1.300pt}{0.118pt}}
\multiput(1201.30,629.17)(-30.302,-11.000){2}{\rule{0.650pt}{0.400pt}}
\multiput(1122.89,617.94)(-4.722,-0.468){5}{\rule{3.400pt}{0.113pt}}
\multiput(1129.94,618.17)(-25.943,-4.000){2}{\rule{1.700pt}{0.400pt}}
\put(1137.0,619.0){\rule[-0.200pt]{8.191pt}{0.400pt}}
\multiput(1058.63,613.93)(-3.604,-0.477){7}{\rule{2.740pt}{0.115pt}}
\multiput(1064.31,614.17)(-27.313,-5.000){2}{\rule{1.370pt}{0.400pt}}
\multiput(1025.29,608.93)(-3.716,-0.477){7}{\rule{2.820pt}{0.115pt}}
\multiput(1031.15,609.17)(-28.147,-5.000){2}{\rule{1.410pt}{0.400pt}}
\put(969,603.17){\rule{6.900pt}{0.400pt}}
\multiput(988.68,604.17)(-19.679,-2.000){2}{\rule{3.450pt}{0.400pt}}
\multiput(963.60,601.92)(-1.534,-0.492){19}{\rule{1.300pt}{0.118pt}}
\multiput(966.30,602.17)(-30.302,-11.000){2}{\rule{0.650pt}{0.400pt}}
\put(902,590.67){\rule{8.191pt}{0.400pt}}
\multiput(919.00,591.17)(-17.000,-1.000){2}{\rule{4.095pt}{0.400pt}}
\multiput(883.32,589.95)(-7.160,-0.447){3}{\rule{4.500pt}{0.108pt}}
\multiput(892.66,590.17)(-23.660,-3.000){2}{\rule{2.250pt}{0.400pt}}
\multiput(849.77,586.95)(-7.383,-0.447){3}{\rule{4.633pt}{0.108pt}}
\multiput(859.38,587.17)(-24.383,-3.000){2}{\rule{2.317pt}{0.400pt}}
\multiput(826.76,583.93)(-2.476,-0.485){11}{\rule{1.986pt}{0.117pt}}
\multiput(830.88,584.17)(-28.879,-7.000){2}{\rule{0.993pt}{0.400pt}}
\multiput(782.77,578.61)(-7.383,0.447){3}{\rule{4.633pt}{0.108pt}}
\multiput(792.38,577.17)(-24.383,3.000){2}{\rule{2.317pt}{0.400pt}}
\multiput(759.76,579.93)(-2.476,-0.485){11}{\rule{1.986pt}{0.117pt}}
\multiput(763.88,580.17)(-28.879,-7.000){2}{\rule{0.993pt}{0.400pt}}
\multiput(715.77,572.95)(-7.383,-0.447){3}{\rule{4.633pt}{0.108pt}}
\multiput(725.38,573.17)(-24.383,-3.000){2}{\rule{2.317pt}{0.400pt}}
\put(668,569.67){\rule{7.950pt}{0.400pt}}
\multiput(684.50,570.17)(-16.500,-1.000){2}{\rule{3.975pt}{0.400pt}}
\put(1070.0,615.0){\rule[-0.200pt]{8.191pt}{0.400pt}}
\multiput(586.47,568.94)(-4.868,-0.468){5}{\rule{3.500pt}{0.113pt}}
\multiput(593.74,569.17)(-26.736,-4.000){2}{\rule{1.750pt}{0.400pt}}
\put(533,566.17){\rule{6.900pt}{0.400pt}}
\multiput(552.68,565.17)(-19.679,2.000){2}{\rule{3.450pt}{0.400pt}}
\put(500,568.17){\rule{6.700pt}{0.400pt}}
\multiput(519.09,567.17)(-19.094,2.000){2}{\rule{3.350pt}{0.400pt}}
\multiput(485.47,568.94)(-4.868,-0.468){5}{\rule{3.500pt}{0.113pt}}
\multiput(492.74,569.17)(-26.736,-4.000){2}{\rule{1.750pt}{0.400pt}}
\put(433,565.67){\rule{7.950pt}{0.400pt}}
\multiput(449.50,565.17)(-16.500,1.000){2}{\rule{3.975pt}{0.400pt}}
\multiput(423.18,565.93)(-3.022,-0.482){9}{\rule{2.367pt}{0.116pt}}
\multiput(428.09,566.17)(-29.088,-6.000){2}{\rule{1.183pt}{0.400pt}}
\multiput(387.63,559.93)(-3.604,-0.477){7}{\rule{2.740pt}{0.115pt}}
\multiput(393.31,560.17)(-27.313,-5.000){2}{\rule{1.370pt}{0.400pt}}
\put(332,555.67){\rule{8.191pt}{0.400pt}}
\multiput(349.00,555.17)(-17.000,1.000){2}{\rule{4.095pt}{0.400pt}}
\multiput(313.32,555.95)(-7.160,-0.447){3}{\rule{4.500pt}{0.108pt}}
\multiput(322.66,556.17)(-23.660,-3.000){2}{\rule{2.250pt}{0.400pt}}
\multiput(291.53,552.93)(-2.211,-0.488){13}{\rule{1.800pt}{0.117pt}}
\multiput(295.26,553.17)(-30.264,-8.000){2}{\rule{0.900pt}{0.400pt}}
\multiput(246.32,546.61)(-7.160,0.447){3}{\rule{4.500pt}{0.108pt}}
\multiput(255.66,545.17)(-23.660,3.000){2}{\rule{2.250pt}{0.400pt}}
\multiput(224.53,547.93)(-2.211,-0.488){13}{\rule{1.800pt}{0.117pt}}
\multiput(228.26,548.17)(-30.264,-8.000){2}{\rule{0.900pt}{0.400pt}}
\multiput(179.32,539.95)(-7.160,-0.447){3}{\rule{4.500pt}{0.108pt}}
\multiput(188.66,540.17)(-23.660,-3.000){2}{\rule{2.250pt}{0.400pt}}
\multiput(156.52,536.93)(-2.552,-0.485){11}{\rule{2.043pt}{0.117pt}}
\multiput(160.76,537.17)(-29.760,-7.000){2}{\rule{1.021pt}{0.400pt}}
\put(1439,659){\makebox(0,0){$+$}}
\put(1405,650){\makebox(0,0){$+$}}
\put(1372,648){\makebox(0,0){$+$}}
\put(1338,640){\makebox(0,0){$+$}}
\put(1305,637){\makebox(0,0){$+$}}
\put(1271,636){\makebox(0,0){$+$}}
\put(1238,632){\makebox(0,0){$+$}}
\put(1204,630){\makebox(0,0){$+$}}
\put(1171,619){\makebox(0,0){$+$}}
\put(1137,619){\makebox(0,0){$+$}}
\put(1104,615){\makebox(0,0){$+$}}
\put(1070,615){\makebox(0,0){$+$}}
\put(1037,610){\makebox(0,0){$+$}}
\put(1003,605){\makebox(0,0){$+$}}
\put(969,603){\makebox(0,0){$+$}}
\put(936,592){\makebox(0,0){$+$}}
\put(902,591){\makebox(0,0){$+$}}
\put(869,588){\makebox(0,0){$+$}}
\put(835,585){\makebox(0,0){$+$}}
\put(802,578){\makebox(0,0){$+$}}
\put(768,581){\makebox(0,0){$+$}}
\put(735,574){\makebox(0,0){$+$}}
\put(701,571){\makebox(0,0){$+$}}
\put(668,570){\makebox(0,0){$+$}}
\put(634,570){\makebox(0,0){$+$}}
\put(601,570){\makebox(0,0){$+$}}
\put(567,566){\makebox(0,0){$+$}}
\put(533,568){\makebox(0,0){$+$}}
\put(500,570){\makebox(0,0){$+$}}
\put(466,566){\makebox(0,0){$+$}}
\put(433,567){\makebox(0,0){$+$}}
\put(399,561){\makebox(0,0){$+$}}
\put(366,556){\makebox(0,0){$+$}}
\put(332,557){\makebox(0,0){$+$}}
\put(299,554){\makebox(0,0){$+$}}
\put(265,546){\makebox(0,0){$+$}}
\put(232,549){\makebox(0,0){$+$}}
\put(198,541){\makebox(0,0){$+$}}
\put(165,538){\makebox(0,0){$+$}}
\put(131,531){\makebox(0,0){$+$}}
\put(1349,768){\makebox(0,0){$+$}}
\put(601.0,570.0){\rule[-0.200pt]{16.140pt}{0.400pt}}
\put(131.0,82.0){\rule[-0.200pt]{0.400pt}{187.179pt}}
\put(131.0,82.0){\rule[-0.200pt]{315.097pt}{0.400pt}}
\put(1439.0,82.0){\rule[-0.200pt]{0.400pt}{187.179pt}}
\put(131.0,859.0){\rule[-0.200pt]{315.097pt}{0.400pt}}
\end{picture}
}
   \end{frame}

  \begin{frame}
提升内容贡献者人际动机的仿真结果
\scalebox{0.8}{% GNUPLOT: LaTeX picture
\setlength{\unitlength}{0.240900pt}
\ifx\plotpoint\undefined\newsavebox{\plotpoint}\fi
\begin{picture}(1500,900)(0,0)
\sbox{\plotpoint}{\rule[-0.200pt]{0.400pt}{0.400pt}}%
\put(131.0,82.0){\rule[-0.200pt]{4.818pt}{0.400pt}}
\put(111,82){\makebox(0,0)[r]{ 0}}
\put(1419.0,82.0){\rule[-0.200pt]{4.818pt}{0.400pt}}
\put(131.0,237.0){\rule[-0.200pt]{4.818pt}{0.400pt}}
\put(111,237){\makebox(0,0)[r]{ 1}}
\put(1419.0,237.0){\rule[-0.200pt]{4.818pt}{0.400pt}}
\put(131.0,393.0){\rule[-0.200pt]{4.818pt}{0.400pt}}
\put(111,393){\makebox(0,0)[r]{ 2}}
\put(1419.0,393.0){\rule[-0.200pt]{4.818pt}{0.400pt}}
\put(131.0,548.0){\rule[-0.200pt]{4.818pt}{0.400pt}}
\put(111,548){\makebox(0,0)[r]{ 3}}
\put(1419.0,548.0){\rule[-0.200pt]{4.818pt}{0.400pt}}
\put(131.0,704.0){\rule[-0.200pt]{4.818pt}{0.400pt}}
\put(111,704){\makebox(0,0)[r]{ 4}}
\put(1419.0,704.0){\rule[-0.200pt]{4.818pt}{0.400pt}}
\put(131.0,82.0){\rule[-0.200pt]{0.400pt}{4.818pt}}
\put(131,41){\makebox(0,0){ 1}}
\put(131.0,839.0){\rule[-0.200pt]{0.400pt}{4.818pt}}
\put(198.0,82.0){\rule[-0.200pt]{0.400pt}{4.818pt}}
\put(198,41){\makebox(0,0){ 3}}
\put(198.0,839.0){\rule[-0.200pt]{0.400pt}{4.818pt}}
\put(265.0,82.0){\rule[-0.200pt]{0.400pt}{4.818pt}}
\put(265,41){\makebox(0,0){ 5}}
\put(265.0,839.0){\rule[-0.200pt]{0.400pt}{4.818pt}}
\put(332.0,82.0){\rule[-0.200pt]{0.400pt}{4.818pt}}
\put(332,41){\makebox(0,0){ 7}}
\put(332.0,839.0){\rule[-0.200pt]{0.400pt}{4.818pt}}
\put(399.0,82.0){\rule[-0.200pt]{0.400pt}{4.818pt}}
\put(399,41){\makebox(0,0){ 9}}
\put(399.0,839.0){\rule[-0.200pt]{0.400pt}{4.818pt}}
\put(466.0,82.0){\rule[-0.200pt]{0.400pt}{4.818pt}}
\put(466,41){\makebox(0,0){ 11}}
\put(466.0,839.0){\rule[-0.200pt]{0.400pt}{4.818pt}}
\put(533.0,82.0){\rule[-0.200pt]{0.400pt}{4.818pt}}
\put(533,41){\makebox(0,0){ 13}}
\put(533.0,839.0){\rule[-0.200pt]{0.400pt}{4.818pt}}
\put(601.0,82.0){\rule[-0.200pt]{0.400pt}{4.818pt}}
\put(601,41){\makebox(0,0){ 15}}
\put(601.0,839.0){\rule[-0.200pt]{0.400pt}{4.818pt}}
\put(668.0,82.0){\rule[-0.200pt]{0.400pt}{4.818pt}}
\put(668,41){\makebox(0,0){ 17}}
\put(668.0,839.0){\rule[-0.200pt]{0.400pt}{4.818pt}}
\put(735.0,82.0){\rule[-0.200pt]{0.400pt}{4.818pt}}
\put(735,41){\makebox(0,0){ 19}}
\put(735.0,839.0){\rule[-0.200pt]{0.400pt}{4.818pt}}
\put(802.0,82.0){\rule[-0.200pt]{0.400pt}{4.818pt}}
\put(802,41){\makebox(0,0){ 21}}
\put(802.0,839.0){\rule[-0.200pt]{0.400pt}{4.818pt}}
\put(869.0,82.0){\rule[-0.200pt]{0.400pt}{4.818pt}}
\put(869,41){\makebox(0,0){ 23}}
\put(869.0,839.0){\rule[-0.200pt]{0.400pt}{4.818pt}}
\put(936.0,82.0){\rule[-0.200pt]{0.400pt}{4.818pt}}
\put(936,41){\makebox(0,0){ 25}}
\put(936.0,839.0){\rule[-0.200pt]{0.400pt}{4.818pt}}
\put(1003.0,82.0){\rule[-0.200pt]{0.400pt}{4.818pt}}
\put(1003,41){\makebox(0,0){ 27}}
\put(1003.0,839.0){\rule[-0.200pt]{0.400pt}{4.818pt}}
\put(1070.0,82.0){\rule[-0.200pt]{0.400pt}{4.818pt}}
\put(1070,41){\makebox(0,0){ 29}}
\put(1070.0,839.0){\rule[-0.200pt]{0.400pt}{4.818pt}}
\put(1137.0,82.0){\rule[-0.200pt]{0.400pt}{4.818pt}}
\put(1137,41){\makebox(0,0){ 31}}
\put(1137.0,839.0){\rule[-0.200pt]{0.400pt}{4.818pt}}
\put(1204.0,82.0){\rule[-0.200pt]{0.400pt}{4.818pt}}
\put(1204,41){\makebox(0,0){ 33}}
\put(1204.0,839.0){\rule[-0.200pt]{0.400pt}{4.818pt}}
\put(1271.0,82.0){\rule[-0.200pt]{0.400pt}{4.818pt}}
\put(1271,41){\makebox(0,0){ 35}}
\put(1271.0,839.0){\rule[-0.200pt]{0.400pt}{4.818pt}}
\put(1338.0,82.0){\rule[-0.200pt]{0.400pt}{4.818pt}}
\put(1338,41){\makebox(0,0){ 37}}
\put(1338.0,839.0){\rule[-0.200pt]{0.400pt}{4.818pt}}
\put(1405.0,82.0){\rule[-0.200pt]{0.400pt}{4.818pt}}
\put(1405,41){\makebox(0,0){ 39}}
\put(1405.0,839.0){\rule[-0.200pt]{0.400pt}{4.818pt}}
\put(131.0,82.0){\rule[-0.200pt]{0.400pt}{187.179pt}}
\put(131.0,82.0){\rule[-0.200pt]{315.097pt}{0.400pt}}
\put(1439.0,82.0){\rule[-0.200pt]{0.400pt}{187.179pt}}
\put(131.0,859.0){\rule[-0.200pt]{315.097pt}{0.400pt}}
\put(30,470){\makebox(0,0){\rotatebox{90}{用户贡献}}}
\put(1279,819){\makebox(0,0)[r]{仿真数据}}
\put(1299.0,819.0){\rule[-0.200pt]{24.090pt}{0.400pt}}
\put(1439,640){\usebox{\plotpoint}}
\multiput(1427.29,638.93)(-3.716,-0.477){7}{\rule{2.820pt}{0.115pt}}
\multiput(1433.15,639.17)(-28.147,-5.000){2}{\rule{1.410pt}{0.400pt}}
\multiput(1393.63,633.93)(-3.604,-0.477){7}{\rule{2.740pt}{0.115pt}}
\multiput(1399.31,634.17)(-27.313,-5.000){2}{\rule{1.370pt}{0.400pt}}
\multiput(1360.29,628.93)(-3.716,-0.477){7}{\rule{2.820pt}{0.115pt}}
\multiput(1366.15,629.17)(-28.147,-5.000){2}{\rule{1.410pt}{0.400pt}}
\multiput(1326.63,623.93)(-3.604,-0.477){7}{\rule{2.740pt}{0.115pt}}
\multiput(1332.31,624.17)(-27.313,-5.000){2}{\rule{1.370pt}{0.400pt}}
\multiput(1293.29,618.93)(-3.716,-0.477){7}{\rule{2.820pt}{0.115pt}}
\multiput(1299.15,619.17)(-28.147,-5.000){2}{\rule{1.410pt}{0.400pt}}
\multiput(1256.89,613.94)(-4.722,-0.468){5}{\rule{3.400pt}{0.113pt}}
\multiput(1263.94,614.17)(-25.943,-4.000){2}{\rule{1.700pt}{0.400pt}}
\multiput(1226.29,609.93)(-3.716,-0.477){7}{\rule{2.820pt}{0.115pt}}
\multiput(1232.15,610.17)(-28.147,-5.000){2}{\rule{1.410pt}{0.400pt}}
\multiput(1189.89,604.94)(-4.722,-0.468){5}{\rule{3.400pt}{0.113pt}}
\multiput(1196.94,605.17)(-25.943,-4.000){2}{\rule{1.700pt}{0.400pt}}
\multiput(1156.47,600.94)(-4.868,-0.468){5}{\rule{3.500pt}{0.113pt}}
\multiput(1163.74,601.17)(-26.736,-4.000){2}{\rule{1.750pt}{0.400pt}}
\multiput(1125.63,596.93)(-3.604,-0.477){7}{\rule{2.740pt}{0.115pt}}
\multiput(1131.31,597.17)(-27.313,-5.000){2}{\rule{1.370pt}{0.400pt}}
\multiput(1089.47,591.94)(-4.868,-0.468){5}{\rule{3.500pt}{0.113pt}}
\multiput(1096.74,592.17)(-26.736,-4.000){2}{\rule{1.750pt}{0.400pt}}
\multiput(1051.32,587.95)(-7.160,-0.447){3}{\rule{4.500pt}{0.108pt}}
\multiput(1060.66,588.17)(-23.660,-3.000){2}{\rule{2.250pt}{0.400pt}}
\multiput(1022.47,584.94)(-4.868,-0.468){5}{\rule{3.500pt}{0.113pt}}
\multiput(1029.74,585.17)(-26.736,-4.000){2}{\rule{1.750pt}{0.400pt}}
\multiput(988.47,580.94)(-4.868,-0.468){5}{\rule{3.500pt}{0.113pt}}
\multiput(995.74,581.17)(-26.736,-4.000){2}{\rule{1.750pt}{0.400pt}}
\multiput(950.32,576.95)(-7.160,-0.447){3}{\rule{4.500pt}{0.108pt}}
\multiput(959.66,577.17)(-23.660,-3.000){2}{\rule{2.250pt}{0.400pt}}
\multiput(921.47,573.94)(-4.868,-0.468){5}{\rule{3.500pt}{0.113pt}}
\multiput(928.74,574.17)(-26.736,-4.000){2}{\rule{1.750pt}{0.400pt}}
\multiput(883.32,569.95)(-7.160,-0.447){3}{\rule{4.500pt}{0.108pt}}
\multiput(892.66,570.17)(-23.660,-3.000){2}{\rule{2.250pt}{0.400pt}}
\multiput(849.77,566.95)(-7.383,-0.447){3}{\rule{4.633pt}{0.108pt}}
\multiput(859.38,567.17)(-24.383,-3.000){2}{\rule{2.317pt}{0.400pt}}
\multiput(816.32,563.95)(-7.160,-0.447){3}{\rule{4.500pt}{0.108pt}}
\multiput(825.66,564.17)(-23.660,-3.000){2}{\rule{2.250pt}{0.400pt}}
\multiput(782.77,560.95)(-7.383,-0.447){3}{\rule{4.633pt}{0.108pt}}
\multiput(792.38,561.17)(-24.383,-3.000){2}{\rule{2.317pt}{0.400pt}}
\put(735,557.17){\rule{6.700pt}{0.400pt}}
\multiput(754.09,558.17)(-19.094,-2.000){2}{\rule{3.350pt}{0.400pt}}
\multiput(715.77,555.95)(-7.383,-0.447){3}{\rule{4.633pt}{0.108pt}}
\multiput(725.38,556.17)(-24.383,-3.000){2}{\rule{2.317pt}{0.400pt}}
\multiput(682.32,552.95)(-7.160,-0.447){3}{\rule{4.500pt}{0.108pt}}
\multiput(691.66,553.17)(-23.660,-3.000){2}{\rule{2.250pt}{0.400pt}}
\put(634,549.17){\rule{6.900pt}{0.400pt}}
\multiput(653.68,550.17)(-19.679,-2.000){2}{\rule{3.450pt}{0.400pt}}
\put(601,547.17){\rule{6.700pt}{0.400pt}}
\multiput(620.09,548.17)(-19.094,-2.000){2}{\rule{3.350pt}{0.400pt}}
\put(567,545.17){\rule{6.900pt}{0.400pt}}
\multiput(586.68,546.17)(-19.679,-2.000){2}{\rule{3.450pt}{0.400pt}}
\put(533,543.17){\rule{6.900pt}{0.400pt}}
\multiput(552.68,544.17)(-19.679,-2.000){2}{\rule{3.450pt}{0.400pt}}
\put(500,541.17){\rule{6.700pt}{0.400pt}}
\multiput(519.09,542.17)(-19.094,-2.000){2}{\rule{3.350pt}{0.400pt}}
\put(466,539.17){\rule{6.900pt}{0.400pt}}
\multiput(485.68,540.17)(-19.679,-2.000){2}{\rule{3.450pt}{0.400pt}}
\put(433,537.17){\rule{6.700pt}{0.400pt}}
\multiput(452.09,538.17)(-19.094,-2.000){2}{\rule{3.350pt}{0.400pt}}
\put(399,535.67){\rule{8.191pt}{0.400pt}}
\multiput(416.00,536.17)(-17.000,-1.000){2}{\rule{4.095pt}{0.400pt}}
\put(366,534.67){\rule{7.950pt}{0.400pt}}
\multiput(382.50,535.17)(-16.500,-1.000){2}{\rule{3.975pt}{0.400pt}}
\put(332,533.17){\rule{6.900pt}{0.400pt}}
\multiput(351.68,534.17)(-19.679,-2.000){2}{\rule{3.450pt}{0.400pt}}
\put(299,531.67){\rule{7.950pt}{0.400pt}}
\multiput(315.50,532.17)(-16.500,-1.000){2}{\rule{3.975pt}{0.400pt}}
\put(265,530.67){\rule{8.191pt}{0.400pt}}
\multiput(282.00,531.17)(-17.000,-1.000){2}{\rule{4.095pt}{0.400pt}}
\put(232,529.67){\rule{7.950pt}{0.400pt}}
\multiput(248.50,530.17)(-16.500,-1.000){2}{\rule{3.975pt}{0.400pt}}
\put(165,528.67){\rule{7.950pt}{0.400pt}}
\multiput(181.50,529.17)(-16.500,-1.000){2}{\rule{3.975pt}{0.400pt}}
\put(198.0,530.0){\rule[-0.200pt]{8.191pt}{0.400pt}}
\put(131.0,529.0){\rule[-0.200pt]{8.191pt}{0.400pt}}
\put(1279,768){\makebox(0,0)[r]{改进数据}}
\put(1299.0,768.0){\rule[-0.200pt]{24.090pt}{0.400pt}}
\put(131,548){\usebox{\plotpoint}}
\multiput(131.00,548.59)(2.211,0.488){13}{\rule{1.800pt}{0.117pt}}
\multiput(131.00,547.17)(30.264,8.000){2}{\rule{0.900pt}{0.400pt}}
\multiput(165.00,556.59)(2.932,0.482){9}{\rule{2.300pt}{0.116pt}}
\multiput(165.00,555.17)(28.226,6.000){2}{\rule{1.150pt}{0.400pt}}
\put(198,562.17){\rule{6.900pt}{0.400pt}}
\multiput(198.00,561.17)(19.679,2.000){2}{\rule{3.450pt}{0.400pt}}
\multiput(232.00,564.59)(2.932,0.482){9}{\rule{2.300pt}{0.116pt}}
\multiput(232.00,563.17)(28.226,6.000){2}{\rule{1.150pt}{0.400pt}}
\put(265,568.17){\rule{6.900pt}{0.400pt}}
\multiput(265.00,569.17)(19.679,-2.000){2}{\rule{3.450pt}{0.400pt}}
\put(299,568.17){\rule{6.700pt}{0.400pt}}
\multiput(299.00,567.17)(19.094,2.000){2}{\rule{3.350pt}{0.400pt}}
\put(332,570.17){\rule{6.900pt}{0.400pt}}
\multiput(332.00,569.17)(19.679,2.000){2}{\rule{3.450pt}{0.400pt}}
\put(366,571.67){\rule{7.950pt}{0.400pt}}
\multiput(366.00,571.17)(16.500,1.000){2}{\rule{3.975pt}{0.400pt}}
\multiput(399.00,573.61)(7.383,0.447){3}{\rule{4.633pt}{0.108pt}}
\multiput(399.00,572.17)(24.383,3.000){2}{\rule{2.317pt}{0.400pt}}
\put(433,576.17){\rule{6.700pt}{0.400pt}}
\multiput(433.00,575.17)(19.094,2.000){2}{\rule{3.350pt}{0.400pt}}
\multiput(466.00,576.95)(7.383,-0.447){3}{\rule{4.633pt}{0.108pt}}
\multiput(466.00,577.17)(24.383,-3.000){2}{\rule{2.317pt}{0.400pt}}
\multiput(500.00,575.59)(1.893,0.489){15}{\rule{1.567pt}{0.118pt}}
\multiput(500.00,574.17)(29.748,9.000){2}{\rule{0.783pt}{0.400pt}}
\put(533,583.67){\rule{8.191pt}{0.400pt}}
\multiput(533.00,583.17)(17.000,1.000){2}{\rule{4.095pt}{0.400pt}}
\multiput(567.00,583.95)(7.383,-0.447){3}{\rule{4.633pt}{0.108pt}}
\multiput(567.00,584.17)(24.383,-3.000){2}{\rule{2.317pt}{0.400pt}}
\multiput(601.00,582.59)(2.145,0.488){13}{\rule{1.750pt}{0.117pt}}
\multiput(601.00,581.17)(29.368,8.000){2}{\rule{0.875pt}{0.400pt}}
\multiput(634.00,590.61)(7.383,0.447){3}{\rule{4.633pt}{0.108pt}}
\multiput(634.00,589.17)(24.383,3.000){2}{\rule{2.317pt}{0.400pt}}
\put(668,591.67){\rule{7.950pt}{0.400pt}}
\multiput(668.00,592.17)(16.500,-1.000){2}{\rule{3.975pt}{0.400pt}}
\multiput(701.00,592.61)(7.383,0.447){3}{\rule{4.633pt}{0.108pt}}
\multiput(701.00,591.17)(24.383,3.000){2}{\rule{2.317pt}{0.400pt}}
\multiput(735.00,595.61)(7.160,0.447){3}{\rule{4.500pt}{0.108pt}}
\multiput(735.00,594.17)(23.660,3.000){2}{\rule{2.250pt}{0.400pt}}
\put(768,597.67){\rule{8.191pt}{0.400pt}}
\multiput(768.00,597.17)(17.000,1.000){2}{\rule{4.095pt}{0.400pt}}
\put(802,599.17){\rule{6.700pt}{0.400pt}}
\multiput(802.00,598.17)(19.094,2.000){2}{\rule{3.350pt}{0.400pt}}
\multiput(835.00,601.59)(3.716,0.477){7}{\rule{2.820pt}{0.115pt}}
\multiput(835.00,600.17)(28.147,5.000){2}{\rule{1.410pt}{0.400pt}}
\put(869,605.67){\rule{7.950pt}{0.400pt}}
\multiput(869.00,605.17)(16.500,1.000){2}{\rule{3.975pt}{0.400pt}}
\multiput(902.00,607.59)(3.716,0.477){7}{\rule{2.820pt}{0.115pt}}
\multiput(902.00,606.17)(28.147,5.000){2}{\rule{1.410pt}{0.400pt}}
\multiput(936.00,612.61)(7.160,0.447){3}{\rule{4.500pt}{0.108pt}}
\multiput(936.00,611.17)(23.660,3.000){2}{\rule{2.250pt}{0.400pt}}
\put(969,615.17){\rule{6.900pt}{0.400pt}}
\multiput(969.00,614.17)(19.679,2.000){2}{\rule{3.450pt}{0.400pt}}
\put(1003,615.17){\rule{6.900pt}{0.400pt}}
\multiput(1003.00,616.17)(19.679,-2.000){2}{\rule{3.450pt}{0.400pt}}
\multiput(1037.00,615.61)(7.160,0.447){3}{\rule{4.500pt}{0.108pt}}
\multiput(1037.00,614.17)(23.660,3.000){2}{\rule{2.250pt}{0.400pt}}
\put(1070,618.17){\rule{6.900pt}{0.400pt}}
\multiput(1070.00,617.17)(19.679,2.000){2}{\rule{3.450pt}{0.400pt}}
\multiput(1104.00,620.59)(2.932,0.482){9}{\rule{2.300pt}{0.116pt}}
\multiput(1104.00,619.17)(28.226,6.000){2}{\rule{1.150pt}{0.400pt}}
\multiput(1137.00,626.59)(3.022,0.482){9}{\rule{2.367pt}{0.116pt}}
\multiput(1137.00,625.17)(29.088,6.000){2}{\rule{1.183pt}{0.400pt}}
\multiput(1171.00,632.59)(2.145,0.488){13}{\rule{1.750pt}{0.117pt}}
\multiput(1171.00,631.17)(29.368,8.000){2}{\rule{0.875pt}{0.400pt}}
\put(1204,639.67){\rule{8.191pt}{0.400pt}}
\multiput(1204.00,639.17)(17.000,1.000){2}{\rule{4.095pt}{0.400pt}}
\put(1238,641.17){\rule{6.700pt}{0.400pt}}
\multiput(1238.00,640.17)(19.094,2.000){2}{\rule{3.350pt}{0.400pt}}
\multiput(1271.00,643.61)(7.383,0.447){3}{\rule{4.633pt}{0.108pt}}
\multiput(1271.00,642.17)(24.383,3.000){2}{\rule{2.317pt}{0.400pt}}
\put(1305,646.17){\rule{6.700pt}{0.400pt}}
\multiput(1305.00,645.17)(19.094,2.000){2}{\rule{3.350pt}{0.400pt}}
\multiput(1338.00,648.61)(7.383,0.447){3}{\rule{4.633pt}{0.108pt}}
\multiput(1338.00,647.17)(24.383,3.000){2}{\rule{2.317pt}{0.400pt}}
\multiput(1372.00,651.60)(4.722,0.468){5}{\rule{3.400pt}{0.113pt}}
\multiput(1372.00,650.17)(25.943,4.000){2}{\rule{1.700pt}{0.400pt}}
\put(1405,655.17){\rule{6.900pt}{0.400pt}}
\multiput(1405.00,654.17)(19.679,2.000){2}{\rule{3.450pt}{0.400pt}}
\put(131,548){\makebox(0,0){$+$}}
\put(165,556){\makebox(0,0){$+$}}
\put(198,562){\makebox(0,0){$+$}}
\put(232,564){\makebox(0,0){$+$}}
\put(265,570){\makebox(0,0){$+$}}
\put(299,568){\makebox(0,0){$+$}}
\put(332,570){\makebox(0,0){$+$}}
\put(366,572){\makebox(0,0){$+$}}
\put(399,573){\makebox(0,0){$+$}}
\put(433,576){\makebox(0,0){$+$}}
\put(466,578){\makebox(0,0){$+$}}
\put(500,575){\makebox(0,0){$+$}}
\put(533,584){\makebox(0,0){$+$}}
\put(567,585){\makebox(0,0){$+$}}
\put(601,582){\makebox(0,0){$+$}}
\put(634,590){\makebox(0,0){$+$}}
\put(668,593){\makebox(0,0){$+$}}
\put(701,592){\makebox(0,0){$+$}}
\put(735,595){\makebox(0,0){$+$}}
\put(768,598){\makebox(0,0){$+$}}
\put(802,599){\makebox(0,0){$+$}}
\put(835,601){\makebox(0,0){$+$}}
\put(869,606){\makebox(0,0){$+$}}
\put(902,607){\makebox(0,0){$+$}}
\put(936,612){\makebox(0,0){$+$}}
\put(969,615){\makebox(0,0){$+$}}
\put(1003,617){\makebox(0,0){$+$}}
\put(1037,615){\makebox(0,0){$+$}}
\put(1070,618){\makebox(0,0){$+$}}
\put(1104,620){\makebox(0,0){$+$}}
\put(1137,626){\makebox(0,0){$+$}}
\put(1171,632){\makebox(0,0){$+$}}
\put(1204,640){\makebox(0,0){$+$}}
\put(1238,641){\makebox(0,0){$+$}}
\put(1271,643){\makebox(0,0){$+$}}
\put(1305,646){\makebox(0,0){$+$}}
\put(1338,648){\makebox(0,0){$+$}}
\put(1372,651){\makebox(0,0){$+$}}
\put(1405,655){\makebox(0,0){$+$}}
\put(1439,657){\makebox(0,0){$+$}}
\put(1349,768){\makebox(0,0){$+$}}
\put(131.0,82.0){\rule[-0.200pt]{0.400pt}{187.179pt}}
\put(131.0,82.0){\rule[-0.200pt]{315.097pt}{0.400pt}}
\put(1439.0,82.0){\rule[-0.200pt]{0.400pt}{187.179pt}}
\put(131.0,859.0){\rule[-0.200pt]{315.097pt}{0.400pt}}
\end{picture}
}
   \end{frame}


   \begin{frame}
提升自我效能和认知失调的仿真结果
     \scalebox{0.8}{% GNUPLOT: LaTeX picture
\setlength{\unitlength}{0.240900pt}
\ifx\plotpoint\undefined\newsavebox{\plotpoint}\fi
\begin{picture}(1500,900)(0,0)
\sbox{\plotpoint}{\rule[-0.200pt]{0.400pt}{0.400pt}}%
\put(131.0,82.0){\rule[-0.200pt]{4.818pt}{0.400pt}}
\put(111,82){\makebox(0,0)[r]{-1}}
\put(1419.0,82.0){\rule[-0.200pt]{4.818pt}{0.400pt}}
\put(131.0,212.0){\rule[-0.200pt]{4.818pt}{0.400pt}}
\put(111,212){\makebox(0,0)[r]{ 0}}
\put(1419.0,212.0){\rule[-0.200pt]{4.818pt}{0.400pt}}
\put(131.0,341.0){\rule[-0.200pt]{4.818pt}{0.400pt}}
\put(111,341){\makebox(0,0)[r]{ 1}}
\put(1419.0,341.0){\rule[-0.200pt]{4.818pt}{0.400pt}}
\put(131.0,471.0){\rule[-0.200pt]{4.818pt}{0.400pt}}
\put(111,471){\makebox(0,0)[r]{ 2}}
\put(1419.0,471.0){\rule[-0.200pt]{4.818pt}{0.400pt}}
\put(131.0,600.0){\rule[-0.200pt]{4.818pt}{0.400pt}}
\put(111,600){\makebox(0,0)[r]{ 3}}
\put(1419.0,600.0){\rule[-0.200pt]{4.818pt}{0.400pt}}
\put(131.0,730.0){\rule[-0.200pt]{4.818pt}{0.400pt}}
\put(111,730){\makebox(0,0)[r]{ 4}}
\put(1419.0,730.0){\rule[-0.200pt]{4.818pt}{0.400pt}}
\put(131.0,82.0){\rule[-0.200pt]{0.400pt}{4.818pt}}
\put(131,41){\makebox(0,0){ 1}}
\put(131.0,839.0){\rule[-0.200pt]{0.400pt}{4.818pt}}
\put(198.0,82.0){\rule[-0.200pt]{0.400pt}{4.818pt}}
\put(198,41){\makebox(0,0){ 3}}
\put(198.0,839.0){\rule[-0.200pt]{0.400pt}{4.818pt}}
\put(265.0,82.0){\rule[-0.200pt]{0.400pt}{4.818pt}}
\put(265,41){\makebox(0,0){ 5}}
\put(265.0,839.0){\rule[-0.200pt]{0.400pt}{4.818pt}}
\put(332.0,82.0){\rule[-0.200pt]{0.400pt}{4.818pt}}
\put(332,41){\makebox(0,0){ 7}}
\put(332.0,839.0){\rule[-0.200pt]{0.400pt}{4.818pt}}
\put(399.0,82.0){\rule[-0.200pt]{0.400pt}{4.818pt}}
\put(399,41){\makebox(0,0){ 9}}
\put(399.0,839.0){\rule[-0.200pt]{0.400pt}{4.818pt}}
\put(466.0,82.0){\rule[-0.200pt]{0.400pt}{4.818pt}}
\put(466,41){\makebox(0,0){ 11}}
\put(466.0,839.0){\rule[-0.200pt]{0.400pt}{4.818pt}}
\put(533.0,82.0){\rule[-0.200pt]{0.400pt}{4.818pt}}
\put(533,41){\makebox(0,0){ 13}}
\put(533.0,839.0){\rule[-0.200pt]{0.400pt}{4.818pt}}
\put(601.0,82.0){\rule[-0.200pt]{0.400pt}{4.818pt}}
\put(601,41){\makebox(0,0){ 15}}
\put(601.0,839.0){\rule[-0.200pt]{0.400pt}{4.818pt}}
\put(668.0,82.0){\rule[-0.200pt]{0.400pt}{4.818pt}}
\put(668,41){\makebox(0,0){ 17}}
\put(668.0,839.0){\rule[-0.200pt]{0.400pt}{4.818pt}}
\put(735.0,82.0){\rule[-0.200pt]{0.400pt}{4.818pt}}
\put(735,41){\makebox(0,0){ 19}}
\put(735.0,839.0){\rule[-0.200pt]{0.400pt}{4.818pt}}
\put(802.0,82.0){\rule[-0.200pt]{0.400pt}{4.818pt}}
\put(802,41){\makebox(0,0){ 21}}
\put(802.0,839.0){\rule[-0.200pt]{0.400pt}{4.818pt}}
\put(869.0,82.0){\rule[-0.200pt]{0.400pt}{4.818pt}}
\put(869,41){\makebox(0,0){ 23}}
\put(869.0,839.0){\rule[-0.200pt]{0.400pt}{4.818pt}}
\put(936.0,82.0){\rule[-0.200pt]{0.400pt}{4.818pt}}
\put(936,41){\makebox(0,0){ 25}}
\put(936.0,839.0){\rule[-0.200pt]{0.400pt}{4.818pt}}
\put(1003.0,82.0){\rule[-0.200pt]{0.400pt}{4.818pt}}
\put(1003,41){\makebox(0,0){ 27}}
\put(1003.0,839.0){\rule[-0.200pt]{0.400pt}{4.818pt}}
\put(1070.0,82.0){\rule[-0.200pt]{0.400pt}{4.818pt}}
\put(1070,41){\makebox(0,0){ 29}}
\put(1070.0,839.0){\rule[-0.200pt]{0.400pt}{4.818pt}}
\put(1137.0,82.0){\rule[-0.200pt]{0.400pt}{4.818pt}}
\put(1137,41){\makebox(0,0){ 31}}
\put(1137.0,839.0){\rule[-0.200pt]{0.400pt}{4.818pt}}
\put(1204.0,82.0){\rule[-0.200pt]{0.400pt}{4.818pt}}
\put(1204,41){\makebox(0,0){ 33}}
\put(1204.0,839.0){\rule[-0.200pt]{0.400pt}{4.818pt}}
\put(1271.0,82.0){\rule[-0.200pt]{0.400pt}{4.818pt}}
\put(1271,41){\makebox(0,0){ 35}}
\put(1271.0,839.0){\rule[-0.200pt]{0.400pt}{4.818pt}}
\put(1338.0,82.0){\rule[-0.200pt]{0.400pt}{4.818pt}}
\put(1338,41){\makebox(0,0){ 37}}
\put(1338.0,839.0){\rule[-0.200pt]{0.400pt}{4.818pt}}
\put(1405.0,82.0){\rule[-0.200pt]{0.400pt}{4.818pt}}
\put(1405,41){\makebox(0,0){ 39}}
\put(1405.0,839.0){\rule[-0.200pt]{0.400pt}{4.818pt}}
\put(131.0,82.0){\rule[-0.200pt]{0.400pt}{187.179pt}}
\put(131.0,82.0){\rule[-0.200pt]{315.097pt}{0.400pt}}
\put(1439.0,82.0){\rule[-0.200pt]{0.400pt}{187.179pt}}
\put(131.0,859.0){\rule[-0.200pt]{315.097pt}{0.400pt}}
\put(30,470){\makebox(0,0){\rotatebox{90}{用户贡献}}}
\put(1279,819){\makebox(0,0)[r]{仿真数据}}
\put(1299.0,819.0){\rule[-0.200pt]{24.090pt}{0.400pt}}
\put(131,552){\usebox{\plotpoint}}
\multiput(131.58,549.44)(0.498,-0.647){65}{\rule{0.120pt}{0.618pt}}
\multiput(130.17,550.72)(34.000,-42.718){2}{\rule{0.400pt}{0.309pt}}
\multiput(165.00,506.92)(0.661,-0.497){47}{\rule{0.628pt}{0.120pt}}
\multiput(165.00,507.17)(31.697,-25.000){2}{\rule{0.314pt}{0.400pt}}
\multiput(198.00,481.92)(0.775,-0.496){41}{\rule{0.718pt}{0.120pt}}
\multiput(198.00,482.17)(32.509,-22.000){2}{\rule{0.359pt}{0.400pt}}
\multiput(232.00,459.92)(1.113,-0.494){27}{\rule{0.980pt}{0.119pt}}
\multiput(232.00,460.17)(30.966,-15.000){2}{\rule{0.490pt}{0.400pt}}
\multiput(265.00,444.92)(1.581,-0.492){19}{\rule{1.336pt}{0.118pt}}
\multiput(265.00,445.17)(31.226,-11.000){2}{\rule{0.668pt}{0.400pt}}
\multiput(299.00,433.92)(1.041,-0.494){29}{\rule{0.925pt}{0.119pt}}
\multiput(299.00,434.17)(31.080,-16.000){2}{\rule{0.463pt}{0.400pt}}
\multiput(332.00,417.93)(2.552,-0.485){11}{\rule{2.043pt}{0.117pt}}
\multiput(332.00,418.17)(29.760,-7.000){2}{\rule{1.021pt}{0.400pt}}
\multiput(366.00,410.93)(3.604,-0.477){7}{\rule{2.740pt}{0.115pt}}
\multiput(366.00,411.17)(27.313,-5.000){2}{\rule{1.370pt}{0.400pt}}
\multiput(399.00,405.92)(1.746,-0.491){17}{\rule{1.460pt}{0.118pt}}
\multiput(399.00,406.17)(30.970,-10.000){2}{\rule{0.730pt}{0.400pt}}
\multiput(433.00,395.92)(1.290,-0.493){23}{\rule{1.115pt}{0.119pt}}
\multiput(433.00,396.17)(30.685,-13.000){2}{\rule{0.558pt}{0.400pt}}
\multiput(500.00,382.94)(4.722,-0.468){5}{\rule{3.400pt}{0.113pt}}
\multiput(500.00,383.17)(25.943,-4.000){2}{\rule{1.700pt}{0.400pt}}
\multiput(533.00,378.93)(3.022,-0.482){9}{\rule{2.367pt}{0.116pt}}
\multiput(533.00,379.17)(29.088,-6.000){2}{\rule{1.183pt}{0.400pt}}
\multiput(567.00,372.95)(7.383,-0.447){3}{\rule{4.633pt}{0.108pt}}
\multiput(567.00,373.17)(24.383,-3.000){2}{\rule{2.317pt}{0.400pt}}
\multiput(601.00,369.93)(3.604,-0.477){7}{\rule{2.740pt}{0.115pt}}
\multiput(601.00,370.17)(27.313,-5.000){2}{\rule{1.370pt}{0.400pt}}
\multiput(634.00,364.94)(4.868,-0.468){5}{\rule{3.500pt}{0.113pt}}
\multiput(634.00,365.17)(26.736,-4.000){2}{\rule{1.750pt}{0.400pt}}
\multiput(668.00,360.93)(2.145,-0.488){13}{\rule{1.750pt}{0.117pt}}
\multiput(668.00,361.17)(29.368,-8.000){2}{\rule{0.875pt}{0.400pt}}
\multiput(701.00,352.95)(7.383,-0.447){3}{\rule{4.633pt}{0.108pt}}
\multiput(701.00,353.17)(24.383,-3.000){2}{\rule{2.317pt}{0.400pt}}
\multiput(735.00,349.93)(3.604,-0.477){7}{\rule{2.740pt}{0.115pt}}
\multiput(735.00,350.17)(27.313,-5.000){2}{\rule{1.370pt}{0.400pt}}
\multiput(768.00,344.93)(3.022,-0.482){9}{\rule{2.367pt}{0.116pt}}
\multiput(768.00,345.17)(29.088,-6.000){2}{\rule{1.183pt}{0.400pt}}
\put(802,340.17){\rule{6.700pt}{0.400pt}}
\multiput(802.00,339.17)(19.094,2.000){2}{\rule{3.350pt}{0.400pt}}
\multiput(835.00,340.93)(2.552,-0.485){11}{\rule{2.043pt}{0.117pt}}
\multiput(835.00,341.17)(29.760,-7.000){2}{\rule{1.021pt}{0.400pt}}
\multiput(869.00,333.93)(2.932,-0.482){9}{\rule{2.300pt}{0.116pt}}
\multiput(869.00,334.17)(28.226,-6.000){2}{\rule{1.150pt}{0.400pt}}
\put(902,327.17){\rule{6.900pt}{0.400pt}}
\multiput(902.00,328.17)(19.679,-2.000){2}{\rule{3.450pt}{0.400pt}}
\put(936,327.17){\rule{6.700pt}{0.400pt}}
\multiput(936.00,326.17)(19.094,2.000){2}{\rule{3.350pt}{0.400pt}}
\multiput(969.00,327.94)(4.868,-0.468){5}{\rule{3.500pt}{0.113pt}}
\multiput(969.00,328.17)(26.736,-4.000){2}{\rule{1.750pt}{0.400pt}}
\multiput(1003.00,323.93)(3.022,-0.482){9}{\rule{2.367pt}{0.116pt}}
\multiput(1003.00,324.17)(29.088,-6.000){2}{\rule{1.183pt}{0.400pt}}
\multiput(1037.00,317.93)(2.932,-0.482){9}{\rule{2.300pt}{0.116pt}}
\multiput(1037.00,318.17)(28.226,-6.000){2}{\rule{1.150pt}{0.400pt}}
\put(1070,311.67){\rule{8.191pt}{0.400pt}}
\multiput(1070.00,312.17)(17.000,-1.000){2}{\rule{4.095pt}{0.400pt}}
\multiput(1104.00,310.95)(7.160,-0.447){3}{\rule{4.500pt}{0.108pt}}
\multiput(1104.00,311.17)(23.660,-3.000){2}{\rule{2.250pt}{0.400pt}}
\multiput(1137.00,307.94)(4.868,-0.468){5}{\rule{3.500pt}{0.113pt}}
\multiput(1137.00,308.17)(26.736,-4.000){2}{\rule{1.750pt}{0.400pt}}
\put(466.0,384.0){\rule[-0.200pt]{8.191pt}{0.400pt}}
\put(1238,303.17){\rule{6.700pt}{0.400pt}}
\multiput(1238.00,304.17)(19.094,-2.000){2}{\rule{3.350pt}{0.400pt}}
\put(1271,301.67){\rule{8.191pt}{0.400pt}}
\multiput(1271.00,302.17)(17.000,-1.000){2}{\rule{4.095pt}{0.400pt}}
\put(1171.0,305.0){\rule[-0.200pt]{16.140pt}{0.400pt}}
\put(1338,301.67){\rule{8.191pt}{0.400pt}}
\multiput(1338.00,301.17)(17.000,1.000){2}{\rule{4.095pt}{0.400pt}}
\multiput(1372.00,301.93)(2.932,-0.482){9}{\rule{2.300pt}{0.116pt}}
\multiput(1372.00,302.17)(28.226,-6.000){2}{\rule{1.150pt}{0.400pt}}
\put(1405,295.67){\rule{8.191pt}{0.400pt}}
\multiput(1405.00,296.17)(17.000,-1.000){2}{\rule{4.095pt}{0.400pt}}
\put(1305.0,302.0){\rule[-0.200pt]{7.950pt}{0.400pt}}
\put(1279,768){\makebox(0,0)[r]{改进数据}}
\put(1299.0,768.0){\rule[-0.200pt]{24.090pt}{0.400pt}}
\put(131,583){\usebox{\plotpoint}}
\put(131,582.67){\rule{8.191pt}{0.400pt}}
\multiput(131.00,582.17)(17.000,1.000){2}{\rule{4.095pt}{0.400pt}}
\put(198,584.17){\rule{6.900pt}{0.400pt}}
\multiput(198.00,583.17)(19.679,2.000){2}{\rule{3.450pt}{0.400pt}}
\put(165.0,584.0){\rule[-0.200pt]{7.950pt}{0.400pt}}
\put(265,585.67){\rule{8.191pt}{0.400pt}}
\multiput(265.00,585.17)(17.000,1.000){2}{\rule{4.095pt}{0.400pt}}
\put(232.0,586.0){\rule[-0.200pt]{7.950pt}{0.400pt}}
\put(332,586.67){\rule{8.191pt}{0.400pt}}
\multiput(332.00,586.17)(17.000,1.000){2}{\rule{4.095pt}{0.400pt}}
\put(366,588.17){\rule{6.700pt}{0.400pt}}
\multiput(366.00,587.17)(19.094,2.000){2}{\rule{3.350pt}{0.400pt}}
\put(399,589.67){\rule{8.191pt}{0.400pt}}
\multiput(399.00,589.17)(17.000,1.000){2}{\rule{4.095pt}{0.400pt}}
\put(433,590.67){\rule{7.950pt}{0.400pt}}
\multiput(433.00,590.17)(16.500,1.000){2}{\rule{3.975pt}{0.400pt}}
\put(466,592.17){\rule{6.900pt}{0.400pt}}
\multiput(466.00,591.17)(19.679,2.000){2}{\rule{3.450pt}{0.400pt}}
\put(500,593.67){\rule{7.950pt}{0.400pt}}
\multiput(500.00,593.17)(16.500,1.000){2}{\rule{3.975pt}{0.400pt}}
\put(533,595.17){\rule{6.900pt}{0.400pt}}
\multiput(533.00,594.17)(19.679,2.000){2}{\rule{3.450pt}{0.400pt}}
\put(567,597.17){\rule{6.900pt}{0.400pt}}
\multiput(567.00,596.17)(19.679,2.000){2}{\rule{3.450pt}{0.400pt}}
\put(601,599.17){\rule{6.700pt}{0.400pt}}
\multiput(601.00,598.17)(19.094,2.000){2}{\rule{3.350pt}{0.400pt}}
\put(634,601.17){\rule{6.900pt}{0.400pt}}
\multiput(634.00,600.17)(19.679,2.000){2}{\rule{3.450pt}{0.400pt}}
\put(668,603.17){\rule{6.700pt}{0.400pt}}
\multiput(668.00,602.17)(19.094,2.000){2}{\rule{3.350pt}{0.400pt}}
\put(701,604.67){\rule{8.191pt}{0.400pt}}
\multiput(701.00,604.17)(17.000,1.000){2}{\rule{4.095pt}{0.400pt}}
\multiput(735.00,606.61)(7.160,0.447){3}{\rule{4.500pt}{0.108pt}}
\multiput(735.00,605.17)(23.660,3.000){2}{\rule{2.250pt}{0.400pt}}
\multiput(768.00,609.61)(7.383,0.447){3}{\rule{4.633pt}{0.108pt}}
\multiput(768.00,608.17)(24.383,3.000){2}{\rule{2.317pt}{0.400pt}}
\put(802,612.17){\rule{6.700pt}{0.400pt}}
\multiput(802.00,611.17)(19.094,2.000){2}{\rule{3.350pt}{0.400pt}}
\multiput(835.00,614.61)(7.383,0.447){3}{\rule{4.633pt}{0.108pt}}
\multiput(835.00,613.17)(24.383,3.000){2}{\rule{2.317pt}{0.400pt}}
\put(869,617.17){\rule{6.700pt}{0.400pt}}
\multiput(869.00,616.17)(19.094,2.000){2}{\rule{3.350pt}{0.400pt}}
\multiput(902.00,619.61)(7.383,0.447){3}{\rule{4.633pt}{0.108pt}}
\multiput(902.00,618.17)(24.383,3.000){2}{\rule{2.317pt}{0.400pt}}
\multiput(936.00,622.61)(7.160,0.447){3}{\rule{4.500pt}{0.108pt}}
\multiput(936.00,621.17)(23.660,3.000){2}{\rule{2.250pt}{0.400pt}}
\multiput(969.00,625.61)(7.383,0.447){3}{\rule{4.633pt}{0.108pt}}
\multiput(969.00,624.17)(24.383,3.000){2}{\rule{2.317pt}{0.400pt}}
\multiput(1003.00,628.61)(7.383,0.447){3}{\rule{4.633pt}{0.108pt}}
\multiput(1003.00,627.17)(24.383,3.000){2}{\rule{2.317pt}{0.400pt}}
\multiput(1037.00,631.60)(4.722,0.468){5}{\rule{3.400pt}{0.113pt}}
\multiput(1037.00,630.17)(25.943,4.000){2}{\rule{1.700pt}{0.400pt}}
\multiput(1070.00,635.61)(7.383,0.447){3}{\rule{4.633pt}{0.108pt}}
\multiput(1070.00,634.17)(24.383,3.000){2}{\rule{2.317pt}{0.400pt}}
\multiput(1104.00,638.61)(7.160,0.447){3}{\rule{4.500pt}{0.108pt}}
\multiput(1104.00,637.17)(23.660,3.000){2}{\rule{2.250pt}{0.400pt}}
\multiput(1137.00,641.61)(7.383,0.447){3}{\rule{4.633pt}{0.108pt}}
\multiput(1137.00,640.17)(24.383,3.000){2}{\rule{2.317pt}{0.400pt}}
\multiput(1171.00,644.60)(4.722,0.468){5}{\rule{3.400pt}{0.113pt}}
\multiput(1171.00,643.17)(25.943,4.000){2}{\rule{1.700pt}{0.400pt}}
\multiput(1204.00,648.60)(4.868,0.468){5}{\rule{3.500pt}{0.113pt}}
\multiput(1204.00,647.17)(26.736,4.000){2}{\rule{1.750pt}{0.400pt}}
\multiput(1238.00,652.60)(4.722,0.468){5}{\rule{3.400pt}{0.113pt}}
\multiput(1238.00,651.17)(25.943,4.000){2}{\rule{1.700pt}{0.400pt}}
\multiput(1271.00,656.60)(4.868,0.468){5}{\rule{3.500pt}{0.113pt}}
\multiput(1271.00,655.17)(26.736,4.000){2}{\rule{1.750pt}{0.400pt}}
\multiput(1305.00,660.61)(7.160,0.447){3}{\rule{4.500pt}{0.108pt}}
\multiput(1305.00,659.17)(23.660,3.000){2}{\rule{2.250pt}{0.400pt}}
\multiput(1338.00,663.60)(4.868,0.468){5}{\rule{3.500pt}{0.113pt}}
\multiput(1338.00,662.17)(26.736,4.000){2}{\rule{1.750pt}{0.400pt}}
\multiput(1372.00,667.59)(2.932,0.482){9}{\rule{2.300pt}{0.116pt}}
\multiput(1372.00,666.17)(28.226,6.000){2}{\rule{1.150pt}{0.400pt}}
\multiput(1405.00,673.61)(7.383,0.447){3}{\rule{4.633pt}{0.108pt}}
\multiput(1405.00,672.17)(24.383,3.000){2}{\rule{2.317pt}{0.400pt}}
\put(131,583){\makebox(0,0){$+$}}
\put(165,584){\makebox(0,0){$+$}}
\put(198,584){\makebox(0,0){$+$}}
\put(232,586){\makebox(0,0){$+$}}
\put(265,586){\makebox(0,0){$+$}}
\put(299,587){\makebox(0,0){$+$}}
\put(332,587){\makebox(0,0){$+$}}
\put(366,588){\makebox(0,0){$+$}}
\put(399,590){\makebox(0,0){$+$}}
\put(433,591){\makebox(0,0){$+$}}
\put(466,592){\makebox(0,0){$+$}}
\put(500,594){\makebox(0,0){$+$}}
\put(533,595){\makebox(0,0){$+$}}
\put(567,597){\makebox(0,0){$+$}}
\put(601,599){\makebox(0,0){$+$}}
\put(634,601){\makebox(0,0){$+$}}
\put(668,603){\makebox(0,0){$+$}}
\put(701,605){\makebox(0,0){$+$}}
\put(735,606){\makebox(0,0){$+$}}
\put(768,609){\makebox(0,0){$+$}}
\put(802,612){\makebox(0,0){$+$}}
\put(835,614){\makebox(0,0){$+$}}
\put(869,617){\makebox(0,0){$+$}}
\put(902,619){\makebox(0,0){$+$}}
\put(936,622){\makebox(0,0){$+$}}
\put(969,625){\makebox(0,0){$+$}}
\put(1003,628){\makebox(0,0){$+$}}
\put(1037,631){\makebox(0,0){$+$}}
\put(1070,635){\makebox(0,0){$+$}}
\put(1104,638){\makebox(0,0){$+$}}
\put(1137,641){\makebox(0,0){$+$}}
\put(1171,644){\makebox(0,0){$+$}}
\put(1204,648){\makebox(0,0){$+$}}
\put(1238,652){\makebox(0,0){$+$}}
\put(1271,656){\makebox(0,0){$+$}}
\put(1305,660){\makebox(0,0){$+$}}
\put(1338,663){\makebox(0,0){$+$}}
\put(1372,667){\makebox(0,0){$+$}}
\put(1405,673){\makebox(0,0){$+$}}
\put(1439,676){\makebox(0,0){$+$}}
\put(1349,768){\makebox(0,0){$+$}}
\put(299.0,587.0){\rule[-0.200pt]{7.950pt}{0.400pt}}
\put(131.0,82.0){\rule[-0.200pt]{0.400pt}{187.179pt}}
\put(131.0,82.0){\rule[-0.200pt]{315.097pt}{0.400pt}}
\put(1439.0,82.0){\rule[-0.200pt]{0.400pt}{187.179pt}}
\put(131.0,859.0){\rule[-0.200pt]{315.097pt}{0.400pt}}
\end{picture}
}
   \end{frame}

 \begin{frame}
提升其他个体动机的仿真结果
     \scalebox{0.8}{% GNUPLOT: LaTeX picture
\setlength{\unitlength}{0.240900pt}
\ifx\plotpoint\undefined\newsavebox{\plotpoint}\fi
\sbox{\plotpoint}{\rule[-0.200pt]{0.400pt}{0.400pt}}%
\begin{picture}(1500,900)(0,0)
\sbox{\plotpoint}{\rule[-0.200pt]{0.400pt}{0.400pt}}%
\put(131.0,82.0){\rule[-0.200pt]{4.818pt}{0.400pt}}
\put(111,82){\makebox(0,0)[r]{-1}}
\put(1419.0,82.0){\rule[-0.200pt]{4.818pt}{0.400pt}}
\put(131.0,237.0){\rule[-0.200pt]{4.818pt}{0.400pt}}
\put(111,237){\makebox(0,0)[r]{ 0}}
\put(1419.0,237.0){\rule[-0.200pt]{4.818pt}{0.400pt}}
\put(131.0,393.0){\rule[-0.200pt]{4.818pt}{0.400pt}}
\put(111,393){\makebox(0,0)[r]{ 1}}
\put(1419.0,393.0){\rule[-0.200pt]{4.818pt}{0.400pt}}
\put(131.0,548.0){\rule[-0.200pt]{4.818pt}{0.400pt}}
\put(111,548){\makebox(0,0)[r]{ 2}}
\put(1419.0,548.0){\rule[-0.200pt]{4.818pt}{0.400pt}}
\put(131.0,704.0){\rule[-0.200pt]{4.818pt}{0.400pt}}
\put(111,704){\makebox(0,0)[r]{ 3}}
\put(1419.0,704.0){\rule[-0.200pt]{4.818pt}{0.400pt}}
\put(131.0,82.0){\rule[-0.200pt]{0.400pt}{4.818pt}}
\put(131,41){\makebox(0,0){ 1}}
\put(131.0,839.0){\rule[-0.200pt]{0.400pt}{4.818pt}}
\put(198.0,82.0){\rule[-0.200pt]{0.400pt}{4.818pt}}
\put(198,41){\makebox(0,0){ 3}}
\put(198.0,839.0){\rule[-0.200pt]{0.400pt}{4.818pt}}
\put(265.0,82.0){\rule[-0.200pt]{0.400pt}{4.818pt}}
\put(265,41){\makebox(0,0){ 5}}
\put(265.0,839.0){\rule[-0.200pt]{0.400pt}{4.818pt}}
\put(332.0,82.0){\rule[-0.200pt]{0.400pt}{4.818pt}}
\put(332,41){\makebox(0,0){ 7}}
\put(332.0,839.0){\rule[-0.200pt]{0.400pt}{4.818pt}}
\put(399.0,82.0){\rule[-0.200pt]{0.400pt}{4.818pt}}
\put(399,41){\makebox(0,0){ 9}}
\put(399.0,839.0){\rule[-0.200pt]{0.400pt}{4.818pt}}
\put(466.0,82.0){\rule[-0.200pt]{0.400pt}{4.818pt}}
\put(466,41){\makebox(0,0){ 11}}
\put(466.0,839.0){\rule[-0.200pt]{0.400pt}{4.818pt}}
\put(533.0,82.0){\rule[-0.200pt]{0.400pt}{4.818pt}}
\put(533,41){\makebox(0,0){ 13}}
\put(533.0,839.0){\rule[-0.200pt]{0.400pt}{4.818pt}}
\put(601.0,82.0){\rule[-0.200pt]{0.400pt}{4.818pt}}
\put(601,41){\makebox(0,0){ 15}}
\put(601.0,839.0){\rule[-0.200pt]{0.400pt}{4.818pt}}
\put(668.0,82.0){\rule[-0.200pt]{0.400pt}{4.818pt}}
\put(668,41){\makebox(0,0){ 17}}
\put(668.0,839.0){\rule[-0.200pt]{0.400pt}{4.818pt}}
\put(735.0,82.0){\rule[-0.200pt]{0.400pt}{4.818pt}}
\put(735,41){\makebox(0,0){ 19}}
\put(735.0,839.0){\rule[-0.200pt]{0.400pt}{4.818pt}}
\put(802.0,82.0){\rule[-0.200pt]{0.400pt}{4.818pt}}
\put(802,41){\makebox(0,0){ 21}}
\put(802.0,839.0){\rule[-0.200pt]{0.400pt}{4.818pt}}
\put(869.0,82.0){\rule[-0.200pt]{0.400pt}{4.818pt}}
\put(869,41){\makebox(0,0){ 23}}
\put(869.0,839.0){\rule[-0.200pt]{0.400pt}{4.818pt}}
\put(936.0,82.0){\rule[-0.200pt]{0.400pt}{4.818pt}}
\put(936,41){\makebox(0,0){ 25}}
\put(936.0,839.0){\rule[-0.200pt]{0.400pt}{4.818pt}}
\put(1003.0,82.0){\rule[-0.200pt]{0.400pt}{4.818pt}}
\put(1003,41){\makebox(0,0){ 27}}
\put(1003.0,839.0){\rule[-0.200pt]{0.400pt}{4.818pt}}
\put(1070.0,82.0){\rule[-0.200pt]{0.400pt}{4.818pt}}
\put(1070,41){\makebox(0,0){ 29}}
\put(1070.0,839.0){\rule[-0.200pt]{0.400pt}{4.818pt}}
\put(1137.0,82.0){\rule[-0.200pt]{0.400pt}{4.818pt}}
\put(1137,41){\makebox(0,0){ 31}}
\put(1137.0,839.0){\rule[-0.200pt]{0.400pt}{4.818pt}}
\put(1204.0,82.0){\rule[-0.200pt]{0.400pt}{4.818pt}}
\put(1204,41){\makebox(0,0){ 33}}
\put(1204.0,839.0){\rule[-0.200pt]{0.400pt}{4.818pt}}
\put(1271.0,82.0){\rule[-0.200pt]{0.400pt}{4.818pt}}
\put(1271,41){\makebox(0,0){ 35}}
\put(1271.0,839.0){\rule[-0.200pt]{0.400pt}{4.818pt}}
\put(1338.0,82.0){\rule[-0.200pt]{0.400pt}{4.818pt}}
\put(1338,41){\makebox(0,0){ 37}}
\put(1338.0,839.0){\rule[-0.200pt]{0.400pt}{4.818pt}}
\put(1405.0,82.0){\rule[-0.200pt]{0.400pt}{4.818pt}}
\put(1405,41){\makebox(0,0){ 39}}
\put(1405.0,839.0){\rule[-0.200pt]{0.400pt}{4.818pt}}
\put(131.0,82.0){\rule[-0.200pt]{0.400pt}{187.179pt}}
\put(131.0,82.0){\rule[-0.200pt]{315.097pt}{0.400pt}}
\put(1439.0,82.0){\rule[-0.200pt]{0.400pt}{187.179pt}}
\put(131.0,859.0){\rule[-0.200pt]{315.097pt}{0.400pt}}
\put(30,470){\makebox(0,0){\rotatebox{90}{用户贡献}}}
\put(785,-10){\makebox(0,0){月度}}
\put(1279,819){\makebox(0,0)[r]{仿真数据}}
\put(1299.0,819.0){\rule[-0.200pt]{24.090pt}{0.400pt}}
\put(131,646){\usebox{\plotpoint}}
\multiput(131.58,643.05)(0.498,-0.766){65}{\rule{0.120pt}{0.712pt}}
\multiput(130.17,644.52)(34.000,-50.523){2}{\rule{0.400pt}{0.356pt}}
\multiput(165.00,592.92)(0.549,-0.497){57}{\rule{0.540pt}{0.120pt}}
\multiput(165.00,593.17)(31.879,-30.000){2}{\rule{0.270pt}{0.400pt}}
\multiput(198.00,562.92)(0.607,-0.497){53}{\rule{0.586pt}{0.120pt}}
\multiput(198.00,563.17)(32.784,-28.000){2}{\rule{0.293pt}{0.400pt}}
\multiput(232.00,534.92)(0.979,-0.495){31}{\rule{0.876pt}{0.119pt}}
\multiput(232.00,535.17)(31.181,-17.000){2}{\rule{0.438pt}{0.400pt}}
\multiput(265.00,517.92)(1.231,-0.494){25}{\rule{1.071pt}{0.119pt}}
\multiput(265.00,518.17)(31.776,-14.000){2}{\rule{0.536pt}{0.400pt}}
\multiput(299.00,503.92)(0.874,-0.495){35}{\rule{0.795pt}{0.119pt}}
\multiput(299.00,504.17)(31.350,-19.000){2}{\rule{0.397pt}{0.400pt}}
\multiput(332.00,484.93)(2.211,-0.488){13}{\rule{1.800pt}{0.117pt}}
\multiput(332.00,485.17)(30.264,-8.000){2}{\rule{0.900pt}{0.400pt}}
\multiput(366.00,476.93)(3.604,-0.477){7}{\rule{2.740pt}{0.115pt}}
\multiput(366.00,477.17)(27.313,-5.000){2}{\rule{1.370pt}{0.400pt}}
\multiput(399.00,471.92)(1.329,-0.493){23}{\rule{1.146pt}{0.119pt}}
\multiput(399.00,472.17)(31.621,-13.000){2}{\rule{0.573pt}{0.400pt}}
\multiput(433.00,458.92)(1.041,-0.494){29}{\rule{0.925pt}{0.119pt}}
\multiput(433.00,459.17)(31.080,-16.000){2}{\rule{0.463pt}{0.400pt}}
\put(466,443.67){\rule{8.191pt}{0.400pt}}
\multiput(466.00,443.17)(17.000,1.000){2}{\rule{4.095pt}{0.400pt}}
\multiput(500.00,443.93)(3.604,-0.477){7}{\rule{2.740pt}{0.115pt}}
\multiput(500.00,444.17)(27.313,-5.000){2}{\rule{1.370pt}{0.400pt}}
\multiput(533.00,438.93)(2.552,-0.485){11}{\rule{2.043pt}{0.117pt}}
\multiput(533.00,439.17)(29.760,-7.000){2}{\rule{1.021pt}{0.400pt}}
\multiput(567.00,431.93)(3.716,-0.477){7}{\rule{2.820pt}{0.115pt}}
\multiput(567.00,432.17)(28.147,-5.000){2}{\rule{1.410pt}{0.400pt}}
\multiput(601.00,426.93)(3.604,-0.477){7}{\rule{2.740pt}{0.115pt}}
\multiput(601.00,427.17)(27.313,-5.000){2}{\rule{1.370pt}{0.400pt}}
\multiput(634.00,421.93)(3.716,-0.477){7}{\rule{2.820pt}{0.115pt}}
\multiput(634.00,422.17)(28.147,-5.000){2}{\rule{1.410pt}{0.400pt}}
\multiput(668.00,416.93)(1.893,-0.489){15}{\rule{1.567pt}{0.118pt}}
\multiput(668.00,417.17)(29.748,-9.000){2}{\rule{0.783pt}{0.400pt}}
\multiput(701.00,407.94)(4.868,-0.468){5}{\rule{3.500pt}{0.113pt}}
\multiput(701.00,408.17)(26.736,-4.000){2}{\rule{1.750pt}{0.400pt}}
\multiput(735.00,403.93)(2.932,-0.482){9}{\rule{2.300pt}{0.116pt}}
\multiput(735.00,404.17)(28.226,-6.000){2}{\rule{1.150pt}{0.400pt}}
\multiput(768.00,397.93)(2.552,-0.485){11}{\rule{2.043pt}{0.117pt}}
\multiput(768.00,398.17)(29.760,-7.000){2}{\rule{1.021pt}{0.400pt}}
\put(802,391.67){\rule{7.950pt}{0.400pt}}
\multiput(802.00,391.17)(16.500,1.000){2}{\rule{3.975pt}{0.400pt}}
\multiput(835.00,391.93)(2.211,-0.488){13}{\rule{1.800pt}{0.117pt}}
\multiput(835.00,392.17)(30.264,-8.000){2}{\rule{0.900pt}{0.400pt}}
\multiput(869.00,383.93)(2.932,-0.482){9}{\rule{2.300pt}{0.116pt}}
\multiput(869.00,384.17)(28.226,-6.000){2}{\rule{1.150pt}{0.400pt}}
\multiput(902.00,377.95)(7.383,-0.447){3}{\rule{4.633pt}{0.108pt}}
\multiput(902.00,378.17)(24.383,-3.000){2}{\rule{2.317pt}{0.400pt}}
\put(936,376.17){\rule{6.700pt}{0.400pt}}
\multiput(936.00,375.17)(19.094,2.000){2}{\rule{3.350pt}{0.400pt}}
\multiput(969.00,376.94)(4.868,-0.468){5}{\rule{3.500pt}{0.113pt}}
\multiput(969.00,377.17)(26.736,-4.000){2}{\rule{1.750pt}{0.400pt}}
\multiput(1003.00,372.93)(2.211,-0.488){13}{\rule{1.800pt}{0.117pt}}
\multiput(1003.00,373.17)(30.264,-8.000){2}{\rule{0.900pt}{0.400pt}}
\multiput(1037.00,364.93)(2.476,-0.485){11}{\rule{1.986pt}{0.117pt}}
\multiput(1037.00,365.17)(28.879,-7.000){2}{\rule{0.993pt}{0.400pt}}
\put(1070,357.67){\rule{8.191pt}{0.400pt}}
\multiput(1070.00,358.17)(17.000,-1.000){2}{\rule{4.095pt}{0.400pt}}
\multiput(1104.00,356.94)(4.722,-0.468){5}{\rule{3.400pt}{0.113pt}}
\multiput(1104.00,357.17)(25.943,-4.000){2}{\rule{1.700pt}{0.400pt}}
\multiput(1137.00,352.94)(4.868,-0.468){5}{\rule{3.500pt}{0.113pt}}
\multiput(1137.00,353.17)(26.736,-4.000){2}{\rule{1.750pt}{0.400pt}}
\put(1171,348.67){\rule{7.950pt}{0.400pt}}
\multiput(1171.00,349.17)(16.500,-1.000){2}{\rule{3.975pt}{0.400pt}}
\put(1238,347.17){\rule{6.700pt}{0.400pt}}
\multiput(1238.00,348.17)(19.094,-2.000){2}{\rule{3.350pt}{0.400pt}}
\put(1271,345.67){\rule{8.191pt}{0.400pt}}
\multiput(1271.00,346.17)(17.000,-1.000){2}{\rule{4.095pt}{0.400pt}}
\put(1305,345.67){\rule{7.950pt}{0.400pt}}
\multiput(1305.00,345.17)(16.500,1.000){2}{\rule{3.975pt}{0.400pt}}
\put(1204.0,349.0){\rule[-0.200pt]{8.191pt}{0.400pt}}
\multiput(1372.00,345.93)(2.476,-0.485){11}{\rule{1.986pt}{0.117pt}}
\multiput(1372.00,346.17)(28.879,-7.000){2}{\rule{0.993pt}{0.400pt}}
\put(1405,338.17){\rule{6.900pt}{0.400pt}}
\multiput(1405.00,339.17)(19.679,-2.000){2}{\rule{3.450pt}{0.400pt}}
\put(1338.0,347.0){\rule[-0.200pt]{8.191pt}{0.400pt}}
\put(1279,768){\makebox(0,0)[r]{改进数据}}
\put(1299.0,768.0){\rule[-0.200pt]{24.090pt}{0.400pt}}
\put(131,669){\usebox{\plotpoint}}
\multiput(131.00,667.92)(1.009,-0.495){31}{\rule{0.900pt}{0.119pt}}
\multiput(131.00,668.17)(32.132,-17.000){2}{\rule{0.450pt}{0.400pt}}
\multiput(165.00,650.92)(0.589,-0.497){53}{\rule{0.571pt}{0.120pt}}
\multiput(165.00,651.17)(31.814,-28.000){2}{\rule{0.286pt}{0.400pt}}
\multiput(198.00,622.92)(1.009,-0.495){31}{\rule{0.900pt}{0.119pt}}
\multiput(198.00,623.17)(32.132,-17.000){2}{\rule{0.450pt}{0.400pt}}
\multiput(232.00,605.93)(2.932,-0.482){9}{\rule{2.300pt}{0.116pt}}
\multiput(232.00,606.17)(28.226,-6.000){2}{\rule{1.150pt}{0.400pt}}
\multiput(265.00,599.92)(1.581,-0.492){19}{\rule{1.336pt}{0.118pt}}
\multiput(265.00,600.17)(31.226,-11.000){2}{\rule{0.668pt}{0.400pt}}
\multiput(299.00,588.92)(1.041,-0.494){29}{\rule{0.925pt}{0.119pt}}
\multiput(299.00,589.17)(31.080,-16.000){2}{\rule{0.463pt}{0.400pt}}
\multiput(332.00,572.92)(1.444,-0.492){21}{\rule{1.233pt}{0.119pt}}
\multiput(332.00,573.17)(31.440,-12.000){2}{\rule{0.617pt}{0.400pt}}
\multiput(366.00,560.92)(1.113,-0.494){27}{\rule{0.980pt}{0.119pt}}
\multiput(366.00,561.17)(30.966,-15.000){2}{\rule{0.490pt}{0.400pt}}
\multiput(399.00,545.93)(2.211,-0.488){13}{\rule{1.800pt}{0.117pt}}
\multiput(399.00,546.17)(30.264,-8.000){2}{\rule{0.900pt}{0.400pt}}
\put(433,537.17){\rule{6.700pt}{0.400pt}}
\multiput(433.00,538.17)(19.094,-2.000){2}{\rule{3.350pt}{0.400pt}}
\multiput(466.00,535.93)(1.951,-0.489){15}{\rule{1.611pt}{0.118pt}}
\multiput(466.00,536.17)(30.656,-9.000){2}{\rule{0.806pt}{0.400pt}}
\multiput(500.00,526.93)(2.145,-0.488){13}{\rule{1.750pt}{0.117pt}}
\multiput(500.00,527.17)(29.368,-8.000){2}{\rule{0.875pt}{0.400pt}}
\multiput(533.00,518.92)(1.581,-0.492){19}{\rule{1.336pt}{0.118pt}}
\multiput(533.00,519.17)(31.226,-11.000){2}{\rule{0.668pt}{0.400pt}}
\multiput(567.00,507.92)(1.147,-0.494){27}{\rule{1.007pt}{0.119pt}}
\multiput(567.00,508.17)(31.911,-15.000){2}{\rule{0.503pt}{0.400pt}}
\multiput(601.00,492.93)(2.932,-0.482){9}{\rule{2.300pt}{0.116pt}}
\multiput(601.00,493.17)(28.226,-6.000){2}{\rule{1.150pt}{0.400pt}}
\multiput(634.00,486.94)(4.868,-0.468){5}{\rule{3.500pt}{0.113pt}}
\multiput(634.00,487.17)(26.736,-4.000){2}{\rule{1.750pt}{0.400pt}}
\multiput(668.00,482.93)(2.476,-0.485){11}{\rule{1.986pt}{0.117pt}}
\multiput(668.00,483.17)(28.879,-7.000){2}{\rule{0.993pt}{0.400pt}}
\multiput(701.00,475.95)(7.383,-0.447){3}{\rule{4.633pt}{0.108pt}}
\multiput(701.00,476.17)(24.383,-3.000){2}{\rule{2.317pt}{0.400pt}}
\multiput(735.00,472.95)(7.160,-0.447){3}{\rule{4.500pt}{0.108pt}}
\multiput(735.00,473.17)(23.660,-3.000){2}{\rule{2.250pt}{0.400pt}}
\multiput(768.00,469.93)(3.716,-0.477){7}{\rule{2.820pt}{0.115pt}}
\multiput(768.00,470.17)(28.147,-5.000){2}{\rule{1.410pt}{0.400pt}}
\multiput(802.00,464.93)(2.145,-0.488){13}{\rule{1.750pt}{0.117pt}}
\multiput(802.00,465.17)(29.368,-8.000){2}{\rule{0.875pt}{0.400pt}}
\multiput(835.00,456.93)(3.716,-0.477){7}{\rule{2.820pt}{0.115pt}}
\multiput(835.00,457.17)(28.147,-5.000){2}{\rule{1.410pt}{0.400pt}}
\multiput(869.00,451.93)(1.893,-0.489){15}{\rule{1.567pt}{0.118pt}}
\multiput(869.00,452.17)(29.748,-9.000){2}{\rule{0.783pt}{0.400pt}}
\multiput(902.00,442.93)(3.022,-0.482){9}{\rule{2.367pt}{0.116pt}}
\multiput(902.00,443.17)(29.088,-6.000){2}{\rule{1.183pt}{0.400pt}}
\multiput(936.00,436.93)(2.145,-0.488){13}{\rule{1.750pt}{0.117pt}}
\multiput(936.00,437.17)(29.368,-8.000){2}{\rule{0.875pt}{0.400pt}}
\multiput(969.00,428.93)(2.211,-0.488){13}{\rule{1.800pt}{0.117pt}}
\multiput(969.00,429.17)(30.264,-8.000){2}{\rule{0.900pt}{0.400pt}}
\multiput(1003.00,420.94)(4.868,-0.468){5}{\rule{3.500pt}{0.113pt}}
\multiput(1003.00,421.17)(26.736,-4.000){2}{\rule{1.750pt}{0.400pt}}
\multiput(1037.00,416.93)(3.604,-0.477){7}{\rule{2.740pt}{0.115pt}}
\multiput(1037.00,417.17)(27.313,-5.000){2}{\rule{1.370pt}{0.400pt}}
\multiput(1070.00,411.93)(2.211,-0.488){13}{\rule{1.800pt}{0.117pt}}
\multiput(1070.00,412.17)(30.264,-8.000){2}{\rule{0.900pt}{0.400pt}}
\multiput(1104.00,403.92)(1.401,-0.492){21}{\rule{1.200pt}{0.119pt}}
\multiput(1104.00,404.17)(30.509,-12.000){2}{\rule{0.600pt}{0.400pt}}
\put(1137,391.17){\rule{6.900pt}{0.400pt}}
\multiput(1137.00,392.17)(19.679,-2.000){2}{\rule{3.450pt}{0.400pt}}
\put(1171,389.67){\rule{7.950pt}{0.400pt}}
\multiput(1171.00,390.17)(16.500,-1.000){2}{\rule{3.975pt}{0.400pt}}
\multiput(1204.00,388.93)(3.716,-0.477){7}{\rule{2.820pt}{0.115pt}}
\multiput(1204.00,389.17)(28.147,-5.000){2}{\rule{1.410pt}{0.400pt}}
\multiput(1238.00,383.95)(7.160,-0.447){3}{\rule{4.500pt}{0.108pt}}
\multiput(1238.00,384.17)(23.660,-3.000){2}{\rule{2.250pt}{0.400pt}}
\multiput(1271.00,380.93)(3.716,-0.477){7}{\rule{2.820pt}{0.115pt}}
\multiput(1271.00,381.17)(28.147,-5.000){2}{\rule{1.410pt}{0.400pt}}
\multiput(1305.00,375.93)(3.604,-0.477){7}{\rule{2.740pt}{0.115pt}}
\multiput(1305.00,376.17)(27.313,-5.000){2}{\rule{1.370pt}{0.400pt}}
\put(1338,370.67){\rule{8.191pt}{0.400pt}}
\multiput(1338.00,371.17)(17.000,-1.000){2}{\rule{4.095pt}{0.400pt}}
\put(1372,369.17){\rule{6.700pt}{0.400pt}}
\multiput(1372.00,370.17)(19.094,-2.000){2}{\rule{3.350pt}{0.400pt}}
\put(131,669){\makebox(0,0){$+$}}
\put(165,652){\makebox(0,0){$+$}}
\put(198,624){\makebox(0,0){$+$}}
\put(232,607){\makebox(0,0){$+$}}
\put(265,601){\makebox(0,0){$+$}}
\put(299,590){\makebox(0,0){$+$}}
\put(332,574){\makebox(0,0){$+$}}
\put(366,562){\makebox(0,0){$+$}}
\put(399,547){\makebox(0,0){$+$}}
\put(433,539){\makebox(0,0){$+$}}
\put(466,537){\makebox(0,0){$+$}}
\put(500,528){\makebox(0,0){$+$}}
\put(533,520){\makebox(0,0){$+$}}
\put(567,509){\makebox(0,0){$+$}}
\put(601,494){\makebox(0,0){$+$}}
\put(634,488){\makebox(0,0){$+$}}
\put(668,484){\makebox(0,0){$+$}}
\put(701,477){\makebox(0,0){$+$}}
\put(735,474){\makebox(0,0){$+$}}
\put(768,471){\makebox(0,0){$+$}}
\put(802,466){\makebox(0,0){$+$}}
\put(835,458){\makebox(0,0){$+$}}
\put(869,453){\makebox(0,0){$+$}}
\put(902,444){\makebox(0,0){$+$}}
\put(936,438){\makebox(0,0){$+$}}
\put(969,430){\makebox(0,0){$+$}}
\put(1003,422){\makebox(0,0){$+$}}
\put(1037,418){\makebox(0,0){$+$}}
\put(1070,413){\makebox(0,0){$+$}}
\put(1104,405){\makebox(0,0){$+$}}
\put(1137,393){\makebox(0,0){$+$}}
\put(1171,391){\makebox(0,0){$+$}}
\put(1204,390){\makebox(0,0){$+$}}
\put(1238,385){\makebox(0,0){$+$}}
\put(1271,382){\makebox(0,0){$+$}}
\put(1305,377){\makebox(0,0){$+$}}
\put(1338,372){\makebox(0,0){$+$}}
\put(1372,371){\makebox(0,0){$+$}}
\put(1405,369){\makebox(0,0){$+$}}
\put(1439,369){\makebox(0,0){$+$}}
\put(1349,768){\makebox(0,0){$+$}}
\put(1405.0,369.0){\rule[-0.200pt]{8.191pt}{0.400pt}}
\put(131.0,82.0){\rule[-0.200pt]{0.400pt}{187.179pt}}
\put(131.0,82.0){\rule[-0.200pt]{315.097pt}{0.400pt}}
\put(1439.0,82.0){\rule[-0.200pt]{0.400pt}{187.179pt}}
\put(131.0,859.0){\rule[-0.200pt]{315.097pt}{0.400pt}}
\end{picture}
}
   \end{frame}
 \begin{frame}
提升其他人际动机的仿真结果
     \scalebox{0.8}{% GNUPLOT: LaTeX picture
\setlength{\unitlength}{0.240900pt}
\ifx\plotpoint\undefined\newsavebox{\plotpoint}\fi
\begin{picture}(1500,900)(0,0)
\sbox{\plotpoint}{\rule[-0.200pt]{0.400pt}{0.400pt}}%
\put(131.0,82.0){\rule[-0.200pt]{4.818pt}{0.400pt}}
\put(111,82){\makebox(0,0)[r]{-1}}
\put(1419.0,82.0){\rule[-0.200pt]{4.818pt}{0.400pt}}
\put(131.0,237.0){\rule[-0.200pt]{4.818pt}{0.400pt}}
\put(111,237){\makebox(0,0)[r]{ 0}}
\put(1419.0,237.0){\rule[-0.200pt]{4.818pt}{0.400pt}}
\put(131.0,393.0){\rule[-0.200pt]{4.818pt}{0.400pt}}
\put(111,393){\makebox(0,0)[r]{ 1}}
\put(1419.0,393.0){\rule[-0.200pt]{4.818pt}{0.400pt}}
\put(131.0,548.0){\rule[-0.200pt]{4.818pt}{0.400pt}}
\put(111,548){\makebox(0,0)[r]{ 2}}
\put(1419.0,548.0){\rule[-0.200pt]{4.818pt}{0.400pt}}
\put(131.0,704.0){\rule[-0.200pt]{4.818pt}{0.400pt}}
\put(111,704){\makebox(0,0)[r]{ 3}}
\put(1419.0,704.0){\rule[-0.200pt]{4.818pt}{0.400pt}}
\put(131.0,82.0){\rule[-0.200pt]{0.400pt}{4.818pt}}
\put(131,41){\makebox(0,0){ 1}}
\put(131.0,839.0){\rule[-0.200pt]{0.400pt}{4.818pt}}
\put(198.0,82.0){\rule[-0.200pt]{0.400pt}{4.818pt}}
\put(198,41){\makebox(0,0){ 3}}
\put(198.0,839.0){\rule[-0.200pt]{0.400pt}{4.818pt}}
\put(265.0,82.0){\rule[-0.200pt]{0.400pt}{4.818pt}}
\put(265,41){\makebox(0,0){ 5}}
\put(265.0,839.0){\rule[-0.200pt]{0.400pt}{4.818pt}}
\put(332.0,82.0){\rule[-0.200pt]{0.400pt}{4.818pt}}
\put(332,41){\makebox(0,0){ 7}}
\put(332.0,839.0){\rule[-0.200pt]{0.400pt}{4.818pt}}
\put(399.0,82.0){\rule[-0.200pt]{0.400pt}{4.818pt}}
\put(399,41){\makebox(0,0){ 9}}
\put(399.0,839.0){\rule[-0.200pt]{0.400pt}{4.818pt}}
\put(466.0,82.0){\rule[-0.200pt]{0.400pt}{4.818pt}}
\put(466,41){\makebox(0,0){ 11}}
\put(466.0,839.0){\rule[-0.200pt]{0.400pt}{4.818pt}}
\put(533.0,82.0){\rule[-0.200pt]{0.400pt}{4.818pt}}
\put(533,41){\makebox(0,0){ 13}}
\put(533.0,839.0){\rule[-0.200pt]{0.400pt}{4.818pt}}
\put(601.0,82.0){\rule[-0.200pt]{0.400pt}{4.818pt}}
\put(601,41){\makebox(0,0){ 15}}
\put(601.0,839.0){\rule[-0.200pt]{0.400pt}{4.818pt}}
\put(668.0,82.0){\rule[-0.200pt]{0.400pt}{4.818pt}}
\put(668,41){\makebox(0,0){ 17}}
\put(668.0,839.0){\rule[-0.200pt]{0.400pt}{4.818pt}}
\put(735.0,82.0){\rule[-0.200pt]{0.400pt}{4.818pt}}
\put(735,41){\makebox(0,0){ 19}}
\put(735.0,839.0){\rule[-0.200pt]{0.400pt}{4.818pt}}
\put(802.0,82.0){\rule[-0.200pt]{0.400pt}{4.818pt}}
\put(802,41){\makebox(0,0){ 21}}
\put(802.0,839.0){\rule[-0.200pt]{0.400pt}{4.818pt}}
\put(869.0,82.0){\rule[-0.200pt]{0.400pt}{4.818pt}}
\put(869,41){\makebox(0,0){ 23}}
\put(869.0,839.0){\rule[-0.200pt]{0.400pt}{4.818pt}}
\put(936.0,82.0){\rule[-0.200pt]{0.400pt}{4.818pt}}
\put(936,41){\makebox(0,0){ 25}}
\put(936.0,839.0){\rule[-0.200pt]{0.400pt}{4.818pt}}
\put(1003.0,82.0){\rule[-0.200pt]{0.400pt}{4.818pt}}
\put(1003,41){\makebox(0,0){ 27}}
\put(1003.0,839.0){\rule[-0.200pt]{0.400pt}{4.818pt}}
\put(1070.0,82.0){\rule[-0.200pt]{0.400pt}{4.818pt}}
\put(1070,41){\makebox(0,0){ 29}}
\put(1070.0,839.0){\rule[-0.200pt]{0.400pt}{4.818pt}}
\put(1137.0,82.0){\rule[-0.200pt]{0.400pt}{4.818pt}}
\put(1137,41){\makebox(0,0){ 31}}
\put(1137.0,839.0){\rule[-0.200pt]{0.400pt}{4.818pt}}
\put(1204.0,82.0){\rule[-0.200pt]{0.400pt}{4.818pt}}
\put(1204,41){\makebox(0,0){ 33}}
\put(1204.0,839.0){\rule[-0.200pt]{0.400pt}{4.818pt}}
\put(1271.0,82.0){\rule[-0.200pt]{0.400pt}{4.818pt}}
\put(1271,41){\makebox(0,0){ 35}}
\put(1271.0,839.0){\rule[-0.200pt]{0.400pt}{4.818pt}}
\put(1338.0,82.0){\rule[-0.200pt]{0.400pt}{4.818pt}}
\put(1338,41){\makebox(0,0){ 37}}
\put(1338.0,839.0){\rule[-0.200pt]{0.400pt}{4.818pt}}
\put(1405.0,82.0){\rule[-0.200pt]{0.400pt}{4.818pt}}
\put(1405,41){\makebox(0,0){ 39}}
\put(1405.0,839.0){\rule[-0.200pt]{0.400pt}{4.818pt}}
\put(131.0,82.0){\rule[-0.200pt]{0.400pt}{187.179pt}}
\put(131.0,82.0){\rule[-0.200pt]{315.097pt}{0.400pt}}
\put(1439.0,82.0){\rule[-0.200pt]{0.400pt}{187.179pt}}
\put(131.0,859.0){\rule[-0.200pt]{315.097pt}{0.400pt}}
\put(30,470){\makebox(0,0){\rotatebox{90}{用户贡献}}}
\put(1279,819){\makebox(0,0)[r]{仿真数据}}
\put(1299.0,819.0){\rule[-0.200pt]{24.090pt}{0.400pt}}
\put(131,646){\usebox{\plotpoint}}
\multiput(131.58,643.05)(0.498,-0.766){65}{\rule{0.120pt}{0.712pt}}
\multiput(130.17,644.52)(34.000,-50.523){2}{\rule{0.400pt}{0.356pt}}
\multiput(165.00,592.92)(0.549,-0.497){57}{\rule{0.540pt}{0.120pt}}
\multiput(165.00,593.17)(31.879,-30.000){2}{\rule{0.270pt}{0.400pt}}
\multiput(198.00,562.92)(0.607,-0.497){53}{\rule{0.586pt}{0.120pt}}
\multiput(198.00,563.17)(32.784,-28.000){2}{\rule{0.293pt}{0.400pt}}
\multiput(232.00,534.92)(0.979,-0.495){31}{\rule{0.876pt}{0.119pt}}
\multiput(232.00,535.17)(31.181,-17.000){2}{\rule{0.438pt}{0.400pt}}
\multiput(265.00,517.92)(1.231,-0.494){25}{\rule{1.071pt}{0.119pt}}
\multiput(265.00,518.17)(31.776,-14.000){2}{\rule{0.536pt}{0.400pt}}
\multiput(299.00,503.92)(0.874,-0.495){35}{\rule{0.795pt}{0.119pt}}
\multiput(299.00,504.17)(31.350,-19.000){2}{\rule{0.397pt}{0.400pt}}
\multiput(332.00,484.93)(2.211,-0.488){13}{\rule{1.800pt}{0.117pt}}
\multiput(332.00,485.17)(30.264,-8.000){2}{\rule{0.900pt}{0.400pt}}
\multiput(366.00,476.93)(3.604,-0.477){7}{\rule{2.740pt}{0.115pt}}
\multiput(366.00,477.17)(27.313,-5.000){2}{\rule{1.370pt}{0.400pt}}
\multiput(399.00,471.92)(1.329,-0.493){23}{\rule{1.146pt}{0.119pt}}
\multiput(399.00,472.17)(31.621,-13.000){2}{\rule{0.573pt}{0.400pt}}
\multiput(433.00,458.92)(1.041,-0.494){29}{\rule{0.925pt}{0.119pt}}
\multiput(433.00,459.17)(31.080,-16.000){2}{\rule{0.463pt}{0.400pt}}
\put(466,443.67){\rule{8.191pt}{0.400pt}}
\multiput(466.00,443.17)(17.000,1.000){2}{\rule{4.095pt}{0.400pt}}
\multiput(500.00,443.93)(3.604,-0.477){7}{\rule{2.740pt}{0.115pt}}
\multiput(500.00,444.17)(27.313,-5.000){2}{\rule{1.370pt}{0.400pt}}
\multiput(533.00,438.93)(2.552,-0.485){11}{\rule{2.043pt}{0.117pt}}
\multiput(533.00,439.17)(29.760,-7.000){2}{\rule{1.021pt}{0.400pt}}
\multiput(567.00,431.93)(3.716,-0.477){7}{\rule{2.820pt}{0.115pt}}
\multiput(567.00,432.17)(28.147,-5.000){2}{\rule{1.410pt}{0.400pt}}
\multiput(601.00,426.93)(3.604,-0.477){7}{\rule{2.740pt}{0.115pt}}
\multiput(601.00,427.17)(27.313,-5.000){2}{\rule{1.370pt}{0.400pt}}
\multiput(634.00,421.93)(3.716,-0.477){7}{\rule{2.820pt}{0.115pt}}
\multiput(634.00,422.17)(28.147,-5.000){2}{\rule{1.410pt}{0.400pt}}
\multiput(668.00,416.93)(1.893,-0.489){15}{\rule{1.567pt}{0.118pt}}
\multiput(668.00,417.17)(29.748,-9.000){2}{\rule{0.783pt}{0.400pt}}
\multiput(701.00,407.94)(4.868,-0.468){5}{\rule{3.500pt}{0.113pt}}
\multiput(701.00,408.17)(26.736,-4.000){2}{\rule{1.750pt}{0.400pt}}
\multiput(735.00,403.93)(2.932,-0.482){9}{\rule{2.300pt}{0.116pt}}
\multiput(735.00,404.17)(28.226,-6.000){2}{\rule{1.150pt}{0.400pt}}
\multiput(768.00,397.93)(2.552,-0.485){11}{\rule{2.043pt}{0.117pt}}
\multiput(768.00,398.17)(29.760,-7.000){2}{\rule{1.021pt}{0.400pt}}
\put(802,391.67){\rule{7.950pt}{0.400pt}}
\multiput(802.00,391.17)(16.500,1.000){2}{\rule{3.975pt}{0.400pt}}
\multiput(835.00,391.93)(2.211,-0.488){13}{\rule{1.800pt}{0.117pt}}
\multiput(835.00,392.17)(30.264,-8.000){2}{\rule{0.900pt}{0.400pt}}
\multiput(869.00,383.93)(2.932,-0.482){9}{\rule{2.300pt}{0.116pt}}
\multiput(869.00,384.17)(28.226,-6.000){2}{\rule{1.150pt}{0.400pt}}
\multiput(902.00,377.95)(7.383,-0.447){3}{\rule{4.633pt}{0.108pt}}
\multiput(902.00,378.17)(24.383,-3.000){2}{\rule{2.317pt}{0.400pt}}
\put(936,376.17){\rule{6.700pt}{0.400pt}}
\multiput(936.00,375.17)(19.094,2.000){2}{\rule{3.350pt}{0.400pt}}
\multiput(969.00,376.94)(4.868,-0.468){5}{\rule{3.500pt}{0.113pt}}
\multiput(969.00,377.17)(26.736,-4.000){2}{\rule{1.750pt}{0.400pt}}
\multiput(1003.00,372.93)(2.211,-0.488){13}{\rule{1.800pt}{0.117pt}}
\multiput(1003.00,373.17)(30.264,-8.000){2}{\rule{0.900pt}{0.400pt}}
\multiput(1037.00,364.93)(2.476,-0.485){11}{\rule{1.986pt}{0.117pt}}
\multiput(1037.00,365.17)(28.879,-7.000){2}{\rule{0.993pt}{0.400pt}}
\put(1070,357.67){\rule{8.191pt}{0.400pt}}
\multiput(1070.00,358.17)(17.000,-1.000){2}{\rule{4.095pt}{0.400pt}}
\multiput(1104.00,356.94)(4.722,-0.468){5}{\rule{3.400pt}{0.113pt}}
\multiput(1104.00,357.17)(25.943,-4.000){2}{\rule{1.700pt}{0.400pt}}
\multiput(1137.00,352.94)(4.868,-0.468){5}{\rule{3.500pt}{0.113pt}}
\multiput(1137.00,353.17)(26.736,-4.000){2}{\rule{1.750pt}{0.400pt}}
\put(1171,348.67){\rule{7.950pt}{0.400pt}}
\multiput(1171.00,349.17)(16.500,-1.000){2}{\rule{3.975pt}{0.400pt}}
\put(1238,347.17){\rule{6.700pt}{0.400pt}}
\multiput(1238.00,348.17)(19.094,-2.000){2}{\rule{3.350pt}{0.400pt}}
\put(1271,345.67){\rule{8.191pt}{0.400pt}}
\multiput(1271.00,346.17)(17.000,-1.000){2}{\rule{4.095pt}{0.400pt}}
\put(1305,345.67){\rule{7.950pt}{0.400pt}}
\multiput(1305.00,345.17)(16.500,1.000){2}{\rule{3.975pt}{0.400pt}}
\put(1204.0,349.0){\rule[-0.200pt]{8.191pt}{0.400pt}}
\multiput(1372.00,345.93)(2.476,-0.485){11}{\rule{1.986pt}{0.117pt}}
\multiput(1372.00,346.17)(28.879,-7.000){2}{\rule{0.993pt}{0.400pt}}
\put(1405,338.17){\rule{6.900pt}{0.400pt}}
\multiput(1405.00,339.17)(19.679,-2.000){2}{\rule{3.450pt}{0.400pt}}
\put(1338.0,347.0){\rule[-0.200pt]{8.191pt}{0.400pt}}
\put(1279,768){\makebox(0,0)[r]{改进数据}}
\put(1299.0,768.0){\rule[-0.200pt]{24.090pt}{0.400pt}}
\put(131,671){\usebox{\plotpoint}}
\multiput(131.00,669.92)(1.073,-0.494){29}{\rule{0.950pt}{0.119pt}}
\multiput(131.00,670.17)(32.028,-16.000){2}{\rule{0.475pt}{0.400pt}}
\multiput(165.00,653.92)(0.923,-0.495){33}{\rule{0.833pt}{0.119pt}}
\multiput(165.00,654.17)(31.270,-18.000){2}{\rule{0.417pt}{0.400pt}}
\multiput(198.00,635.92)(0.855,-0.496){37}{\rule{0.780pt}{0.119pt}}
\multiput(198.00,636.17)(32.381,-20.000){2}{\rule{0.390pt}{0.400pt}}
\multiput(232.00,615.93)(2.476,-0.485){11}{\rule{1.986pt}{0.117pt}}
\multiput(232.00,616.17)(28.879,-7.000){2}{\rule{0.993pt}{0.400pt}}
\multiput(265.00,608.92)(1.581,-0.492){19}{\rule{1.336pt}{0.118pt}}
\multiput(265.00,609.17)(31.226,-11.000){2}{\rule{0.668pt}{0.400pt}}
\multiput(299.00,597.92)(1.694,-0.491){17}{\rule{1.420pt}{0.118pt}}
\multiput(299.00,598.17)(30.053,-10.000){2}{\rule{0.710pt}{0.400pt}}
\multiput(332.00,587.92)(1.231,-0.494){25}{\rule{1.071pt}{0.119pt}}
\multiput(332.00,588.17)(31.776,-14.000){2}{\rule{0.536pt}{0.400pt}}
\multiput(366.00,573.92)(1.290,-0.493){23}{\rule{1.115pt}{0.119pt}}
\multiput(366.00,574.17)(30.685,-13.000){2}{\rule{0.558pt}{0.400pt}}
\multiput(399.00,560.93)(2.211,-0.488){13}{\rule{1.800pt}{0.117pt}}
\multiput(399.00,561.17)(30.264,-8.000){2}{\rule{0.900pt}{0.400pt}}
\multiput(433.00,552.93)(1.893,-0.489){15}{\rule{1.567pt}{0.118pt}}
\multiput(433.00,553.17)(29.748,-9.000){2}{\rule{0.783pt}{0.400pt}}
\multiput(466.00,543.93)(3.022,-0.482){9}{\rule{2.367pt}{0.116pt}}
\multiput(466.00,544.17)(29.088,-6.000){2}{\rule{1.183pt}{0.400pt}}
\multiput(500.00,537.93)(3.604,-0.477){7}{\rule{2.740pt}{0.115pt}}
\multiput(500.00,538.17)(27.313,-5.000){2}{\rule{1.370pt}{0.400pt}}
\multiput(533.00,532.95)(7.383,-0.447){3}{\rule{4.633pt}{0.108pt}}
\multiput(533.00,533.17)(24.383,-3.000){2}{\rule{2.317pt}{0.400pt}}
\multiput(567.00,529.93)(3.022,-0.482){9}{\rule{2.367pt}{0.116pt}}
\multiput(567.00,530.17)(29.088,-6.000){2}{\rule{1.183pt}{0.400pt}}
\multiput(601.00,523.95)(7.160,-0.447){3}{\rule{4.500pt}{0.108pt}}
\multiput(601.00,524.17)(23.660,-3.000){2}{\rule{2.250pt}{0.400pt}}
\multiput(634.00,520.92)(1.581,-0.492){19}{\rule{1.336pt}{0.118pt}}
\multiput(634.00,521.17)(31.226,-11.000){2}{\rule{0.668pt}{0.400pt}}
\multiput(668.00,509.93)(3.604,-0.477){7}{\rule{2.740pt}{0.115pt}}
\multiput(668.00,510.17)(27.313,-5.000){2}{\rule{1.370pt}{0.400pt}}
\multiput(701.00,504.93)(3.022,-0.482){9}{\rule{2.367pt}{0.116pt}}
\multiput(701.00,505.17)(29.088,-6.000){2}{\rule{1.183pt}{0.400pt}}
\multiput(735.00,498.93)(2.932,-0.482){9}{\rule{2.300pt}{0.116pt}}
\multiput(735.00,499.17)(28.226,-6.000){2}{\rule{1.150pt}{0.400pt}}
\multiput(768.00,492.93)(3.022,-0.482){9}{\rule{2.367pt}{0.116pt}}
\multiput(768.00,493.17)(29.088,-6.000){2}{\rule{1.183pt}{0.400pt}}
\multiput(802.00,486.94)(4.722,-0.468){5}{\rule{3.400pt}{0.113pt}}
\multiput(802.00,487.17)(25.943,-4.000){2}{\rule{1.700pt}{0.400pt}}
\multiput(835.00,482.92)(1.231,-0.494){25}{\rule{1.071pt}{0.119pt}}
\multiput(835.00,483.17)(31.776,-14.000){2}{\rule{0.536pt}{0.400pt}}
\multiput(869.00,468.94)(4.722,-0.468){5}{\rule{3.400pt}{0.113pt}}
\multiput(869.00,469.17)(25.943,-4.000){2}{\rule{1.700pt}{0.400pt}}
\multiput(902.00,464.93)(3.022,-0.482){9}{\rule{2.367pt}{0.116pt}}
\multiput(902.00,465.17)(29.088,-6.000){2}{\rule{1.183pt}{0.400pt}}
\multiput(936.00,458.95)(7.160,-0.447){3}{\rule{4.500pt}{0.108pt}}
\multiput(936.00,459.17)(23.660,-3.000){2}{\rule{2.250pt}{0.400pt}}
\multiput(969.00,455.93)(2.552,-0.485){11}{\rule{2.043pt}{0.117pt}}
\multiput(969.00,456.17)(29.760,-7.000){2}{\rule{1.021pt}{0.400pt}}
\multiput(1003.00,448.94)(4.868,-0.468){5}{\rule{3.500pt}{0.113pt}}
\multiput(1003.00,449.17)(26.736,-4.000){2}{\rule{1.750pt}{0.400pt}}
\multiput(1037.00,444.93)(3.604,-0.477){7}{\rule{2.740pt}{0.115pt}}
\multiput(1037.00,445.17)(27.313,-5.000){2}{\rule{1.370pt}{0.400pt}}
\multiput(1104.00,439.93)(3.604,-0.477){7}{\rule{2.740pt}{0.115pt}}
\multiput(1104.00,440.17)(27.313,-5.000){2}{\rule{1.370pt}{0.400pt}}
\multiput(1137.00,434.93)(3.022,-0.482){9}{\rule{2.367pt}{0.116pt}}
\multiput(1137.00,435.17)(29.088,-6.000){2}{\rule{1.183pt}{0.400pt}}
\multiput(1171.00,428.93)(2.932,-0.482){9}{\rule{2.300pt}{0.116pt}}
\multiput(1171.00,429.17)(28.226,-6.000){2}{\rule{1.150pt}{0.400pt}}
\multiput(1204.00,422.93)(1.951,-0.489){15}{\rule{1.611pt}{0.118pt}}
\multiput(1204.00,423.17)(30.656,-9.000){2}{\rule{0.806pt}{0.400pt}}
\multiput(1238.00,413.93)(2.476,-0.485){11}{\rule{1.986pt}{0.117pt}}
\multiput(1238.00,414.17)(28.879,-7.000){2}{\rule{0.993pt}{0.400pt}}
\multiput(1271.00,406.95)(7.383,-0.447){3}{\rule{4.633pt}{0.108pt}}
\multiput(1271.00,407.17)(24.383,-3.000){2}{\rule{2.317pt}{0.400pt}}
\multiput(1305.00,403.95)(7.160,-0.447){3}{\rule{4.500pt}{0.108pt}}
\multiput(1305.00,404.17)(23.660,-3.000){2}{\rule{2.250pt}{0.400pt}}
\multiput(1338.00,400.93)(3.022,-0.482){9}{\rule{2.367pt}{0.116pt}}
\multiput(1338.00,401.17)(29.088,-6.000){2}{\rule{1.183pt}{0.400pt}}
\put(1372,394.17){\rule{6.700pt}{0.400pt}}
\multiput(1372.00,395.17)(19.094,-2.000){2}{\rule{3.350pt}{0.400pt}}
\put(1405,392.67){\rule{8.191pt}{0.400pt}}
\multiput(1405.00,393.17)(17.000,-1.000){2}{\rule{4.095pt}{0.400pt}}
\put(131,671){\makebox(0,0){$+$}}
\put(165,655){\makebox(0,0){$+$}}
\put(198,637){\makebox(0,0){$+$}}
\put(232,617){\makebox(0,0){$+$}}
\put(265,610){\makebox(0,0){$+$}}
\put(299,599){\makebox(0,0){$+$}}
\put(332,589){\makebox(0,0){$+$}}
\put(366,575){\makebox(0,0){$+$}}
\put(399,562){\makebox(0,0){$+$}}
\put(433,554){\makebox(0,0){$+$}}
\put(466,545){\makebox(0,0){$+$}}
\put(500,539){\makebox(0,0){$+$}}
\put(533,534){\makebox(0,0){$+$}}
\put(567,531){\makebox(0,0){$+$}}
\put(601,525){\makebox(0,0){$+$}}
\put(634,522){\makebox(0,0){$+$}}
\put(668,511){\makebox(0,0){$+$}}
\put(701,506){\makebox(0,0){$+$}}
\put(735,500){\makebox(0,0){$+$}}
\put(768,494){\makebox(0,0){$+$}}
\put(802,488){\makebox(0,0){$+$}}
\put(835,484){\makebox(0,0){$+$}}
\put(869,470){\makebox(0,0){$+$}}
\put(902,466){\makebox(0,0){$+$}}
\put(936,460){\makebox(0,0){$+$}}
\put(969,457){\makebox(0,0){$+$}}
\put(1003,450){\makebox(0,0){$+$}}
\put(1037,446){\makebox(0,0){$+$}}
\put(1070,441){\makebox(0,0){$+$}}
\put(1104,441){\makebox(0,0){$+$}}
\put(1137,436){\makebox(0,0){$+$}}
\put(1171,430){\makebox(0,0){$+$}}
\put(1204,424){\makebox(0,0){$+$}}
\put(1238,415){\makebox(0,0){$+$}}
\put(1271,408){\makebox(0,0){$+$}}
\put(1305,405){\makebox(0,0){$+$}}
\put(1338,402){\makebox(0,0){$+$}}
\put(1372,396){\makebox(0,0){$+$}}
\put(1405,394){\makebox(0,0){$+$}}
\put(1439,393){\makebox(0,0){$+$}}
\put(1349,768){\makebox(0,0){$+$}}
\put(1070.0,441.0){\rule[-0.200pt]{8.191pt}{0.400pt}}
\put(131.0,82.0){\rule[-0.200pt]{0.400pt}{187.179pt}}
\put(131.0,82.0){\rule[-0.200pt]{315.097pt}{0.400pt}}
\put(1439.0,82.0){\rule[-0.200pt]{0.400pt}{187.179pt}}
\put(131.0,859.0){\rule[-0.200pt]{315.097pt}{0.400pt}}
\end{picture}
}
   \end{frame}

 \begin{frame}
提升自我效能和归属感的仿真结果
     \scalebox{0.8}{% GNUPLOT: LaTeX picture
\setlength{\unitlength}{0.240900pt}
\ifx\plotpoint\undefined\newsavebox{\plotpoint}\fi
\begin{picture}(1500,900)(0,0)
\sbox{\plotpoint}{\rule[-0.200pt]{0.400pt}{0.400pt}}%
\put(211.0,82.0){\rule[-0.200pt]{4.818pt}{0.400pt}}
\put(191,82){\makebox(0,0)[r]{-0.02}}
\put(1419.0,82.0){\rule[-0.200pt]{4.818pt}{0.400pt}}
\put(211.0,143.0){\rule[-0.200pt]{4.818pt}{0.400pt}}
\put(191,143){\makebox(0,0)[r]{-0.015}}
\put(1419.0,143.0){\rule[-0.200pt]{4.818pt}{0.400pt}}
\put(211.0,203.0){\rule[-0.200pt]{4.818pt}{0.400pt}}
\put(191,203){\makebox(0,0)[r]{-0.01}}
\put(1419.0,203.0){\rule[-0.200pt]{4.818pt}{0.400pt}}
\put(211.0,264.0){\rule[-0.200pt]{4.818pt}{0.400pt}}
\put(191,264){\makebox(0,0)[r]{-0.005}}
\put(1419.0,264.0){\rule[-0.200pt]{4.818pt}{0.400pt}}
\put(211.0,325.0){\rule[-0.200pt]{4.818pt}{0.400pt}}
\put(191,325){\makebox(0,0)[r]{ 0}}
\put(1419.0,325.0){\rule[-0.200pt]{4.818pt}{0.400pt}}
\put(211.0,386.0){\rule[-0.200pt]{4.818pt}{0.400pt}}
\put(191,386){\makebox(0,0)[r]{ 0.005}}
\put(1419.0,386.0){\rule[-0.200pt]{4.818pt}{0.400pt}}
\put(211.0,446.0){\rule[-0.200pt]{4.818pt}{0.400pt}}
\put(191,446){\makebox(0,0)[r]{ 0.01}}
\put(1419.0,446.0){\rule[-0.200pt]{4.818pt}{0.400pt}}
\put(211.0,507.0){\rule[-0.200pt]{4.818pt}{0.400pt}}
\put(191,507){\makebox(0,0)[r]{ 0.015}}
\put(1419.0,507.0){\rule[-0.200pt]{4.818pt}{0.400pt}}
\put(211.0,568.0){\rule[-0.200pt]{4.818pt}{0.400pt}}
\put(191,568){\makebox(0,0)[r]{ 0.02}}
\put(1419.0,568.0){\rule[-0.200pt]{4.818pt}{0.400pt}}
\put(211.0,628.0){\rule[-0.200pt]{4.818pt}{0.400pt}}
\put(191,628){\makebox(0,0)[r]{ 0.025}}
\put(1419.0,628.0){\rule[-0.200pt]{4.818pt}{0.400pt}}
\put(211.0,689.0){\rule[-0.200pt]{4.818pt}{0.400pt}}
\put(191,689){\makebox(0,0)[r]{ 0.03}}
\put(1419.0,689.0){\rule[-0.200pt]{4.818pt}{0.400pt}}
\put(211.0,82.0){\rule[-0.200pt]{0.400pt}{4.818pt}}
\put(211,41){\makebox(0,0){ 1}}
\put(211.0,839.0){\rule[-0.200pt]{0.400pt}{4.818pt}}
\put(274.0,82.0){\rule[-0.200pt]{0.400pt}{4.818pt}}
\put(274,41){\makebox(0,0){ 3}}
\put(274.0,839.0){\rule[-0.200pt]{0.400pt}{4.818pt}}
\put(337.0,82.0){\rule[-0.200pt]{0.400pt}{4.818pt}}
\put(337,41){\makebox(0,0){ 5}}
\put(337.0,839.0){\rule[-0.200pt]{0.400pt}{4.818pt}}
\put(400.0,82.0){\rule[-0.200pt]{0.400pt}{4.818pt}}
\put(400,41){\makebox(0,0){ 7}}
\put(400.0,839.0){\rule[-0.200pt]{0.400pt}{4.818pt}}
\put(463.0,82.0){\rule[-0.200pt]{0.400pt}{4.818pt}}
\put(463,41){\makebox(0,0){ 9}}
\put(463.0,839.0){\rule[-0.200pt]{0.400pt}{4.818pt}}
\put(526.0,82.0){\rule[-0.200pt]{0.400pt}{4.818pt}}
\put(526,41){\makebox(0,0){ 11}}
\put(526.0,839.0){\rule[-0.200pt]{0.400pt}{4.818pt}}
\put(589.0,82.0){\rule[-0.200pt]{0.400pt}{4.818pt}}
\put(589,41){\makebox(0,0){ 13}}
\put(589.0,839.0){\rule[-0.200pt]{0.400pt}{4.818pt}}
\put(652.0,82.0){\rule[-0.200pt]{0.400pt}{4.818pt}}
\put(652,41){\makebox(0,0){ 15}}
\put(652.0,839.0){\rule[-0.200pt]{0.400pt}{4.818pt}}
\put(715.0,82.0){\rule[-0.200pt]{0.400pt}{4.818pt}}
\put(715,41){\makebox(0,0){ 17}}
\put(715.0,839.0){\rule[-0.200pt]{0.400pt}{4.818pt}}
\put(778.0,82.0){\rule[-0.200pt]{0.400pt}{4.818pt}}
\put(778,41){\makebox(0,0){ 19}}
\put(778.0,839.0){\rule[-0.200pt]{0.400pt}{4.818pt}}
\put(841.0,82.0){\rule[-0.200pt]{0.400pt}{4.818pt}}
\put(841,41){\makebox(0,0){ 21}}
\put(841.0,839.0){\rule[-0.200pt]{0.400pt}{4.818pt}}
\put(904.0,82.0){\rule[-0.200pt]{0.400pt}{4.818pt}}
\put(904,41){\makebox(0,0){ 23}}
\put(904.0,839.0){\rule[-0.200pt]{0.400pt}{4.818pt}}
\put(967.0,82.0){\rule[-0.200pt]{0.400pt}{4.818pt}}
\put(967,41){\makebox(0,0){ 25}}
\put(967.0,839.0){\rule[-0.200pt]{0.400pt}{4.818pt}}
\put(1030.0,82.0){\rule[-0.200pt]{0.400pt}{4.818pt}}
\put(1030,41){\makebox(0,0){ 27}}
\put(1030.0,839.0){\rule[-0.200pt]{0.400pt}{4.818pt}}
\put(1093.0,82.0){\rule[-0.200pt]{0.400pt}{4.818pt}}
\put(1093,41){\makebox(0,0){ 29}}
\put(1093.0,839.0){\rule[-0.200pt]{0.400pt}{4.818pt}}
\put(1156.0,82.0){\rule[-0.200pt]{0.400pt}{4.818pt}}
\put(1156,41){\makebox(0,0){ 31}}
\put(1156.0,839.0){\rule[-0.200pt]{0.400pt}{4.818pt}}
\put(1219.0,82.0){\rule[-0.200pt]{0.400pt}{4.818pt}}
\put(1219,41){\makebox(0,0){ 33}}
\put(1219.0,839.0){\rule[-0.200pt]{0.400pt}{4.818pt}}
\put(1282.0,82.0){\rule[-0.200pt]{0.400pt}{4.818pt}}
\put(1282,41){\makebox(0,0){ 35}}
\put(1282.0,839.0){\rule[-0.200pt]{0.400pt}{4.818pt}}
\put(1345.0,82.0){\rule[-0.200pt]{0.400pt}{4.818pt}}
\put(1345,41){\makebox(0,0){ 37}}
\put(1345.0,839.0){\rule[-0.200pt]{0.400pt}{4.818pt}}
\put(1408.0,82.0){\rule[-0.200pt]{0.400pt}{4.818pt}}
\put(1408,41){\makebox(0,0){ 39}}
\put(1408.0,839.0){\rule[-0.200pt]{0.400pt}{4.818pt}}
\put(211.0,82.0){\rule[-0.200pt]{0.400pt}{187.179pt}}
\put(211.0,82.0){\rule[-0.200pt]{295.825pt}{0.400pt}}
\put(1439.0,82.0){\rule[-0.200pt]{0.400pt}{187.179pt}}
\put(211.0,859.0){\rule[-0.200pt]{295.825pt}{0.400pt}}
\put(30,470){\makebox(0,0){\rotatebox{90}{用户贡献}}}
\put(1279,819){\makebox(0,0)[r]{仿真数据}}
\put(1299.0,819.0){\rule[-0.200pt]{24.090pt}{0.400pt}}
\put(211,423){\usebox{\plotpoint}}
\put(211,421.17){\rule{6.300pt}{0.400pt}}
\multiput(211.00,422.17)(17.924,-2.000){2}{\rule{3.150pt}{0.400pt}}
\multiput(242.00,419.94)(4.575,-0.468){5}{\rule{3.300pt}{0.113pt}}
\multiput(242.00,420.17)(25.151,-4.000){2}{\rule{1.650pt}{0.400pt}}
\multiput(305.00,415.94)(4.575,-0.468){5}{\rule{3.300pt}{0.113pt}}
\multiput(305.00,416.17)(25.151,-4.000){2}{\rule{1.650pt}{0.400pt}}
\multiput(337.00,413.60)(4.429,0.468){5}{\rule{3.200pt}{0.113pt}}
\multiput(337.00,412.17)(24.358,4.000){2}{\rule{1.600pt}{0.400pt}}
\put(274.0,417.0){\rule[-0.200pt]{7.468pt}{0.400pt}}
\multiput(400.00,415.93)(2.013,-0.488){13}{\rule{1.650pt}{0.117pt}}
\multiput(400.00,416.17)(27.575,-8.000){2}{\rule{0.825pt}{0.400pt}}
\multiput(431.00,407.94)(4.575,-0.468){5}{\rule{3.300pt}{0.113pt}}
\multiput(431.00,408.17)(25.151,-4.000){2}{\rule{1.650pt}{0.400pt}}
\multiput(463.00,403.93)(3.382,-0.477){7}{\rule{2.580pt}{0.115pt}}
\multiput(463.00,404.17)(25.645,-5.000){2}{\rule{1.290pt}{0.400pt}}
\multiput(494.00,400.59)(3.493,0.477){7}{\rule{2.660pt}{0.115pt}}
\multiput(494.00,399.17)(26.479,5.000){2}{\rule{1.330pt}{0.400pt}}
\multiput(526.00,403.93)(3.382,-0.477){7}{\rule{2.580pt}{0.115pt}}
\multiput(526.00,404.17)(25.645,-5.000){2}{\rule{1.290pt}{0.400pt}}
\multiput(557.00,398.93)(2.079,-0.488){13}{\rule{1.700pt}{0.117pt}}
\multiput(557.00,399.17)(28.472,-8.000){2}{\rule{0.850pt}{0.400pt}}
\multiput(589.00,390.93)(3.382,-0.477){7}{\rule{2.580pt}{0.115pt}}
\multiput(589.00,391.17)(25.645,-5.000){2}{\rule{1.290pt}{0.400pt}}
\multiput(620.00,385.94)(4.575,-0.468){5}{\rule{3.300pt}{0.113pt}}
\multiput(620.00,386.17)(25.151,-4.000){2}{\rule{1.650pt}{0.400pt}}
\multiput(652.00,381.93)(2.013,-0.488){13}{\rule{1.650pt}{0.117pt}}
\multiput(652.00,382.17)(27.575,-8.000){2}{\rule{0.825pt}{0.400pt}}
\multiput(683.00,373.95)(6.937,-0.447){3}{\rule{4.367pt}{0.108pt}}
\multiput(683.00,374.17)(22.937,-3.000){2}{\rule{2.183pt}{0.400pt}}
\multiput(715.00,370.93)(3.382,-0.477){7}{\rule{2.580pt}{0.115pt}}
\multiput(715.00,371.17)(25.645,-5.000){2}{\rule{1.290pt}{0.400pt}}
\multiput(746.00,365.93)(1.834,-0.489){15}{\rule{1.522pt}{0.118pt}}
\multiput(746.00,366.17)(28.841,-9.000){2}{\rule{0.761pt}{0.400pt}}
\multiput(778.00,358.60)(4.429,0.468){5}{\rule{3.200pt}{0.113pt}}
\multiput(778.00,357.17)(24.358,4.000){2}{\rule{1.600pt}{0.400pt}}
\multiput(809.00,360.92)(1.250,-0.493){23}{\rule{1.085pt}{0.119pt}}
\multiput(809.00,361.17)(29.749,-13.000){2}{\rule{0.542pt}{0.400pt}}
\multiput(841.00,349.60)(4.429,0.468){5}{\rule{3.200pt}{0.113pt}}
\multiput(841.00,348.17)(24.358,4.000){2}{\rule{1.600pt}{0.400pt}}
\multiput(872.00,351.92)(1.009,-0.494){29}{\rule{0.900pt}{0.119pt}}
\multiput(872.00,352.17)(30.132,-16.000){2}{\rule{0.450pt}{0.400pt}}
\put(904,335.67){\rule{7.468pt}{0.400pt}}
\multiput(904.00,336.17)(15.500,-1.000){2}{\rule{3.734pt}{0.400pt}}
\multiput(935.00,334.93)(2.841,-0.482){9}{\rule{2.233pt}{0.116pt}}
\multiput(935.00,335.17)(27.365,-6.000){2}{\rule{1.117pt}{0.400pt}}
\multiput(967.00,328.92)(1.590,-0.491){17}{\rule{1.340pt}{0.118pt}}
\multiput(967.00,329.17)(28.219,-10.000){2}{\rule{0.670pt}{0.400pt}}
\put(998,318.17){\rule{6.500pt}{0.400pt}}
\multiput(998.00,319.17)(18.509,-2.000){2}{\rule{3.250pt}{0.400pt}}
\multiput(1030.00,316.92)(0.977,-0.494){29}{\rule{0.875pt}{0.119pt}}
\multiput(1030.00,317.17)(29.184,-16.000){2}{\rule{0.438pt}{0.400pt}}
\multiput(1061.00,300.93)(2.079,-0.488){13}{\rule{1.700pt}{0.117pt}}
\multiput(1061.00,301.17)(28.472,-8.000){2}{\rule{0.850pt}{0.400pt}}
\multiput(1093.00,292.93)(2.013,-0.488){13}{\rule{1.650pt}{0.117pt}}
\multiput(1093.00,293.17)(27.575,-8.000){2}{\rule{0.825pt}{0.400pt}}
\put(1124,284.67){\rule{7.709pt}{0.400pt}}
\multiput(1124.00,285.17)(16.000,-1.000){2}{\rule{3.854pt}{0.400pt}}
\multiput(1156.00,283.92)(1.590,-0.491){17}{\rule{1.340pt}{0.118pt}}
\multiput(1156.00,284.17)(28.219,-10.000){2}{\rule{0.670pt}{0.400pt}}
\multiput(1187.00,273.92)(1.642,-0.491){17}{\rule{1.380pt}{0.118pt}}
\multiput(1187.00,274.17)(29.136,-10.000){2}{\rule{0.690pt}{0.400pt}}
\multiput(1219.00,263.95)(6.714,-0.447){3}{\rule{4.233pt}{0.108pt}}
\multiput(1219.00,264.17)(22.214,-3.000){2}{\rule{2.117pt}{0.400pt}}
\multiput(1250.00,260.92)(1.642,-0.491){17}{\rule{1.380pt}{0.118pt}}
\multiput(1250.00,261.17)(29.136,-10.000){2}{\rule{0.690pt}{0.400pt}}
\multiput(1282.00,250.92)(1.439,-0.492){19}{\rule{1.227pt}{0.118pt}}
\multiput(1282.00,251.17)(28.453,-11.000){2}{\rule{0.614pt}{0.400pt}}
\multiput(1313.00,239.92)(1.642,-0.491){17}{\rule{1.380pt}{0.118pt}}
\multiput(1313.00,240.17)(29.136,-10.000){2}{\rule{0.690pt}{0.400pt}}
\multiput(1345.00,229.93)(2.013,-0.488){13}{\rule{1.650pt}{0.117pt}}
\multiput(1345.00,230.17)(27.575,-8.000){2}{\rule{0.825pt}{0.400pt}}
\multiput(1376.00,221.93)(3.493,-0.477){7}{\rule{2.660pt}{0.115pt}}
\multiput(1376.00,222.17)(26.479,-5.000){2}{\rule{1.330pt}{0.400pt}}
\multiput(1408.00,216.92)(1.315,-0.492){21}{\rule{1.133pt}{0.119pt}}
\multiput(1408.00,217.17)(28.648,-12.000){2}{\rule{0.567pt}{0.400pt}}
\put(368.0,417.0){\rule[-0.200pt]{7.709pt}{0.400pt}}
\put(1279,768){\makebox(0,0)[r]{改进数据}}
\put(1299.0,768.0){\rule[-0.200pt]{24.090pt}{0.400pt}}
\put(211,526){\usebox{\plotpoint}}
\multiput(211.58,523.28)(0.497,-0.695){59}{\rule{0.120pt}{0.655pt}}
\multiput(210.17,524.64)(31.000,-41.641){2}{\rule{0.400pt}{0.327pt}}
\multiput(242.00,481.92)(0.697,-0.496){43}{\rule{0.657pt}{0.120pt}}
\multiput(242.00,482.17)(30.637,-23.000){2}{\rule{0.328pt}{0.400pt}}
\multiput(274.00,458.92)(0.919,-0.495){31}{\rule{0.829pt}{0.119pt}}
\multiput(274.00,459.17)(29.279,-17.000){2}{\rule{0.415pt}{0.400pt}}
\multiput(305.00,441.92)(1.486,-0.492){19}{\rule{1.264pt}{0.118pt}}
\multiput(305.00,442.17)(29.377,-11.000){2}{\rule{0.632pt}{0.400pt}}
\multiput(337.00,430.93)(1.776,-0.489){15}{\rule{1.478pt}{0.118pt}}
\multiput(337.00,431.17)(27.933,-9.000){2}{\rule{0.739pt}{0.400pt}}
\multiput(368.00,421.92)(0.847,-0.495){35}{\rule{0.774pt}{0.119pt}}
\multiput(368.00,422.17)(30.394,-19.000){2}{\rule{0.387pt}{0.400pt}}
\multiput(431.00,402.92)(1.158,-0.494){25}{\rule{1.014pt}{0.119pt}}
\multiput(431.00,403.17)(29.895,-14.000){2}{\rule{0.507pt}{0.400pt}}
\multiput(463.00,388.95)(6.714,-0.447){3}{\rule{4.233pt}{0.108pt}}
\multiput(463.00,389.17)(22.214,-3.000){2}{\rule{2.117pt}{0.400pt}}
\put(494,386.67){\rule{7.709pt}{0.400pt}}
\multiput(494.00,386.17)(16.000,1.000){2}{\rule{3.854pt}{0.400pt}}
\multiput(526.00,386.93)(1.776,-0.489){15}{\rule{1.478pt}{0.118pt}}
\multiput(526.00,387.17)(27.933,-9.000){2}{\rule{0.739pt}{0.400pt}}
\multiput(557.00,377.93)(2.079,-0.488){13}{\rule{1.700pt}{0.117pt}}
\multiput(557.00,378.17)(28.472,-8.000){2}{\rule{0.850pt}{0.400pt}}
\multiput(589.00,369.93)(3.382,-0.477){7}{\rule{2.580pt}{0.115pt}}
\multiput(589.00,370.17)(25.645,-5.000){2}{\rule{1.290pt}{0.400pt}}
\put(400.0,404.0){\rule[-0.200pt]{7.468pt}{0.400pt}}
\multiput(652.00,366.61)(6.714,0.447){3}{\rule{4.233pt}{0.108pt}}
\multiput(652.00,365.17)(22.214,3.000){2}{\rule{2.117pt}{0.400pt}}
\multiput(683.00,367.93)(1.834,-0.489){15}{\rule{1.522pt}{0.118pt}}
\multiput(683.00,368.17)(28.841,-9.000){2}{\rule{0.761pt}{0.400pt}}
\multiput(715.00,358.93)(3.382,-0.477){7}{\rule{2.580pt}{0.115pt}}
\multiput(715.00,359.17)(25.645,-5.000){2}{\rule{1.290pt}{0.400pt}}
\put(746,353.67){\rule{7.709pt}{0.400pt}}
\multiput(746.00,354.17)(16.000,-1.000){2}{\rule{3.854pt}{0.400pt}}
\multiput(778.00,352.93)(3.382,-0.477){7}{\rule{2.580pt}{0.115pt}}
\multiput(778.00,353.17)(25.645,-5.000){2}{\rule{1.290pt}{0.400pt}}
\put(809,348.67){\rule{7.709pt}{0.400pt}}
\multiput(809.00,348.17)(16.000,1.000){2}{\rule{3.854pt}{0.400pt}}
\multiput(841.00,350.60)(4.429,0.468){5}{\rule{3.200pt}{0.113pt}}
\multiput(841.00,349.17)(24.358,4.000){2}{\rule{1.600pt}{0.400pt}}
\multiput(872.00,352.92)(1.642,-0.491){17}{\rule{1.380pt}{0.118pt}}
\multiput(872.00,353.17)(29.136,-10.000){2}{\rule{0.690pt}{0.400pt}}
\multiput(904.00,344.59)(2.751,0.482){9}{\rule{2.167pt}{0.116pt}}
\multiput(904.00,343.17)(26.503,6.000){2}{\rule{1.083pt}{0.400pt}}
\multiput(935.00,348.93)(3.493,-0.477){7}{\rule{2.660pt}{0.115pt}}
\multiput(935.00,349.17)(26.479,-5.000){2}{\rule{1.330pt}{0.400pt}}
\multiput(967.00,343.94)(4.429,-0.468){5}{\rule{3.200pt}{0.113pt}}
\multiput(967.00,344.17)(24.358,-4.000){2}{\rule{1.600pt}{0.400pt}}
\multiput(998.00,341.60)(4.575,0.468){5}{\rule{3.300pt}{0.113pt}}
\multiput(998.00,340.17)(25.151,4.000){2}{\rule{1.650pt}{0.400pt}}
\multiput(1030.00,343.93)(2.751,-0.482){9}{\rule{2.167pt}{0.116pt}}
\multiput(1030.00,344.17)(26.503,-6.000){2}{\rule{1.083pt}{0.400pt}}
\multiput(1061.00,339.60)(4.575,0.468){5}{\rule{3.300pt}{0.113pt}}
\multiput(1061.00,338.17)(25.151,4.000){2}{\rule{1.650pt}{0.400pt}}
\multiput(1093.00,341.93)(2.323,-0.485){11}{\rule{1.871pt}{0.117pt}}
\multiput(1093.00,342.17)(27.116,-7.000){2}{\rule{0.936pt}{0.400pt}}
\multiput(1124.00,336.59)(3.493,0.477){7}{\rule{2.660pt}{0.115pt}}
\multiput(1124.00,335.17)(26.479,5.000){2}{\rule{1.330pt}{0.400pt}}
\multiput(1156.00,339.95)(6.714,-0.447){3}{\rule{4.233pt}{0.108pt}}
\multiput(1156.00,340.17)(22.214,-3.000){2}{\rule{2.117pt}{0.400pt}}
\put(1187,336.17){\rule{6.500pt}{0.400pt}}
\multiput(1187.00,337.17)(18.509,-2.000){2}{\rule{3.250pt}{0.400pt}}
\multiput(1219.00,336.59)(3.382,0.477){7}{\rule{2.580pt}{0.115pt}}
\multiput(1219.00,335.17)(25.645,5.000){2}{\rule{1.290pt}{0.400pt}}
\put(1250,339.17){\rule{6.500pt}{0.400pt}}
\multiput(1250.00,340.17)(18.509,-2.000){2}{\rule{3.250pt}{0.400pt}}
\multiput(1282.00,337.95)(6.714,-0.447){3}{\rule{4.233pt}{0.108pt}}
\multiput(1282.00,338.17)(22.214,-3.000){2}{\rule{2.117pt}{0.400pt}}
\put(1313,335.67){\rule{7.709pt}{0.400pt}}
\multiput(1313.00,335.17)(16.000,1.000){2}{\rule{3.854pt}{0.400pt}}
\put(1345,335.17){\rule{6.300pt}{0.400pt}}
\multiput(1345.00,336.17)(17.924,-2.000){2}{\rule{3.150pt}{0.400pt}}
\multiput(1376.00,333.94)(4.575,-0.468){5}{\rule{3.300pt}{0.113pt}}
\multiput(1376.00,334.17)(25.151,-4.000){2}{\rule{1.650pt}{0.400pt}}
\put(1408,329.67){\rule{7.468pt}{0.400pt}}
\multiput(1408.00,330.17)(15.500,-1.000){2}{\rule{3.734pt}{0.400pt}}
\put(211,526){\makebox(0,0){$+$}}
\put(242,483){\makebox(0,0){$+$}}
\put(274,460){\makebox(0,0){$+$}}
\put(305,443){\makebox(0,0){$+$}}
\put(337,432){\makebox(0,0){$+$}}
\put(368,423){\makebox(0,0){$+$}}
\put(400,404){\makebox(0,0){$+$}}
\put(431,404){\makebox(0,0){$+$}}
\put(463,390){\makebox(0,0){$+$}}
\put(494,387){\makebox(0,0){$+$}}
\put(526,388){\makebox(0,0){$+$}}
\put(557,379){\makebox(0,0){$+$}}
\put(589,371){\makebox(0,0){$+$}}
\put(620,366){\makebox(0,0){$+$}}
\put(652,366){\makebox(0,0){$+$}}
\put(683,369){\makebox(0,0){$+$}}
\put(715,360){\makebox(0,0){$+$}}
\put(746,355){\makebox(0,0){$+$}}
\put(778,354){\makebox(0,0){$+$}}
\put(809,349){\makebox(0,0){$+$}}
\put(841,350){\makebox(0,0){$+$}}
\put(872,354){\makebox(0,0){$+$}}
\put(904,344){\makebox(0,0){$+$}}
\put(935,350){\makebox(0,0){$+$}}
\put(967,345){\makebox(0,0){$+$}}
\put(998,341){\makebox(0,0){$+$}}
\put(1030,345){\makebox(0,0){$+$}}
\put(1061,339){\makebox(0,0){$+$}}
\put(1093,343){\makebox(0,0){$+$}}
\put(1124,336){\makebox(0,0){$+$}}
\put(1156,341){\makebox(0,0){$+$}}
\put(1187,338){\makebox(0,0){$+$}}
\put(1219,336){\makebox(0,0){$+$}}
\put(1250,341){\makebox(0,0){$+$}}
\put(1282,339){\makebox(0,0){$+$}}
\put(1313,336){\makebox(0,0){$+$}}
\put(1345,337){\makebox(0,0){$+$}}
\put(1376,335){\makebox(0,0){$+$}}
\put(1408,331){\makebox(0,0){$+$}}
\put(1439,330){\makebox(0,0){$+$}}
\put(1349,768){\makebox(0,0){$+$}}
\put(620.0,366.0){\rule[-0.200pt]{7.709pt}{0.400pt}}
\put(211.0,82.0){\rule[-0.200pt]{0.400pt}{187.179pt}}
\put(211.0,82.0){\rule[-0.200pt]{295.825pt}{0.400pt}}
\put(1439.0,82.0){\rule[-0.200pt]{0.400pt}{187.179pt}}
\put(211.0,859.0){\rule[-0.200pt]{295.825pt}{0.400pt}}
\end{picture}
}
   \end{frame}

   \begin{frame}{主要创新之处}
\small
     \begin{enumerate}
\item 虚拟实践社区中的协同行为研究。通过深入分析虚拟实践社区中的用户行
  为,明确知识协同行为的定义和特点,并确定协同行为的度量方法,将协同行
  为进行量化。
\item 虚拟实践社区中的用户分类研究。虚拟实践社区中存在着不同类型的用
  户,这些用户间的协同行为和模式各不相同。利用量化的用户贡献度和用户参
  与水平两个维度,将不同类型的用户区分开。
\item 知识协同的动机因素模型研究。协同活动的特点决定了协同的动机不仅仅
包括个人动机,也包括人际动机。两种动机的共同作用影响了人的实际行为。这
一部分研究将这两类动机融入到动机模型中,建立适合分析协同活动的动机模型。

\item 模型仿真与结果分析。使用实际数据验证模型的有效性,并利用模型分析
  动机因素是如何影响用户行为的。在此基础上,提出相应的管理建议,促进社
  区提升管理水平,达到良性发展的目的。

\end{enumerate}
   \end{frame}
\end{document}
%%% Local Variables: 
%%% mode: latex
%%% TeX-master: t
%%% End: 
