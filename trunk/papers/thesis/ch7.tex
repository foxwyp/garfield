
\chapter*{结论}

研究用户参与虚拟实践社区的动机对于理解用户在社区中的行为具有重要的意义。
特别是在用户没有明显回报的情况下,自身动机水平的高低直接影响了用户参与
社区的程度,最终决定了社区能够延续并进一步发展。

本研究利用已有的动机理论和虚拟实践社区相关理论,以中文维基百科社区为
例,深入研究了用户参与虚拟
实践社区知识协同活动的动机,揭示了动机与协同行为的关系。研究取得的主要
成果如下:

\begin{enumerate}
\item 提出了衡量用户贡献度的方法。
用户的协同动机会转化为协同行为,行为在一定程度上反映了动机的水平。因
此,分析用户的协同行为并对其量化,使得不同用户的行为能有一个共同的基础
进行比较,对于研究用户的动机是十分必要的。本研究提出了一种基于文本相似
度的协同贡献计算方法,根据用户所贡献内容是否能在最终版本里得以保留为标
准,将每次协同的结果加以量化。量化的结果充分反映了用户的各种协同活动所
产生的结果,同时又具有较高的稳健性,算法没有明显的疏漏。
\item 以研究用户的动机为目的对用户进行了分类,并详细分析了每一类用户的
  协同行为特点,总结出两种协同模式。不同的用户其参与社区知识协同的动机
  水平不同,主导的动机因素也不同,这就导致其最终的行为具有不同的行为模
  式。将社区的用户进行分类既有利于深入地理解每种用户的动机于行为间的关
  系,也有利于针对不同用户的特点实施管理措施。本研究从用户的贡献度水平
  和参与的广泛程度两个维度将用户分为五种类型,并利用社会网络分析方法研
  究了其行为特点。
\item 分析了虚拟实践社区中影响用户协同行为的动机因素,在此基础上建立了
  动机与行为的系统动力学模型。本文将用户的动机分为两种类型:个体动机与
  人际动机。这两种动机起作用的方式是不同的。另外,本文还进一步研究了行
  为对于动机的影响以及动机因素间的相互作用。根据不同用户的协同特点,本
  文提出了两个动机因素模型,分别用户分析不同类型的用户。利用因果关系模
  型,完整地刻画了各个因素变量与协同行为间复杂的、多反馈、非线性的关系。
\item 使用维基百科提供的数据,对动机因素模型进行了模拟仿真。仿真的结果
  表明模型是有效的、可以用来解释用户在社区中的行为特征。利用仿真模型,
  本文还进一步模拟了不同条件下用户协同行为水平的变化,并分析了产生这些
  变化的原因。在此基础上进一步提出了管理实务建议供社区的管理者参考。
\end{enumerate}

文本的主要研究目的是分析虚拟实践社区中知识协同的动机因素,尽管之前有过
类似的研究,由其是动机理论方面已经非常成熟,但是由于虚拟实践社区是近些
年来随着网络技术的快速发展而产生的新生事物,因此本文的研究成果不可避免
地带有一定的局限性,还不能完整地解决虚拟实践社区中的现实问题。因此本研
究只是针对特定问题所进行的初步探索,还存在许多不足之处,有待在下一步的
研究中继续深入分析和加以修正。未来的主要研究方向包括以下几点:
\begin{enumerate}
\item 将用户的讨论内容纳入到研究范围。用户的讨论是内容协同编辑的重要补
  充。根据维基百科的统计,用户参与到讨论中的比重逐年增加,讨论的重要性
  正在逐渐体现出来。尽管本文没有将讨论纳入到研究中来并给出了实际的理
  由,但是对于讨论的分析有助于更深入地理解协同用户的行为方式,从而使得
  动机研究的结果更准确、更完善。
\item 本文研究的对象是用户参与知识协同的动机。采取的方式是分析维基百科
  的编辑数据,这和传统的基于问卷调查的方式分析动机因素有较大区别。采用
  实际数据的方式分析动机是问卷调查的有力补充、但却不是替代方法。只有将
  二者的优势集合起来才能最大限度地发挥作用。尽管本文提出的动机模型也参
  考了其他研究中问卷调查的结果,但是如果能根据本研究的目的进行一次有针
  对性的问卷调查,厘清用户的主观倾向和意愿,同时结合数据进行分析应会取
  得更好的结果。
\item 动机模型仍有改进空间。本文采用系统动力学方法建立动机模型。尽管模
  型较好地反映了行为变化的趋势,但是仍然有许多行为变化模型没有反应出来。
  这说明模型可能忽略了某型变量,或者变量间的关系还比较模糊,定义比较粗
  糙。另外系统动力学模型的特点决定了参数的取值不会非常精确,这些问题都
  需要在将来的研究中解决,进一步分解、细化模型,改进模型的不足。
\item 为管理建议寻找实际数据支持。管理建议毕竟只是构建在模型仿真的基础
  上提出的,其真实性和有效性还有待验证。由于现实情况千差万别,需要有实际数据进一步支持所提出的各种
  建议。这样既可以进一步验证模型的有效性,也可以进一步加深对用户行为的
  理解,为进一步深入研究打下基础。
\end{enumerate}
%%% Local Variables: 
%%% mode: latex
%%% TeX-master: t
%%% End: 
