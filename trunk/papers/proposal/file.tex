\input{ctex4xetex.cfg}
\documentclass[12pt,a4paper]{ctexart}
\usepackage{setspace}
\usepackage{fontspec}
\usepackage[CJKaddspaces]{xeCJK}
\usepackage{amsmath}
%\usepackage[numbers,sort\&compress]{natbib}
    %    \setlength{\bibsep}{0.5ex}  % vertical spacing between references
\setmainfont{Times New Roman}
\setCJKmainfont{Adobe Song Std}
\setCJKfamilyfont{hei}{Adobe Heiti Std}
\setCJKfamilyfont{song}{Adobe Song Std}

\CTEXsetup[format={\CJKfamily{hei}\zihao{3}\centering}]{chapter}
\CTEXsetup[format+={\CJKfamily{hei}\zihao{4}\flushleft}]{section}
\CTEXsetup[format+={\CJKfamily{hei}\zihao{-4}\flushleft}]{subsection}


\usepackage{geometry}
\usepackage[dvips]{xcolor}
\geometry{vmargin={25mm},hmargin={30mm,20mm}}


\newcommand{\chuhao}{\fontsize{42pt}{\baselinestrench}\selectfont} % 字号设置
\newcommand{\xiaochuhao}{\fontsize{36pt}{\baselineskip}\selectfont} % 字号设置
\newcommand{\yihao}{\fontsize{28pt}{\baselineskip}\selectfont} % 字号设置
\newcommand{\erhao}{\fontsize{21pt}{\baselineskip}\selectfont} % 字号设置
\newcommand{\xiaoerhao}{\fontsize{18pt}{\baselineskip}\selectfont} % 字号设置
\newcommand{\sanhao}{\fontsize{15.75pt}{\baselineskip}\selectfont} % 字号设置
\newcommand{\sihao}{\fontsize{14pt}{\baselineskip}\selectfont} % 字号设置
\newcommand{\xiaosihao}{\fontsize{12pt}{\baselineskip}\selectfont} % 字号设置
\newcommand{\wuhao}{\fontsize{10.5pt}{\baselineskip}\selectfont} % 字号设置
\newcommand{\xiaowuhao}{\fontsize{9pt}{\baselineskip}\selectfont} % 字号设置
\newcommand{\liuhao}{\fontsize{7.875pt}{\baselineskip}\selectfont} % 字号设置
\newcommand{\qihao}{\fontsize{5.25pt}{\baselineskip}\selectfont} % 字号设置


\begin{document}
% fisrt page
  \title{虚拟实践社区中知识共享的内源性动机因
      素研究 }
  \author {吴云鹏}
   \maketitle
% directory
 \newpage
  \tableofcontents
% text  



\newpage
  \section{论文选题依据}

  \subsection{研究背景}
人类的实践活动始终伴随着各类知识的创造,运用,丰富与发展,人们认为,占
有更多的知识能够提高生产的效率,创造更大的价值。如今,越来越多的企业认
识到,要想获得持续成功就必须管理好组织中的各种知识。正如温特
\cite{Williamson1994}指出的:“企业就是知道如何做事的组织。”不论企业是
生产产品还是提供服务或者二者兼有,都依赖于企业知道“如何做事”,也就是企
业中的知识。在当今知识经济的时代,知识已经成为最重要的经济资源,人们对
知识的追求与竞争正变得愈发激烈起来。对于企业来说,对知识的追求就是对超
额利润的追求。随着社会分工和社会化大生产变得更加精细与完善,传统的生产
资源已经不能再为企业贡献超额利润了,知识几乎成为超额利润的唯一来源。企业来要想获得长
久稳定的竞争优势,就必须在知识的占有和管理上取得领先。

尽管知识总是被不自觉地同信息和数据混淆,但是人们普遍认为知识要比信
息、数据具有更广泛更深刻的内涵和更丰富的内容。企业中的知识构成非常复
杂,它是一种混合
体,包括成体系的经验、企业价值、与环境相关的信息和专家的洞察力等等。企
业中的知识不仅存在于文档和数据库中,也存在于企业每日的例行事务,企业过
程和标准规范中。随着劳动分工的日益精细,导致了知识分工的出现。一方面,
人们在越来越窄的范围内知道的越来越多;另一方面,也随之出现了越来越多的
“知识岛屿”,使得如何进行知识沟通成为管理者面临的重大问题。企业中知识复
杂性为知识管理知识带来了巨大的困难,现有的管理理论和技术手段已经不能满足企
业对管理知识的巨大需求,因此,研究在互联网条件下如何进行有效的知识沟通
就成为知识管理研究者紧迫的任务。

知识管理的主要研究领域目前主要还集中在知识管理的“前端”,即知识获取的方
法和技术上。知识抽取、知识表示、知识存贮等方面是知识管理研究最多的方
面,尽管这些领域很重要,研究也取得了丰硕的成果,但是研究表明,对于企业
来说,进行有效的知识管理仍然是困难的。一份对欧洲和北美的研究报告显示,只有13\%的经理认为
知识在企业各组织单元中有效地进行传递,大部分的知识管理都是不尽如人意的\cite{Ruggles1998}。知识管理所面临的困境,根本原因就在于没能更好地理
解人们是如何共享知识的。正如
Sveiby所认为的:“如果我们承认知识是人的一部分,知识管理的目的就是组织
如何最好地培育、利用和激励人们改进和分享他们的行动能
力”\cite{Seviby1997}。因此,研究组织中知识共享的模式和动机因素不论是对理论还是现
实均具有重要意义。

\subsection{选题依据}
当前,对于组织中的知识共享,大部分还集中在对知识共享的模式和共享能力的影响因素方
面。而对于动机因素的研究,也就是人们为什么共享知识,共享的动机有哪些这
类问题的研究较少。尽管在心理学和其他行为科学等学科中,动机因素的研究已
经进行了相当长的时间,但是对知识共享领域的动机因素研究还属于初级阶段。

知识共享可以视为一个“将一组复杂的,有时是含混的惯例重建并延续于另一个
新的环境的过程“\cite{szulanski2000pkt},或者说一个“一个个体受到另一个
个体的经验影响的过程”\cite{Argote2000}。有以上两个定义可以看到,知识共
享的核心在于它是共享双方的有效交互。不论是“重建并延续”还是“受到经验的
影响”,都反映了知识的接受一方所做出的相应。因此,Davenport和Prusak给出
了一个更为简洁的定义:“知识共享=传递+吸收”\cite{davenport1998wko}。

尽管对知识共享的定义准确地描述了一个动态的共享过程,但是当我们涉及到研
究知识共享的动机因素的时候却陷入了矛盾中。首先,不是所有的分享行为都可
以视为知识共享。因为共享需要知识的接收端正向反应:“吸收”,也就是说,接
收端需要理解知识,并且成功地融入自身的知识体系。对于那些有分享的行为,
却没有达到共享的效果的过程,尽管知识的发送端有明确的目的,但是却无法划
归到知识共享的范畴内,其动机因素因此也就不在研究范围内了。另一方面,由
于知识本身的特性,隐性知识的共享往往是在无意识的情况下,潜移默化地为接
收端所吸收的。达到知识共享的结果可能只是其他行为的副产品。Sveiby指出
“问题在于共享不是有意的或者是通过某种正式的渠道进行的$\cdots$ 知识
总是以一种最经济的、无意识无目的的方式一代一代传承下去的
\cite{sveiby1996tka}。

实际上,在以上知识共享的定义中,共享的过程可以大致分为两个部分,即知识
源传递知识和接收端接受知识。我们真正感兴趣的动机因素应该是知识源的动
机,也就是人们为什么愿意向他人分享,传递自己的知识。因此,对于动机因素
的研究应该聚焦在知识源的动作---传递上。本文对知识传递将重新进行定义:
知识传递表示知识源使得某件知识可以为接收端可认而且可得,如果知识仅仅是
可以获得,但是获得的是一种不可认的形式,不能称之为知识传递。只有两个条件同时满足,才
可以成为知识传递。

动机因素可以分为两种:外源性动机和内源性动机。外源性动机是对与他人而言
可以接触和观察到的,通过和外界进行物质、能量和信息的交换而获得的。外源
性动机是由其他人或机构进行分配,包括工资、福利和晋升。同时也包括避免惩罚的驱动力。外在的奖励通常是以一致性为基础,即外在的激励因子需要与绩效的提高保持一致。外源性激励因素经常被用于鼓励员工达到更高的绩效水平或者新的目标。但是外源性动机不能完全解释员工个体所进行的每一项努力。
 内源性动机是自内部产生的。换句话说,是这些激励因子将个体与任务或者工
 作本身联系起来的。内源性回报包括责任感、成功感、成就感,他们是通过经
 验习得的。还包括挑战感或竞争感。他们涉及某些吸引人的任务或目标。长久
 以来,完成有意义的工作总是与内在动机相联系的\cite{Luthans2001}。

对于外源性因素来说,尽管通过复杂的薪酬制度设计可以较长时间地提高员工的
工作动机,但是很多学者认为通过激励制度不能带来长期、稳定、持续的动机。“在过去三十年里至少两打以上的研究论文已经得出的结论表明,希望因为完成一项任务或者成功地完成该项任务而得到奖赏的人就是没有根本不想获取报酬的人干得好“\cite{93120316461993}。
另一方面,一些组织是在几乎完全没有外源性
因素的作用下持续运作的,仅仅依靠诸如热情、信仰之类的内源性因素就可以不
断发展壮大下去。因此,本文的研究将针对内源性因素开展,具体研究有哪些内
源性因素会影响组织中的知识传递以及各个因素之间有哪些联系。

外源性动机与内源性动机并不是完全独立的,而是常常交织在一起并相互作用的。
外源性动机即可以强化内源性动机,也可以削弱内源性动机,有时候二者之间还可
能互不影响\cite{deci1971eem}。为此,为了研究内源性因素,希望能够选择一
个比较“纯净”的研究对象,该对象应该较少或者根本不受外源性因素的形象。同
时,内源性因素往往和心理上的矛盾、冲突相联系,因此如果选取在时空、组织、
和背景上差异很大的虚拟组织应该比实体组织具有更大的研究意义,得到更明显
的研究结果。

目前,对于虚拟社区的知识共享主要集中与两类:开源社区(类似于Apache)和
内容协同社区(例如维基百科)。开源社区是针对创建软件代码而展开的共享协
同,内容协同社区则以创建开放的内容而进行知识共享。对于开源社区中知识共享动机因素的研究要
比对内容协同社区的研究丰富的多,研究结果也要复杂的多。有学者的研究结果
显示外源性动机在开源社区中占主导地位\cite{10.1109/HICSS.2001.927045},
另一些学者研究发现内源性因素起重要作用\cite{Lakhani2003}。Xiaoquan等认
为开源社区不论是从软件的功能性和复杂性还是社区的规模、力度上都迥异不同。
不同的开源社区采用不同的协作方式,不同的开源许可,导致了很难通过对一小
部分社区的研究而获得足够丰富的信息以体现这些差异\cite{Zhang2006}。因
此,本文将以内容协同社区为研究对象,具体深入地研究内源性动机因素对知识
传递的影响。


\section{国内外研究现状}
目前,从事知识管理的学者对知识共享进行了大量的研究;同时,动机因
素研究在心理学和行为科学领域也取得了丰富的理论成果。而伴随着虚拟实践社区的兴
起,知识协同行为随之大量涌现,群体行为如何影响人的共享动机成为了新的。本文拟从这两部分总
结并分析国内外的研究现状。



\section{实践社区与知识协同}


实践社区的概念最早由Lave和Wenger提出\cite{lave1991sll}。他们分析了组织
中成员共享知识的行为,认为“情景学习”是知识传递的有效方式。用过这种方
式,知识在员工中(尤其是从老员工到新员工)共享和传递,各种新的想法和解
决方法不断涌向出来。另一方面,知识管理实践在组织中遇到了困难。组织管理者一直试图能找到一个
有效的途径使得员工的经验可以共享,即使在员工离开组织他们的宝贵技能也能
留在组织中。然而以将知识编码化作为共享的手段似乎并不能真正解决如何进行
有效的知识共享的问题。Huysman等分析了其中的原因:人们很难显性化地表述
他们的经验和见识\cite{huysman2002ksp}。一旦脱离了相应的环境,知识很难
通过常用的表达手段展现出来。因此,组织成员更愿意将自己的思想经验等记录
在本地环境中,而不是共享给其他人。由此代来的结果就是人们不愿意向企业中的知识管理系
统贡献内容。Kwan等将这种情况归结为知识管理系统本身为员工带来了额外的负
担,因为他们需要将隐藏在实践中的知识“文档化”;而且不同的人关注知识的角
度不同,缺乏同一的知识结构的表述方式也是重要的原因\cite{MillieKwan2003}。实践社区实际上提供了一个知识共享的基本环境,在这个
环境下成员通过实践理解、体验新的知识,从而更好地实现知识的共享。这种松
散的组织结构增强了组织性,又保存了知识共享特有的个人性。Kogut认为实践社区的作用在于连接了个人与组织间
的隐性知识共享和传递过程\cite{443473219920801},从而当社区成员通过不断
的知识创造实践活动改进自身能力的同时,组织也获得了组织学习和持续创新的
能力\cite{wenger1999cpl}。Millen等从三个层次总结了实践社区的好处:对个
人来讲,个人可以通过实践活动接触到领域专家和信息资源;对社区来说,可以
促进新想法的诞生、改进知识的质量、提供公共的认知基础等;对整个组织来
讲,通过实践社区的活动获得了商业价值\cite{millen2002uba}。这些研究表明,实践
社区在知识创造、
知识交流、知识共享和促进参与者学习方面具有不可低估的作用,被认为是组织内部和组织之间
支持知识共享和知识创新的一种特别有效的组织形式,是正式的组织机构有益的补充和延伸。



在实践社区中,社区成员分享他们的兴趣以及特定主题的问题,针对某一话题
展开交互从而获得知识和专业技能\cite{Wenger2002}。新加入的成
员通过参与社区中相关的实践活动从而获得新的知识,随着时间的推移,新成员
从社区的边缘人物变为社区的积极参与者。Wenger等认为实践社区
不同于组织内部其他形式的内部小组,例如项目小组、运作小组等。首先,社区
成员没有一个被赋予的组织角色,也不会有相应的任务分工;第二,成员在组织
中的角色也并不象在项目小组中那样明确。衡量实践社区成功与否不在于它完成
了多少既定目标,而在于社区中产生了多少可以改进组织绩效的实践行为。即使
一个实践社区达到了先前规划的目标,也不会象项目组一样随即停止。第三:社
区没有明确的约定成员应该履行何种义务,担负什么样的责任。最后实践社区的
主题非常明确,种类也很单一,致力于成员对知识“知其所以然”。
Wenger 有进一步归纳出实践社区三个基本特征:领域、社区以及实践
\cite{wenger2004kmd}。领域知识凝聚了社区成员,并且赋予社区独特的标志。
成员关注本领域的关键问题,寻找解决问题的办法。实践社区总是关注某一主题
的,这是她和个人网络的本质区别。社区的特征既不是由任务或者团队决定
的,而是由成员探索与创造的领域知识决定的。社区由与领域相关的一组成员组
成。成员之间开展交互并建立联系,以共同解决问题并分享知识。实践是指社区
成员创造、分享知识、方法、工具、案例、文档等活动。成员从实践活动中改进
和提高自己的能力。这三种特性使得实践社区可以有效地管理知识。时间社区实
际上解决了两个关键问题:如何寻找相关的知识以及如何寻找知识的交互对象\cite{Yang2008}。

实践社区对组织可以产生积极的影响。它一方面为组织中知识共享和创造提供了良好的环境,从而提高组织中各类实践
的效果\cite{lesser2001ucp},同时还提供了一种良好的知识重用的机
制,可以降低新进员工的学习曲线\cite{dube2005isc}。一些研究已经证明实践
社区是一种有效的解决非结构化为题的工具\cite{vonwartburg2004csa}。实践
社区的成功之处就在于:移除了个人参与实践活动的壁垒;支持组织成员在特
定环境下交互自己独特的知识并将这种独特性与社区的目标结合起来
\cite{ardichvili2003mab}。实践社区将组织的结构重新进行了编排,不再是从
组织功能的角度,而是从增强组织的核心竞争力方面进行重组\cite{Pan2003}。
因此从传统的角度来看实践社区是一个没有组织结构的组织形式\cite{lesser2001cpa}。
Lin等人从组织结构的角度分析了实践社区,认为知识地图、社会网络和助记功
能是实践社区的关键部分\cite{lin2001cmv}。知识地图描述了知识对象之间的关
系,社会网络则够成了成员之间的关系,助记功能则包括知识分配、社会网络更
新、知识维护以及协同知识获取。这是这三者提升了组织学习的能力。从总体上看, 实
践社区中的创新过程是在实践社区自我组织、自我管理、自我调节的过
程中进行的。通过整合组织的内外部知识资源, 使组织学习、利
用和创造知识的整体效益大于各独立组成部分总和
的效应。实践社区的两大功能正是社会化的知识构建并为其提供创新环境。

Chu等人指出:虚拟社区中蕴藏的知识资产可以改变人的行为,反过来使组织从
中获利\cite{Chu2008}。当实践社区更倾向于共享显性知识时,组织的实现会侧
重于知识重用,强调知识的存储和访问。通过对知识加以整理和分类,成员共享
的效率会得到改进,工作效率随之提高。当实践社区偏向于隐形知识的共享时,
社区的任务就是提供适合群体学习的环境,推动成员于专家的交流和互动。这种
类型的社区会促进组织创新能力的提高。他们还进一步提出了实践社区对企业战
略的影响:1)可以提升创新学习的能力。跨领域的知识学习和共享有利于提升
知识的创新能力,建立共同知识。2)提高反应能力。因为社区的主题明确,因
此专业领域的专家会非常容易地加入到社区中,并且给出面向问题的解决方法。
3)增强核心竞争力。组织成员在社区中提高了知识水平,同时专家的经验也得
以保存和传承。4)增强工作效率。通过知识重用,减少重复工作使员工的效率
得到提高。另外一个实践社区的成员可能同时还属于其他社区,在实践中他会将
其他组织的引入到本社区中来,在客观上实现了组织间的知识贡献促进了组织整
体效能的提升\cite{Garrety2004}。

实践社区社区中主要实践是知识的共享。但是实践社区的本意并不在于如何更好
地促进知识共享,而是如何利用现有的知识创造出新的知识,也就是知识协同。
管理学领域的“协同”概念最早由Ansoff, H .I
( 1965) 提出。他认为, 企业组织的协同可使整体价值大
于各部分价值之合。\cite{Ansoff1965}。Karlenzig将知识协同定义为: “它是一种组
织战略方法, 可以动态集结内部和外部系统、商业过程、
技术和关系( 社区、客户、伙伴、供应商) , 以最大化商业绩
效\cite{karlenzig2002tip}。研究表明知识协同可以促进资源的交换与融合,
从而提升组织的竞争优势\cite{ghoshal1998sci}\cite{galunic1998rrf}。Anklam 指出, 知识协同就是知识管理的
协同化发展阶段, 他将知识管理的发展划分为三个主要
阶段: 第一阶段是以数据/信息传递为主要标志; 第二个
发展阶段是以知识共享/隐性知识管理为主要标志; 第三
个发展阶段是以知识协同为主要标志。并指出在第三个
发展阶段, 大多数公司是以协同/协作、共享、合作创新为
主题, 通过实践社区、学习社区, 兴趣社区、目的社区等进
行知识的协同和交互\cite{anklam2002kmc}。可见,知识协同虽然包括大量的知识共
享活动,但目的是在更高层次上实现知识创造\cite{Heiman2004}。Fischer等认
为人的创新能
力来自于同他人的交互和协同\cite{Fischer2005}。而实践社区中的实践行为正是社区成员通过对其拥有知识的共享与交流,共同实践某一知识型任
务的过程。知识协同具有以下特征:① 实质上是一种知识开发活动;② 在合作协议或合作结果中这些活动是“可
见的”,例如发明专利或科学论文;③ 有多方参与其中\cite{mckelvey2003dcl}。
社区中某一个知识请求者首先认识到自己没有能力解决
某个问题,而另一个知识提供者恰好有这方面的能力,如果双方能够达成共识,则可以整合双方的
知识,以弥补知识请求者的知识需求,最终解决问题\cite{Leijen2002}。通过“协同”的方式进行知
识创新, 能够弥补知识缺口, 有效的解决知识情景嵌入和
路径依赖的问题, 消除“知识孤岛”, 并可获得多主体、多
目标、多任务间的“1+1>2”的知识协同效应\cite{fanzhiping2007}。而实践社区
则为知识协同提供了良好的支持环境。




随着信息技术的发展,实践社区已经不再局限于本地的成员了。异地的、不同背
景的成员不断加入到实践社区中,使得实践社区呈现出越来越多的虚拟性
\cite{preece2004eea}。Dube认为实践社区的组织形式应该是与组织的管理需求相关
的\cite{dube2006ttv}。当虚拟组织的好处被更多的管理着所认同,实践社区想
虚拟组织转化的趋势也就愈发明显了。  然而组织的虚拟性却为知识的共享带
来了阻力。ZIGU RS认为团队分散的维度越多, 团队的虚拟性就越大。她提出了4 个影
响虚拟性的维度:地理分散、时间分散、组织分散和文化分散\cite{ZIGURS2003}。
组织分散意味着成员加入和退出非常自由,而组织的建立和消亡也非常迅速
\cite{655269}。文化分散代表这虚拟团队成员的多样性
\cite{huangyouliandliutuanjie}。这些分离性影响了人们共享知识的动机。Davenport将这些问题归因为知识的本地性:人们总是
从身边的人获取知识,而这些人往往是他们信任的人。当人们进行非面对面的交流
时,这种信任往往消失掉,从而减弱了知识的转移\cite{davenport1998wko}。
信任问题对知识共享至关重要。从个人角度说,实质共享的障碍源自缺乏沟通的
技巧和成员间的社会关系、文化背景上的差异、以及缺乏沟通的时间与相互信任。
从组织的角度讲,障碍通常来自缺乏必要的基础设施和资源
\cite{riege2005tdk}。而社区的分散性实际增强了这些负面因素的影响。
Sole等人也认为,组织的分散会带来解释性的障碍,也就是说人么丧失了交流的
共同基础\cite{sole2000bkg}。尽管现代科技可以在一定程度上弥补这种缺陷,
通过各种通信媒体,并结合面对面的沟通,能够构建工作实践的情景\cite{robey2000slc},
但是消除时空距离带来的摩擦需要靠技术于社会虚拟组织中的各个社会单元共同
作用\cite{sarker2004isa}。为了应对组织的分散性,需要组织建立一个系统的
共享参考框架和一种互信机制。Sarker等\cite{sarker2000uvt}对虚拟社区的发展最早进行了比较充分和细致的
研究分析。他们调差了12个由美国和加拿大学生组成的信息系统开发的虚拟团
队,研究分析后认为:不同领域背景和专业水平的组织成员要想有效地开展协
作,仅仅通过广泛地沟通是不够的。学习过程的发生不但需要信息的交互,更重
要的是将信息置于每日工作的实践环境中,将结果及时反馈,才能是接收方有效
掌握相关知识。理解团队的知识协同必须同时理解团队的结构、沟通的模式和业
务形态三个方面。这些因素应该同沟通方式匹配且均是有效的,才能保证知识协
同的有效性。Sackmann等认为文化背景的差异会在三个层次上影响知识共享:来自感情的影响、
认知影响和来自经验的影响\cite{sackmann2007eci}。其中来自情感的影响最为
显著。而互信则够成了共同认识的基础。通过情感的交流与相互信任,可以较好
地克服文化差异。Lin从四个方面讨论了文化分散对知识共享的影响:员工之间
的和谐、组织的义务、任务独立性以及参与决策的程度\cite{lin2007son}。和
谐代表了组织成员的相似性。人们总是试图寻找与自己的想法一致的观点并最大
化一致的部分\cite{vianen2000pof},因此成员之间的相似程度越高,知识共享
就越容易进行。Kane也证明了一个群体对于外来成员的接纳程度取决于该成员与
群体间的共同度又多大,群体总是愿意吸收那些拥有共同社会认知的成员\cite{Kanea2005}。组织义务体现了成员对工作环境的认同程度
\cite{testa2001ocj},对组织越是依赖,越是有认同,那么员工分享知识的意
愿也就越强烈。尽管这两个因素都可以对知识共享起到积极作用,但是对于不同
文化背景的社区,其作用可能是不同的。对于那些共享意识淡薄的社区,可能会
起到明显的作用,因为他们之前可能较少关注知识贡献的作用
\cite{witt2001iep}。而对那些共享意识比较浓厚的社区,成员更关注的是共享
哪些内容和如何共享,因此他们共享的行为不会有明显的改变\cite{eisenberger2001rpo}。






\section{工作动机理论}
工作动机理论一直是心理学和组织行为学
领域的一个重要主题。根据不同的研究侧重,可以分为三个流派:内容理论、过
程理论和当代理论。

\subsection{内容理论}
内容理论试图找出工作中的激励因素,从而解释工作动机与激励因素之间的关系。
马斯洛的需求层次理论是研究组织激励时应用得最广泛的理论。
马斯洛理论把需求分成生理需求、安全需求、社交需求、尊重需求和自我实现需
求五类,依次由较低层次到较高层次。需求层次理论既是解释人格的理论,也是
解释动机的重要理论。马斯洛认为动机是由多种不同层次与性质的需求组成的,
每个层次的需求与满足的程度,将决定个体的人格的发展境界
\cite{Maslow1943}。

Herzberg对马斯洛的理论进行了拓展,建立了一个更为具
体的关于工作动机的内容理论。Herzberg通过实证研究发现:属于工作本身或
者工作内容方面的因素(例如挑战性的工作、认可、责任等)可以满足工人对于成
就、地位、自我认知等方面的需求,从而使工人感到满意;而缺乏这些因素并不
会导致公认的不满。另一方面,工人的不满主要来自于与工作环境或是工作关系
方面的因素,诸如(地位、工作安全感、薪水、福利等)。
Herzberg将前者成为
激励因素,后者称为保健因素。如果管理者希望提高工人的满意度,就应该关注
于激励因素\cite{hertzberg1959mw}。Alferder同样扩展了马斯洛的理论。他确
立了三种核心的需要类型:存在、关系和成长。同马斯洛和Herzberg不同的是,
他并不认为只有在低层次的需求满足之后才可以进一步满足高层次的需求,也不
认可缺失是激发需求的唯一途径\cite{alderfer:era}。

通过对人的需求和动机进行研究,McClelland提出了成就需求理论。McClelland
认为,具有强烈的成就需求的人渴望将事情做得更为完美,提高工作效率,获得
更大的成功,他们追求的是在争取成功的过程中克服困难、解决难题、努力奋斗
的乐趣,以及成功之后的个人的成就感,他们并不看重成功所带来的物质奖励。
个体的成就需求与他们所处的经济、文化、社会、政府的发展程度有关,社会风
气也制约着人们的成就需求。而金钱刺激对高成就需求者的影响很复杂,他们一般总以自己的最高效率工作,所以金钱固然是成就和能力的鲜明标志,但是由于他们觉得这配不上他们的贡献,所以可能引起不满\cite{mcclelland1976am}。

斯金纳的强化理论认为, 人的行为是由外部因素控制的,
控制行为的因素称为强化物。强化物是在行为结果之后紧接
着的一个反应, 它提高了该行为重复的可能性。按照斯金纳的
观点, 当人们因采取某种理想行为而受到奖励时, 他们最有可
能重复这种行为。当这种奖励紧跟在理想行为之后, 则奖励最
为有效; 当某种行为没有受到奖励或者是受到惩罚时, 其重复
的可能性则非常小。因此, 强化理论者们认为行为是其结果的
函数\cite{skinner1938boe}\cite{ferster1957sr}。强化理论的致命弱点是在于它忽视了诸如目标、期望、
需要等个体因素, 而仅仅注重当人们采取行为时会带来什么
样的后果。

强化理论在Bundura的研究中进一步发展,使得强化理论被用来解释许多社会心
理学问题,社会心理学第一次拥有了改造社会的理论。社会学习理论认为,不仅
加诸于个体本身的刺激物可以让其获得或失去某种行为,观察别的个体的社教化
在
学习过程也可以获得同样的效果。Bundura认为观察学习是规则和创造性行为的
主要来源,通过对社会榜样的模仿、学习,个人可以做出创造性的反应
\cite{bundura1977slt}。

X理论和Y理论(Theory X and Theory Y),是管理学中关于人们工作源动力的理
论,由美国心理学家Douglas McGregor1960年提出的。这是一对完全基于两种完
全相反假设的理论,X理论认为人们有消极的工作源动力,而Y理论则认为人们有
积极的工作源动力。根据X理论,企业管理的唯一激励办法,就是以经济报酬来
激励生产,只要增加金钱奖励,便能取得更高的产量。同时也很重视惩罚,认为
惩罚是最有效的管理工具。McGregor认为X理论是极为片面的,这种软硬兼施的
管理办法,其后果是导致职工的敌视与反抗。针对X理论的错误假设,他提出了
相反的Y理论。Y理论指将个人目标与组织目标融合的观点,与X理论相对立。通
过扩大工作范围,尽可能把职工工作安排得富有意义并具挑战性,工作之后引起
自豪,满足其自尊和自我实现的需要,从而使职工达到自己激励。只要启发内
因,员工就可以实行自我控制和自我指导\cite{mcgregor2006hse}。

\subsection{过程理论}
相对内容理论,过程理论更关注动机因素的作用过程。Vroom在1964年提出了期
望模型。该模型包含三个要素:效价、工具性和期望。人的工作动机来源于效价
(对具体结果的偏好)和期望的双重作用。Vroom将其定义为二者的乘积,这两
个因素增大任何一个,都可以提高人的动机。当人们达到了预期的结果后,该结
果则体现出工具性,即该结果是为了达到另一个新的合意结果而达成的。如果新
的结果无法达到,即当前结果的工具性不强,则反过来有影响到动机本身\cite{vroom1964wam}。
Vroom的理论清晰地反映了个体工作动机的差异。

Locke从20实际60年代中期开始从事目标设定理论的研究。该理论认为绩效与目
标是有直接关系的。易达成的目标总是伴随着较低的绩效,而模糊的目标也很难
提高绩效水平。目标通过三种方式影响绩效。首先目标将人的注意力聚焦于目标
相关的活动,人所做的努力大部分是为达成目标;其次目标具有维持努力程度的
作用;最后明确的目标有助于人们会想起相关的知识和经验\cite{locke1990tgs}。

波特-劳勒模型从跟本上解决了工作满意度与绩效关系的问题。他们认为,一个
人在作出了成绩后,得到两类报酬。一是外在报酬,包括工资、地位、提升、安
全感等。按照马斯洛需求层次理论,外在报酬往往满足的是一些低层次的需要。
由于一个人的成绩,特别是非定量化的成绩往往难于精确衡量,而工资、地位、
提升等报酬的取得也包含多种因素的考虑,不完全取决于个人成绩,所以二者并
非直接的、必然的因果关系。另一种报酬是内在报酬。即一个人由于工作成绩良
好而给予自己的报酬,如感到对社会作出了贡献,对自我存在意义及能力的肯定
等等。它对应的是一些高层次的需要的满足,而且与工作成绩是直接相关
的,“内在报酬”与“外在报酬” 经过“所理解的公正报酬”的调节,最终形成了满
意度。也就是说,一个人要把自己所得到的报酬同自己认为应该得到的报酬相比
较。如果他认为相符合,他就会感到满足,并激励他以后更好地努力。如果他认
为自己得到的报酬低于“所理解的公正报酬”,那么,即使事实上他得到的报酬量
并不少,他也会感到不满足,甚至失落,从而影响他以后的努力
\cite{porter1968maa}。 

\subsection{公平和程序公正理论}
公平理论是由Adams于1965年提出的:员工的激励程度来源于对自己和参照对象
的报酬和投入的比例的主观比较感觉。当以下情况发生的时候,员工会感到公平:
$$\frac{\text{本人获得的结果}}{\text{本人的投入}} = \frac{\text{本人认为的他人获得的结
  果}}{\text{本人认为的他人的投入}}$$
否则,当比例不相等是,不论是大于还是小于,本人都会通过一定的努力来调整
比率的值力图使其相等。这种努力就可以认为是工作的动机。努力的方式包括改
变自己的投入、改变自己的所得、扭曲对自己的认知、扭曲对他人的认知、改变
参考对象、改变目前的工作这几方面\cite{adams1965ise}。

Festinger提出的认知失调理论是另一个非常重要的动机理论。认知失调是一个心理学上的名词,用来描述在同一时间有着两种相矛盾的想法,因而产生了一种不甚舒适的紧张状态。更精确一点来说,是两种认知中所产生的一种不兼容的知觉,这里的“认知”指的是任何一种知识的型式,包含看法、情绪、信仰,以及行为等。认知失调的理论表示相冲突的认知是一种原动力,会强迫心灵去寻求或发明新的思想或信仰,或是去修改已在心里存在的信仰,好让认知间相冲突的程度减到最低。已有实验试图去量化此一理论上的趋动力\cite{aronson1969tcd}。

\subsection{内源性动机因素}
动机因素对组织来说至关重要,正如McConnell指出的:”动机是一种软因素,它
难于量化而且总是排在那些次要而容易测度的因素之后。每个组织都知道动机很
重要,但是很少有人想到利用动机。很多日常的管理活动小处精打细算、大处大
手大脚,为了改进一些不重要的管理思想或者节约一些成本就把动机因素所能带来的
巨大价值白白忽略掉了“\cite{663801}。

尽管相关理论和学说众多,但是几乎所有的相
关学者都不约而同地把动机分为两大类:内在动机
和外在动机\cite{sansone2000iae}。内
部动机的主要特征为对活动本身的注意和兴
趣,而外部动机的主要特征为关注外在的奖励,外在认同和外在的指导
\cite{collins1999mac}。
对希望采取的行为进行外部激励是外部动机的特征。当员工只有在获得物质回报
的时候才愿意和他人分享知识时,外部动机就占据了主导地位。而内部动机来自
任务本身,并且以特定活动的内部激励为先决条件。而且,行为并不因报酬的缘
故而发生,而是其活动使人有成就感。严格来说,内部动机不能自发产生;只能创造合
适的发展环境。大前提是任务的设计,以唤起内部动机。
\cite{KaiMertins2003}



然而,关于哪种动机因素能够更好地激
励人们完成工作这个问题一直以来都没有确定的答案。从经济学的角度来看,个
体会对外界刺激作出反应,而正向的刺激可望获得正向的结果。在经济学中一个
中心理念是激励能够提升努力的程度和绩效水平,并且在研究中得到证实
\cite{388530120001201}。虽然经济学家们同时也承认内部动机的存在,但是内
部动机相对来说更难为管理者把握,结果也更加不确定,因此管理者在实际上更
偏重于建立各种奖惩制度和规章\cite{argyris1998ee}。心理学领域给出的研究
论断则正好相反,研究表明内部动机可以有效地提升员工的绩效,达到很多正向
激励难以实现的效果。内部动机有助于解决企业的多重任务问题\cite{gibbons1998io}\cite{holmstrom1991mpa}\cite{prendergast1999pif},即员工只关注
于与奖励直接挂钩的工作,而对那些与奖励没有直接关系的任务视而不见。Austin指出:企业在很大程度上是依赖
于内源因素而存续的\cite{austin1996mam}。很多研究表明,在有激励的情况下工作的
人反而没有那些不受任何激励的人做得好
\cite{Deci1975}\cite{wilson1981aas}\cite{kruglanski1971eei}\cite{lepper1973ucs}
。例如:一些学者通过对软件工程师的动机因素进行研究发现,
他们的工作动机并不在外界的奖励和承认方面,而是享受工作本身的乐趣,例如技术上的
新发现或者是挑战技术难题等\cite{1252263}\cite{1125221}。
激励可以改变人的行为,却不能改变人的态度。


外部动机与内部动机并不是互不干涉的,而是常常交织在一起同时做用的。这种
作用被称为群集效应(Crowding Effect)。如果外部动机对内部动机有正向的
影响,则称为挤入效应,反之称为挤出效应。从已有的文献来看,对挤出效应的
研究数量远大于对挤入效应的研究。Kruglanski发现激励
可能在实际上降低绩效水平,尤其是从长期来看,反向强化的作用更加明显
\cite{Kruglanski1978}。Deci等进行了一组全面的实验验证了:所有的奖励,
不论是实物奖励还是期望奖励,都严重地降低了自我选择的内源性动机。尽管奖
励改变了人们的行为,却阻碍了自我约束发挥效用\cite{deci1999mar}。即使员
工开始是因为内在动机而从事工作,当外部的奖励
反复出现时,个体很容易忽视那些重要的内在价值、
需要和道德因素\cite{deci2000agp}。当外部
激励因素强化到一定程度时,个体开始完全将外部动机作为惟一的动机,而取代
内部动机\cite{kasser2002hpm}。在外部动机激励下的员工更倾向于重复已经做
过的工作,而不愿意从事更具创新性的工作\cite{amabile1998kc}\cite{schwartz1993cad}。在组织中,同奖励随之而来
的还有更严格的监督,更频繁的评审以及更激烈的竞争\cite{Kohn1993},同样
损害了内源动机\cite{deci1985ima}。尽管外源性动机会损害内源动机
已经被多次证明,但是仍有研究证明了挤入效应的存在。Eisenberger和Cameron分析认为:通过恰当的激励手段可
以降低外源性动机的负面影响,同时提高个体的创造能力;外部动机
和内部动机之间可以有不显著甚至是正向的关联。
\cite{eisenberger1996der}。Vansteenkiste等认为这是由于尽管实
物性奖励增强了外部控制感,会导致对自主性产生负面影响,但是任务性奖励却对个体的能力进行了
肯定,从而促进了其兴趣水平,因此个体反而会从外部激励中获得正向反应\cite{vansteenkiste2003ccr}。Eisenberger和Armeli发现,对于非
创造性任务非显著绩效的报酬奖励会降低内在动
机,但对于创造性任务高绩效的报酬奖励会增强内
在动机\cite{eisenberger1997csr}。奖励可以增加个人的自决能
力,使人感到着我们不再依赖于外界的施舍,从而提高了人的自主性\cite{eisenberger1999dpp},\cite{eisenberger1999eri}。从以上内容可以看到外部动机与内部动机
之间的复杂关系,看似矛盾的结论实际上说明:适当的外部激励可以提高人的动
机,而不适当的外部激励会降低人的动机。



对于内源性动机,不同的学者给出了不同的定义。Hull认为习得性行为均源自基
本需要的满足,凡是内在驱动的行为都跟当事人的基本需要有密切的关系,许多
场合正是这些基本需要导致了行为的内在驱动的\cite{hull1943pbi}。White认为:人们经常参与某些活动只不过为了
体验效能或能力\cite{white66wmr}。类似的,deCharms提出一个人天生有一种
原始驱动的本性,它关乎自身从事活动的缘由\cite{decharms1968pci}。这类观
点认为内在动机主要与人们的某些精神需要相联系\cite{Kanfer1990}。另外一些学者主要从个体的行为归因来对内在
动机进行界定。他们认为,如果个体认为某活动或
工作是自己本身愿意去做的,那么其主要的工作动
机就是内在动机。受到内在动机激励的员工,他们
往往觉得自身能力在工作中得到了发挥,具有很大
的工作自主性。他们往往不是追求一些明显的外在
报酬而是由于对活动或工作本身感兴趣而产生强烈
的工作动机\cite{chenandwu2008}。Hackman和Oldman将内在动机定义为员工在
工作过程中通过自我激励而达到的有效程度,这种程度越高,那么员工的工作体
验越好\cite{hackman1975djd}。

内在动机的影响因素很多。Amabile对内在动机的相关研究进
行了总结,确定了内在动机的五种主要构成要素,
它们分别是自我决定、胜任感、工作参与、好奇心
和兴趣\cite{amabile1996cc}。。Hackman 和
Oldham指出,技能多样性、工作完整性、重要性、
自主性及回馈性等五种工作特性激励性较高,有助
于提高员工工作动机好\cite{hackman1975djd}。Waterman等人通过实证研究认为,自我决定以及
任务的挑战性与技能的平衡是影响内在动机的前因
变量\cite{AlanSWaterman11012003}。Eisenberger, Jones, Stinglhamber等人的实证
研究表明,对于成就导向的员工来说,高技能与挑
战性工作的结合有助于提高其对任务的兴趣和积极
情绪体验;但对于低成就需要的员工来说则不存在
这种现象\cite{Eisenberger2005}。。Deci和Ryan认为自主性(autonomy)即个体的自由选择性是内在动
机的一个关键性因素。无论对于何种行
为,自主性都只是一个程度问题,是内在性与外在
性连续维度上的某一点。当个体的自主性达到内在
性的最高值,这时候的行为动机就是内在动机\cite{ryan2000sdt}。Guay, Boggiano和Vallerand研究表明,
自主性有助于个体自我能力评估的提高,进而提高
个体的内在动机。当自主性被破坏时,员工的内在
动机也随之降低,直接导致其效率降低成本增高,
这种现象在需要创造性和灵活性的任务中尤其明显
\cite{FredericGuay06012001}。

自我效能也是影响内在动机的重要因素。自我
效能是一种个人信念,即个人可以以某种方式达成某个目标
\cite{ormrod2003epd}。自我效能高的人会更加努力地投入某项工作,工作热情
也会更加持久\cite{schunk1990gsa}。研究表明,高自我效能感的员工会对自己的
能力表现出更强的信心也在工作过程中拥有更高的
内在动机。如果员工自我效能感较高,由此激发的
内在动机会促使其选择充满挑战性的工作;相反,
如果员工对自己顺利完成某项任务的能力表示怀
疑,自然倾向于逃避相应工作\cite{David2007}\cite{bandura2003nse}。Brown 和Dutton 的研究发现,
具有较低自我效能感的个体往往对失败具有更强的
消极体验,而且往往倾向于把自己某一方面的失败
泛化到生活中其他非相关领域。而具有较高自我效
能感的个体对于自己在某种特定情境下的失败有清
晰的情境认知,不会轻易泛化到生活中其他情境,
因而其消极体验就相对较弱\cite{brown1995tvc}。即使工作有比较大的难度,
承担起来有一定的压力,自我效能所激发的自信也能抵消这些不利因素。

个人目标也会影响内源性动机因素。个人目标可以大致分为两类:一类主要是通过某项工作任务的完成来展示
自己的工作能力,从而得到领导或同事对其正向的
评价,此类目标属于绩效型目标(performance goal)
或任务导向型目标(task-oriented goal);另一类工
作目标主要是通过从事某项活动来发展和提高自己
的工作技能与能力,此类工作目标属于学习型目标
(learning goal)或掌握型目标(mastery
goal)\cite{LairdJRawsthorne11011999}\cite{Pajares2000}。Utman证明了学
习型目标同内在动机的关系更为密切:学习目标型
员工在工作过程中会体验到更多乐趣,具有较高的
工作满意度;而任务目标型员工则体验到较少甚至
体验不到乐趣,相反他们体验到的是更多的压力\cite{ChristopherHUtman05011997}。Potosk和Ramakrishna研究发现,学习型目
标与自我效能感呈显著的正相关,任务型目标则与
自我效能感呈显著的负相关,学习型或任务型目标
的设置可能通过自我效能感的中介作用对内在动机
产生间接影响\cite{potosky2002mru}。Dweck发现一个人未
来的成功与天分和当前的成就没有太大关系,而是与个人的目标紧密相关
\cite{dweck2000stt}。她提出了两种目标倾向:成绩目标取向(performance
goal orientation),致力于通过寻求关于自身能力的肯定性评价、避免否定性
评价来展示自身能力的高水平;学习目标取向(learning goal
orientation),致力于发展掌握新技能、适应新环境的能力。Elliot进一步提出了三因素的目标取向模型,即把成绩目标再分成两类:进取(proving)或趋近(approach)型成绩目标和回避型(avoidance)成绩目标\cite{elliot1996aaa}。成就趋向型目
标个体关注的是赢得对能力的正向评价,成就逃避
型目标个体逃避对能力的负面评价。只要是希望成功而不是逃避失败,不管个体
的目的是为了掌握任务本身还是为了获取一个好的
结果,都可以提高内在动机水平 。

除了个体的动机因素之外,动机因素还包括个体间的动机因素\cite{Hackman1975}\cite{Hackman1980}。个人动机因素在个人独立工作的时候
起作用,而人际间的动机因素只在个体间相互协作的时候才发挥作用。个体间的动机
因素包括能力直觉和相属感。能力知觉是指一个人相信自己做好了某事,或者能够做好某件事情的
程度\cite{harter1981nsr}\cite{bandura1982sem}。能力知觉越高,个体就越
愿意向组织贡献自己的劳动。反之,如果一个人觉得自己的工作不受到别人的重
视,他的工作动机就会显著下降\cite{Hertel2003}。正向的反馈可以有效提升
工作动机,而负向的反馈则会削弱工作动机。


\subsection{实践社区知识共享的动机因素}


知识资产已经成为企业发展最重要的驱动力。Leventhal和March将其描述为“由
组织内部的个人或团体持有的一组特殊的竞争力\cite{levinthal1993ml}。知识管理的重要目标是力图在最恰当的时间将最恰当的知识传递给最恰当的人,
以期能做出最好的决策\cite{Petrash1996}。广义地讲,知识共
享就是个体之间交换知识并协同产生新知识的过程。Alavi指出:知识创造和编码不一定能够改进企业绩效和企业价值
\cite{alavi2000mok}。知识必须被他人共享才能
最大化其价值。然而,组织中的知识共享并不是容易的事情。Szulanski分析了
知识共享的各个阶段,提出了共享的“粘性”问题:知识共享并不是自动发生的,
而且在共享过程中存在着许多阻碍因素\cite{szulanski2000pkt}。因此,研究
知识共享的动机因素,已经成为当前的研究热点。Choi等指出:推动有效的知识
共享社会因素(信任、奖励等)远比技术因素重要
\cite{SueYoungChoi10012008}。因此,越来越多的学者开始从动机因素的角度
研究如何推动知识共享。

在不断进步的信息技术的支持下,虚拟社区这种松散的组织形式越来越多地涌现
出来。由于虚拟社区本身有许多特点不同于实体社区,虚拟社区中知识共享的动
机因素也发生了一些变化。当前对于虚拟社区中知识共享的动机因素研究主要以
这两类虚拟社区为目标:一种是开源软件开发社区;另一种是开放的内容生产社
区。前一种以Apache和Sourceforge等为代表,后一种以维基百科为代表。而对
于开源软件社区的研究数量有远远大于对内容生产社区的研究,研究的结果也更
为复杂。

知识共享可以具有多种形式。例如以公共品渠道发布
知识,非经济的方式转知识以及知识交易。传统的研究认为:在这三种方式中,只有知识交易使
能够成为长期稳定的知识传播方式。尽管前两种方式在现实中并不少见,但是正如达
文波特指出的:“那些认为知识无需要经济激励就可以自
然扩散的想法是乌托邦式的;在没有回报预期的条件下人们不太可能贡献出具有
价值的知识”\cite{davenport1998wko}。每个理性人都会根据自己的判断,对于
自身所掌握的知识估计其价值然后进行交易。知识的拥有者通过知识转移进行“排他性”控制获得经济
利益,交易结果是实现知识转移。\cite{zhoubo2006}。诸葛海等总结了组织中
知识共享的动机因素,除了利他主义外,主要包括:希望获得物质奖励、希望获
得组织中他人的赞扬、认同、名誉等以及希望能与组织中的其他成员互惠互利,
在自己需要知识的时候可以从他人那里获取三个动机因素\cite{Zhugea}。企业
可以通过设计有效的奖励机制(不论是基于个人贡献还是基于组织绩效),从而
达到促进知识共享的目的\cite{Lee2007}。在组
织中还同时存在着大量的无形回报形式的共享
行为。组织成员的互惠互利实际上可以看作是一种知识分工。一个人的时间和精
力都是有限的,即使是自己的专业知识要想全部掌握也几乎是不可能的。因此,
通过与他人的交易,通过互换的手段来达到目标就成了组织成员的必然选择。回
报不一定会马上发生,知识的出让方确信自己会在将来的某一时刻可以从别人那
里获得相应的回报,以弥补自己出让知识的损失。对于追求名誉的人来说,尽管
名誉本身不能为其带来实际的利益,但是名誉可以保障工作稳定,帮助职位的晋
升以及提供同其他专家交流的机会。而且,如果一个人拥有知识渊博且乐于分享
这样的名声的话,会有更多的人与他进行互利的知识交易。因此,应该在企业内部建立高
效的知识市场,提高知识共享的效率\cite{Andreas2007}。外部因
素以上观点实际上反映
了知识共享的外源性动机,并且长期作为知识共享的主要动机。



以开源软件社区为代表的虚拟实践社区的兴起对用外部动机解释知识共享行为的
论断带来了挑战。虚拟社区中的成员大部分都不会从这些社区中获得任何收益,
社区也不会雇用这些人员。他们是完全自愿为社区做义工的
\cite{lerner2002sse}。一些学者开始从内源性动机那里寻找答案,认为他们的
动机来源于利他主义以及对自己心理需求上的满足,或者是追求某种道德准则
\cite{Wu2007}。其中,纯粹的利他主义被一些学者视为人们参与开源社区的主
要动机。利他主义是人性的一部分,每个人都或多或少地以某种形式表现出来。
开源社区中的参与者其实是非常乐于向需要的人伸出援手的,同时他们也会想方
设法回报帮助他们的人。在他们的世界里,决定社会地位的因素不是个人拥有什
么,而是个人向社会回馈了什么\cite{raymond1999cab}。然而这种观点却受到了经济学家的质疑,他们认为这不足以解
释为什么有如此数量巨大参与者投入到这种活动中并且不计报酬,尽管收益者中
有很多人可以支付的起这些报酬。开源社区在本质上和其他产业并无不同,但是却很
难在其他产业 中见到这种普遍的自愿行为\cite{schmidt2002pso}。Hars等人认
为尽管人们为社区免费贡献自己的成果,其实是希望通过间接的方式获得回报。
这实际上是一种投资行为\cite{hars2002wfm}。Hars等将人们参与社区的动机分为三类:
\begin{enumerate}
\item  希望从相关的实体产品和服务中获得收益。比如开源软件社区中许多开
  源软件的开发实际上增加了相关实体企业的软硬件销售收入。通过软件开源,
  而从服务上获得利润已经成为越来越多的企业的盈利模式。
\item 提升个人能力。通过参与实际的开源项目,以及同社区中的专家进行交
  互,学习并掌握更多的工作技巧可以提升个人的竞争力,进一步谋求更好的职
  业和更优厚的待遇。Ye and Kishida认为学习是人们参与虚拟社区的主要原因。
  社区中的开发者和用户构成了一个实践社区。在不断的共享知识与协同工作的
  同时,社区中每个成员都学到了新的知识。尤其是那些新手更有机会从专业人
  员那里习得经验和诀窍。社区成员从社区学到的东西越多,就越会深入参与各
  种活动中去\cite{1201220}。
\item 自我推销。虚拟社区创造出的“产品”由于其自由性必然会受到大量的关
  注,这也成为了一些人自我展示的舞台。通过这个平台参与者可以向外界展示
  自己的能力和技术,从而在将来获得某种回报。
\end{enumerate}
Lerner and Tirole对外源性动机作出了全面的总结:只有当参与者能够从社区
活动中获得净收益的时候才会为社区贡献自己的力量。这种净收益包括即时的收
益和延迟的收益\cite{lerner2002sse}。这样,社区参与者的动机因素又重新
纳入到经济学的轨道中来。

希望通过参与知识共享而获得声名、机会等动因确实获得了一批实证研究的支
持,但是这里仍然有一个问题无法解释:为什么那些领域中的顶级专家也会参与
到社区中来?显然这些专家并不需要通过这种形式来证明自己,他们已经获得了
足够的名声。另外,如果社区中的参与者真的是受到声名的驱动,那么在社区中
应该有许多的挑战项目领导者权威的行动,或者干脆脱离社区另立门户以获得领导地
位\cite{WeberWeberSteven}。但是现实中这样的情况却很少发生,除非是成员
之间因为理念不合才会分道扬镳。另一方面,如果共享可以换来名声,那么共享
的内容本身将会成为影响参与者动机的重要因素,对于那些不那么重要的、几乎不能带
来任何名誉的知识自然会没人有动力共享,而对那些“有价值”的知识,应该会有很多人
竞争。但是一些研究发现,有很多人恰恰就是在从事共享那些“价值较小”的知识,例
如一些人会为开源软件撰写文档,对程序本身给与评论和描述,以这种方式向社区贡献。Rossi指出,这证明名
声已经不再是社区参与者主要的动机因素了\cite{Rossi2004}。

另一种广为接受的动机来自于用户自身的需求,参与者存在对他人知识的需求。
他首先贡献自身的知识,以换取更多的人分享知识。这在开源软件社区中尤为明
显,例如有的人创建某一功能的初始代码,后来者会解决程序中的错误或者改进
该项
功能。由于虚拟社区的开放性,任何人都可以自由地在用户和贡献者两种角色之
间自由转化,造就了群策群力共同完成目标的模式。在这种情况下,个人可以借
助集体的力量,以较少的投入获得高质量的回报。而且,同在知识集市中常出现
的知识垄断现象不同,创新的共享者并不将创新结果的采用者视为对手,反而欢
迎别人体验、利用自己的成果。开放的成果越多,越能吸引更多的人
参与进来。而不同背景的、异质的参与者带来了不同的用户需求,反过来又促进
了更多有用的成果的产生。同样,那些从事文档工作或者热衷于回答各种用户提
问的热心人实际上也是希望能够通过这些活动获取有价值的信息
\cite{lakhani2003fos}。



众多的实证研究表明,外部动机确实在虚拟社区中,尤其是开源软件开发社区中
扮演了重要角色。但是,我们很难说到底那种因素占主导地位,也没有证据表明
一个参与者仅仅因为某一个动机而加入到社区中。更重要的是,象维基百科这类
社区是在几乎没有外部激励的刺激下,依靠作者的热情而不断发展壮大的。因
此,内源性因素对虚拟社区实际上是至关重要的。

自我效能是一个重要的内部动机。自我效能有助于激发个人与组织中的其他成员
共享知识。Hsiu-Fen通过对台湾50家大企业172个员工的
调查,证明了自我效能越高的人知识共享的态度越积极,意图也更加明显
\cite{Hsiu-FenLin04012007}。Wasko等研究了网络上人们共享知识的动机因
素,认为在专业水平和共享知识的程度呈正相关关系。一个人的专业水平越高,
共享的动机也就越强。同时,知识的保有量也是共享的重要影响因素。一个人占
有的知识越多,他越是可能向其他人共享\cite{1631335820050301}。不论是专
业水平还是知识的保有量,都决定了一个人自我效能的高低。Kankanhalli等研
究了企业员工向企业知识库共享知识的动机因素,也
认为如果组织中的成员认为自己的知识不会给组织带来什么实际用处的话,共享
自己知识的动机就会降低\cite{1631337020050301}。Bock等调查了4个大型公共
组织中的476个雇员,发现员工在共享知识前都会作出评估,给出一个“期望贡
献”,如果员工认为自己贡献的内容能够对企业绩效有提升的话,知识共享的态
度会更加积极\cite{631757820020401}。Hsu等研究了虚拟社区中知识共享的动
机因素,也得到了类似的结论:自我效能是知识共享的重要动机\cite{Hsu2007}。

个人义务是另一个经常出现的内部动机。个人义务和利他主义看上去很像,都不
是个人心中的一种信念:即我应该帮助其他人。但是与利他主义不同的是,个人
义务不是从个体心理活动的角度,而是从社会交换的角度表现出来的。社会交换
会对个体产生非特定的义务。例如一个新进员工收到了老员工的帮助,新进员工
会感到有义务在将来设法回报老员工。这种义务会促进彼此之间的信任,从而增
强人与人之间的社会联系\cite{gouldner1960nrp}。

利他主义尽管收到了一定程度的批评,但是仍然是知识共享非常重要的内部动机。
Dixon指出,我们本质上都是乐于帮助别人的\cite{dixon2000ckc}。利他主义者
在知识的共享过程中获得心理上的满足。这些心理上的满足包括:通过展现自己
的专业技能来显示自身的价值甚至是对组织的影响力;或者是满足自身的自豪感
以及对组织的归属感;或者是显现自己在组织中的竞争力
\cite{443078119941201}。Constant等研究了组织中异地员工的共享行为
\cite{44348771996}。研究发现尽管相隔遥远且彼此不认识,仍然有许多人愿意
向陌生的信息寻求者提供有用的建议。而且问题越是分散,参与的人数就越多,
共享的行为也就更加明显。Ye等人以虚拟社区为研究对象,发现乐于帮助他人是虚拟
社区中知识共享最重要的因素,其显著程度大大超过其他动机因素\cite{Ye2006}。

个人态度与主观规范也是知识共享动机研究中经常提及的动机因素。根据理性行
为理论\cite{fishbein1975bai}和计划行为理论\cite{Ajzenbw2002},人的态度
决定其意图,而主观规范是个人决定做或者不做某事时感受到的社会压力。
Jarvanpaa提出人们共享知识的行为反映了他们对共享的偏好态度
\cite{Jarvenpaaee2000}。Kolekofski指出知识共享的态度是影响共享行为的重
要因素。相反,如果人们对知识共享的积极性不高,那么他们就不太愿意和别人
共享知识\cite{cabrera2002ksd}。De Long等的研究表明在不太崇尚知识共享的企业中,主观规范成
了影响知识共享的阻碍因素\cite{DeLong2000}。如果一个企业的企业文化不太能接受错误的产生,
那么知识共享会受到很大的削弱。

大量的研究从不同角度论证了外部动机和内部动机对知识共享的影响。同时我们
也常常看到外部动机与内部动机相互影响。Christensen认为组织中的成员共享
知识的动机实际上是内部动机和外部动机的混合体,进而构成了一个连续体,从
最纯粹的机会主义(外部动机)到完全的利他主义(内部动机)\cite{Christensen2005}。Kwok等人对P2P社区中的共享行为进行了分析,提出
了四种动机因素:奖励、个人需求、利他主义和名誉,这几种因素综合作用构成
了社区中的共享动机\cite{kwok2004ksc}。Roberts等人通过对Apache社区的研
究发现开发者的动机不是独立的,而是相互影响的。一种动机可能会提升另一种
动机水平,同时降低其他动机的水平。动机作用的方式也可能互不相同:有的动
机升高开发者的共享程度而另一种动机可能会减弱开发者的共享程度
\cite{2151758320060701}。Chiu等人的研究也证明了不同的动机因素有不同的
作用。他们发现组织的预期收益会对共享知识的数量和质量邮政向影响,而个人
的预期收益同知识共享的数量呈负相关。社会交互和互惠互利可以增加知识共享
的数量,对知识的质量却没有显著影响。而影响知识共享的信任因素同样作用不
显著\cite{Chiu2006}。可见,外部动机与内部因素同时作用是非常普遍的,而
且不同的因素作用的方式也不尽相同。

动机因素同时也容易受到外界因素的影响,从而在不同的时刻可能呈现出不同的
动机。Ardichvili等发现企业员工共享知识的动机同他们对知识本身的态度有关
\cite{ardichvili2003mab}。如果员工认为知识是公共产品,那么知识共享在企
业中就会频繁进行。这时候员工会认为共享知识是自己的道德义务,是为了企业
的利益而共享的。另外资深的员工和高级管理者还常常基于对自我的认同而共享
知识。这些人认为自己已经到了一个回馈的阶段,从而愿意把在组织中学到、掌
握的知识同其他人分享。Bock等人发现对知识共享的态度以及主观规范以及组织
的风气影响者知识共享的意愿,而员工对知识共享的态度受到互惠关系的影响,
主观规范受到企业文化和自我价值的影响\cite{1631336620050301}。对知识共
享的态度越积极,主观规范越强大,那么共享的动机也就越是明显。Gottschalg
等观察到企业类型决定了哪种动机因素占主导地位\cite{Gottschalg2006}。如果一个企业是基于管理优先
的理念,那么企业就会强化激励、监管和控制,从而外部动机会占主导地位。而基于
能力有限的企业会将企业视为一个社群,每个成员力求从中获得认同,这种情况
下内部动机成为主要因素。Song等将知识共享分为了两种形式:请求型与自愿型。
请求型的共享者期望将来能够获得对方的帮助,而自愿型的共享者则没有这种要
求。他们的研究发现,工作的结构对请求型的知识共享显著影响,越是需要沟通、
交互的工作,请求型的知识共享越频繁。但是工作结构对自愿型的知识共享去没
有显著影响。同时,开放的、鼓励交流的企业文化对自愿型的知识共享有很大的
促进作用,但是对请求型的知识共享几乎没有影响\cite{4439047}。Huang等人
研究了中国知识共享的动机因素,发现文化特点对共享动机有显著影响。在中国
人的观念里,“脸面”是非常重要的。如果共享的知识可能是错误的或者是初级
的,那么共享行为将被认为是一件“丢脸”的事情。研究证明,如果共享者认为知
识共享有可能导致丢脸,那么共享的意愿将会降低,反之,如果共享者认为知识
共享是一件“长脸”的事情,那么他共享的意愿会提升\cite{huang2008ipa}。

除了个人动机会影响知识共享外,个体间的动机对知识共享也有非常重要的影响。
随着对知识协同的行为越来与普遍,知识共享行为已经从个人行为发展到群体间
协同共享,因而研究个体间的动机也就变得越来越重要了。当前对个体间的动机
因素研究较少。Xiaoquan等研究了维基百科中知识共享的动机因素,发现如果一
个条目被修改的次数越多,则条目的原始作者贡献的动机会降低
\cite{Zhang2006}。群体的反应对原作者带来了负效应。Tong等人通过对在线反
馈系统的研究发现,评论的数量会对评论者带来不同的影响。最开始评论数量较少
的时候评论者的动机最强,因为这使得评论能为大多数人看到,所以往往是最有
价值的。随着评论的数量增加,评论者评论的意愿开始下降,因为他们担心自己
的评论会淹没在众多的评论中\cite{tong2007uii}。

另外,动机因素因素与实际行为可能存在偏差,而这一点在许多文献中被忽略掉
了。 Kuo等人研究了虚拟社区中知识共享的动机与行为,发现了二者的不一致性。
在许多文献中认为的共享意图同共享行为正相关的结论在他们的研究中收到了否
定,数据显示两种因素的关系并不显著。仅有共享的意图并不一定导致共享的行
为\cite{Kuowiley2008}。在Kuo等人的另一项研究中还揭示了知觉行为控制与共
享行为的不一致性,同时控制因素同共享行为也没有必然联系\cite{Kuo2008}。

\section{研究目标}
实践社区作为一种重要的组织形式,连接了组织内部与组织之间的知识共享与协
同活动。以往的研究将注意力集中在社区成员知识共享的动机因素研究上,而对
知识协同的动机因素研究相对缺乏。随着大量的知识协同行为的产生,理解协同
的动机、找到影响动机的因素对于组织的管理者来说是至关重要的。本文以虚拟
社区为研究目标,对实践社区开展调查研究,运用心理学和组织行为学的基本理
论,通过对主客观数据的分析,从而找到实践社区中知识协同的动机因素。同时
利用社区的数据进行实证分析、验证理论假设,从而提出实践社区知识协同的动
机因素模型。
\section{研究内容}
当前对实践社区的研究主要以开源软件社区和内容创造社区为研究对象。论文将
以维基百科为研究对象,研究内容贡献者的动机因素,找到影响内容贡献者知识
协同的影响因素。

\subsection{行为主体的内部动机}
行为主体在没有任何外部激励的情况下,自觉自愿地向整个社区贡献自己的知识。
本研究拟通过问卷的形式,调查行为主题进行知识协同行为的动机,并通过实证
研究,验证模型假设。
\subsection{行为主体的协同模式}
实践社区的各种活动可以被记录下来,从而反映出行为主题的行为模式。根据理
性行为理论,每个人的活动反映了个人的理性动机。通过对客观数据的分析与归
纳,整理出实践社区中行为主题的理性动机,找出群体间的行为模式对行为主题
的动机的影响。
\subsection{行为主体的动机-行为研究}
人的动机与其行为可能并不一致。通过对比主观动机因素与客观行为模式,分析
二者之间可能的不一致,进而找到,影响、扭曲行为主体初始动机的因素。对于
在实证研究部分未得到验证的假设,寻找客观数据的解释。

\section{拟解决的关键问题 }
\label{sec:-}


\section{研究方法}

\section{技术路线}

\section{可行性分析}

\section{预期的创新点}

\section{预期目标}

\section{预期的研究成果}

\section{论文工作计划}





\bibliographystyle{unsrt}
\bibliography{../../bibtex/elsevier,../../bibtex/emerald,../../bibtex/chinese,../../bibtex/jstor,../../bibtex/citeseer,../../bibtex/acm,../../bibtex/wiley,../../bibtex/book,../../bibtex/thesis,../../bibtex/ebsco,../../bibtex/old,../../bibtex/ieee,../../bibtex/internet,../../bibtex/ssrn,../../bibtex/apa,../../bibtex/blackwell,../../bibtex/sage,../../bibtex/springer,../../bibtex/MESharp,../../bibtex/taylor}

\end{document}


